%!TEX root = x.tex

\rSec0[pattern]{模式匹配}
\indextext{模式匹配}

\begin{bnf}{Pattern}
    PatternBody PatternAssertion\bnfs
\end{bnf}

\begin{bnf}{PatternBody}
    NullPattern \br
    ExprPattern \br
    BindPattern \br
    SomePattern \br
    EnumPattern \br
    ArrayPattern \br
    TuplePattern \br
    StructPattern \br
    AltPattern
\end{bnf}

\begin{bnf}{PatternAssertion}
    TypeAssertion \br
    IncludeAssertion \br
    CondAssertion
\end{bnf}

\pnum
模式匹配用于检验一个值是否符合特定的\term{模式},以及在符合特定的模式时从中提取某些成分。本节中,$\mathcal{v}$表示值,$\mathcal{p}$表示模式。值符合特定的模式称为这个值\term{匹配}这个模式,记作$\mathcal{v}\mathrel{\Vert}\mathcal{p}$。

\pnum
模式$\mathcal{p}$由\term{模式主体}和\term{模式断言}构成。模式主体规定匹配的结构与操作,模式断言则对值的特征进行断言。一个主体可以带有任意数量的断言。

\rSec1[pattern.null]{空模式}
\indextext{模式匹配!空}

\begin{bnf}{NullPattern}
    \tcode{_}
\end{bnf}

\pnum
空模式能够匹配任意值。匹配成功后,$\mathcal{v}$的值将被丢弃。

\rSec1[pattern.expr]{表达式模式}
\indextext{模式匹配!表达式}

\begin{bnf}{ExprPattern}
    RangeExpr
\end{bnf}

\pnum
$\mathcal{v}$和$\mathcal{e}$必须可比较。$\mathcal{v}\mathrel{\Vert}\mathcal{e}$当且仅当$\mathcal{v}\mathrel{\tcode{==}}\mathcal{e}$。

\pnum
\enternote 表达式模式的语法可能与其他模式产生歧义。此时总是优先解析为其他模式。 \exitnote

\rSec1[pattern.some]{\tcode{some}模式}
\indextext{模式匹配!\idxcode{some}}

\begin{bnf}{SomePattern}
    \terminal{some} Pattern
\end{bnf}

\pnum
\tcode{some}模式匹配可空类型。如果该值为非空,则匹配成功。

\rSec1[pattern.enum]{枚举模式}
\indextext{模式匹配!枚举}

\begin{bnf}{EnumPattern}
    EntityID EnumPatternPayload\bnfq \br
    \terminal{.} Identifier EnumPatternPayload\bnfq
\end{bnf}

\begin{bnf}{EnumPatternPayload}
    ArrayPattern \br
    TuplePattern \br
    StructPattern
\end{bnf}

\pnum
枚举模式匹配枚举项。如果$\mathcal{v}$是$\mathcal{p}$中的枚举项,且其载荷匹配$\mathcal{p}$中的载荷模式(如果有),则匹配成功。

\pnum
枚举模式中的枚举符可以使用匿名枚举符,此时会进行静态推导。

\rSec1[pattern.array]{数组模式}
\indextext{模式匹配!数组}

\begin{bnf}{ArrayPattern}
    \terminal{[} AnyPattern \bnflp\terminal{,} AnyPattern\bnfrp\bnfs\ \terminal{]}
\end{bnf}

\begin{bnf}{AnyPattern}
    Pattern\br
    \terminal{...} \br
    BindKeyword \terminal{...} Identifier
\end{bnf}

\pnum
数组模式匹配序列中的元素。其中\tcode{...}项(称作\term{任意项模式})只能出现至多一次,否则这是一个编译错误。$\mathcal{v}$必须实现\tcode{Sequence},否则这是一个编译错误。

\begin{enumerate}
    \item 如果模式不包含任意项,且\tcode{$\mathcal{v}$.size}与模式中项的数量不相等,则匹配失败。
    \item 如果模式包含任意项,且\tcode{$\mathcal{v}$.size}小于模式中非任意项的数量,则匹配失败。
\end{enumerate}

\pnum
在那之后,将按如下规则依次对$\mathcal{v}$的元素进行匹配。如果每个匹配都成功,则整个模式$\mathcal{p}$匹配$\mathcal{v}$。

\begin{enumerate}
    \item 对任意项模式之前的模式(如果不存在任意项则对每个子模式),$\mathcal{p}_i$匹配\tcode{$\mathcal{v}$[$i$]},其中$i$是子模式的索引(从0开始)。
    \item 对任意项模式之后的模式,$\mathcal{p}_r$匹配\tcode{$\mathcal{v}$[\$-$r$]},其中$r$是子模式从后向前数的索引(从0开始)。
\end{enumerate}

\pnum
如果任意项包含一个绑定,则该任意项匹配到的所有元素将被绑定到相应的标识符上。该标识符将具有与$\mathcal{v}$相同的类型。

\rSec1[pattern.tuple]{元组模式}
\indextext{模式匹配!元组}

\begin{bnf}{TuplePattern}
    \terminal{(} AnyPattern \bnflp\terminal{,} AnyPattern\bnfrp\bnfs\ \terminal{)}
\end{bnf}

\pnum
与数组模式类似,\term{元组模式}匹配元组。但匹配能否成功都是能在编译期决定的。

\pnum
如果元组模式的任意项包含绑定,则该绑定对应的值将具有对应的元组类型。

\rSec1[pattern.struct]{结构模式}
\indextext{模式匹配!结构}

\begin{bnf}{StructPattern}
    \terminal{\{} StructPatternBody \terminal{\}}
\end{bnf}

\begin{bnf}{StructPatternBody}
    StructItem \bnflp\terminal{,} StructItem\bnfrp\bnfs
\end{bnf}

\begin{bnf}{StructItem}
    Identifier \terminal{:} Pattern
\end{bnf}

\pnum
\term{结构模式}对结构进行匹配。如果对于每个对$(k, \mathcal{p}_k)$而言,\tcode{$\mathcal{v}$.$k$}匹配$\mathcal{p}_k$都成立,则整个模式匹配成功。

\pnum
如果结构模式中存在没有匹配的字段,且不包含任意项模式,则这是一个编译错误。

\pnum
结构模式的任意项匹配所有未被其他项匹配的字段。如果该模式被绑定到一个标识符,则其对应的值为这些剩余字段的集合。

\rSec1[pattern.bind]{绑定模式}
\indextext{模式匹配!绑定}

\begin{bnf}{BindPattern}
    BindKeyword PatternBind
\end{bnf}

\begin{bnf}{BindKeyword}
    \terminal{let} \br
    \terminal{let} \terminal{mut} \br
    \terminal{let} \terminal{const}
\end{bnf}

\begin{bnf}{PatternBind}
    Identifier PatternAssertion \br
    SomePatternBind \br
    ArrayPatternBind \br
    TuplePatternBind \br
    StructPatternBind
\end{bnf}

\begin{bnf}{SomePatternBind}
    \terminal{some} AnyPatternBind
\end{bnf}

\begin{bnf}{ArrayPatternBind}
    \terminal{[} AnyPatternBind \bnflp\terminal{,} AnyPatternBind\bnfrp\bnfs \terminal{]}
\end{bnf}

\begin{bnf}{TuplePatternBind}
    \terminal{(} AnyPatternBind \bnflp\terminal{,} AnyPatternBind\bnfrp\bnfs \terminal{)}
\end{bnf}

\begin{bnf}{AnyPatternBind}
    PatternBind \br
    \terminal{...} Identifier\bnfq \br
    NullPattern \br
    ExprPattern
\end{bnf}

\begin{bnf}{StructPatternBind}
    \terminal{\{} StructPatternBodyBind \terminal{\}}
\end{bnf}

\begin{bnf}{StructPatternBodyBind}
    StructItemBind \bnflp\terminal{,} StructItemBind\bnfrp\bnfs
\end{bnf}

\begin{bnf}{StructItemBind}
    Identifier \terminal{:} PatternBind \br
    Identifier
\end{bnf}

\pnum
绑定模式可以匹配任意值。匹配成功后,该标识符将作为一个变量插入到当前作用域中。

\pnum
如果绑定使用关键字\tcode{const},则这是一个常量绑定。参见~\ref{qual.const}。

\pnum
绑定模式可以使用简写,表~\ref{tab:binding-shorthand}~列出了一些常见的简写形式。

\begin{floattable}{绑定简写与其完整形式}{tab:binding-shorthand}{l|l}
    \topline
    \tcode{let [a, b]} & \tcode{[let a, let b]} \\
    \tcode{let (v, _)} & \tcode{(let v, _)} \\
    \tcode{let [x, ...y]} & \tcode{[let x, let... y]} \\
    \tcode{let \{ x \}} & \tcode{\{ x: let x \}} \\
\end{floattable}

\rSec2[pattern.alt]{选择模式}

\begin{bnf}{AltPattern}
    Pattern \terminal{|} Pattern \br
    AltPattern \terminal{|} Pattern
\end{bnf}

\pnum
选择模式同时匹配多个模式。如果其中有模式匹配成功,则整个模式匹配成功。匹配将从左到右进行。

\pnum
选择模式各分支可以包含绑定模式,但绑定的变量必须在每个分支中都出现且类型相同,否则这是一个编译错误。

\rSec1[pattern.type]{类型断言}
\indextext{模式匹配!类型断言}

\begin{bnf}{TypeAssertion}
    \terminal{is} Type \br
    \terminal{:} Type \br
    \terminal{as} Type
\end{bnf}

\pnum
\term{类型断言}对值的类型进行约束。它包括以下类型:

\begin{itemize}
    \item \tcode{is T}要求值的类型与\tcode{T}完全一致。
    \item \tcode{: T}要求值的类型是\tcode{T}的子类型。
    \item \tcode{as T}要求值的类型能够转换到\tcode{T},无论显式或隐式。
    \end{itemize}

\rSec1[pattern.include]{包含断言}
\indextext{模式匹配!包含断言}

\begin{bnf}{IncludeAssertion}
    \terminal{in} Expression
\end{bnf}

\pnum
\term{包含断言}要求值包含在某个集合$\mathcal{e}$中。如果\tcode{$\mathcal{v}$ !in $\mathcal{e}$},则匹配失败。

\rSec1[pattern.cond]{条件断言}
\indextext{模式匹配!条件断言}

\begin{bnf}{CondAssertion}
    \terminal{if} Expression
\end{bnf}

\pnum
\term{条件断言}要求值满足某个条件。