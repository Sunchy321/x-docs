%!TEX root = x.tex

\rSec0[pattern]{模式匹配}
\indextext{模式匹配}

\begin{bnf}{Pattern}
    PatternBody PatternAssterion\bnfs
\end{bnf}

\begin{bnf}{PatternBody}
    NullPattern \br
    ExprPattern \br
    BindPattern \br
    ArrayPattern \br
    TuplePattern \br
    ObjectPattern \br
\end{bnf}

\begin{bnf}{PatternAssterion}
    TypeAssertion
    IncludeAssertion
    CondAssertion
\end{bnf}

\pnum
模式匹配用于检验一个值是否符合特定的\term{模式},以及在符合特定的模式时从中提取某些成分。值符合特定的模式称为这个值\term{匹配}这个模式。本节中,$\mathcal{v}$表示待匹配的值,$\mathcal{p}$表示待匹配的模式。

\pnum
模式$\mathcal{p}$由\term{模式主体}和\term{模式断言}构成。模式主体规定匹配的结构与操作,模式断言则对值的特征进行断言。一个主体可以带有任意数量的断言。

\rSec1[pattern.null]{空模式}
\indextext{模式匹配!空}

\begin{bnf}{NullPattern}
    \tcode{_}
\end{bnf}

\pnum
空模式能够匹配任意值。匹配成功后,$\mathcal{v}$的值将被丢弃。

\rSec1[pattern.expr]{表达式模式}
\indextext{模式匹配!表达式}

\begin{bnf}{ExprPattern}
    Expression
\end{bnf}

\pnum
表达式模式中的表达式必须实现了 \tcode{core.Equtable}。$\mathcal{v}$匹配表达式模式$\mathcal{p}$当且仅当$\mathcal{v}\tcode{==}\mathcal{p}$。

\rSec1[pattern.array]{数组模式}
\indextext{模式匹配!数组}

\begin{bnf}{ArrayPattern}
    \terminal{[} AnyPattern \bnflp\terminal{,} AnyPattern\bnfrp\bnfs\ \terminal{]}
\end{bnf}

\begin{bnf}{AnyPattern}
    Pattern\br
    \terminal{...}
\end{bnf}

\pnum
数组模式匹配序列中的元素。其中\tcode{...}项(称作\term{任意项模式})只能出现至多一次,否则这是一个编译错误。$\mathcal{v}$必须实现\tcode{core.Sequence},否则这是一个编译错误。

\begin{enumerate}
    \item 如果模式不包含任意项,且$\mathcal{v}$\tcode{.size}与模式中项的数量不相等,则匹配失败。
    \item 如果模式包含任意项,且$\mathcal{v}$\tcode{.size}小于模式中非任意项的数量,则匹配失败。
\end{enumerate}

\pnum
在那之后,将按如下规则依次对$\mathcal{v}$的元素进行匹配。如果每个匹配都成功,则整个模式$\mathcal{p}$匹配$\mathcal{v}$。

\begin{enumerate}
    \item 对任意项模式之前的模式(如果不存在任意项则对每个子模式),$\mathcal{p}_i$匹配$\mathcal{v}$\tcode{[}$i$\tcode{]},其中$i$是子模式的索引(从0开始)。
    \item 对任意项模式之后的模式,$\mathcal{p}_r$匹配$\mathcal{v}$\tcode{[\$-}$r$\tcode{]},其中$r$是子模式从后向前数的索引(从0开始)。
\end{enumerate}

\rSec1[pattern.type]{元组模式}
\indextext{模式匹配!元组}

\begin{bnf}{TuplePattern}
    \terminal{(} AnyPattern \bnflp\terminal{,} AnyPattern\bnfrp\bnfs\ \terminal{)}
\end{bnf}

\pnum
与数组模式类似,\term{元组模式}匹配元组。

\rSec1[pattern.object]{对象模式}
\indextext{模式匹配!对象}

\begin{bnf}{ObjectPattern}
    \terminal{\{} ObjectPatternBody \terminal{\}}
\end{bnf}

\begin{bnf}{ObjectPatternBody}
    ObjectItem \bnflp\terminal{,} ObjectItem\bnfrp\bnfs
\end{bnf}

\begin{bnf}{ObjectItem}
    Identifier \terminal{:} Pattern
\end{bnf}

\pnum
\term{对象模式}对对象进行匹配。如果对于每个对$(k, \mathcal{p})$而言,$\mathcal{v}$\tcode{.}$k$匹配$\mathcal{p}$都成立,则整个模式匹配成功。

\pnum
与数组和元组匹配不同,对象匹配是开放的,即$\mathcal{v}$可以包含未在模式中列出的项。

\rSec1[pattern.bind]{绑定模式}
\indextext{模式匹配!绑定}

\begin{bnf}{BindPattern}
    \terminal{var} PatternBind TypeNotation\bnfq \br
    \terminal{let} PatternBind TypeNotation\bnfq
\end{bnf}

\begin{bnf}{PatternBind}
    Identifier \br
    ArrayPatternBind \br
    TuplePatternBind \br
    ObjectPatternBind
\end{bnf}

\begin{bnf}{ArrayPatternBind}
    \terminal{[} AnyPatternBind \bnflp\terminal{,} AnyPatternBind\bnfrp\bnfs \terminal{]}
\end{bnf}

\begin{bnf}{TuplePatternBind}
    \terminal{(} AnyPatternBind \bnflp\terminal{,} AnyPatternBind\bnfrp\bnfs \terminal{)}
\end{bnf}

\begin{bnf}{AnyPatternBind}
    PatternBind \br
    \terminal{...} \br
    NullPattern \br
    ExprPattern
\end{bnf}

\begin{bnf}{ObjectPatternBind}
    \terminal{\{} ObjectPatternBodyBind \terminal{\}}
\end{bnf}

\begin{bnf}{ObjectPatternBodyBind}
    ObjectItemBind \bnflp\terminal{,} ObjectItemBind\bnfrp\bnfs
\end{bnf}

\begin{bnf}{ObjectItemBind}
    Identifier \terminal{:} PatternBind
\end{bnf}

\pnum
绑定模式可以匹配任意值。匹配成功后,该标识符将作为一个变量插入到当前作用域中。如果绑定使用的是\tcode{var},则该变量具有\tcode{mut}修饰。

\pnum
绑定模式可以使用简写:\tcode{let [a, b]}等价于\tcode{[let a, let b]};\tcode{let [i, _]}等价于\tcode{[let i, _]}。

\rSec1[pattern.type]{类型断言}
\indextext{模式匹配!类型断言}

\begin{bnf}{TypeAssertion}
    \terminal{is} Type \br
    \terminal{:} Type \br
    \terminal{as} Type
\end{bnf}

\pnum
\term{类型断言}对值的类型进行约束。它包括以下类型:

\begin{itemize}
    \item \tcode{is T}要求值的类型与\tcode{T}完全一致。
    \item \tcode{: T}要求值的类型是\tcode{T}的子类型。
    \item \tcode{as T}要求值的类型能够转换到\tcode{T},无论显式或隐式。
    \end{itemize}

\rSec1[pattern.include]{包含断言}
\indextext{模式匹配!包含断言}

\begin{bnf}{IncludeAssertion}
    \terminal{in} Expression
\end{bnf}

\pnum
\term{包含断言}要求值包含在某个集合$\mathcal{e}$中。如果$\mathcal{v}$\tcode{ !in }$\mathcal{e}$,则匹配失败。

\rSec1[pattern.cond]{条件断言}
\indextext{模式匹配!条件断言}

\begin{bnf}{CondAssertion}
    \terminal{if} Expression
\end{bnf}

\pnum
\term{条件断言}要求值满足某个条件。