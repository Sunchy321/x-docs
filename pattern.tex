%!TEX root = x.tex

\rSec0[pattern]{模式匹配}

\begin{bnf}
\nontermdef{Pattern} \br
    NullPattern \br
    ExprPattern \br
    BindPattern
\end{bnf}

\pnum
模式匹配用于检验一个值是否符合特定的模式,以及在符合特定的模式时从中提取某些成分。值符合特定的模式称为这个值\term{匹配}这个模式。

\rSec1[pattern.null]{空模式}

\begin{bnf}
\nontermdef{NullPattern} \br
    \tcode{_}
\end{bnf}

\pnum
空模式始终认为匹配成功。

\rSec1[pattern.expr]{表达式模式}

\begin{bnf}
\nontermdef{ExprPattern} \br
    Expression
\end{bnf}

\pnum
表达式模式中的表达式必须实现了 \tcode{core::Equtable};它必须是无副作用的。值 \tcode{x} 匹配模式 \tcode{p} 当且仅当 \tcode{x == p}。

\rSec1[pattern.bind]{绑定模式}

\begin{bnf}
\nontermdef{BindPattern} \br
    \terminal{var} Identifier TypeNotation\bnfq \br
    \terminal{let} Identifier TypeNotation\bnfq
\end{bnf}

\begin{bnf}
\nontermdef{TypeNotation} \br
    \terminal{is} Type \br
    \terminal{:} Type
\end{bnf}

\pnum
如果绑定模式中不具有类型标记,那么任意值 $\langle T, Q, v \rangle$ 都可以匹配模式 \tcode{let n} 或者 \tcode{var n};之后 \tcode{n} 将作为一个变量名称插入到当前作用域中,其类型为 \tcode{T};如果使用的是 \tcode{let} 绑定,则具有限定符 \tcode{immut}。

\pnum
如果绑定模式中具有类型标记,那么值匹配成功当且仅当其类型也满足类型标记的约束;如果能够匹配,接下来将会像没有类型标记那样继续。类型标记的约束如下:

\begin{itemize}
\item \tcode{is T} 要求值的类型为 \tcode{T}。
\item \tcode{: T} 要求值的类型能够隐式转换到 \tcode{T}。
\end{itemize}
