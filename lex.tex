%!TEX root = x.tex

\rSec0[lex]{词法约定}

\pnum
\term{程序文本}指将被翻译为\X{}程序的文本的整体或者一部分。它存储在\term{源文件}中,并以UTF-8编码读取。

\rSec1[lex.comment]{注释}

\pnum
有两种形式的\term{注释}:以\tcode{/*} 开始,\tcode{*/}结束的\term{块注释}和以\tcode{//} 开始,到行末结束的\term{行注释}。注释内部的 \tcode{/*} 或 \tcode{//} 等不具有特殊含义。注释在将程序文本分割为标记以后和空白一起被删除。

\rSec1[lex.identifier]{标识符}

\begin{bnf}{Identifier}
    IdentifierHead IdentifierTail\bnfs
\end{bnf}

\begin{bnf}{IdentifierHead}
    Unicode(XID_Start) \br
    \terminal{_}
\end{bnf}

\begin{bnf}{IdentifierTail}
    IdentifierHead \br
    Unicode(XID_Continue)
\end{bnf}

\pnum
\term{标识符}以具有Unicode XID_Start属性的字符或\tcode{_}开始,后跟零个或数个具有Unicode XID_Continue属性的字符,但不能与关键字相同。标识符区分大小写。

\rSec1[lex.keyword]{关键字}

\pnum
表~\ref{tab:keyword}~中的标识符被保留做关键字。关键字不能作为标识符使用。

\begin{floattable}{关键字}{tab:keyword}{lllll}
\topline
\tcode{_}         &
\tcode{bool}      &
\tcode{cmp}       &
\tcode{else}      &
\tcode{enum}      \\
\tcode{false}     &
\tcode{float}     &
\tcode{for}       &
\tcode{if}        &
\tcode{int}       \\
\tcode{let}       &
\tcode{match}     &
\tcode{mut}       &
\tcode{never}     &
\tcode{nil}       \\
\tcode{return}    &
\tcode{self}      &
\tcode{shl}       &
\tcode{shl_eq}    &
\tcode{shr}       \\
\tcode{shr_rq}    &
\tcode{string}    &
\tcode{then}      &
\tcode{this}      &
\tcode{throw}     \\
\tcode{true}      &
\tcode{type}      &
\tcode{typeof}    &
\tcode{uint}      &
\tcode{var}       \\
\tcode{void}      &
\tcode{while}     &&&\\
\end{floattable}

\pnum
表~\ref{tab:context-keyword}~中的标识符称为\term{上下文关键字}。在特定的语法结构中它将被解析为关键字,在其他位置可以当作一般标识符使用。

\begin{floattable}{上下文关键字}{tab:context-keyword}{lllll}
\topline
\tcode{then} &&&&\\
\end{floattable}

\rSec1[lex.op]{运算符}

\begin{bnf}{Operator}
    RawOperator
\end{bnf}

\begin{bnf}{RawOperator}[\oneof]
    CustomOperator \terminal{, ; : ( ) [ ] \{ \}}
\end{bnf}

\begin{bnf}{CustomOperator}
    CustomOperatorPart\bnfp
\end{bnf}

\begin{bnf}{CustomOperatorPart}[\oneof]
    \terminal{\~ ! \# \% \^ \& * - | + = / ? < > . '}
\end{bnf}

\pnum
\tcode{.},\tcode{...} 和 \tcode{=} 被保留不能重载为运算符。

\rSec1[lex.literal]{字面量}

\begin{bnf}{Literal}
    IntegerLiteral \br
    FloatingLiteral \br
    StringLiteral \br
    SymbolLiteral
\end{bnf}

\rSec2[literal.integer]{整数字面量}

\begin{bnf}{IntegerLiteral}
    DecimalLiteral Suffix\bnfq  \br
    BinaryLiteral Suffix\bnfq \br
    HexadecimalLiteral Suffix\bnfq
\end{bnf}

\begin{bnf}{DecimalLiteral}
    Digit \bnflp\terminal{_}\bnfq Digit\bnfrp \bnfs
\end{bnf}

\begin{bnf}{Digit}[\oneof]
    \terminal{0 1 2 3 4 5 6 7 8 9}
\end{bnf}

\begin{bnf}{BinaryLiteral}
    \terminal{0b} BinaryDigit \bnflp\terminal{_}\bnfq BinaryDigit\bnfrp\bnfs
\end{bnf}

\begin{bnf}{BinaryDigit}
    \terminal{0} \br
    \terminal{1}
\end{bnf}

\begin{bnf}{HexadecimalLiteral}
    \terminal{0x} HexadecimalDigit \bnflp\terminal{_}\bnfq HexadecimalDigit\bnfrp\bnfs
    \terminal{0X} HexadecimalDigit \bnflp\terminal{_}\bnfq HexadecimalDigit\bnfrp\bnfs
\end{bnf}

\begin{bnf}{HexadecimalDigit}[\oneof]
    \terminal{0 1 2 3 4 5 6 7 8 9} \br
    \terminal{A B C D E F} \br
    \terminal{a b c d e f}
\end{bnf}

\pnum
整数字面量由一系列数字构成。其中的单引号用作分隔并且不影响字面量的值。字面量的前缀用于指示它的进制。最左边的数字具有最大的权重值。\term{十进制字面量}由若干十进制数字构成;\term{十六进制字面量}由前缀\tcode{0x}或\tcode{0X}后跟若干十六进制数字构成;\term{二进制字面量}前缀 \tcode{0b} 后跟若干二进制数字构成。\X 不支持八进制字面量。

\rSec2[literal.floating]{浮点字面量}

\pnum
浮点字面量用于表示浮点数。与很多其他语言不同的是,浮点字面量的小数点前后不允许省略数字。

\rSec2[literal.string]{字符串字面量}

\begin{bnf}{StringLiteral}
    \terminal{"} Schar\bnfs\ \terminal{"} Suffix\bnfq \br
    \terminal{@"} Rchar\bnfs\ \terminal{"} Suffix\bnfq \br
    \terminal{\$"} SIchar\bnfs\ \terminal{"} Suffix\bnfq \br
    \terminal{\$@"} RIchar\bnfs\ \terminal{"} Suffix\bnfq \br
    Mdelim Mchar\bnfs Suffix\bnfq Mdelim
\end{bnf}

\begin{bnf}{Schar}
    \textnormal{除了\terminal{\textbackslash}和\terminal{"}以外的非换行可打印字符} \br
    EscapeSeq
\end{bnf}

\begin{bnf}{EscapeSeq}
    \terminal{\textbackslash} SimpleEscape \br
    \terminal{\textbackslash{}u\{} HexadecimalDigit\bnfp \terminal{\}}
\end{bnf}

\begin{bnf}{SimpleEscape}[one of]
    \terminal{' " ? \textbackslash a b f n r t v}
\end{bnf}

\begin{bnf}{Rchar}
    \textnormal{除了\terminal{"}以外的非换行可打印字符} \br
    \terminal{""}
\end{bnf}

\begin{bnf}{SIchar}
    \textnormal{除了\terminal{\textbackslash}、\terminal{"}、\terminal{\{}和\terminal{\}}以外的非换行可打印字符} \br
    EscapeSeq \br
    \terminal{\{} Expression \terminal{\}} \br
    \terminal{\{\{} \br
    \terminal{\}\}}
\end{bnf}

\begin{bnf}{RIchar}
    \textnormal{除了\terminal{"}、\terminal{\{}和\terminal{\}}以外的非换行可打印字符} \br
    "" \br
    \terminal{\{} Expression \terminal{\}} \br
    \terminal{\{\{} \br
    \terminal{\}\}}
\end{bnf}

\begin{bnf}{Mdelim}
    """\ "\bnfs
\end{bnf}

\begin{bnf}{Mchar}
    \textnormal{任意可打印字符}
\end{bnf}

\pnum
有三种类型的字符串字面量:普通字符串字面量、原始字符串字面量、多行字符串字面量。\term{普通字符串字面量}使用反斜杠开始的转义序列表示其他字符。\term{原始字符串字面量}中所有可打印字符将会表示这个字符本身,除了\tcode{""}表示单个双引号的转义序列之外。这两类字符串字面量不允许包含换行符。

\pnum
上述两类字符串字面量可以前加\tcode{\$}表示插值字符串。插值字符串中的$\tcode{\{}e\tcode{\}}$序列会被解释为一个字符串插值,其中$e$为表达式。该表达式将被求值后转换为字符串插入当前位置。

\pnum
\term{多行字符串字面量}表示横跨多行的字符串。它以任意数量但不少于三个的\tcode{"}开始,并以等量的\tcode{"}结束。每一行包含换行符都属于该字符串字面量,但是除了以下字符:

\begin{itemize}
    \item 开头引号序列之后紧邻的换行会被删除。
    \item 如果结尾引号序列所在行之前全都是空白字符,则这些字符连同上一行的换行会被删除。
    \item 如果每一行的空白字符都以结尾引号序列之前的空白字符为前缀,则这些字符都会被删除。
\end{itemize}

如果结尾行包含空白字符且之前的某一行的空白字符不以这些空白字符为前缀,则这是一个编译错误。

\enterexample

如下字面量为合法的多行字符串字面量:

\begin{codeblock}
"""
abc
"""
\end{codeblock}

其等价于\tcode{"abc"}。

\exitexample

\rSec2[literal.symbol]{符号字面量}

\begin{bnf}{SymbolLiteral}
    \terminal{'} Identifier
\end{bnf}

符号字面量后跟的标识符可以与关键字相同。

\rSec2[literal.suffix]{字面量后缀}

\begin{bnf}{Suffix}
    \terminal{_}\bnfq SuffixIdentifier
\end{bnf}

\begin{bnf}{SuffixIdentifier}
    SuffixIdentifierHead SuffixIdentifierTail\bnfs
\end{bnf}

\begin{bnf}{SuffixIdentifierHead}
    Unicode(XID_Start)
\end{bnf}

\begin{bnf}{SuffixIdentifierTail}
    SuffixIdentifierHead \br
    Unicode(XID_Continue)
\end{bnf}

\pnum
整数、浮点数与字符串能带有后缀,指示不同的类别。后缀由字母开始,后跟任意数量的字母或数字。字面量与后缀之间可以添加\tcode{_}分隔。后缀可能是内建的或是用户自定义的。内建后缀将该字面量的类型确定为对应的类型。

\pnum
整数字面量可以有以下内建后缀:

\begin{floattable}{整数字面量后缀}{tab:integer-suffix}{c|c||c|c}
    \topline
    后缀 & 对应的类型 & 后缀 & 对应的类型 \\
    \hline
    \tcode{i8} & \tcode{int<8>} &
    \tcode{i16} & \tcode{int<16>} \\
    \tcode{i32} & \tcode{int<32>} &
    \tcode{i64} & \tcode{int<64>} \\
    \tcode{u8} & \tcode{uint<8>} &
    \tcode{u16} & \tcode{uint<16>} \\
    \tcode{u32} & \tcode{uint<32>} &
    \tcode{u64} & \tcode{uint<64>} \\
    \tcode{f} 或 \tcode{f64} & \tcode{float<64>} &
    \tcode{f32} & \tcode{float<32>} \\
\end{floattable}

\pnum
浮点字面量可以有以下内建后缀:

\begin{floattable}{浮点字面量后缀}{tab:floating-suffix}{c|c}
    \topline
    后缀 & 对应的类型 \\
    \hline
    \tcode{f64} & \tcode{float<64>} \\
    \tcode{f32} & \tcode{float<32>} \\
\end{floattable}