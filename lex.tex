%!TEX root = x.tex

\rSec0[lex]{词法约定}

\pnum
\term{程序文本}指将被翻译为 \X 程序的文本的整体或者一部分。它存储在\term{源文件}中。

\rSec1[lex.indent]{缩进}

\pnum
每一行的第一个非空白字符之前的所有空白字符称为\term{缩进}。两个缩进\term{相等}当且仅当它们由相同的字符构成;缩进 $a$ \term{小于}缩进 $b$ 当且仅当 $a \ne b$ 且 $a$ 是 $b$ 的前缀。实现可以对在同一源文件中的缩进不可比较的情况给出一个警告。

\rSec1[lex.comment]{注释}

\pnum
有两种形式的\term{注释}:以 \tcode{/*} 开始, \tcode{*/} 结束的\term{块注释} 和以 \tcode{//} 开始,到行末结束的\term{行注释}。注释内部的 \tcode{/*} 或 \tcode{//} 等不具有特殊含义。注释在将程序文本分割为标记以后和空白一起被删除。

\pnum
以 \tcode{/**} 开始、以 \tcode{*/} 结束,或者以 \tcode{///} 开始,到行末结束的注释称为\term{文档注释}。文档注释用于表示一个程序实体对应的文档。它按照如下规则确定应用到的实体:

\begin{itemize}

\item
如果对于文档注释 $c$ ,存在程序实体 $e$ 满足

\begin{itemize}

\item
$c$ 的最后一行处于 $e$ 的程序文本的第一行的上一行;

\item
$c$ 的每一行的缩进都不小于 $e$ 的第一行的缩进;

\item
$c$ 的第一行与最后一行的缩进等于 $e$ 的第一行的缩进;以及

\item
$c$ 和 $e$ 的第一个标记之间只有一个空白。

\end{itemize}

那么 $c$ 是 $e$ 的文档注释。\enterexample

\begin{codeblock}
/// \tcode{X} 的文档注释
type X { }

/// 不是 \tcode{t} 的文档注释:间隔了一行

var t = 0

/* 不是 @!\tcode{T}@! 的文档注释:不是 /** 的形式 */
type T { }

/** 不是 @!\tcode{x}@! 的文档注释:有多余的标记 */ /* 多余的注释 */
var x = 0

/// 不是 \tcode{m} 的文档注释:缩进不一致
    var m = 0
\end{codeblock}

\exitexample

\item
否则,$c$ 不是任何实体的文档注释并且被当做普通注释处理。实现可以对此给出一个警告。

\end{itemize}

\rSec1[lex.identifier]{标识符}

\begin{bnf}
\nontermdef{Identifier} \br
    NormalIdentifier \br
    LambdaIdentifier
\end{bnf}

\begin{bnf}
\nontermdef{NormalIdentifier} \br
    IdentifierHead IdentifierTail\bnfs
\end{bnf}

\begin{bnf}
\nontermdef{IdentifierHead} \br
    Unicode(Lu, Ll, Lt, Lm, Lo, Nl) \br
    \terminal{_}
\end{bnf}

\begin{bnf}
\nontermdef{Alnum} \br
    \terminal{a} \textnormal{至} \terminal{z} \br
    \terminal{A} \textnormal{至} \terminal{Z} \br
    Digit
\end{bnf}

\begin{bnf}
\nontermdef{IdentifierTail} \br
    IdentifierHead \br
    Unicode(Mn, Mc, Nd, Pc, Cf)
\end{bnf}

\begin{bnf}
\nontermdef{LambdaIdentifier} \br
    \terminal{\$} Digit\bnfp \br
    \terminal{\$} Identifier
\end{bnf}

\pnum
任意长的由字母和数字组成的序列是标识符。大写和小写字母被认为是不同的。每个字符都是有效的。

\pnum
以 \tcode{\$} 开始后接十进制数字或标识符的序列也是标识符。这些标识符只能在 lambda 作用域中使用。

\rSec1[lex.keyword]{关键字}

\pnum
表~\ref{tab:keyword}~中的标识符被保留做关键字;此外,以 \tcode{\_\_} 开始的标识符被保留作为关键字。只有在某些特定语法结构中的关键字可以作为标识符使用。其他情况下的关键字不能作为标识符。

\begin{floattable}{关键字}{tab:keyword}{lllll}
\topline
\tcode{_}         &
\tcode{bitand}    &
\tcode{bitand_eq} &
\tcode{bitor}     &
\tcode{bitor_eq}  \\
\tcode{bool}      &
\tcode{clone}     &
\tcode{concept}   &
\tcode{div}       &
\tcode{div_eq}    \\
\tcode{else}      &
\tcode{false}     &
\tcode{float}     &
\tcode{if}        &
\tcode{int}       \\
\tcode{let}       &
\tcode{lvalue}    &
\tcode{mod}       &
\tcode{mod_eq}    &
\tcode{mut}       \\
\tcode{move}      &
\tcode{nil}       &
\tcode{return}    &
\tcode{scope}     &
\tcode{string}    \\
\tcode{switch}    &
\tcode{then}      &
\tcode{this}      &
\tcode{throw}     &
\tcode{throws}    \\
\tcode{true}      &
\tcode{type}      &
\tcode{typeof}    &
\tcode{never}     &
\tcode{var}       \\
\tcode{void}      &
\tcode{xor}       &
\tcode{xor_eq}    &
& \\
\end{floattable}

\rSec1[lex.op]{运算符}

\begin{bnf}
\nontermdef{Operator} \br
    \terminal{\lq}\bnfq RawOperator
\end{bnf}

\begin{bnf}
\nontermdef{RawOperator} \textnormal{one of} \br
    CustomOperator \terminal{, ; : ( ) [ ] \{ \}}
\end{bnf}

\begin{bnf}
\nontermdef{CustomOperator} \br
    CustomOperatorPart\bnfp
\end{bnf}

\begin{bnf}
\nontermdef{CustomOperatorPart} \textnormal{one of} \br
    \terminal{\~ ! \# \% \^ \& * - | + = / ? < > .}
\end{bnf}

\pnum
\tcode{.},\tcode{...} 和 \tcode{=} 被保留不能重载为运算符。

\rSec1[lex.literal]{字面量}

\begin{bnf}
\nontermdef{Literal} \br
    IntegerLiteral Suffix\bnfq \br
    FloatingLiteral Suffix\bnfq \br
    StringLiteral Suffix\bnfq
\end{bnf}

\rSec2[literal.integer]{整数字面量}

\begin{bnf}
\nontermdef{IntegerLiteral} \br
    DecimalLiteral \br
    BinaryLiteral \br
    HexadecimalLiteral
\end{bnf}

\begin{bnf}
\nontermdef{DecimalLiteral} \br
    Digit \bnflp\terminal{\textquotesingle}\bnfq Digit\bnfrp\bnfs
\end{bnf}

\begin{bnf}
\nontermdef{Digit} \textnormal{one of} \br
    \terminal{0 1 2 3 4 5 6 7 8 9}
\end{bnf}

\begin{bnf}
\nontermdef{BinaryLiteral} \br
    \terminal{0b} BinaryDigit \bnflp\terminal{\textquotesingle}\bnfq BinaryDigit\bnfrp\bnfs
\end{bnf}

\begin{bnf}
\nontermdef{BinaryDigit} \br
    \terminal{0} \br
    \terminal{1}
\end{bnf}

\begin{bnf}
\nontermdef{HexadecimalLiteral} \br
    \terminal{0x} HexadecimalDigit \bnflp\terminal{\textquotesingle}\bnfq HexadecimalDigit\bnfrp\bnfs
    \terminal{0X} HexadecimalDigit \bnflp\terminal{\textquotesingle}\bnfq HexadecimalDigit\bnfrp\bnfs
\end{bnf}

\begin{bnf}
\nontermdef{HexadecimalDigit} \textnormal{one of} \br
    \terminal{0 1 2 3 4 5 6 7 8 9} \br
    \terminal{A B C D E F} \br
    \terminal{a b c d e f}
\end{bnf}

\pnum
整数字面量由一系列数字构成。其中的单引号用作分隔并且不影响字面量的值。字面量的前缀用于指示它的进制。最左边的数字具有最大的权重值。\term{十进制字面量}由若干十进制数字构成;\term{十六进制字面量}由前缀\tcode{0x}或\tcode{0X}后跟若干十六进制数字构成;\term{二进制字面量}前缀 \tcode{0b} 后跟若干二进制数字构成。\X 不支持八进制字面量。

\pnum
如果整数字面量对应的数学值为$i$,则整数字面量的值为$\tcode{int}_i$。如果$i$太大,则这个值是非法的。

\rSec2[literal.floating]{浮点字面量}

\rSec2[literal.string]{字符串字面量}

\begin{bnf}
\nontermdef{StringLiteral} \br
    \terminal{\textquotedbl} ScharSeq\bnfq \terminal{\textquotedbl} \br
    \terminal{@\textquotedbl} RcharSeq\bnfq \terminal{\textquotedbl}
\end{bnf}

\begin{bnf}
\nontermdef{ScharSeq} \br
    Schar \br
    EscapeSeq \br
    ScharSeq Schar \br
    ScharSeq EscapeSeq
\end{bnf}

\begin{bnf}
\nontermdef{Schar} \br
    \textnormal{除了 \textbackslash 和 \textquotedbl 以外的非空白可打印字符}
\end{bnf}

\begin{bnf}
\nontermdef{RcharSeq} \br
    Rchar \br
    \terminal{\textquotedbl\textquotedbl} \br
    RcharSeq Rchar \br
    RcharSeq \terminal{\textquotedbl\textquotedbl}
\end{bnf}

\begin{bnf}
\nontermdef{Rchar} \br
    \textnormal{除了 \textquotedbl 以外的非空白可打印字符}
\end{bnf}

\rSec2[literal.custom]{用户定义字面量}

\begin{bnf}
\nontermdef{Suffix} \br
    \terminal{\_} Alnums
\end{bnf}
