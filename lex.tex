%!TEX root = x.tex

\rSec0[lex]{词法约定}

\begin{bnf}{Token}
    Identifier \br
    Keyword \br
    Punctuator \br
    Literal \br
    LambdaParameter \br
    MacroInvocation
\end{bnf}

\begin{bnf}{TokenDelimited}
    \terminal{(} TokenList\bnfq \terminal{)} \br
    \terminal{[} TokenList\bnfq \terminal{]} \br
    \terminal{\{} TokenList\bnfq \terminal{\}}
\end{bnf}

\begin{bnf}{TokenList}
    TokenListItem\bnfp
\end{bnf}

\begin{bnf}{TokenListItem}
    Token \textnormal{但不是} \terminal{( ) [ ] \{ \}} \br
    TokenDelimited
\end{bnf}

\pnum
\term{程序文本}指将被翻译为\X{}程序的文本的整体或者一部分。它存储在\term{源文件}中,并以UTF-8编码读取。

\pnum
程序文本将被分割为\term{标记}的序列。标记是程序文本中的最小单元,包括标识符、关键字、标点符号、字面量、lambda参数。
除了少数地方,标记之间包含的空白字符或注释会被忽略。它们不影响程序含义。

\pnum
宏调用是特殊的标记,它将在编译时展开为标记序列。

\rSec1[lex.comment]{注释}

\pnum
有两种形式的\term{注释}:以\tcode{/*} 开始,\tcode{*/}结束的\term{块注释}和以\tcode{//} 开始,到行末结束的\term{行注释}。注释可以嵌套。在将程序文本分割为标记以后,注释和空白一起被删除。

\rSec1[lex.identifier]{标识符}

\begin{bnf}{Identifier}
    NormalIdentifier \br
    RawIdentifier
\end{bnf}

\begin{bnf}{NormalIdentifier}
    IdentifierHead IdentifierTail\bnfs
\end{bnf}

\begin{bnf}{IdentifierHead}
    Unicode(XID_Start) \br
    \terminal{_}
\end{bnf}

\begin{bnf}{IdentifierTail}
    IdentifierHead \br
    Unicode(XID_Continue)
\end{bnf}

\begin{bnf}{RawIdentifier}
    \terminal{`} NormalIdentifier \terminal{`}
\end{bnf}

\pnum
\term{标识符}以具有Unicode XID_Start属性的字符或\tcode{_}开始,后跟零个或数个具有Unicode XID_Continue属性的字符,但不能与关键字相同。标识符区分大小写。

\pnum
标识符可以使用反引号包围,这种标识符称为\term{原始标识符}。原始标识符与其不带引号的版本完全相同,但不会被识别为关键字。

\rSec1[lex.keyword]{关键字}

\pnum
\begin{floattable}{关键字}{tab:keyword}{lllll}
\topline
\tcode{_}         & \tcode{any}       & \tcode{as}        & \tcode{async}     & \tcode{await}     \\
\tcode{auto}      & \tcode{bool}      & \tcode{break}     & \tcode{catch}     & \tcode{char}      \\
\tcode{class}     & \tcode{cmp}       & \tcode{const}     & \tcode{continue}  & \tcode{deinit}    \\
\tcode{defer}     & \tcode{do}        & \tcode{else}      & \tcode{enum}      & \tcode{extern}    \\
\tcode{false}     & \tcode{float}     & \tcode{for}       & \tcode{func}      & \tcode{if}        \\
\tcode{impl}      & \tcode{import}    & \tcode{in}        & \tcode{init}      & \tcode{infer}     \\
\tcode{int}       & \tcode{internal}  & \tcode{is}        & \tcode{lazy}      & \tcode{let}       \\
\tcode{macro}     & \tcode{match}     & \tcode{mut}       & \tcode{never}     & \tcode{nil}       \\
\tcode{operator}  & \tcode{partial}   & \tcode{private}   & \tcode{public}    & \tcode{ref}       \\
\tcode{return}    & \tcode{self}      & \tcode{shl}       & \tcode{shl_eq}    & \tcode{shr}       \\
\tcode{shr_eq}    & \tcode{some}      & \tcode{static}    & \tcode{string}    & \tcode{this}      \\
\tcode{throw}     & \tcode{true}      & \tcode{try}       & \tcode{type}      & \tcode{typeof}    \\
\tcode{uint}      & \tcode{void}      & \tcode{while}     &                   &                   \\
\end{floattable}

\pnum
表~\ref{tab:context-keyword}~中的标识符称为\term{上下文关键字}。在特定的语法结构中它将被解析为关键字,在其他位置可以当作一般标识符使用。
\begin{floattable}{上下文关键字}{tab:context-keyword}{lllll}
\topline
\tcode{didSet}  & \tcode{get}     & \tcode{infix}   & \tcode{prefix}  & \tcode{root}    \\
\tcode{set}     & \tcode{suffix}  & \tcode{super}   & \tcode{then}    & \tcode{willSet} \\
\end{floattable}

\rSec1[lex.punc]{标点符号}
\indextext{标点符号}

\begin{bnf}{Punctuator}
    PunctuatorPart\bnfp
\end{bnf}

\begin{bnf}{PunctuatorPart}[\oneof]
    \terminal{\~ ! @ \# \$ \% \^{} \& * ( ) - + = [ ] \{ \} | ; : ' < > , . ? /}
\end{bnf}

\pnum
\term{标点符号}由一个或数个符号组成,其中的一部分称为\term{运算符},参见~\ref{op}。

\pnum
除了多字符标点符号外,标点符号都由单个字符组成。表~\ref{tab:multichar-punc}~列出了内建的多字符标点,但不包含运算符。解析标点符号时,应尽可能长地匹配多字符标点符号。用户也可以自定义多字符运算符。

\begin{floattable}{多字符标点符号}{tab:multichar-punc}{lllll}
\topline
\tcode{'(}  &
\tcode{->}  &
\tcode{=>}  \\
\tcode{::}  &
\tcode{...} &&&\\
\end{floattable}

\rSec1[lex.literal]{字面量}

\begin{bnf}{Literal}
    IntegerLiteral \br
    FloatingLiteral \br
    StringLiteral \br
    CharacterLiteral \br
    SymbolLiteral \br
    BooleanLiteral
\end{bnf}

\rSec2[literal.integer]{整数字面量}

\begin{bnf}{IntegerLiteral}
    DecimalLiteral Suffix\bnfq \br
    BinaryLiteral Suffix\bnfq \br
    HexadecimalLiteral Suffix\bnfq
\end{bnf}

\begin{bnf}{DecimalLiteral}
    Digits
\end{bnf}

\begin{bnf}{Digits}
    Digit \br
    Digits \terminal{'}\bnfq Digit
\end{bnf}

\begin{bnf}{Digit}[\oneof]
    \terminal{0 1 2 3 4 5 6 7 8 9}
\end{bnf}

\begin{bnf}{BinaryLiteral}
    \terminal{0b} BinaryDigit \bnflp\terminal{'}\bnfq BinaryDigit\bnfrp\bnfs
\end{bnf}

\begin{bnf}{BinaryDigit}
    \terminal{0} \br
    \terminal{1}
\end{bnf}

\begin{bnf}{HexadecimalLiteral}
    \terminal{0x} HexadecimalDigits
\end{bnf}

\begin{bnf}{HexadecimalDigits}
    HexadecimalDigit \br
    HexadecimalDigits \terminal{'}\bnfs HexadecimalDigit
\end{bnf}

\begin{bnf}{HexadecimalDigit}[\oneof]
    \terminal{0 1 2 3 4 5 6 7 8 9} \br
    \terminal{A B C D E F} \br
    \terminal{a b c d e f}
\end{bnf}

\pnum
整数字面量由一系列数字构成。可以使用单引号作分隔并且不影响字面量的值。字面量的前缀用于指示它的进制。\term{十进制字面量}由若干十进制数字构成;\term{十六进制字面量}由前缀\tcode{0x}后跟若干十六进制数字构成;\term{二进制字面量}前缀\tcode{0b}后跟若干二进制数字构成。\X 不支持八进制字面量。

\pnum
整数字面量的值为其数字序列表示的值,依不同前缀分别为十进制、十六进制或二进制。最左侧的数字为最高位。字面量的类型参见表格~\ref{tab:integer-suffix},其中$i$为其字面值:

\begin{floattable}{整数字面量后缀}{tab:integer-suffix}{c|c||c|c}
    \topline
    后缀 & 对应的类型 & 后缀 & 对应的类型 \\
    \capsep
    无          & $\tcode{int}_i$     & \tcode{u}   & $\tcode{uint}_i$     \\
    \tcode{i8}  & $\tcode{int<8>}_i$  & \tcode{u8}  & $\tcode{uint<8>}_i$  \\
    \tcode{i16} & $\tcode{int<16>}_i$ & \tcode{u16} & $\tcode{uint<16>}_i$ \\
    \tcode{i32} & $\tcode{int<32>}_i$ & \tcode{u32} & $\tcode{uint<32>}_i$ \\
    \tcode{i64} & $\tcode{int<64>}_i$ & \tcode{u64} & $\tcode{uint<64>}_i$ \\
    \tcode{f} 或 \tcode{f64} & \tcode{float<64>} &
    \tcode{f32} & \tcode{float<32>} \\
\end{floattable}

如果字面量的字面值超出了其类型的约束范围,则这是一个编译错误。

\rSec2[literal.floating]{浮点字面量}

\begin{bnf}{FloatingLiteral}
    DecimalFloatingLiteral Suffix\bnfq \br
    HexadecimalFloatingLiteral Suffix\bnfq
\end{bnf}

\begin{bnf}{DecimalFloatingLiteral}
    Digits \terminal{.} Digits ExponentPart\bnfq \br
    Digits ExponentPart
\end{bnf}

\begin{bnf}{HexadecimalFloatingLiteral}
    HexadecimalPrefix HexadecimalDigits \terminal{.} HexadecimalDigits BinaryExponentPart\bnfq \br
    HexadecimalPrefix HexadecimalDigits BinaryExponentPart
\end{bnf}

\begin{bnf}{ExponentPart}
    \terminal{e} Sign\bnfq Digit\bnfp \br
    \terminal{E} Sign\bnfq Digit\bnfp
\end{bnf}

\begin{bnf}{BinaryExponentPart}
    \terminal{p} Sign\bnfq Digit\bnfp \br
    \terminal{P} Sign\bnfq Digit\bnfp
\end{bnf}

\begin{bnf}{Sign}[\oneof]
    \terminal{+ -}
\end{bnf}

\pnum
浮点字面量用于表示浮点数,其中的下划线用作分隔并且不影响字面量的值。浮点字面量的小数点前后不允许省略数字。

\pnum
浮点字面量的类型按表~\ref{tab:floating-suffix}~确定。其值依如下方式确定:如果它包含指数部分,则命$e$为指数部分按十进制数字解析得到的数;否则,$e$为0。对十进制浮点字面量而言,命$s$为除去指数的部分按十进制数字解析得到的数,则令$f=s\times 10^e$。对十六进制浮点字面量而言,命$s$为除去指数的部分按十六进制数字解析得到的数,则令$f=s\times 2^e$。浮点字面量的值为其类型中最接近$f$的值。如果$f$太大,则值为对应的正无限大;如果$f$太小,则值为0。

\begin{floattable}{浮点字面量后缀}{tab:floating-suffix}{c|c}
    \topline
    后缀 & 对应的类型 \\
    \capsep
    无或\tcode{f64} & \tcode{float<64>} \\
    \tcode{f32} & \tcode{float<32>} \\
\end{floattable}

\rSec2[literal.string]{字符串字面量}

\begin{bnf}{StringLiteral}
    \terminal{"} Schar\bnfs \terminal{"} Suffix\bnfq \br
    \terminal{@}\bnfp \terminal{"} Rchar\bnfs \terminal{"} \terminal{@}\bnfp Suffix\bnfq
\end{bnf}

\begin{bnf}{Schar}
    \textnormal{除了\terminal{\textbackslash}和\terminal{"}以外的非换行可打印字符} \br
    EscapeSeq \br
    TextInterpolation
\end{bnf}

\begin{bnf}{EscapeSeq}
    \terminal{\textbackslash} SimpleEscape \br
    \terminal{\textbackslash{}u\{} HexadecimalDigit\bnfp \terminal{\}}
\end{bnf}

\begin{bnf}{TextInterpolation}
    \terminal{\textbackslash{}(} Expression \terminal{)}
\end{bnf}

\begin{bnf}{SimpleEscape}[\oneof]
    \terminal{0 ' " \textbackslash{} a b f n r t v}
\end{bnf}

\begin{bnf}{Rchar}
    \textnormal{非换行可打印字符} \br
    RawTextInterpolation
\end{bnf}

\begin{bnf}{RawTextInterpolation}
    \terminal{\textbackslash} \terminal{@}\bnfp \terminal{(}  Expression \terminal{)}
\end{bnf}

\pnum
字符串字面量表示UTF-8字符串,其类型为\tcode{string}。
普通字符串字面量被双引号包围。其中可以使用反斜杠开始的转义序列表示其他字符。
普通字符串字面量也可以前后加上等量的\tcode{@},此时它被称为\term{原始字符串字面量}。原始字符串字面量中的反斜杠不会被解释为转义序列,而是字符本身。

\pnum
字符串字面量可以横跨多行,其中每一行包含换行符都属于该字符串字面量,但是除了以下字符:

\begin{itemize}
    \item 开头引号序列之后紧邻的换行会被删除。
    \item 除了开始行之外的行的公共空白字符前缀会被删除。
    \item 如果结尾行除了公共前缀之外没有其它字符,则上一行的换行会被删除。
    \item 如果一行末尾有反斜杠,则这个反斜杠和其后的换行符会被删除。
\end{itemize}

\enterexample
表~\ref{tab:multiline-string}~是一些多行字符串字面量及其等价的单行表示:

\begin{floattable}{多行字符串字面量示例}{tab:multiline-string}{l|l}
\topline
\begin{codeblock}
"
abc
"
\end{codeblock}
&\tcode{"abc"}\\
\hline

\begin{codeblock}
"
abc
   "
\end{codeblock}
&\tcode{"abc\textbackslash{}n\ \ \ "}\\
\hline

\begin{codeblock}
"abc
def
"
\end{codeblock}
&\tcode{"abc\textbackslash{}ndef"}\\
\hline

\begin{codeblock}
"
abc\
def
"
\end{codeblock}
&\tcode{"abcdef"}\\
\end{floattable}
\exitexample

\pnum
字符串字面量可以包含字符串插值,其形式为$\tcode{\textbackslash(}e\tcode{)}$,其中$e$为表达式。字符串插值会被求值后转换为字符串插入当前位置。对原始字符串字面量而言,反斜杠和括号之间需要插入等量的\tcode{@},否则仍然会被解释为字面符号。

\rSec2[literal.char]{字符字面量}

\begin{bnf}{CharacterLiteral}
    \terminal{'} Character \terminal{'}
\end{bnf}

\begin{bnf}{Character}
    \textnormal{除了\terminal{\textbackslash}和\terminal{'}以外的非换行可打印字符} \br
    EscapeSeq
\end{bnf}

\pnum
字符字面量表示单个字符,其类型为\tcode{char}。字符字面量由单引号包围,其中的字符可以是除了单引号和反斜杠以外的任意字符,或者转义序列。

\rSec2[literal.symbol]{符号字面量}

\begin{bnf}{SymbolLiteral}
    \terminal{'} Identifier
\end{bnf}

\pnum
符号字面量用于标识成员名称,其类型和其值为其标识符的值。

\rSec2[literal.boolean]{布尔字面量}

\begin{bnf}{BooleanLiteral}
    \terminal{true} \br
    \terminal{false}
\end{bnf}

$$ \tcode{true} \coloneqq \langle \mathrm{true}, \tcode{bool} \rangle $$
$$ \tcode{false} \coloneqq \langle \mathrm{false}, \tcode{bool} \rangle $$

\pnum
布尔字面量的类型为\tcode{bool}。\tcode{true}和\tcode{false}分别对应其真值与假值。

\rSec2[literal.suffix]{字面量后缀}

\begin{bnf}{Suffix}
    \terminal{_}\bnfq SuffixIdentifier
\end{bnf}

\begin{bnf}{SuffixIdentifier}
    SuffixIdentifierHead SuffixIdentifierTail\bnfs
\end{bnf}

\begin{bnf}{SuffixIdentifierHead}
    Unicode(XID_Start)
\end{bnf}

\begin{bnf}{SuffixIdentifierTail}
    SuffixIdentifierHead \br
    Unicode(XID_Continue)
\end{bnf}

\pnum
整数、浮点数与字符串能带有内建或用户自定义的后缀。后缀由字母开始,后跟任意数量的字母或数字,字面量与后缀之间可以添加\tcode{_}分隔。用户自定义的后缀不能与内建后缀相同,否则这是一个编译错误。

\pnum
自定义后缀的规则与标识符相同,但会自动去除前导的\tcode{_}。如果前导的下划线多于一个,这是一个编译错误。
自定义后缀不能与内建后缀相同。

\enterexample
\begin{codeblock}
IntegerLiteral<'s>; // 后缀为\tcode{s}
IntegerLiteral<'_s>; // 错误,后缀不能包含前导下划线

0x0123ABC; // 没有后缀
0x0123_ABC; // 后缀为\tcode{ABC},下划线用作区分
\end{codeblock}
\exitexample

\pnum
概念\tcode{IntegerLiteral}、\tcode{FloatingLiteral}、\tcode{StringLiteral}、\tcode{CharacterLiteral}用于实现具有特定后缀的字面量。
如果有多于一个类型实现了相同的后缀或者提供了非法的后缀,这是一个编译错误。

\rSec1[lex.lambda-param]{Lambda参数}

\begin{bnf}{LambdaParameter}
    \terminal{\$} Digit\bnfp \br
    \terminal{\$} Identifier
\end{bnf}

\pnum
Lambda参数只能在lambda作用域(\ref{scope.lambda})中使用,用于引用匿名参数。其类型是待推导的。
不在lambda作用域中使用lambda参数,或在显式指定参数的lambda表达式中使用lambda参数,是一个编译错误。