%!TEX root = x.tex

\rSec0[enum]{枚举}

\begin{bnf}{EnumDecl}
    \terminal{enum} Identifier EnumBaseType\bnfq \terminal{\{} Enumerators \terminal{\}}
\end{bnf}

\begin{bnf}{EnumBaseType}
    \terminal{:} Type
\end{bnf}

\begin{bnf}{Enumerators}
    Enumerator \bnflp \terminal{,} Enumerator \bnfrp\bnfs \terminal{,}\bnfq
\end{bnf}

\begin{bnf}{Enumerator}
    Attribute\bnfq Identifier EnumeratorTail\bnfq
\end{bnf}

\begin{bnf}{EnumeratorTail}
    \terminal{=} Expression \br
    \terminal{[} Type \terminal{]} \br
    StructType
\end{bnf}

\pnum
\term{枚举类型}用来表示一组孤立值,在其定义中使用\term{枚举符}表示。枚举符还可以带有载荷,以表示同一枚举符下的一系列值。

\enterexample
\begin{codeblock}

enum E {
    A,
    B(int),
    C[int],
    D(name: string)
}

\end{codeblock}

上述代码定义了一个枚举类型\tcode{E},它包含四个枚举符,可以以如下方式访问:\tcode{E.A}、\tcode{E.B(0)}、\tcode{E.C[1, 2, 3]}及\tcode{E.D(name: "Hello")}。
\exitexample

\rSec1[enum.trad]{传统枚举类型}
\indextext{枚举!传统}

\pnum
只包含单独枚举符的枚举类型称作\term{传统枚举类型}。传统枚举类型可以指定基底类型$B$,也可以为其枚举符指定值。如果一个传统枚举类型没有显式指定基底类型,则$B$为\tcode{int}。$B$必须实现\tcode{Equtable}。

\pnum
在传统枚举类型中,每个枚举符都有对应的值。其确定如下:

\begin{itemize}
    \item 如果该枚举符被指定值,则其值为被指定的表达式隐式转换到$B$的结果;
    \item 否则,如果该枚举符是第一个值,且基底类型实现了\tcode{core::Default},则其值为\tcode{$B$::default()};
    \item 否则,假设该枚举符的前一个值为$v$,则其值为\tcode{$v$+!}。
\end{itemize}

每个枚举符都可以显式转换为对应的基底类型。