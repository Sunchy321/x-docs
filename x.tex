\documentclass[ebook,10pt,oneside,openany,final]{memoir}

\usepackage[iso,american]
           {isodate}      % use iso format for dates
\usepackage[final]
           {listings}     % code listings
\usepackage{longtable}    % auto-breaking tables
\usepackage{ltcaption}    % fix captions for long tables
\usepackage{booktabs}     % fancy tables
\usepackage{relsize}      % provide relative font size changes
\usepackage{textcomp}     % provide \text{l,r}angle
\usepackage{underscore}   % remove special status of '_' in ordinary text
\usepackage{verbatim}     % improved verbatim environment
\usepackage{parskip}      % handle non-indented paragraphs "properly"
\usepackage{array}        % new column definitions for tables
\usepackage[normalem]{ulem}
\usepackage{color}        % define colors for strikeouts and underlines
\usepackage{amsmath}      % additional math symbols
\usepackage{mathrsfs}     % mathscr font
\usepackage{microtype}
\usepackage{multicol}
\usepackage{xspace}
\usepackage{fixme}
\usepackage{lmodern}
\usepackage[T1]{fontenc}
\usepackage[final]{graphicx}
\usepackage[pdftitle={X 文档},
            pdfsubject={X 文档},
            pdfcreator={Sunchy321},
            bookmarks=true,
            bookmarksnumbered=true,
            pdfpagelabels=true,
            pdfpagemode=UseOutlines,
            pdfstartview=FitH,
            linktocpage=true,
            colorlinks=true,
            linkcolor=blue,
            plainpages=false
           ]{hyperref}
\usepackage{memhfixc}     % fix interactions between hyperref and memoir
\usepackage{ctex}
\usepackage{mathtools}
\usepackage{amsfonts}
\usepackage{dsfont}    % 空心的整数
\usepackage[cal=boondoxo]{mathalfa}

%!TEX root = x.tex
%% layout.tex -- set overall page appearance

%%--------------------------------------------------
%%  set page size, type block size, type block position

\setstocksize{11in}{8.5in}
\settrimmedsize{11in}{8.5in}{*}
\setlrmarginsandblock{1in}{1in}{*}
\setulmarginsandblock{1in}{*}{1.618}

%%--------------------------------------------------
%%  set header and footer positions and sizes

\setheadfoot{\onelineskip}{2\onelineskip}
\setheaderspaces{*}{2\onelineskip}{*}

%%--------------------------------------------------
%%  make miscellaneous adjustments, then finish the layout
\setmarginnotes{7pt}{7pt}{0pt}
\checkandfixthelayout

%%--------------------------------------------------
%% Paragraph and bullet numbering

\newcounter{Paras}
\counterwithin{Paras}{chapter}
\counterwithin{Paras}{section}
\counterwithin{Paras}{subsection}
\counterwithin{Paras}{subsubsection}
\counterwithin{Paras}{paragraph}
\counterwithin{Paras}{subparagraph}

\newcounter{Bullets1}[Paras]
\newcounter{Bullets2}[Bullets1]
\newcounter{Bullets3}[Bullets2]
\newcounter{Bullets4}[Bullets3]

\makeatletter
\newcommand{\parabullnum}[2]{%
\stepcounter{#1}%
\noindent\makebox[0pt][l]{\makebox[#2][r]{%
\scriptsize\raisebox{.7ex}%
{%
\ifnum \value{Paras}>0
\ifnum \value{Bullets1}>0 (\fi%
                          \arabic{Paras}%
\ifnum \value{Bullets1}>0 .\arabic{Bullets1}%
\ifnum \value{Bullets2}>0 .\arabic{Bullets2}%
\ifnum \value{Bullets3}>0 .\arabic{Bullets3}%
\fi\fi\fi%
\ifnum \value{Bullets1}>0 )\fi%
\fi%
}%
\hspace{\@totalleftmargin}\quad%
}}}
\makeatother

\def\pnum{\parabullnum{Paras}{0pt}}

% Leave more room for section numbers in TOC
\cftsetindents{section}{1.5em}{3.0em}

% For compatibility only.  We no longer need this environment.
\newenvironment{paras}{}{}

%!TEX root = x.tex
%% styles.tex -- set styles for:
%     chapters
%     pages
%     footnotes

%%--------------------------------------------------
%%  create chapter style

\makechapterstyle{xstd}{%
  \renewcommand{\beforechapskip}{\onelineskip}
  \renewcommand{\afterchapskip}{\onelineskip}
  \renewcommand{\chapternamenum}{}
  \renewcommand{\chapnamefont}{\chaptitlefont}
  \renewcommand{\chapnumfont}{\chaptitlefont}
  \renewcommand{\printchapternum}{\chapnumfont\thechapter\quad}
  \renewcommand{\afterchapternum}{}
}

%%--------------------------------------------------
%%  create page styles

\makepagestyle{xpage}
\makeevenhead{xpage}{\X}{}{\textbf{\docno}}
\makeoddhead{xpage}{\X}{}{\textbf{\docno}}
\makeevenfoot{xpage}{\leftmark}{}{\thepage}
\makeoddfoot{xpage}{\leftmark}{}{\thepage}

\makeatletter
\makepsmarks{xpage}{%
  \let\@mkboth\markboth
  \def\chaptermark##1{\markboth{##1}{##1}}%
  \def\sectionmark##1{\markboth{%
    \ifnum \c@secnumdepth>\z@
      \textsection\space\thesection
    \fi
    }{\rightmark}}%
  \def\subsectionmark##1{\markboth{%
    \ifnum \c@secnumdepth>\z@
      \textsection\space\thesubsection
    \fi
    }{\rightmark}}%
  \def\subsubsectionmark##1{\markboth{%
    \ifnum \c@secnumdepth>\z@
      \textsection\space\thesubsubsection
    \fi
    }{\rightmark}}%
  \def\paragraphmark##1{\markboth{%
    \ifnum \c@secnumdepth>\z@
      \textsection\space\theparagraph
    \fi
    }{\rightmark}}}
\makeatother

\aliaspagestyle{chapter}{xpage}

%%--------------------------------------------------
%%  set heading styles for main matter
\newcommand{\beforeskip}{-.7\onelineskip plus -1ex}
\newcommand{\afterskip}{.3\onelineskip minus .2ex}

\setbeforesecskip{\beforeskip}
\setsecindent{0pt}
\setsecheadstyle{\large\bfseries\raggedright}
\setaftersecskip{\afterskip}

\setbeforesubsecskip{\beforeskip}
\setsubsecindent{0pt}
\setsubsecheadstyle{\large\bfseries\raggedright}
\setaftersubsecskip{\afterskip}

\setbeforesubsubsecskip{\beforeskip}
\setsubsubsecindent{0pt}
\setsubsubsecheadstyle{\normalsize\bfseries\raggedright}
\setaftersubsubsecskip{\afterskip}

\setbeforeparaskip{\beforeskip}
\setparaindent{0pt}
\setparaheadstyle{\normalsize\bfseries\raggedright}
\setafterparaskip{\afterskip}

\setbeforesubparaskip{\beforeskip}
\setsubparaindent{0pt}
\setsubparaheadstyle{\normalsize\bfseries\raggedright}
\setaftersubparaskip{\afterskip}

%%--------------------------------------------------
% set heading style for annexes
\newcommand{\Annex}[3]{\chapter[#2]{(#3)\protect\\#2\hfill[#1]}\relax\label{#1}}
\newcommand{\infannex}[2]{\Annex{#1}{#2}{informative}}
\newcommand{\normannex}[2]{\Annex{#1}{#2}{normative}}

%%--------------------------------------------------
%%  set footnote style
\footmarkstyle{\smaller#1) }

%%--------------------------------------------------
% set style for main text
\setlength{\parindent}{0pt}
\setlength{\parskip}{1ex}
\setlength{\partopsep}{0pt}

% set style for lists (itemizations, enumerations)
\setlength{\partopsep}{0pt}
\newlist{indenthelper}{itemize}{1}
\newlist{bnflist}{itemize}{1}
\setlist[itemize]{parsep=\parskip, partopsep=0pt, itemsep=0pt, topsep=0pt,
                  beginpenalty=10 }
\setlist[enumerate]{parsep=\parskip, partopsep=0pt, itemsep=0pt, topsep=0pt}
\setlist[indenthelper]{parsep=\parskip, partopsep=0pt, itemsep=0pt, topsep=0pt, label={}}
\setlist[bnflist]{parsep=\parskip, partopsep=0pt, itemsep=0pt, topsep=0pt, label={},
                  leftmargin=\bnfindentrest, listparindent=-\bnfindentinc, itemindent=\listparindent}


%%--------------------------------------------------
%%  set caption style and delimiter
\captionstyle{\centering}
\captiondelim{ --- }
% override longtable's caption delimiter to match
\makeatletter
\def\LT@makecaption#1#2#3{%
  \LT@mcol\LT@cols c{\hbox to\z@{\hss\parbox[t]\LTcapwidth{%
    \sbox\@tempboxa{#1{#2 --- }#3}%
    \ifdim\wd\@tempboxa>\hsize
      #1{#2 --- }#3%
    \else
      \hbox to\hsize{\hfil\box\@tempboxa\hfil}%
    \fi
    \endgraf\vskip\baselineskip}%
  \hss}}}
\makeatother

%%--------------------------------------------------
%% set global styles that get reset by \mainmatter
\newcommand{\setglobalstyles}{
  \counterwithout{footnote}{chapter}
  \counterwithout{table}{chapter}
  \counterwithout{figure}{chapter}
  \renewcommand{\chaptername}{}
  \renewcommand{\appendixname}{附录 }
}

%%--------------------------------------------------
%% change list item markers to number and em-dash

\renewcommand{\labelitemi}{---\parabullnum{Bullets1}{\labelsep}}
\renewcommand{\labelitemii}{---\parabullnum{Bullets2}{\labelsep}}
\renewcommand{\labelitemiii}{---\parabullnum{Bullets3}{\labelsep}}
\renewcommand{\labelitemiv}{---\parabullnum{Bullets4}{\labelsep}}

%%--------------------------------------------------
%% set section numbering limit, toc limit
\maxsecnumdepth{subparagraph}
\setcounter{tocdepth}{1}

%!TEX root = x.tex
% Definitions and redefinitions of special commands

%%--------------------------------------------------
%% Difference markups
\definecolor{addclr}{rgb}{0,.6,.6}
\definecolor{remclr}{rgb}{1,0,0}
\definecolor{noteclr}{rgb}{0,0,1}

\renewcommand{\added}[1]{\textcolor{addclr}{\uline{#1}}}
\newcommand{\removed}[1]{\textcolor{remclr}{\sout{#1}}}
\renewcommand{\changed}[2]{\removed{#1}\added{#2}}

\newcommand{\nbc}[1]{[#1]\ }
\newcommand{\addednb}[2]{\added{\nbc{#1}#2}}
\newcommand{\removednb}[2]{\removed{\nbc{#1}#2}}
\newcommand{\changednb}[3]{\removednb{#1}{#2}\added{#3}}
\newcommand{\remitem}[1]{\item\removed{#1}}

\newenvironment{addedblock}
{
\color{addclr}
}
{
\color{black}
}
\newenvironment{removedblock}
{
\color{remclr}
}
{
\color{black}
}

%%--------------------------------------------------
%% Sectioning macros.
% Each section has a depth, an automatically generated section
% number, a name, and a short tag.  The depth is an integer in
% the range [0,5].  (If it proves necessary, it wouldn't take much
% programming to raise the limit from 5 to something larger.)


% The basic sectioning command.  Example:
%    \Sec1[intro.scope]{Scope}
% defines a first-level section whose name is "Scope" and whose short
% tag is intro.scope.  The square brackets are mandatory.
\def\Sec#1[#2]#3{%
\ifcase#1\let\s=\chapter
      \or\let\s=\section
      \or\let\s=\subsection
      \or\let\s=\subsubsection
      \or\let\s=\paragraph
      \or\let\s=\subparagraph
      \fi%
\s[#3]{#3\hfill[#2]}\label{#2}}

% A convenience feature (mostly for the convenience of the Project
% Editor, to make it easy to move around large blocks of text):
% the \rSec macro is just like the \Sec macro, except that depths
% relative to a global variable, SectionDepthBase.  So, for example,
% if SectionDepthBase is 1,
%   \rSec1[temp.arg.type]{Template type arguments}
% is equivalent to
%   \Sec2[temp.arg.type]{Template type arguments}
\newcounter{SectionDepthBase}
\newcounter{scratch}

\def\rSec#1[#2]#3{%
\setcounter{scratch}{#1}
\addtocounter{scratch}{\value{SectionDepthBase}}
\Sec{\arabic{scratch}}[#2]{#3}}

\newcommand{\synopsis}[1]{\textbf{#1}}

%%--------------------------------------------------
% Indexing

% locations
\newcommand{\indextext}[1]{\index[generalindex]{#1}}
\newcommand{\indexlibrary}[1]{\index[libraryindex]{#1}}
\newcommand{\indexgram}[1]{\index[grammarindex]{#1}}

\newcommand{\indexdefn}[1]{\indextext{#1}}
\newcommand{\indexgrammar}[1]{\indexgram{#1}}

% appearance
\newcommand{\idxcode}[1]{#1@\tcode{#1}}
\newcommand{\idxhdr}[1]{#1@\tcode{<#1>}}
\newcommand{\idxgram}[1]{#1@\textit{#1}}

%%--------------------------------------------------
% General code style
\newcommand{\CodeStyle}{\ttfamily}
\newcommand{\CodeStylex}[1]{\texttt{#1}}

\definecolor{grammar-gray}{gray}{0.2}

% General grammar style
\newcommand{\GrammarStylex}[1]{\textcolor{grammar-gray}{\textsf{\textit{#1}}}}

% Code and definitions embedded in text.
\newcommand{\tcode}[1]{\CodeStylex{#1}}
\newcommand{\term}[1]{\textit{#1}}
\newcommand{\gterm}[1]{\GrammarStylex{#1}}
\newcommand{\fakegrammarterm}[1]{\gterm{#1}}
\newcommand{\keyword}[1]{\texorpdfstring{\tcode{#1}\protect\indextext{\idxcode{#1}!keyword}}{#1}}                % macro length: 8
\newcommand{\grammarterm}[1]{\texorpdfstring{\protect\indexgram{\idxgram{#1}}\gterm{#1}}{#1}}
\newcommand{\grammartermnc}[1]{\indexgram{\idxgram{#1}}\gterm{#1\nocorr}}
\newcommand{\regrammarterm}[1]{\textit{#1}}
\newcommand{\placeholder}[1]{\textit{#1}}                                   % macro length: 12
\newcommand{\placeholdernc}[1]{\textit{#1\nocorr}}                          % macro length: 14
\newcommand{\exposid}[1]{\tcode{\placeholder{#1}}}                          % macro length: 8
\newcommand{\exposidnc}[1]{\tcode{\placeholdernc{#1}}\itcorr[-1]}           % macro length: 10
\newcommand{\defnxname}[1]{\indextext{\idxxname{#1}}\xname{#1}}
\newcommand{\defnlibxname}[1]{\indexlibrary{\idxxname{#1}}\xname{#1}}

%%--------------------------------------------------
%% allow line break if needed for justification
\newcommand{\brk}{\discretionary{}{}{}}
%  especially for scope qualifier

%%--------------------------------------------------
%% Macros for funky text
\newcommand{\X}{$X$}
\newcommand{\bnflp}{\texttt{(}}
\newcommand{\bnfrp}{\texttt{)}}
\newcommand{\bnfq}{\texttt{?}\ }
\newcommand{\bnfs}{\texttt{*}\ }
\newcommand{\bnfp}{\texttt{+}\ }
\newcommand{\bnfv}{\texttt{|}}
\newcommand{\oneof}{\emph{以下之一}}

\newcommand{\bigoh}[1]{\ensuremath{\mathscr{O}(#1)}}

% Make all tildes a little larger to avoid visual similarity with hyphens.
% FIXME: Remove \tilde in favour of \~.
\renewcommand{\tilde}{\textasciitilde}
\renewcommand{\~}{\textasciitilde}
\let\OldTextAsciiTilde\textasciitilde
\renewcommand{\textasciitilde}{\protect\raisebox{-0.17ex}{\larger\OldTextAsciiTilde}}

%%--------------------------------------------------
%% Notes and examples
\newcommand{\EnterBlock}[1]{[\,\textit{#1:}}
\newcommand{\ExitBlock}{]\xspace}
\newcommand{\enternote}{\EnterBlock{注}}
\newcommand{\exitnote}{\ExitBlock}
\newcommand{\enterexample}{\EnterBlock{例}}
\newcommand{\exitexample}{\ExitBlock}

%% Library function descriptions
\newcommand{\Fundesc}[1]{\textit{#1}:\xspace}
\newcommand{\require}{\Fundesc{约束}}
\newcommand{\effect}{\Fundesc{效果}}
\newcommand{\return}{\Fundesc{返回值}}
\newcommand{\values}{\Fundesc{值}}
\newcommand{\note}{\Fundesc{注}}

%% Cross reference
\newcommand{\xref}{\textsc{See also:}\xspace}
\newcommand{\xsee}{\textsc{See:}\xspace}

%% Code annotations
\newcommand{\EXPO}[1]{\textit{#1}}
\newcommand{\expos}{\EXPO{exposition only}}

\newcommand{\UNSP}[1]{\textit{\texttt{#1}}}
\newcommand{\UNSPnc}[1]{\textit{\texttt{#1}\nocorr}}
\newcommand{\seebelow}{\UNSP{see below}}
\newcommand{\seebelownc}{\UNSPnc{see below}}

%% Manual insertion of italic corrections, for aligning in the presence
%% of the above annotations.
\newlength{\itcorrwidth}
\newlength{\itletterwidth}
\newcommand{\itcorr}[1][]{%
 \settowidth{\itcorrwidth}{\textit{x\/}}%
 \settowidth{\itletterwidth}{\textit{x\nocorr}}%
 \addtolength{\itcorrwidth}{-1\itletterwidth}%
 \makebox[#1\itcorrwidth]{}%
}

%% Double underscore
\newcommand{\ungap}{\kern.5pt}
\newcommand{\unun}{\_\ungap\_}
\newcommand{\xname}[1]{\unun\ungap#1}
\newcommand{\mname}[1]{\tcode{\unun\ungap#1\ungap\unun}}

%% Miscellaneous
\newcommand{\uniquens}{\textrm{\textit{\textbf{unique}}}}
\newcommand{\stage}[1]{\item{\textbf{Stage #1:}}\xspace}

%%--------------------------------------------------
%% Environments for code listings.

% We use the 'listings' package, with some small customizations.  The
% most interesting customization: all TeX commands are available
% within comments.  Comments are set in italics, keywords and strings
% don't get special treatment.

\lstdefinelanguage{X} {
    % morekeywords={int, uint, bool, never, void, self, immut},
    % morekeywords={concept, const, enum, extend, func, impl, let, qual, type, var},
    % morekeywords={pure},
    morekeywords={if, while, for},
    morekeywords={enum},
    morecomment=[l]{//},
    morecomment=[s]{/*}{*/}
}

\lstset{
    language=X,
    basicstyle=\small\CodeStyle,
    keywordstyle=\color{blue},
    stringstyle=,
    xleftmargin=1em,
    showstringspaces=true,
    commentstyle=\itshape\rmfamily,
    columns=flexible,
    keepspaces=true,
    texcl=true
}

% Our usual abbreviation for 'listings'.  Comments are in
% italics.  Arbitrary TeX commands can be used if they're
% surrounded by @ signs.
\newcommand{\CodeBlockSetup}{%
    \lstset{escapechar=@, aboveskip=\parskip, belowskip=0pt,
            % midpenalty=500, endpenalty=-50,
            % emptylinepenalty=-250, semicolonpenalty=0,
            upquote=true}%
    \renewcommand{\tcode}[1]{\textup{\CodeStylex{##1}}}
    \renewcommand{\term}[1]{\textit{##1}}%
    \renewcommand{\grammarterm}[1]{\gterm{##1}}%
}

\lstnewenvironment{codeblock}{\CodeBlockSetup}{}

% A code block in which single-quotes are digit separators
% rather than character literals.
\lstnewenvironment{codeblockdigitsep}{
 \CodeBlockSetup
 \lstset{deletestring=[b]{'}}
}{}

%%--------------------------------------------------
%% Indented text
\newenvironment{indented}[1][]
{\begin{indenthelper}[#1]\item\relax}
{\end{indenthelper}}

%%--------------------------------------------------
%% Library item descriptions
\lstnewenvironment{itemdecl}
{
 \lstset{escapechar=@!,
 xleftmargin=0em,
 aboveskip=2ex,
 belowskip=0ex	% leave this alone: it keeps these things out of the
				% footnote area
 }
}
{
}

\newenvironment{itemdescr}
{
 \begin{indented}}
{
 \end{indented}
}


%%--------------------------------------------------
%% Bnf environments
\newlength{\BnfIndent}
\setlength{\BnfIndent}{\leftmargini}
\newlength{\BnfInc}
\setlength{\BnfInc}{\BnfIndent}
\newlength{\BnfRest}
\setlength{\BnfRest}{2\BnfIndent}
\newcommand{\BnfNontermshape}{\small\rmfamily\itshape}
\newcommand{\BnfTermshape}{\small\ttfamily\upshape}
\newcommand{\nonterminal}[1]{{\BnfNontermshape #1}}

\newenvironment{bnfbase}
 {
 \newcommand{\nontermdef}[1]{{\BnfNontermshape##1\itcorr}\indexgrammar{\idxgram{##1}}\textnormal{:}}
 \newcommand{\terminal}[1]{{\BnfTermshape ##1}}
 \renewcommand{\keyword}[1]{\terminal{##1}\indextext{\idxcode{##1}!keyword}}
 \renewcommand{\exposid}[1]{\terminal{\textit{##1}}}
 \renewcommand{\placeholder}[1]{\textrm{\textit{##1}}}
 \newcommand{\descr}[1]{\textnormal{##1}}
 \newcommand{\bnfindent}{\hspace*{\bnfindentfirst}}
 \newcommand{\bnfindentfirst}{\BnfIndent}
 \newcommand{\bnfindentinc}{\BnfInc}
 \newcommand{\bnfindentrest}{\BnfRest}
 \newcommand{\br}{\hfill\\*}
 \widowpenalties 1 10000
 \frenchspacing
 }
 {
 \nonfrenchspacing
 }

\newenvironment{simplebnf}
{
 \begin{bnfbase}
 \BnfNontermshape
 \begin{indented}[before*=\setlength{\rightmargin}{-\leftmargin}]
}
{
 \end{indented}
 \end{bnfbase}
}

\DeclareDocumentEnvironment{bnf}{m O{}}{
    \begin{bnfbase}
    \begin{bnflist}
    \BnfNontermshape
    \item\relax
    \nontermdef{#1} \textnormal{#2}\br
}{
    \end{bnflist}
    \end{bnfbase}
}

%%--------------------------------------------------
%% Drawing environment
%
% usage: \begin{drawing}{UNITLENGTH}{WIDTH}{HEIGHT}{CAPTION}
\newenvironment{drawing}[4]
{
\newcommand{\mycaption}{#4}
\begin{figure}[h]
\setlength{\unitlength}{#1}
\begin{center}
\begin{picture}(#2,#3)\thicklines
}
{
\end{picture}
\end{center}
\caption{\mycaption}
\end{figure}
}

%%--------------------------------------------------
%% Environment for imported graphics
% usage: \begin{importgraphic}{CAPTION}{TAG}{FILE}

\newenvironment{importgraphic}[3]
{%
\newcommand{\cptn}{#1}
\newcommand{\lbl}{#2}
\begin{figure}[htp]\centering%
\includegraphics[scale=.35]{#3}
}
{
\caption{\cptn}\label{\lbl}%
\end{figure}}

%% enumeration display overrides
% enumerate with lowercase letters
\newenvironment{enumeratea}
{
 \renewcommand{\labelenumi}{\alph{enumi})}
 \begin{enumerate}
}
{
 \end{enumerate}
}

% enumerate with arabic numbers
\newenvironment{enumeraten}
{
 \renewcommand{\labelenumi}{\arabic{enumi})}
 \begin{enumerate}
}
{
 \end{enumerate}
}

%%--------------------------------------------------
%% Definitions section
% usage: \definition{name}{xref}
%\newcommand{\definition}[2]{\rSec2[#2]{#1}}
% for ISO format, use:
\newcommand{\definition}[2]{%
\subsection[#1]{\hfill[#2]}\vspace{-.3\onelineskip}\label{#2}\textbf{#1}\\%
}
\newcommand{\definitionx}[2]{%
\subsubsection[#1]{\hfill[#2]}\vspace{-.3\onelineskip}\label{#2}\textbf{#1}\\%
}
\newcommand{\defncontext}[1]{\textlangle#1\textrangle}

%!TEX root = x.tex
% Definitions of table environments

%%--------------------------------------------------
%% Table environments

% Set parameters for floating tables
\setcounter{totalnumber}{10}

% Base definitions for tables
\newenvironment{TableBase}
{
 \renewcommand{\tcode}[1]{{\CodeStylex{##1}}}
 \newcommand{\topline}{\hline}
 \newcommand{\capsep}{\hline\hline}
 \newcommand{\rowsep}{\hline}
 \newcommand{\bottomline}{\hline}

%% vertical alignment
 \newcommand{\rb}[1]{\raisebox{1.5ex}[0pt]{##1}}	% move argument up half a row

%% header helpers
 \newcommand{\hdstyle}[1]{\textbf{##1}}				% set header style
 \newcommand{\Head}[3]{\multicolumn{##1}{##2}{\hdstyle{##3}}}	% add title spanning multiple columns
 \newcommand{\lhdrx}[2]{\Head{##1}{|c}{##2}}		% set header for left column spanning #1 columns
 \newcommand{\chdrx}[2]{\Head{##1}{c}{##2}}			% set header for center column spanning #1 columns
 \newcommand{\rhdrx}[2]{\Head{##1}{c|}{##2}}		% set header for right column spanning #1 columns
 \newcommand{\ohdrx}[2]{\Head{##1}{|c|}{##2}}		% set header for only column spanning #1 columns
 \newcommand{\lhdr}[1]{\lhdrx{1}{##1}}				% set header for single left column
 \newcommand{\chdr}[1]{\chdrx{1}{##1}}				% set header for single center column
 \newcommand{\rhdr}[1]{\rhdrx{1}{##1}}				% set header for single right column
 \newcommand{\ohdr}[1]{\ohdrx{1}{##1}}
 \newcommand{\br}{\hfill\break}						% force newline within table entry

%% column styles
 \newcolumntype{x}[1]{>{\raggedright\let\\=\tabularnewline}p{##1}}	% word-wrapped ragged-right
 																	% column, width specified by #1
 % \newcolumntype{m}[1]{>{\CodeStyle}l{##1}}              % variable width column, all entries in CodeStyle
 \newcolumntype{m}[1]{l{##1}}							% variable width column, all entries in CodeStyle

  % do not number bullets within tables
  \renewcommand{\labelitemi}{---}
  \renewcommand{\labelitemii}{---}
  \renewcommand{\labelitemiii}{---}
  \renewcommand{\labelitemiv}{---}
}
{
}

% General Usage: TITLE is the title of the table, XREF is the
% cross-reference for the table. LAYOUT is a sequence of column
% type specifiers (e.g. cp{1.0}c), without '|' for the left edge
% or right edge.

% usage: \begin{floattablebase}{TITLE}{XREF}{COLUMNS}{PLACEMENT}
% produces floating table, location determined within limits
% by LaTeX.
\newenvironment{floattablebase}[4]
{
 \begin{TableBase}
 \begin{table}[#4]
 \caption{\label{#2}#1}
 \begin{center}
 \begin{tabular}{|#3|}
}
{
 \bottomline
 \end{tabular}
 \end{center}
 \end{table}
 \end{TableBase}
}

% usage: \begin{floattable}{TITLE}{XREF}{COLUMNS}
% produces floating table, location determined within limits
% by LaTeX.
\newenvironment{floattable}[3]
{
 \begin{floattablebase}{#1}{#2}{#3}{htbp}
}
{
 \end{floattablebase}
}

% usage: \begin{tokentable}{TITLE}{XREF}{HDR1}{HDR2}
% produces six-column table used for lists of replacement tokens;
% the columns are in pairs -- left-hand column has header HDR1,
% right hand column has header HDR2; pairs of columns are separated
% by vertical lines. Used in the "Alternative tokens" table.
\newenvironment{tokentable}[4]
{
 \begin{floattablebase}{#1}{#2}{cc|cc|cc}{htbp}
 \topline
 #3   &   #4    &
 #3   &   #4    &
 #3   &   #4    \\ \capsep
}
{
 \end{floattablebase}
}

% usage: \begin{libsumtabbase}{TITLE}{XREF}{HDR1}{HDR2}
% produces three-column table with column headers HDR1 and HDR2.
% Used in "Library Categories" table in standard, and used as
% base for other library summary tables.
\newenvironment{libsumtabbase}[4]
{
 \begin{floattable}{#1}{#2}{lll}
 \topline
 \lhdrx{2}{#3}	&	\hdstyle{#4}	\\ \capsep
}
{
 \end{floattable}
}

% usage: \begin{libsumtab}{TITLE}{XREF}
% produces three-column table with column headers "Subclause" and "Header(s)".
% Used in "C++ Headers for Freestanding Implementations" table in standard.
\newenvironment{libsumtab}[2]
{
 \begin{libsumtabbase}{#1}{#2}{Subclause}{Header(s)}
}
{
 \end{libsumtabbase}
}

% usage: \begin{LibSynTab}{CAPTION}{TITLE}{XREF}{COUNT}{LAYOUT}
% produces table with COUNT columns. Used as base for
% C library description tables
\newcounter{LibSynTabCols}
\newcounter{LibSynTabWd}
\newenvironment{LibSynTabBase}[5]
{
 \setcounter{LibSynTabCols}{#4}
 \setcounter{LibSynTabWd}{#4}
 \addtocounter{LibSynTabWd}{-1}
 \newcommand{\centry}[1]{\textbf{##1}:}
 \newcommand{\macro}{\centry{Macro}}
 \newcommand{\macros}{\centry{Macros}}
 \newcommand{\function}{\centry{Function}}
 \newcommand{\functions}{\centry{Functions}}
 \newcommand{\mfunctions}{\centry{Math Functions}}
 \newcommand{\cfunctions}{\centry{Classification/comparison Functions}}
 \newcommand{\type}{\centry{Type}}
 \newcommand{\types}{\centry{Types}}
 \newcommand{\values}{\centry{Values}}
 \newcommand{\struct}{\centry{Struct}}
 \newcommand{\cspan}[1]{\multicolumn{\value{LibSynTabCols}}{|l|}{##1}}
 \begin{floattable}{#1 \tcode{<#2>} synopsis}{#3}
 {#5}
 \topline
 \lhdr{Type}	&	\rhdrx{\value{LibSynTabWd}}{Name(s)}	\\ \capsep
}
{
 \end{floattable}
}

% usage: \begin{LibSynTab}{TITLE}{XREF}{COUNT}{LAYOUT}
% produces table with COUNT columns. Used as base for description tables
% for C library
\newenvironment{LibSynTab}[4]
{
 \begin{LibSynTabBase}{Header}{#1}{#2}{#3}{#4}
}
{
 \end{LibSynTabBase}
}

% usage: \begin{LibSynTabAdd}{TITLE}{XREF}{COUNT}{LAYOUT}
% produces table with COUNT columns. Used as base for description tables
% for additions to C library
\newenvironment{LibSynTabAdd}[4]
{
 \begin{LibSynTabBase}{Additions to header}{#1}{#2}{#3}{#4}
}
{
 \end{LibSynTabBase}
}

% usage: \begin{libsyntabN}{TITLE}{XREF}
%        \begin{libsyntabaddN}{TITLE}{XREF}
% produces a table with N columns for C library description tables
\newenvironment{libsyntab2}[2]
{
 \begin{LibSynTab}{#1}{#2}{2}{ll}
}
{
 \end{LibSynTab}
}

\newenvironment{libsyntab3}[2]
{
 \begin{LibSynTab}{#1}{#2}{3}{lll}
}
{
 \end{LibSynTab}
}

\newenvironment{libsyntab4}[2]
{
 \begin{LibSynTab}{#1}{#2}{4}{llll}
}
{
 \end{LibSynTab}
}

\newenvironment{libsyntab5}[2]
{
 \begin{LibSynTab}{#1}{#2}{5}{lllll}
}
{
 \end{LibSynTab}
}

\newenvironment{libsyntab6}[2]
{
 \begin{LibSynTab}{#1}{#2}{6}{llllll}
}
{
 \end{LibSynTab}
}

\newenvironment{libsyntabadd2}[2]
{
 \begin{LibSynTabAdd}{#1}{#2}{2}{ll}
}
{
 \end{LibSynTabAdd}
}

\newenvironment{libsyntabadd3}[2]
{
 \begin{LibSynTabAdd}{#1}{#2}{3}{lll}
}
{
 \end{LibSynTabAdd}
}

\newenvironment{libsyntabadd4}[2]
{
 \begin{LibSynTabAdd}{#1}{#2}{4}{llll}
}
{
 \end{LibSynTabAdd}
}

\newenvironment{libsyntabadd5}[2]
{
 \begin{LibSynTabAdd}{#1}{#2}{5}{lllll}
}
{
 \end{LibSynTabAdd}
}

\newenvironment{libsyntabadd6}[2]
{
 \begin{LibSynTabAdd}{#1}{#2}{6}{llllll}
}
{
 \end{LibSynTabAdd}
}

% usage: \begin{libsyntabfN}{TITLE}{XREF}
% produces a fixed width table with N columns for C library description tables
\newenvironment{libsyntabf2}[2]
{
 \begin{LibSynTab}{#1}{#2}{2}{p{1in}p{4in}}
}
{
 \end{LibSynTab}
}

\newenvironment{libsyntabf3}[2]
{
 \begin{LibSynTab}{#1}{#2}{3}{p{1in}p{.9in}p{2.9in}}
}
{
 \end{LibSynTab}
}

\newenvironment{libsyntabf5}[2]
{
 \begin{LibSynTab}{#1}{#2}{5}{p{.9in}p{.9in}p{.9in}p{.9in}p{.9in}}
}
{
 \end{LibSynTab}
}

\newenvironment{libsyntabf4}[2]
{
 \begin{LibSynTab}{#1}{#2}{4}{p{1in}p{.9in}p{.9in}p{1.8in}}
}
{
 \end{LibSynTab}
}

% usage: \begin{concepttable}{TITLE}{TAG}{LAYOUT}
% produces table at current location
\newenvironment{concepttable}[3]
{
 \begin{TableBase}
 \begin{table}[!htb]
 \caption[#1]{\label{tab:#2}\label{#2}#1 \textbf{[#2]}}
 \begin{center}
 \begin{tabular}{|#3|}
}
{
 \bottomline
 \end{tabular}
 \end{center}
 \end{table}
 \end{TableBase}
}

% usage: \begin{simpletypetable}{TITLE}{TAG}{LAYOUT}
% produces table at current location
\newenvironment{simpletypetable}[3]
{
 \begin{TableBase}
 \begin{table}[!htb]
 \caption{#1}\label{#2}
 \begin{center}
 \begin{tabular}{|#3|}
}
{
 \bottomline
 \end{tabular}
 \end{center}
 \end{table}
 \end{TableBase}
}

% usage: \begin{LongTable}{TITLE}{XREF}{LAYOUT}
% produces table that handles page breaks sensibly.
\newenvironment{LongTable}[3]
{
 \newcommand{\continuedcaption}{\caption[]{#1 (continued)}}
 \begin{TableBase}
 \begin{longtable}
 {|#3|}\caption{#1}\label{#2}
}
{
 \bottomline
 \end{longtable}
 \end{TableBase}
}

% usage: \begin{twocol}{TITLE}{XREF}
% produces a two-column breakable table. Used in
% "simple-type-specifiers and the types they specify" in the standard.
\newenvironment{twocol}[2]
{
 \begin{concepttable}
 {#1}{#2}
 {ll}
}
{
 \end{LongTable}
}

% usage: \begin{libreqtabN}{TITLE}{XREF}
% produces an N-column breakable table. Used in
% most of the library Clauses for requirements tables.
% Example at "Position type requirements" in the standard.

\newenvironment{libreqtab1}[2]
{
 \begin{LongTable}
 {#1}{#2}
 {x{.55\hsize}}
}
{
 \end{LongTable}
}

\newenvironment{libreqtab2}[2]
{
 \begin{LongTable}
 {#1}{#2}
 {lx{.55\hsize}}
}
{
 \end{LongTable}
}

\newenvironment{libreqtab2a}[2]
{
 \begin{LongTable}
 {#1}{#2}
 {x{.30\hsize}x{.68\hsize}}
}
{
 \end{LongTable}
}

\newenvironment{libreqtab3}[2]
{
 \begin{LongTable}
 {#1}{#2}
 {x{.28\hsize}x{.18\hsize}x{.43\hsize}}
}
{
 \end{LongTable}
}

\newenvironment{libreqtab3a}[2]
{
 \begin{LongTable}
 {#1}{#2}
 {x{.28\hsize}x{.33\hsize}x{.29\hsize}}
}
{
 \end{LongTable}
}

\newenvironment{libreqtab3b}[2]
{
 \begin{LongTable}
 {#1}{#2}
 {x{.40\hsize}x{.25\hsize}x{.25\hsize}}
}
{
 \end{LongTable}
}

\newenvironment{libreqtab3c}[2]
{
 \begin{LongTable}
 {#1}{#2}
 {x{.30\hsize}x{.25\hsize}x{.35\hsize}}
}
{
 \end{LongTable}
}

\newenvironment{libreqtab3d}[2]
{
 \begin{LongTable}
 {#1}{#2}
 {x{.32\hsize}x{.27\hsize}x{.36\hsize}}
}
{
 \end{LongTable}
}

\newenvironment{libreqtab3e}[2]
{
 \begin{LongTable}
 {#1}{#2}
 {x{.38\hsize}x{.27\hsize}x{.25\hsize}}
}
{
 \end{LongTable}
}

\newenvironment{libreqtab3f}[2]
{
 \begin{LongTable}
 {#1}{#2}
 {x{.40\hsize}x{.22\hsize}x{.31\hsize}}
}
{
 \end{LongTable}
}

\newenvironment{libreqtab4}[2]
{
 \begin{LongTable}
 {#1}{#2}
}
{
 \end{LongTable}
}

\newenvironment{libreqtab4a}[2]
{
 \begin{LongTable}
 {#1}{#2}
 {x{.14\hsize}x{.30\hsize}x{.30\hsize}x{.14\hsize}}
}
{
 \end{LongTable}
}

\newenvironment{libreqtab4b}[2]
{
 \begin{LongTable}
 {#1}{#2}
 {x{.13\hsize}x{.15\hsize}x{.29\hsize}x{.27\hsize}}
}
{
 \end{LongTable}
}

\newenvironment{libreqtab4c}[2]
{
 \begin{LongTable}
 {#1}{#2}
 {x{.16\hsize}x{.21\hsize}x{.21\hsize}x{.30\hsize}}
}
{
 \end{LongTable}
}

\newenvironment{libreqtab4d}[2]
{
 \begin{LongTable}
 {#1}{#2}
 {x{.22\hsize}x{.22\hsize}x{.30\hsize}x{.15\hsize}}
}
{
 \end{LongTable}
}

\newenvironment{libreqtab5}[2]
{
 \begin{LongTable}
 {#1}{#2}
 {x{.14\hsize}x{.14\hsize}x{.20\hsize}x{.20\hsize}x{.14\hsize}}
}
{
 \end{LongTable}
}

% usage: \begin{libtab2}{TITLE}{XREF}{LAYOUT}{HDR1}{HDR2}
% produces two-column table with column headers HDR1 and HDR2.
% Used in "seekoff positioning" in the standard.
\newenvironment{libtab2}[5]
{
 \begin{floattable}
 {#1}{#2}{#3}
 \topline
 \lhdr{#4}	&	\rhdr{#5}	\\ \capsep
}
{
 \end{floattable}
}

% usage: \begin{longlibtab2}{TITLE}{XREF}{LAYOUT}{HDR1}{HDR2}
% produces two-column table with column headers HDR1 and HDR2.
\newenvironment{longlibtab2}[5]
{
 \begin{LongTable}{#1}{#2}{#3}
 \\ \topline
 \lhdr{#4}	&	\rhdr{#5}	\\ \capsep
 \endfirsthead
 \continuedcaption\\
 \topline
 \lhdr{#4}	&	\rhdr{#5}	\\ \capsep
 \endhead
}
{
  \end{LongTable}
}

% usage: \begin{LibEffTab}{TITLE}{XREF}{HDR2}{WD2}
% produces a two-column table with left column header "Element"
% and right column header HDR2, right column word-wrapped with
% width specified by WD2.
\newenvironment{LibEffTab}[4]
{
 \begin{libtab2}{#1}{#2}{lp{#4}}{Element}{#3}
}
{
 \end{libtab2}
}

% Same as LibEffTab except that it uses a long table.
\newenvironment{longLibEffTab}[4]
{
 \begin{longlibtab2}{#1}{#2}{lp{#4}}{Element}{#3}
}
{
 \end{longlibtab2}
}

% usage: \begin{libefftab}{TITLE}{XREF}
% produces a two-column effects table with right column
% header "Effect(s) if set", width 4.5 in. Used in "fmtflags effects"
% table in standard.
\newenvironment{libefftab}[2]
{
 \begin{LibEffTab}{#1}{#2}{Effect(s) if set}{4.5in}
}
{
 \end{LibEffTab}
}

% Same as libefftab except that it uses a long table.
\newenvironment{longlibefftab}[2]
{
 \begin{longLibEffTab}{#1}{#2}{Effect(s) if set}{4.5in}
}
{
 \end{longLibEffTab}
}

% usage: \begin{libefftabmean}{TITLE}{XREF}
% produces a two-column effects table with right column
% header "Meaning", width 4.5 in. Used in "seekdir effects"
% table in standard.
\newenvironment{libefftabmean}[2]
{
 \begin{LibEffTab}{#1}{#2}{Meaning}{4.5in}
}
{
 \end{LibEffTab}
}

% Same as libefftabmean except that it uses a long table.
\newenvironment{longlibefftabmean}[2]
{
 \begin{longLibEffTab}{#1}{#2}{Meaning}{4.5in}
}
{
 \end{longLibEffTab}
}

% usage: \begin{libefftabvalue}{TITLE}{XREF}
% produces a two-column effects table with right column
% header "Value", width 3 in. Used in "basic_ios::init() effects"
% table in standard.
\newenvironment{libefftabvalue}[2]
{
 \begin{LibEffTab}{#1}{#2}{Value}{3in}
}
{
 \end{LibEffTab}
}

% usage: \begin{libefftabvaluenarrow}{TITLE}{XREF}
% produces a two-column effects table with right column
% header "Value", width 1.5 in. Used in basic_string_view effects
% tables in standard.
\newenvironment{libefftabvaluenarrow}[2]
{
 \begin{LibEffTab}{#1}{#2}{Value}{1.3in}
}
{
 \end{LibEffTab}
}

% Same as libefftabvalue except that it uses a long table and a
% slightly wider column.
\newenvironment{longlibefftabvalue}[2]
{
 \begin{longLibEffTab}{#1}{#2}{Value}{3.5in}
}
{
 \end{longLibEffTab}
}

% usage: \begin{liberrtab}{TITLE}{XREF} produces a two-column table
% with left column header ``Value'' and right header "Error
% condition", width 4.5 in. Used in regex Clause in the TR.

\newenvironment{liberrtab}[2]
{
 \begin{libtab2}{#1}{#2}{lp{4.5in}}{Value}{Error condition}
}
{
 \end{libtab2}
}

% Like liberrtab except that it uses a long table.
\newenvironment{longliberrtab}[2]
{
 \begin{longlibtab2}{#1}{#2}{lp{4.5in}}{Value}{Error condition}
}
{
 \end{longlibtab2}
}

% usage: \begin{lib2dtab2base}{TITLE}{XREF}{HDR1}{HDR2}{WID1}{WID2}{WID3}
% produces a table with one heading column followed by 2 data columns.
% used for 2D requirements tables, such as optional::operator= effects
% tables.
\newenvironment{lib2dtab2base}[7]
{
 %% no lines in the top-left cell, and leave a gap around the headers
 %% FIXME: I tried to use hhline here, but it doesn't appear to support
 %% the join between the leftmost top header and the topmost left header,
 %% so we fake it with an empty row and column.
 \newcommand{\topline}{\cline{3-4}}
 \newcommand{\rowsep}{\cline{1-1}\cline{3-4}}
 \newcommand{\capsep}{
  \topline
  \multicolumn{4}{c}{}\$-0.8\normalbaselineskip]
  \rowsep
 }
 \newcommand{\bottomline}{\rowsep}
 \newcommand{\hdstyle}[1]{\textbf{##1}}
 \newcommand{\rowhdr}[1]{\hdstyle{##1}&}
 \newcommand{\colhdr}[1]{\multicolumn{1}{|>{\centering}m{#6}|}{\hdstyle{##1}}}
 %% FIXME: figure out a way to reuse floattable here
 \begin{table}[htbp]
 \caption{\label{#2}#1}
 \begin{center}
 \begin{tabular}{|>{\centering}m{#5}|@{}p{0.2\normalbaselineskip}@{}|m{#6}|m{#7}|}
 %% table header
 \topline
 \multicolumn{1}{c}{}&&\colhdr{#3}&\colhdr{#4}\\
 \capsep
}
{
 \bottomline
 \end{tabular}
 \end{center}
 \end{table}
}

\newenvironment{lib2dtab2}[4]{
 \begin{lib2dtab2base}{#1}{#2}{#3}{#4}{1.2in}{1.8in}{1.8in}
}{
 \end{lib2dtab2base}
}


\makeindex[generalindex]
\makeindex[libraryindex]
\makeindex[grammarindex]

%%--------------------------------------------------
%% turn off all ligatures inside \texttt
\begin{document}
\chapterstyle{xstd}
\pagestyle{xpage}

%%--------------------------------------------------
%% configuration
%!TEX root = x.tex
%%--------------------------------------------------
%% Version numbers
\newcommand{\docno}{0.1}

%% Release date
\newcommand{\reldate}{\today}

%% Library chapters
\newcommand{\firstlibchapter}{core.misc}
\newcommand{\lastlibchapter}{core.sequence}


%%--------------------------------------------------
%% front matter
\frontmatter
%!TEX root = x.tex
%!TEX root = x.tex
%%--------------------------------------------------
%% Title page


\thispagestyle{empty}
\begingroup
\def\hd{\begin{tabular}{ll}
          \textbf{文档编号:} & {\larger\docno}             \\
          \textbf{日期:}            & \reldate                    \\
          \end{tabular}
}
\newlength{\hdwidth}
\settowidth{\hdwidth}{\hd}
\hfill\begin{minipage}{\hdwidth}\hd\end{minipage}
\endgroup

\vspace{2.5cm}
\begin{center}
\textbf{\Huge 编程语言 \X}
\end{center}
\newpage

% \input{cover-reg}

%%--------------------------------------------------
%% The table of contents, list of tables, and list of figures
\markboth{\contentsname}{}

%%--------------------------------------------------
%% Make a bit more room for our long page numbers.
\makeatletter
\renewcommand\@pnumwidth{2.5em}
\makeatother

\renewcommand{\figurename}{表}%
\renewcommand{\contentsname}{目录}
\renewcommand{\listtablename}{表格列表}
\renewcommand{\listfigurename}{图片列表}

\tableofcontents
\setcounter{tocdepth}{5}
\newpage
\listoftables
\newpage
\listoffigures


%%--------------------------------------------------
%% main body of the document
\mainmatter
\setglobalstyles

%!TEX root = x.tex

\rSec0[lex]{词法约定}

\begin{bnf}{Token}
    Identifier \br
    Keyword \br
    Punctuator \br
    Literal \br
    LambdaParameter \br
    MacroInvocation
\end{bnf}

\begin{bnf}{TokenDelimited}
    \terminal{(} TokenList\bnfq \terminal{)} \br
    \terminal{[} TokenList\bnfq \terminal{]} \br
    \terminal{\{} TokenList\bnfq \terminal{\}}
\end{bnf}

\begin{bnf}{TokenList}
    TokenListItem\bnfp
\end{bnf}

\begin{bnf}{TokenListItem}
    Token \textnormal{但不是} \terminal{( ) [ ] \{ \}} \br
    TokenDelimited
\end{bnf}

\pnum
\term{程序文本}指将被翻译为\X{}程序的文本的整体或者一部分。它存储在\term{源文件}中,并以UTF-8编码读取。

\pnum
程序文本将被分割为\term{标记}的序列。标记是程序文本中的最小单元,包括标识符、关键字、标点符号、字面量、lambda参数。
除了少数地方,标记之间包含的空白字符或注释会被忽略。它们不影响程序含义。

\pnum
宏调用是特殊的标记,它将在编译时展开为标记序列。

\rSec1[lex.comment]{注释}

\pnum
有两种形式的\term{注释}:以\tcode{/*} 开始,\tcode{*/}结束的\term{块注释}和以\tcode{//} 开始,到行末结束的\term{行注释}。注释可以嵌套。在将程序文本分割为标记以后,注释和空白一起被删除。

\rSec1[lex.identifier]{标识符}

\begin{bnf}{Identifier}
    NormalIdentifier \br
    RawIdentifier
\end{bnf}

\begin{bnf}{NormalIdentifier}
    IdentifierHead IdentifierTail\bnfs
\end{bnf}

\begin{bnf}{IdentifierHead}
    Unicode(XID_Start) \br
    \terminal{_}
\end{bnf}

\begin{bnf}{IdentifierTail}
    IdentifierHead \br
    Unicode(XID_Continue)
\end{bnf}

\begin{bnf}{RawIdentifier}
    \terminal{`} RIchar\bnfp \terminal{`}
\end{bnf}

\begin{bnf}{RIchar}
    \textnormal{除了\terminal{`}以外的非空白可打印字符} \br
    \textnormal{空格}
\end{bnf}

\pnum
\term{标识符}以具有Unicode XID_Start属性的字符或\tcode{_}开始,后跟零个或数个具有Unicode XID_Continue属性的字符,但不能与关键字相同。标识符区分大小写。

\pnum
使用反引号包围的字符序列称为\term{原始标识符}。原始标识符可以用于将无法作为普通标识符的字符序列作为标识符使用。

\pnum
如果原始标识符中的字符序列是普通标识符的字符系列,则它与不带反引号的形式完全相同,但它不会被认为是关键字。

\pnum
原始标识符中不能包含空格以外的空白字符以及\tcode{`}。原始标识符不能以\tcode{\$}开头。

\rSec1[lex.keyword]{关键字}

\pnum
\begin{floattable}{关键字}{tab:keyword}{lllll}
\topline
\tcode{_}         & \tcode{any}       & \tcode{as}        & \tcode{assert}    & \tcode{async}     \\
\tcode{await}     & \tcode{auto}      & \tcode{bool}      & \tcode{borrow}    & \tcode{break}     \\
\tcode{catch}     & \tcode{char}      & \tcode{class}     & \tcode{cmp}       & \tcode{const}     \\
\tcode{continue}  & \tcode{defer}     & \tcode{do}        & \tcode{dyn}       & \tcode{else}      \\
\tcode{enum}      & \tcode{extern}    & \tcode{false}     & \tcode{float}     & \tcode{for}       \\
\tcode{func}      & \tcode{if}        & \tcode{impl}      & \tcode{import}    & \tcode{in}        \\
\tcode{infer}     & \tcode{init}      & \tcode{int}       & \tcode{internal}  & \tcode{is}        \\
\tcode{let}       & \tcode{macro}     & \tcode{match}     & \tcode{module}    & \tcode{mut}       \\
\tcode{never}     & \tcode{nil}       & \tcode{operator}  & \tcode{partial}   & \tcode{private}   \\
\tcode{public}    & \tcode{ref}       & \tcode{return}    & \tcode{self}      & \tcode{shl}       \\
\tcode{shl_eq}    & \tcode{shr}       & \tcode{shr_eq}    & \tcode{some}      & \tcode{static}    \\
\tcode{string}    & \tcode{this}      & \tcode{throw}     & \tcode{trait}     & \tcode{true}      \\
\tcode{try}       & \tcode{type}      & \tcode{typeof}    & \tcode{uint}      & \tcode{unsafe}    \\
\tcode{void}      & \tcode{while}     &                   &                   &                   \\
\end{floattable}

\pnum
表~\ref{tab:context-keyword}~中的标识符称为\term{上下文关键字}。在特定的语法结构中它将被解析为关键字,在其他位置可以当作一般标识符使用。
\begin{floattable}{上下文关键字}{tab:context-keyword}{lllll}
\topline
\tcode{didSet}  & \tcode{get}     & \tcode{infix}   & \tcode{lazy}    & \tcode{once}    \\
\tcode{prefix}  & \tcode{root}    & \tcode{set}     & \tcode{suffix}  & \tcode{super}   \\
\tcode{then}    & \tcode{willSet} &                 &                 &                 \\
\end{floattable}

\rSec1[lex.punc]{标点符号}
\indextext{标点符号}

\begin{bnf}{Punctuator}
    PunctuatorPart\bnfp
\end{bnf}

\begin{bnf}{PunctuatorPart}[\oneof]
    \terminal{\~ ! @ \# \$ \% \textasciicircum \& * ( ) - + = [ ] \{ \} | ; : ' < > , . ? /}
\end{bnf}

\pnum
\term{标点符号}由一个或数个符号组成,其中的一部分称为\term{运算符},参见~\ref{op}。

\pnum
除了多字符标点符号外,标点符号都由单个字符组成。表~\ref{tab:multichar-punc}~列出了内建的多字符标点,但不包含运算符。解析标点符号时,应尽可能长地匹配多字符标点符号。用户也可以自定义多字符运算符。

\begin{floattable}{多字符标点符号}{tab:multichar-punc}{lllll}
\topline
\tcode{'(}  &
\tcode{->}  &
\tcode{=>}  &
\tcode{!!}  &\\
\tcode{::}  &
\tcode{...} &&&\\
\end{floattable}

\rSec1[lex.literal]{字面量}

\begin{bnf}{Literal}
    IntegerLiteral \br
    FloatingLiteral \br
    StringLiteral \br
    CharacterLiteral \br
    SymbolLiteral \br
    BooleanLiteral
\end{bnf}

\rSec2[literal.integer]{整数字面量}

\begin{bnf}{IntegerLiteral}
    DecimalLiteral Suffix\bnfq \br
    BinaryLiteral Suffix\bnfq \br
    HexadecimalLiteral Suffix\bnfq
\end{bnf}

\begin{bnf}{DecimalLiteral}
    Digits
\end{bnf}

\begin{bnf}{Digits}
    Digit \br
    Digits \terminal{'}\bnfq Digit
\end{bnf}

\begin{bnf}{Digit}[\oneof]
    \terminal{0 1 2 3 4 5 6 7 8 9}
\end{bnf}

\begin{bnf}{BinaryLiteral}
    \terminal{0b} BinaryDigit \bnflp\terminal{'}\bnfq BinaryDigit\bnfrp\bnfs
\end{bnf}

\begin{bnf}{BinaryDigit}
    \terminal{0} \br
    \terminal{1}
\end{bnf}

\begin{bnf}{HexadecimalLiteral}
    \terminal{0x} HexadecimalDigits
\end{bnf}

\begin{bnf}{HexadecimalDigits}
    HexadecimalDigit \br
    HexadecimalDigits \terminal{'}\bnfs HexadecimalDigit
\end{bnf}

\begin{bnf}{HexadecimalDigit}[\oneof]
    \terminal{0 1 2 3 4 5 6 7 8 9} \br
    \terminal{A B C D E F} \br
    \terminal{a b c d e f}
\end{bnf}

\pnum
整数字面量由一系列数字构成。可以使用单引号作分隔并且不影响字面量的值。字面量的前缀用于指示它的进制。\term{十进制字面量}由若干十进制数字构成;\term{十六进制字面量}由前缀\tcode{0x}后跟若干十六进制数字构成;\term{二进制字面量}前缀\tcode{0b}后跟若干二进制数字构成。\X 不支持八进制字面量。

\pnum
整数字面量的值为其数字序列表示的值,依不同前缀分别为十进制、十六进制或二进制。最左侧的数字为最高位。字面量的类型参见表格~\ref{tab:integer-suffix},其中$i$为其字面值:

\begin{floattable}{整数字面量后缀}{tab:integer-suffix}{c|c||c|c}
\topline
后缀 & 对应的类型 & 后缀 & 对应的类型 \\
\capsep
无          & $\tcode{int}_i$     & \tcode{u}   & $\tcode{uint}_i$     \\
\tcode{i8}  & $\tcode{int<8>}_i$  & \tcode{u8}  & $\tcode{uint<8>}_i$  \\
\tcode{i16} & $\tcode{int<16>}_i$ & \tcode{u16} & $\tcode{uint<16>}_i$ \\
\tcode{i32} & $\tcode{int<32>}_i$ & \tcode{u32} & $\tcode{uint<32>}_i$ \\
\tcode{i64} & $\tcode{int<64>}_i$ & \tcode{u64} & $\tcode{uint<64>}_i$ \\
\tcode{i128} & $\tcode{int<128>}_i$ & \tcode{u128} & $\tcode{uint<128>}_i$ \\
\tcode{f16} & \tcode{float<16>} &
\tcode{f32} & \tcode{float<32>} \\
\tcode{f} 或 \tcode{f64} & \tcode{float<64>} &
\tcode{f128} & \tcode{float<128>} \\
\end{floattable}

如果字面量的字面值超出了其类型的约束范围,则这是一个编译错误。

\rSec2[literal.floating]{浮点字面量}

\begin{bnf}{FloatingLiteral}
    DecimalFloatingLiteral Suffix\bnfq \br
    HexadecimalFloatingLiteral Suffix\bnfq
\end{bnf}

\begin{bnf}{DecimalFloatingLiteral}
    Digits \terminal{.} Digits ExponentPart\bnfq \br
    Digits ExponentPart
\end{bnf}

\begin{bnf}{HexadecimalFloatingLiteral}
    HexadecimalPrefix HexadecimalDigits \terminal{.} HexadecimalDigits BinaryExponentPart\bnfq \br
    HexadecimalPrefix HexadecimalDigits BinaryExponentPart
\end{bnf}

\begin{bnf}{ExponentPart}
    \terminal{e} Sign\bnfq Digit\bnfp \br
    \terminal{E} Sign\bnfq Digit\bnfp
\end{bnf}

\begin{bnf}{BinaryExponentPart}
    \terminal{p} Sign\bnfq Digit\bnfp \br
    \terminal{P} Sign\bnfq Digit\bnfp
\end{bnf}

\begin{bnf}{Sign}[\oneof]
    \terminal{+ -}
\end{bnf}

\pnum
浮点字面量用于表示浮点数,其中的单引号用作分隔并且不影响字面量的值。浮点字面量的小数点前后不允许省略数字。

\pnum
浮点字面量的类型按表~\ref{tab:floating-suffix}~确定。其值依如下方式确定:如果它包含指数部分,则命$e$为指数部分按十进制数字解析得到的数;否则,$e$为0。对十进制浮点字面量而言,命$s$为除去指数的部分按十进制数字解析得到的数,则令$f=s\times 10^e$。对十六进制浮点字面量而言,命$s$为除去指数的部分按十六进制数字解析得到的数,则令$f=s\times 2^e$。浮点字面量的值为其类型中最接近$f$的值。如果$f$太大,则值为对应的正无限大;如果$f$太小,则值为0。

\begin{floattable}{浮点字面量后缀}{tab:floating-suffix}{c|c}
\topline
后缀 & 对应的类型 \\
\capsep
\tcode{f16} & \tcode{float<16>} \\
\tcode{f32} & \tcode{float<32>} \\
无或\tcode{f64} & \tcode{float<64>} \\
\tcode{f128} & \tcode{float<128>} \\
\end{floattable}

\rSec2[literal.string]{字符串字面量}

\begin{bnf}{StringLiteral}
    \terminal{"} Schar\bnfs \terminal{"} Suffix\bnfq \br
    \terminal{@}\bnfp \terminal{"} Rchar\bnfs \terminal{"} \terminal{@}\bnfp Suffix\bnfq
\end{bnf}

\begin{bnf}{Schar}
    \textnormal{除了\terminal{\textbackslash}和\terminal{"}以外的可打印字符} \br
    EscapeSeq \br
    TextInterpolation
\end{bnf}

\begin{bnf}{EscapeSeq}
    \terminal{\textbackslash} SimpleEscape \br
    \terminal{\textbackslash{}u\{} HexadecimalDigit\bnfp \terminal{\}}
\end{bnf}

\begin{bnf}{TextInterpolation}
    \terminal{\textbackslash{}(} Expression \terminal{)}
\end{bnf}

\begin{bnf}{SimpleEscape}[\oneof]
    \terminal{0 ' " \textbackslash{} a b f n r t v}
\end{bnf}

\begin{bnf}{Rchar}
    \textnormal{可打印字符} \br
    RawTextInterpolation
\end{bnf}

\begin{bnf}{RawTextInterpolation}
    \terminal{\textbackslash} \terminal{@}\bnfp \terminal{(}  Expression \terminal{)}
\end{bnf}

\pnum
字符串字面量表示UTF-8字符串,其类型为\tcode{string}。
普通字符串字面量被双引号包围。其中可以使用反斜杠开始的转义序列表示其他字符。
普通字符串字面量也可以前后加上等量的\tcode{@},此时它被称为\term{原始字符串字面量}。原始字符串字面量中的反斜杠不会被解释为转义序列,而是字符本身。

\pnum
字符串字面量可以横跨多行,但其开头引号必须为其所在行的最后一个字符,且结尾的引号必须为其所在行的第一个非空白字符。
结尾引号之前的所有空白字符将作为这个字符串字面量的\term{行前缀}。
开头引号到结尾引号之间的所有字符(包括换行符)按顺序成为该字符串字面量的内容,但除了以下字符:

\begin{itemize}
    \item 开头引号之后紧邻的换行;
    \item 每一行的行前缀;
    \item 结尾引号前一行的换行符;
    \item 如果一行末尾有反斜杠,这个反斜杠和其后的换行符。
\end{itemize}

换行符会变成\tcode{'\textbackslash{}n'}。
如果开始引号后不是换行符,或者结束引号前有非空白字符,这是一个编译错误。
如果行前缀不是某一行的前缀,这是一个编译错误。

\enterexample
表~\ref{tab:multiline-string}~是一些多行字符串字面量及其等价的单行表示:

\begin{floattable}{多行字符串字面量示例}{tab:multiline-string}{l|l}
\topline
\begin{codeblock}
"
abc
"
\end{codeblock}
&\tcode{"abc"}\\
\hline

\begin{codeblock}
"
abc\
def
"
\end{codeblock}
&\tcode{"abcdef"}\\
\hline

\begin{codeblock}
"
abc
   "
\end{codeblock}
&编译错误,空格不是公共前缀\\
\hline

\begin{codeblock}
"

abc

"
\end{codeblock}
&\tcode{"\textbackslash{}nabc\textbackslash{}n"}\\
\hline

\end{floattable}
\exitexample

\pnum
字符串字面量可以包含字符串插值,其形式为\tcode{\textbackslash($e$)},其中$e$为表达式。字符串插值会被求值后转换为字符串插入当前位置。对原始字符串字面量而言,反斜杠和括号之间需要插入等量的\tcode{@},否则仍然会被解释为字面符号。

\rSec2[literal.char]{字符字面量}

\begin{bnf}{CharacterLiteral}
    \terminal{'} Character \terminal{'}
\end{bnf}

\begin{bnf}{Character}
    \textnormal{除了\terminal{\textbackslash}和\terminal{'}以外的非换行可打印字符} \br
    EscapeSeq
\end{bnf}

\pnum
字符字面量表示单个字符,其类型为\tcode{char}。字符字面量由单引号包围,其中的字符可以是除了单引号和反斜杠以外的任意字符,或者转义序列。

\rSec2[literal.symbol]{符号字面量}

\begin{bnf}{SymbolLiteral}
    \terminal{'} Identifier
\end{bnf}

\pnum
符号字面量用于标识成员名称,其类型和其值为其标识符的值。

\rSec2[literal.boolean]{布尔字面量}

\begin{bnf}{BooleanLiteral}
    \terminal{true} \br
    \terminal{false}
\end{bnf}

$$ \tcode{true} \coloneqq \langle \mathrm{true}, \tcode{bool} \rangle $$
$$ \tcode{false} \coloneqq \langle \mathrm{false}, \tcode{bool} \rangle $$

\pnum
布尔字面量的类型为\tcode{bool}。\tcode{true}和\tcode{false}分别对应其真值与假值。

\rSec2[literal.suffix]{字面量后缀}

\begin{bnf}{Suffix}
    \terminal{_}\bnfq SuffixIdentifier
\end{bnf}

\begin{bnf}{SuffixIdentifier}
    SuffixIdentifierHead SuffixIdentifierTail\bnfs
\end{bnf}

\begin{bnf}{SuffixIdentifierHead}
    Unicode(XID_Start)
\end{bnf}

\begin{bnf}{SuffixIdentifierTail}
    SuffixIdentifierHead \br
    Unicode(XID_Continue)
\end{bnf}

\pnum
整数、浮点数与字符串能带有内建或用户自定义的后缀。后缀由字母开始,后跟任意数量的字母或数字,字面量与后缀之间可以添加\tcode{_}分隔。用户自定义的后缀不能与内建后缀相同,否则这是一个编译错误。

\pnum
自定义后缀的规则与标识符相同,但会自动去除前导的\tcode{_}。如果前导的下划线多于一个,这是一个编译错误。
自定义后缀不能与内建后缀相同。

\enterexample
\begin{codeblock}
IntegerLiteral<'s>; // 后缀为\tcode{s}
IntegerLiteral<'_s>; // 错误,后缀不能包含前导下划线

0x0123ABC; // 没有后缀
0x0123_ABC; // 后缀为\tcode{ABC},下划线用作区分
\end{codeblock}
\exitexample

\pnum
特征\tcode{IntegerLiteral}、\tcode{FloatingLiteral}、\tcode{StringLiteral}、\tcode{CharacterLiteral}用于实现具有特定后缀的字面量。
如果有多于一个类型实现了相同的后缀或者提供了非法的后缀,这是一个编译错误。

\rSec1[lex.lambda-param]{Lambda参数}

\begin{bnf}{LambdaParameter}
    \terminal{\$} Digit\bnfp \br
    \terminal{\$} Identifier
\end{bnf}

\pnum
Lambda参数只能在lambda作用域(\ref{scope.lambda})中使用,用于引用匿名参数。其类型是待推导的。
不在lambda作用域中使用lambda参数,或在显式指定参数的lambda表达式中使用lambda参数,是一个编译错误。
%!TEX root = x.tex

\rSec0[basic]{基本概念}

\pnum
\term{实体}包括对象、函数、类型、模块、运算符、扩展。

\rSec1[scope]{作用域}
\indextext{作用域}

\pnum
\term{作用域}是一段程序文本。特定的语言功能可能只能在特定的作用域中生效。不同的作用域具有不同的类型,分别被不同的语言功能所引用。作用域可以互相包含。

\pnum
\term{全局作用域}是整个程序文本代表的作用域,包含所有其他作用域。

\rSec2[scope.decl]{声明作用域}
\indextext{作用域!声明}

\pnum
声明作用域限制声明的范围。一个声明或绑定将会被插入到最近的声明作用域中,并且在该作用域内可以使用该名称引用被声明的实体。
在离开该作用域之后,被声明的实体将不能被使用该方式引用。

\pnum
所有的语句都具有声明作用域。全局作用域也是声明作用域。

\rSec2[scope.func]{函数作用域}
\indextext{作用域!函数}

\pnum
函数作用域限制\tcode{return}语句的使用。参见~\ref{stmt.control}。

\pnum
函数定义的块、lambda表达式的块或表达式、\tcode{do}表达式的块和函数调用表达式的lambda块具有函数作用域。

\rSec2[scope.lambda]{Lambda 作用域}
\indextext{作用域!lambda}

\pnum
Lambda作用域限制lambda参数的使用。参见~\ref{expr.lambda.param}。

\pnum
lambda表达式的块或表达式以及函数调用表达式的lambda块具有lambda作用域。

\rSec2[scope.sequence]{序列作用域}
\indextext{作用域!序列}

\pnum
序列作用域限制\tcode{\$}的使用,参见~\ref{expr.dollar}。序列作用域有一个关联的当前序列值,无论它是否实现\tcode{core.Sequence}。

\pnum
下标运算符$s\tcode{[}\ldots\tcode{]}$的两个方括号之间具有序列作用域。其当前序列为$s$。

\pnum
函数调用表达式$s\tcode{(}\ldots\tcode{)}$的括号之间具有序列作用域。如果$s$形如$o\tcode{.}f$,且$f$是一个方法,则其当前序列为$o$,否则当前序列为$s$。\enternote 此处只能进行一次拆分,即$a\tcode{.}b\tcode{.}c$的当前序列不可能为$a$。\exitnote

\pnum
如果函数调用表达式带有一个lambda块,则这整个块也具有序列作用域,其当前序列确定方法同上。

\enterexample

\begin{codeblock}
let a = [1, 2, 3, 4, 5];
let o = { a };
func operator()(this: int[]) -> int[] { [] }
func v(this: int[], index: int) -> int[] { [] }

a[$ - 1] // 当前序列为a
a($ - 1) // 当前序列为a
o.a[$ - 1] // 成员访问,当前序列为o.a
a.v($ - 1) // 方法调用,当前序列为a

\end{codeblock}

\exitexample

\rSec1[name]{名称}
\indextext{名称}

\begin{bnf}{UnqualID}
    Identifier \br
    SymbolLiteral \br
    \terminal{init} \br
    \terminal{deinit} \br
    \terminal{operator} Operator \br
    \terminal{operator} StringLiteral \br
    \terminal{operator} \terminal{(} \terminal{)} \br
    \terminal{operator} \terminal{[} \terminal{]} \br
    \terminal{operator} \terminal{as}
\end{bnf}

\pnum
\term{名称}用于引用程序实体。一个名称可能是一个标识符、\tcode{init}、\tcode{deinit}或一个运算符名称。

\rSec2[name.lookup]{名称查找}
\indextext{名称查找}

\pnum
\term{名称查找}用于解析一个 \grammarterm{IDExpr} 具体指代的实体。

%!TEX root = x.tex

\rSec0[typesystem]{类型系统}

\rSec1[type]{类型、值和对象}
\indextext{类型}

\begin{bnf}{Type}
    NormalType \br
    ResultType
\end{bnf}

\begin{bnf}{NormalType}
    \terminal{(} Type \terminal{)} \br
    FundaType \br
    SpecialType \br
    CompType \br
    OpaqueType \br
    SomeType \br
    AnyType \br
    \terminal{typeof} \terminal{(} Expression \terminal{)} \br
    Type TypeQualifier
\end{bnf}

\begin{bnf}{TypeQualifier}
    \terminal{mut} \br
    \terminal{const}
\end{bnf}

$$ \mathcal{V} = \{ \langle v, T, Q \rangle \mid T \in \mathcal{T}, v \in T, Q \subset \mathcal{Q} \} $$

\pnum
\term{类型}是一个集合。\term{值}是类型、类型的成员和修饰符集合的元组。$T$称为值$v$的\term{类型}。

\rSec2[type.funda]{基本类型}
\indextext{类型!基本类型}

\begin{bnf}{FundaType}
    \terminal{void} \br
    \terminal{never} \br
    \terminal{bool} \br
    \terminal{int} \br
    \terminal{uint} \br
    \terminal{int} \terminal{<} Expression \terminal{>} \br
    \terminal{uint} \terminal{<} Expression \terminal{>} \br
    \terminal{float} \br
    \terminal{float} \terminal{<} Expression \terminal{>} \br
    \terminal{char} \br
    \terminal{string} \br
    SymbolType
\end{bnf}

$$\tcode{void} \coloneqq \{ \mathrm{void} \}$$

\pnum
\tcode{void}标识只有唯一一个值的类型。

$$\tcode{never} \coloneqq \{ \}$$

\pnum
\tcode{never}标识没有值的类型。

$$\tcode{bool} \coloneqq \{ \mathrm{true}, \mathrm{false} \}$$

\pnum
\tcode{bool}标识具有真或假两个值的类型。

$$\tcode{int}_{l,h} \coloneqq \{ x \in \mathcal{Z} \mid l \le x \le h \} $$

\pnum
$\tcode{int}_{l,h}$称作\term{整数类型},其中$l$和$h$为待推导常数。在本规范中,如果$l = h$,则记作$\tcode{int}_l$。存在实现定义的常数$m$和$M$。$l$和$h$须满足
$$ l \ge m $$
$$ h \le M $$
$$ 0 \le h - l \le M $$

\tcode{uint}是$\tcode{int}_{l,h}$的别名,但满足$l\ge0$。

\pnum
\tcode{int<$w$>}是$\tcode{int}_{l,h}$的别名,但满足$l\ge-2^{w-1}$且$h\le2^{w-1}-1$。\tcode{uint<$w$>}是$\tcode{int}_{l,h}$的别名,但满足$l\ge0$且$h\le2^w-1$。其中$w$可以取8、16、32、64或128。它们称作\tcode{定长整数类型},表示长度固定的整数。只能在定长整数类型上进行位运算。

$$ \tcode{float<}s\tcode{>}^\ast \subset \mathcal{R} $$
$$ \tcode{float<}s\tcode{>}^\dagger \subset \{ +\infty, -\infty, \mathrm{NaN}s \} $$
$$ \tcode{float<}s\tcode{>} \coloneqq \tcode{float<}s\tcode{>}^\ast \cup \tcode{float<}s\tcode{>}^\dagger $$

\pnum
\tcode{float<$s$>}称作\term{浮点类型},其中$s$可以取实现定义的值(典型值为16、32、64或128)。
\tcode{float}为\tcode{float<64>}的别名。

\pnum
整数类型、定长整数类型和浮点类型称为\term{算术类型}。

$$\tcode{char} \coloneqq \left\{ \textnormal{Any Unicode scalar value} \right\}$$

\pnum
\tcode{char}表示一个Unicode标量值。值不在Unicode标量值范围内的\tcode{char}(例如,其值为代理项)是未定义的。

$$\tcode{string} \coloneqq \left\{ \textnormal{Any UTF-8 String} \right\} $$

\pnum
\tcode{string}表示任意长度的UTF-8字符串,包括空字符串。

\rSec3[type.symbol]{符号类型}
\indextext{类型!符号}

\begin{bnf}{SymbolType}
    SymbolLiteral \br
    \terminal{'(} UnqualID \terminal{)}
\end{bnf}

$$ \tcode{'}s \coloneqq \left\{ \langle \tcode{'}s, \tcode{'}s \rangle \right\} $$

$$ \tcode{Symbol} \coloneqq \bigcup_{\tcode{'}s} \left\{  \tcode{'}s \right\} $$

\pnum
符号字面量的类型与其值相同。\tcode{Symbol}是符号字面量的公共类型。参见~\ref{core.type}。

\rSec2[type.special]{特殊类型}
\indextext{类型!特殊类型}

\begin{bnf}{SpecialType}
    \terminal{self}
\end{bnf}

\pnum
\tcode{self}在特征或实现中代表当前类型。在此之外使用\tcode{self}是一个编译错误。

\rSec2[type.comp]{复合类型}
\indextext{类型!复合类型}

\begin{bnf}{CompType}
    Type \terminal{?} \br
    Type \terminal{[} \terminal{]} \br
    Type \terminal{[} Type \terminal{]} \br
    \terminal{(} TupleTypes\bnfs \terminal{)} \br
    \terminal{\{} StructTypes \terminal{\}} \br
    \terminal{(} TupleTypes\bnfs \terminal{)} \terminal{->} Type \br
    UnionType \br
    Type \terminal{\&} \br
    FuncType \br
    OpaqueType
\end{bnf}

\rSec3[type.optional]{可空类型}
\indextext{类型!可空}

$$ T\tcode{?} \coloneqq \{ \langle t \rangle \mid t \in T \} \cup \{ \mathrm{nil} \} $$

\pnum
\tcode{$T$?}为\term{可空类型}。
\tcode{$T$?}要么包含一个$T$的值(使用\tcode{some}标识),要么不包含任何值(使用\tcode{nil}标识)。

\pnum
可空类型在核心库中对应\tcode{core::Optional}。参见~\ref{core.type}。

\rSec3[type.array]{数组类型}
\indextext{类型!数组}

$$ T\tcode{[]} \coloneqq \bigcup^\infty_{i=1} T^i $$

\pnum
\tcode{$T$[]}为\term{数组类型},代表有限个类型$T$的值的序列。

\pnum
数组类型在核心库中对应\tcode{core::Array}。参见~\ref{core.type}。

\rSec3[type.dict]{字典类型}
\indextext{类型!字典}

$$ T\tcode{[}K\tcode{]} \coloneqq T^K $$

\pnum
\tcode{$T$[$K$]}为\term{字典类型},代表键类型$K$到值类型$T$的映射。

\pnum
字典类型在核心库中对应\tcode{core::Dictionary}。参见~\ref{core.type}。

\rSec3[type.tuple]{元组类型}
\indextext{类型!元组}

\begin{bnf}{TupleType}
    \terminal{(} TupleTypeList \terminal{)}
\end{bnf}

\begin{bnf}{TupleTypeList}
    Type \terminal{,} \br
    TupleTypeListNoComma \terminal{,}\bnfq
\end{bnf}

\begin{bnf}{TupleTypeListNoComma}
    Type \terminal{,} Type \br
    TupleTypeListNoComma \terminal{,} Type
\end{bnf}

$$ \tcode{(}T_1\tcode{,} \ldots\tcode{,} T_n\tcode{,}\tcode{)} \coloneqq \prod^n_{i=1} T_i $$
$$ \tcode{()} \coloneqq \tcode{void} $$

\pnum
\tcode{($T_1$, $\ldots$, $T_n$)}称作\term{元组类型},代表有限个值的序列。

\pnum
特别地,只有一个元素的元组需表示为\tcode{($T$,)}。没有元素的元组与\tcode{void}等价。

\rSec3[type.struct]{结构类型}
\indextext{类型!结构}

\begin{bnf}{StructTypes}
    StructType \br
    StructTypes \terminal{,} StructType
\end{bnf}

\begin{bnf}{StructType}
    StructTypeQualifier\bnfs Identifier \terminal{:} Type
\end{bnf}

\begin{bnf}{StructTypeQualifier}
    \terminal{mut}
\end{bnf}

$$ \tcode{\{} K_1\tcode{:}T_1\tcode{,} \ldots\tcode{,} K_n\tcode{:}T_n \tcode{\}} \coloneqq \prod^n_{i=1} T_i $$

\pnum
\tcode{\{$K_1$:$T_1$, $\ldots$, $K_n$:$T_n$\}}称作\term{结构类型}。结构类型可以使用标识符访问其成员。

\rSec3[type.ref]{引用类型}
\indextext{类型!引用}

\pnum
引用类型是另一个值的引用。

\rSec3[type.func]{函数类型}
\indextext{类型!函数}

\begin{bnf}{FuncType}
    ParameterInType FuncTypeQual\bnfs ReturnType \br
    Type FuncTypeQual\bnfs ReturnType
\end{bnf}

\begin{bnf}{ParameterInType}
    \terminal{(} ParamListInType\bnfq \terminal{)}
\end{bnf}

\begin{bnf}{ParamListInType}
    ThisParamDeclInType \br
    ThisParamDeclInType \terminal{,} NamedParamListInType \br
    ThisParamDeclInType \terminal{,} UnnamedParamListInType \br
    ThisParamDeclInType \terminal{,} UnnamedParamListInType \terminal{,} NamedParamListInType \br
    UnnamedParamListInType \br
    UnnamedParamListInType \terminal{,} NamedParamListInType \br
    NamedParamListInType
\end{bnf}

\begin{bnf}{UnnamedParamListInType}
    UnnamedParamDeclInType \br
    UnnamedParamListInType \terminal{,} UnnamedParamDeclInType
\end{bnf}

\begin{bnf}{NamedParamListInType}
    NamedParamDeclInType \br
    NamedParamListInType \terminal{,} NamedParamDeclInType
\end{bnf}

\begin{bnf}{UnnamedParamDeclInType}
    ParamQual\bnfq Type \br
    ParamQual\bnfq Identifier \terminal{:} Type\bnfq
\end{bnf}

\begin{bnf}{NamedParamDeclInType}
    ParamQual\bnfq \terminal{(} Identifier \terminal{)} Type\bnfq \br
    ParamQual\bnfq \terminal{(} Identifier \terminal{)} Identifier \terminal{:} Type\bnfq
\end{bnf}

\begin{bnf}{ThisParamDeclInType}
    \terminal{this} \terminal{:} Type
\end{bnf}

\begin{bnf}{FuncTypeQual}
    ThrowQual \br
    \terminal{async} \br
    \terminal{unsafe} \br
    \terminal{mut} \br
    \terminal{once}
\end{bnf}

\pnum
\tcode{($T_1$, $\ldots$, $T_n$) -> $R$}称作\term{函数类型}。函数类型标识能以函数方式调用的值。
如果函数只有一个位置参数,则可以省略括号。

\pnum
函数类型可以包含显式指定类型的\tcode{this}参数。在这种情况下,函数可以在动态成员访问表达式中作为方法被调用。

\pnum
如果函数类型包含\tcode{once}修饰符,则它只能调用一次,随后值会被消耗,它实现了\tcode{core::ops::FunctorOnce}。
如果函数类型包含\tcode{mut}修饰符,则对它的调用将会更改其值,它实现了\tcode{core::ops::FunctorMut}。
如果函数类型不包含这两种修饰符,则对它的调用不会更改其值,它实现了\tcode{core::ops::Functor}。函数均具有此类函数类型。

\rSec2[type.opaque]{不透明类型}
\indextext{类型!不透明}

\begin{bnf}{OpaqueType}
    \terminal{class} Type
\end{bnf}

\pnum
不透明类型用于从现有的类型出发构造一个相同但不能混用的类型。类型$T$与其构造的不透明类型$U$之间不具有隐式转换,但可以显式转换。同一个类型构造的复数个不透明类型之间也不能混用。

\enterexample
\begin{codeblock}

type T = class int;
type U = class int;

let i = 0;
let j: T = i; // 错误,\tcode{int}不能隐式转换为\tcode{class int}
let k: T = i as T; // 可以,显式转换
let l: U = j; // 错误,\tcode{T}和\tcode{U}之间没有隐式转换

\end{codeblock}
\exitexample

\pnum
如果创建一个元组或结构类型的不透明类型,则其各成员的默认访问级别为\tcode{private}。

\rSec2[type.impl]{\tcode{impl}类型}
\indextext{类型!\tcode{impl}}

\begin{bnf}{SomeType}
    \terminal{impl} Type \br
    \terminal{impl} \terminal{_}
\end{bnf}

\pnum
\tcode{impl $T$}标识一个静态待推导类型,但保证该类型为$T$的子类型。
它可以出现在具有初始化值的绑定的类型中或函数返回值处。
\tcode{impl _}代表一个基础类型待推导的\tcode{impl}类型。

\pnum
\tcode{impl}还能用于简化泛型函数声明。参见~\ref{generic.impl}。

\enterexample
\begin{codeblock}

trait T { }
type A { }
type B { }

impl A : T { }
impl B : T { }

let x: impl T = A { }; // \tcode{x}的类型是\tcode{A}
let mut y: impl T = B { }; // \tcode{y}的类型是\tcode{B}

y = x; // 错误,\tcode{B}不能赋给\tcode{A}

\end{codeblock}
\exitexample

\rSec2[type.dyn]{\tcode{dyn}类型}
\indextext{类型!\tcode{dyn}}

\begin{bnf}{AnyType}
    \terminal{dyn} Type \br
    \terminal{dyn} \terminal{_} \br
    \terminal{dyn}
\end{bnf}

\pnum
\tcode{dyn $T$}对$T$的子类型进行包装,保证在运行时可以接受任何为$T$的子类型的值。
\tcode{dyn _}代表一个基础类型待推导的\tcode{dyn}类型。
\tcode{dyn}表示对任意类型的包装。

\enterexample
\begin{codeblock}

trait T { }
type A { }
type B { }

impl A : T { }
impl B : T { }

let x: dyn T = A { }; // \tcode{x}的类型是\tcode{dyn T}
let mut y: dyn T = B { }; // \tcode{y}的类型是\tcode{dyn T}

y = x; // 正确,\tcode{dyn T}之间可以互相赋值

\end{codeblock}
\exitexample

\rSec2[type.result]{结果类型}
\indextext{类型!结果类型}

\begin{bnf}{ResultType}
    NormalType \terminal{!!} NormalType
\end{bnf}

\pnum
结果类型\tcode{$T$ !! $E$}标识一个可能产生错误的值,其中$T$是正常的返回值类型,$E$是错误类型,且必须实现\tcode{ErrorCode}。

\rSec2[type.named]{具名类型}

\begin{bnf}{TypeName}
    \terminal{(} Type \terminal{)} \br
    EntityID \br
    FundaType \br
    SpecialType
\end{bnf}

\pnum
\term{类型名称}在特定的语法位置表示类型,以避免潜在的语法歧义。

\rSec1[subtype]{子类型}
\indextext{类型!子类型}

\pnum
类型$A$可能是类型$B$的\term{子类型},记作$A \preceq B$。$A$可以在需要$B$的上下文中隐式转换到$B$。

\pnum
子类型关系具有自反性和传递性,即对任意类型$A$、$B$和$C$有$A \preceq A$和$A \preceq B \land B \preceq C \implies A \preceq C$成立。

\begin{equation*}
\begin{aligned}
    T \preceq \tcode{void},& T \in \mathcal{T} \\
    \tcode{never} \preceq T,& T \in \mathcal{T}
\end{aligned}
\end{equation*}

\pnum
任何类型都是\tcode{void}的子类型。\tcode{never}是任何类型的子类型。

\begin{equation*}
\begin{aligned}
I_{l_1,h_1} &\preceq J_{l_2,h_2} \mathrel{\mathrm{if}} l_1 \geq l_2 \lor h_1 \leq h_2 \\
    \tcode{float<}s_1\tcode{>} &\preceq \tcode{float<}s_2\tcode{>} \mathrel{\mathrm{if}} s_1 \leq s_2 \\
I_{l,h} &\preceq \tcode{float<}s\tcode{>} \\
    I, J &\in \left\{ \tcode{int}, \tcode{uint}, \tcode{int<}w\tcode{>}, \tcode{uint<}w\tcode{>} \right\}
\end{aligned}
\end{equation*}

\pnum
范围更小的整数类型是范围更大的整数类型的子类型。长度更小的浮点类型是长度更大的浮点类型的子类型。

\pnum
整数类型是浮点类型的子类型。当整数被隐式转换为浮点数时,其值将被转换为最接近的浮点数。\enternote 虽然整数转换到浮点数可能无法保持值不变,但出于习惯仍然保持这个隐式转换。 \exitnote \enternote 浮点类型不能隐式转换到整数类型,但可以显式转换。 \exitnote

$$ \tcode{'}s \preceq \tcode{Symbol} $$

\pnum
所有符号字面量类型都是\tcode{Symbol}的子类型。

\pnum
\tcode{bool}没有语言内建的子类型约束。但是,其他类型的值可以在\tcode{if}表达式中当作条件使用,这通过实现\tcode{Condition}完成。

\begin{equation*}
\begin{aligned}
T &\preceq T\tcode{?} \\
T\tcode{[]} &\preceq U\tcode{[]} \mathrel{\mathrm{if}} T \preceq U \\
T\tcode{[}K\tcode{]} &\preceq U\tcode{[}L\tcode{]} \mathrel{\mathrm{if}} T \preceq U \mathrel{\mathrm{and}} K \preceq L \\
\tcode{(}T_1\tcode{,} \ldots\tcode{,} T_n\tcode{)} &\preceq \tcode{(}U_1\tcode{,} \ldots\tcode{,} U_n\tcode{)} \mathrel{\mathrm{if}} T_i \preceq U_i \mathrel{\mathrm{for}} 1 \leq i \leq n \\
\end{aligned}
\end{equation*}

\pnum
任意类型是其对应可空类型的子类型。如果$T_i$是$U_i$的子类型,则\tcode{$T_0$[]}、\tcode{$T_0$[$T_1$]}、\tcode{($T_1$, $\ldots$, $T_n$)}分别是\tcode{$U_0$[]}、\tcode{$U_0$[$U_1$]}、\tcode{($U_1$, $\ldots$, $U_n$)}的子类型。\enternote 这意味着数组、字典和元组的元素类型是协变的。 \exitnote

\pnum
对两个结构类型$T$和$U$而言,如果$U$的每个成员都有对应的$T$的成员且类型是该成员的子类型,则$T$是$U$的子类型。

\pnum
对两个函数类型$T$和$U$而言,如果$U$的每个参数类型都是对应$T$的参数的子类型,且$T$的返回类型是$U$的返回类型的子类型,则$T$是$U$的子类型。\enternote 这意味着函数类型的参数类型是逆变的,返回类型是协变的。 \exitnote $T$可以拥有比$U$更多的顺序参数,或者$U$不包含的命名参数。

\pnum
类类型可以定义类型转换函数。每个这样的函数定义了一个子类型关系。

\pnum
对类型表达式而言,$T\mathbin{\tcode{\&}}U$是$T$和$U$的子类型;$T$和$U$是$T\mathbin{\tcode{|}}U$的子类型。

\rSec1[type.common]{公共类型}
\indextext{类型!公共类型}

\pnum
对两个类型$A$和$B$而言,存在一个唯一的类型$C$称为$A$和$B$的\term{公共类型}。$C$满足:

\begin{equation}
\begin{aligned}
    A &\preceq C \\
    B &\preceq C \\
    \forall D \in \mathcal{T}, A &\preceq D \land B \preceq D \Rightarrow C \preceq D
\end{aligned}
\end{equation}

记作 $C = A \otimes B$。公共类型满足交换律。

\pnum
如果$A \preceq B$,则$A \otimes B = B$。

\pnum
如果$A$和$B$之间没有子类型关系,则$A \otimes B = A \mathbin{\tcode{|}} B$。

\rSec1[qualifier]{修饰符}
\indextext{修饰符}

\pnum
值除了总是具有类型之外,还可能带有一个或数个修饰符。修饰符指示了值的其他属性。

\pnum
类型可以带有修饰符,指定该值需有特定的修饰符约束。

\rSec2[qual.mut]{\tcode{mut}}
\indextext{修饰符!\idxcode{mut}}

\pnum
\tcode{mut}表示该值是可变的。具有\tcode{mut}的值才能成为赋值操作符的左操作数。参见~\ref{expr.assign}。

\rSec2[qual.const]{\tcode{const}}
\indextext{修饰符!\idxcode{const}}

\pnum
\tcode{const}表示该值是一个常量值,于编译期间确定且不可变。部分语言功能只允许常量值。

\pnum
\tcode{const}绑定创建一个常量并插入当前作用域。该绑定的初始值必须是常量表达式。

%!TEX root = x.tex

\rSec0[expr]{表达式}
\indextext{表达式}

\begin{bnf}{Expression}
    PrimaryExpr \br
    Operator Expression \br
    Expression Operator \br
    Expression Operator Expression
\end{bnf}

% \pnum
% \term{表达式}由\term{运算符}与操作数按照顺序组合在一起,它表示一个计算过程。运算符是若干标点符号的组合、一个标识符或是语言规定的特殊结构。运算符可以是前缀、后缀或者是中缀的,这决定了运算符与其操作数的结合方式。运算符组是运算符的集合,每一个运算符都属于某个运算符组。运算符组之间有一个弱偏序关系,决定了它们的\term{优先级}。完全由中缀运算符构成的运算符组有\term{结合性}:左结合的组每个运算符从左到右选择操作数;右结合的组从右到左;无结合的组两个运算符不能选择同一个操作数。

\pnum
\ref{expr.suffix}到\ref{expr.semi}各节按照优先级从高到低依次对运算符组进行描述。若无特别说明,每一节描述一个运算符组。用户也能定义新的运算符组或在当前的组中添加新的运算符,参见\ref{op.user}。

$$ \mathrm{\rhd}: \mathcal{E} \times \Omega \rightarrow (\mathcal{V} \cup \mathcal{V}^\dag \cup \{\ast\}) \times \Omega $$

\indextext{求值}
\pnum
$\mathrm{e \rhd \omega}$ 称作\term{在环境 $\omega$ 下对 $e$ 求值}。设 $\mathrm{e \rhd \omega} = \langle v, \omega^\prime \rangle$。如果 $v \in \mathcal{V}$,称\term{求值正常结束},$v$ 是 $e$ 的\term{值};否则,称\term{求值以抛出$v$异常结束}。$v = \ast$ 意味着求值过程中程序终止了。若无特别说明,对表达式$e$的子表达式$e_0$求值以抛出$v$异常结束也会导致对$e$的求值以抛出$v$异常结束。

\pnum
$\omega$ 是求值之前的环境,$\omega^\prime$ 是求值之后的环境。如果 $\omega = \omega^\prime$,称 $e$ 是\term{无副作用}的;如果 $e$ 是无副作用的且 $v$ 与 $\omega$ 无关,称 $e$ 是\term{纯}的。

\pnum
对含有子表达式的表达式求值时,总是先对其子表达式按出现次序从左到右求值。

\pnum
\term{丢弃表达式 $e$ 的结果}指,在决定 $e$ 的类型时,直接将它确定为 \tcode{never} 而跳过所有步骤;在对 $e$ 求值时,进行所有步骤,但是如果求值正常结束,丢弃最后的值。

\pnum
完整表达式指下列情况之一:

\begin{itemize}
    \item 表达式语句中的表达式;
    \item 一个语句表达式;
    \item 绑定语句中的初始化表达式;
    \item lambda表达式的函数体表达式;
    \item \tcode{if}、\tcode{match}、\tcode{while}的条件表达式;
    \item \tcode{for}语句的迭代表达式;
    \item \tcode{return}、\tcode{throw}、\tcode{break}表达式的子表达式;
    \item \tcode{assert}表达式的子表达式;
    \item 函数参数的默认值表达式;
    \item 函数体与属性体表达式;
    \item 枚举的初始化表达式;
    \item 模式匹配与泛型的约束表达式;
    \item \tcode{typeof}的参数表达式;
    \item 泛型参数的非类型表达式。
\end{itemize}

\rSec1[expr.primary]{基本表达式}
\indextext{表达式!基本}

\begin{bnf}{PrimaryExpr}
    \terminal{(} Expression \terminal{)} \br
    Statement \br
    LiteralExpr \br
    TypeLiteral \br
    Identifier \br
    DeductedEnumerator \br
    \terminal{this} \br
    \terminal{\$} \br
    LambdaParameter \br
    LambdaExpr \br
    DoExpr \br
    TryExpr
\end{bnf}

\pnum
表达式\tcode{($e$)}等价于$e$,括号只作分组用途。\enternote 注意括号表达式不是一元元组。\exitnote

\pnum
语句都是基本表达式。但由于某些语句可以在末尾包含一个表达式,这些语句实际上只能在另一侧当作基本表达式。

\enterexample
\begin{codeblock}

return 1 + 2 // 等价于\tcode{return (1 + 2)}
1 + return 2 // 等价于\tcode{1 + (return 2)}

\end{codeblock}
\exitexample

\rSec2[expr.lit]{字面量表达式}
\indextext{表达式!字面量}

\begin{bnf}{LiteralExpr}
    IntegerLiteral \br
    FloatingLiteral \br
    StringLiteral \br
    CharacterLiteral \br
    SymbolLiteral \br
    \terminal{'(} UnqualID \terminal{)} \br
    BooleanLiteral \br
    \terminal{nil} \br
    ArrayLiteral \br
    StructLiteral \br
    DictLiteral
\end{bnf}

\begin{bnf}{ArrayLiteral}
    \terminal{[} ExprList\bnfq \terminal{]}
\end{bnf}

\begin{bnf}{ExprList}
    ExprListNoComma \terminal{,}\bnfq
\end{bnf}

\begin{bnf}{ExprListNoComma}
    ExprItem \br
    ExprListNoComma \terminal{,} ExprItem
\end{bnf}

\begin{bnf}{ExprItem}
    Expression \br
    \terminal{...} Expression
\end{bnf}

\begin{bnf}{StructLiteral}
    \terminal{(} StructItems\bnfq \terminal{)}
\end{bnf}

\begin{bnf}{StructItems}
    StructItemsNoComma \terminal{,}\bnfq
\end{bnf}

\begin{bnf}{StructItemsNoComma}
    StructItem \br
    StructItemsNoComma \terminal{,} StructItem
\end{bnf}

\begin{bnf}{StructItem}
    Identifier \terminal{:} Expression \br
    Identifier \br
    \terminal{...} Expression
\end{bnf}

\begin{bnf}{DictLiteral}
    \terminal{[} DictItems \terminal{]} \br
    \terminal{[} \terminal{:} \terminal{]}
\end{bnf}

\begin{bnf}{DictItems}
    DictItemsNoComma \terminal{,}\bnfq
\end{bnf}

\begin{bnf}{DictItemsNoComma}
    DictItem \br
    DictItemsNoComma \terminal{,} DictItem
\end{bnf}

\begin{bnf}{DictItem}
    Expression \terminal{:} Expression \br
    \terminal{...} Expression
\end{bnf}

\pnum
字面量本身是基本表达式。整数字面量、浮点字面量、字符串字面量和布尔字面量的值分别参见\ref{literal.integer}、\ref{literal.floating}、\ref{literal.string}和\ref{literal.boolean}节的定义。

\pnum
\tcode{'($id$)}表示与$id$对应的符号表达式。$id$本身不能是符号字面量。

\pnum
\tcode{nil}的类型为\tcode{$T$?},其中$T$为待推导的类型参数。

\pnum
数组字面量\tcode{[$e_1$, $\ldots$, $e_n$]}表示一个显式写出其各元素的数组值。
其类型为\tcode{$T$[]},其中$T$为各表达式的公共类型。
如果其中包含形如\tcode{...$e$}的项,则视同将$e$的各元素显式插入在该位置。$e$必须可迭代。
如果数组字面量不包含任何成员,则$T$是一个待推导的类型。

\pnum
结构字面量\tcode{($e_1$, $\ldots$, $e_m$, $k_1$:$f_1$, $\ldots$, $x_n$:$f_n$)}表示一个显式写出其各元素的结构值。
其类型为\tcode{($T_1$, $\ldots$, $T_m$, $k_1$:$U_1$, $\ldots$, $x_n$:$U_n$)},其中$T_i$为各$e_i$的类型,$U_i$为各$f_i$的类型。
特别的,\tcode{($e$,)}表示一元元组,此时逗号不能省略。\tcode{()}的类型为\tcode{void},是\tcode{void}的唯一值。

\pnum
结构字面量必须保持顺序成员在前、命名成员在后的顺序。
如果结构字面量中包含形如\tcode{...$e$}的项,则视同将$e$的各成员插入在该位置。$e$必须是结构类型。
如果该项后面有顺序成员,则该项的类型必须不包含任何命名成员;如果该项前面有命名成员,则该项的类型必须不包含任何顺序成员。

\pnum
字典字面量\tcode{[$k_1$:$v_1$, $k_2$:$v_2$, $\ldots$, $k_n$:$v_n$]}表示一个显式写出其各元素的字典值。
其类型为\tcode{$T$[$K$]},其中$T$为各$v_i$的公共类型,$K$为各$k_i$的公共类型。
如果其中包含形如\tcode{...$e$}的项,则视同将$e$的各元素对显式插入在该位置,$e$必须是字典类型。
如果字典字面量不包含任何表达式,则需写作\tcode{[:]}。此时$T$和$K$是待推导的类型。

\pnum
如果以方括号界定的字面量只包含元素展开,则其总是被识别为数组字面量。\enternote 如果需要创建一个只包含元素展开的字典字面量,请将其中一个元素改写为键值对的展开。\exitnote

\rSec2[expr.lit.init]{初始化字面量}
\indextext{表达式!字面量!初始化}

\begin{bnf}{TypeLiteral}
    TypeParenLiteral \br
    TypeArrayLiteral \br
    TypeDictLiteral
\end{bnf}

\begin{bnf}{TypeParenLiteral}
    TypeName \terminal{(} Arguments\bnfq \terminal{)} Block\bnfq \br
    TypeName Block
\end{bnf}

\begin{bnf}{TypeArrayLiteral}
    TypeName \terminal{[} ExprList\bnfq \terminal{]}
\end{bnf}

\begin{bnf}{TypeDictLiteral}
    TypeName DictLiteral
\end{bnf}

\pnum
可以使用与字面量类似的语法来创建指定类型的值。被创建的类型必须是一个类型名称。

\enterexample
\begin{codeblock}

let a = int[] [1, 2, 3]; // 错误,不能用\tcode{[]}初始化\tcode{int}

type IntArray = int[];

let a = IntArray[1, 2, 3]; // 可以

let a = (int[])[1, 2, 3]; // 可以

\end{codeblock}
\exitexample

\pnum
\tcode{$T$($a_1$, $\ldots$, $a_n$)}以给定的参数创建$T$的对象。创建对象时,会以这些参数调用给定的初始化器。以这种方式初始化的对象也可以带有lambda块。如果这个对象的初始化器只接受这一个参数,也可以省略括号。
如果不存在对应的初始化器,且该参数符合结构字面量的语法(并不包含lambda块),则会依次初始化对应的顺序成员和命名成员。

\pnum
\tcode{$T$[$v_1$, $\ldots$, $v_n$]}以对应的数组字面量为参数创建$T$的对象。
\tcode{$T$[$k_1$:$v_1$, $k_2$:$v_2$, $\ldots$, $k_n$:$v_n$]}以对应的字典字面量创建$T$的对象。

\pnum
\tcode{$T$[...$e$]}的语法歧义将按与普通字面量类似的规则解决(见上)。

\pnum
如果使用了内建的初始化方式或者采用了不抛出异常的初始化器,则结果类型为$T$;否则,结果类型为\tcode{$T$ !! $E$},其中$E$为抛出的异常类型。

\rSec2[expr.enum]{匿名静态成员表达式}

\begin{bnf}{DeductedEnumerator}
    \terminal{.} Identifier \br
    \terminal{.} Identifier \terminal{(} Arguments\bnfq \terminal{)} Block\bnfs \br
    \terminal{.} Identifier Block \br
    \terminal{.} Identifier \terminal{[} ExprList\bnfq \terminal{]}
\end{bnf}

\pnum
静态成员和枚举符可以在适当的上下文中省略类型名称而使用自动推导。参见~\ref{deduct.static}。

\rSec2[expr.this]{\tcode{this}}
\indextext{表达式!\idxcode{this}}

\pnum
表达式\tcode{this}在类方法或扩展方法中表示当前方法的调用者。如果没有在参数中显式指定\tcode{this}的类型,则其类型为\tcode{self}。

\rSec2[expr.dollar]{\tcode{\$}}
\indextext{表达式!\idxcode{\$}}

\pnum
表达式\tcode{\$}只能在序列作用域(\ref{scope.sequence})中使用。
如果当前序列为$s$,则\tcode{\$}等价于\tcode{$s$.opeartor\$()}。
如果$s$实现了\tcode{Sequence}(例如内建数组\tcode{$T$[]}),其值为\tcode{$s$.size}。
在其他位置使用\tcode{\$}是一个编译错误。

\rSec2[expr.lambda]{Lambda表达式}
\indextext{表达式!lambda}

\begin{bnf}{LambdaExpr}
    LambdaParameter LambdaQual\bnfs ReturnType\bnfq \terminal{=>} LambdaBody
\end{bnf}

\begin{bnf}{LambdaQual}
    \terminal{async} \br
    ThrowQual
\end{bnf}

\begin{bnf}{LambdaParameter}
    ParamDecl
\end{bnf}

\begin{bnf}{LambdaBody}
    Expression
\end{bnf}

\pnum
\tcode{\$}后跟数字$i$指代第$i$个参数。\tcode{\$}后跟标识符$n$指代具名参数$n$。

\rSec3[expr.do]{\tcode{do}表达式}
\indextext{表达式!\idxcode{do}}

\begin{bnf}{DoExpr}
    \terminal{do} LambdaQual\bnfs Block \br
    \terminal{do!} LambdaQual\bnfs Block
\end{bnf}

\pnum
\tcode{do}后跟一个块创建一个没有显式指定参数的lambda表达式。

\pnum
\tcode{do!}后跟一个块创建一个无参的lambda表达式并立即调用它,将其值作为整个表达式的值。\enternote \tcode{do}和后面的\tcode{!}之间必须没有空白,否则会被识别成两个分开的运算符。\exitnote

\enterexample
\begin{codeblock}
let arr = [1, 2, 3];

let first1 = arr.first(do { $0 > 2 }); // 获取第一个满足条件的元素

let firstFinder = do {
    for let v : $0 {
        if v > 2 { return v; }
    }
    nil
};

let first2 = firstFinder(arr); // 与上面等价

let first3 = do! {
    for let v : arr {
        if v > 2 { return v; }
    }
    nil
}; // 与上面等价

\end{codeblock}
\exitexample

\rSec1[expr.suffix]{后缀运算符}
\indextext{运算符!后缀}

\begin{bnf}{SuffixExpr}
    PrimaryExpr \br
    IndexExpr \br
    FuncCallExpr \br
    MemberAccessExpr \br
    AwaitExpr \br
    NullCheckExpr \br
    PrevNextExpr \br
    IncDecExpr
\end{bnf}

\rSec2[expr.index]{下标运算符}
\indextext{运算符!下标}

\begin{bnf}{IndexExpr}
    SuffixExpr \terminal{[} ExprList\bnfq \terminal{]}
\end{bnf}

\pnum
下标运算符用于对数组或字典进行访问。用户自定义的下标运算符可以接受多于一个参数。

\pnum
对数组\tcode{$T$[]}而言,\tcode{$a$[$i$]}表示数组$a$的第$i$个元素。如果$i$超出可索引的范围,则会\tcode{panic}。结果类型是$T$。

\pnum
对字典\tcode{$T$[$K$]}而言,\tcode{$d$[$k$]}表示字典$d$的键$k$对应的值。如果字典中没有这个键,则会返回\tcode{nil}。结果类型是\tcode{$T$?}。
此外,在赋值语境下,\tcode{$d$[$k$] = $e$}表示将键$k$对应的值设为$e$。如果字典中没有这个键,则会插入一个新的键值对。$e$的类型需要为$T$。

\pnum
下标运算可以在可变和不可变的情况下具有不同的重载语义。参见~\ref{op.over.index}。

\rSec2[expr.call]{函数调用运算符}
\indextext{运算符!函数调用}

\begin{bnf}{FuncCallExpr}
    SuffixExpr \terminal{(} Arguments\bnfq \terminal{)} Block\bnfs \br
    SuffixExpr Block
\end{bnf}

\begin{bnf}{Arguments}
    UnnamedArgs \br
    NamedArgs \br
    UnnamedArgs \terminal{,} NamedArgs
\end{bnf}

\begin{bnf}{UnnamedArgs}
    Argument \br
    UnnamedArgs \terminal{,} Argument
\end{bnf}

\begin{bnf}{NamedArgs}
    Identifier \terminal{:} Argument \br
    NamedArgs \terminal{,} Identifier \terminal{:} Argument
\end{bnf}

\begin{bnf}{Argument}
    ParamQual\bnfq Expression
\end{bnf}

\pnum
函数调用运算符用于调用函数。括号内的项作为参数传递给函数。

\pnum
如果函数调用运算符的左操作数形如\tcode{$o$.$f$},其中$f$是一个名称且不是$o$的成员名称,则这称作\term{方法调用}。此时,$o$将作为$f$的\tcode{this}参数传递给函数$f$。

\pnum
函数调用运算符可以后跟一个块。这个块将作为一个匿名lambda块,创建一个lambda表达式并作为函数的最后一个顺序参数传递给函数。若此时函数没有任何其他参数,则函数调用的小括号可以省略。

\pnum
如果函数参数指定了修饰符,则传递实参时必须显式传递相同的修饰符。

\rSec2[expr.member]{成员访问运算符}
\indextext{运算符!成员访问}

\begin{bnf}{MemberAccessExpr}
    SuffixExpr \terminal{.} UnqualID \br
    SuffixExpr \terminal{.} IntegerLiteral \br
    SuffixExpr \terminal{.} \terminal{(} Expression \terminal{)}
\end{bnf}

\pnum
成员访问运算符用于访问对象的成员。

\pnum
\tcode{$o$.$m$}表示对象$o$的命名成员$m$。如果$o$没有命名成员$m$,若它后跟一个函数调用运算符,则作为方法调用处理;否则这是一个编译错误。

\pnum
\tcode{$o$.$i$}用于结构顺序成员的访问,表示$o$的第$i$个顺序成员。$i$必须是一个不包含前缀或后缀的十进制字面量。如果$i$大于等于结构顺序成员的数量,这是一个编译错误。

\pnum
\tcode{$o$.($e$)}首先对$e$求值。如果得到一个\tcode{symbol}类型的值,则对$o$进行命名成员访问。如果得到一个整数类型的值,则对$o$进行顺序成员访问。此时$o$必须是一个常量表达式。否则,$o$必须是一个带有\tcode{this}参数的函数类型,且其必须后跟一个函数调用运算符,此时将视为对$e$调用方法$o$。

\enterexample
\begin{codeblock}
let o = (a: 1, b: 2);
let s = 'a;

o.(s) // 等价于\tcode{o.a}

let t = (1, 2);
let k = 0;

t.(k) // 等价于\tcode{t.0}

impl int {
    func test(this) => this;
}

let f: (this: int) -> int = int::test;

0.(f)() // 等价于\tcode{0.test()}
\end{codeblock}
\exitexample

\rSec2[expr.await]{\tcode{await}运算符}
\indextext{运算符!\idxcode{await}}

\begin{bnf}{AwaitExpr}
    SuffixExpr \terminal{.} \terminal{await}
\end{bnf}

\pnum
\tcode{await}运算符用于等待一个异步表达式的结果。\tcode{$e$.await}挂起当前计算,直到$e$的值可用。如果$e$的类型是\tcode{Future<$T$>},则\tcode{$e$.await}的类型是$T$。只能在具有\tcode{async}修饰的函数作用域中使用\tcode{await}运算符。

\rSec2[expr.null]{空值检测运算符}
\indextext{运算符!空值检测}

\begin{bnf}{NullCheckExpr}
    SuffixExpr \terminal{?} \br
    SuffixExpr \terminal{!}
\end{bnf}

\pnum
\tcode{$e$?}对$e$进行空值检测。如果$e$的值不为\tcode{nil},则\tcode{$e$?}的值为$e$;否则,该表达式直到空值检测运算符为止的整个表达式的值为\tcode{nil}。$e$的类型必须是\tcode{$T$?}。即使空值检测运算符检测为空,表达式的其它部分仍然会被求值。

\enterexample
\begin{codeblock}
let a: int? = nil;

a + 1 // 编译错误,不存在接受\tcode{int?}和\tcode{int}的加法运算符
a? + 1 // \tcode{nil}
(a? + 1)? // \tcode{nil},多余的运算符
a? + 1 ?? 1 // $1$,\tcode{?}不会越过\tcode{??}运算符
(a? + 1) ?? 1 // 同上,但是更清晰
a? + 1 == b // 等号也会停止传播,等价于\tcode{(a? + 1) == b}

func f(i: int) => i + 1;

f(a?) // 整个表达式的类型为\tcode{int?},值为\tcode{nil}
f(a ?? 2) // 被空值检测运算符截断,整个表达式的类型为\tcode{int},值为\tcode{3}

let mut b: int? = 1;

b = a?; // 被赋值运算符截断,\tcode{b}成为\tcode{nil}

match a? { some let t -> true, nil -> false } // 被\tcode{match}的条件截断,不会传播到整个\tcode{match}表达式级别
\end{codeblock}
\exitexample

\pnum
后缀\tcode{!}与后缀\tcode{?}类似,但是它将空值直接返回而不是传播给完整表达式。

\pnum
如果空值检测运算符的操作数类型是\tcode{$T$?},则整个表达式的类型为$T$。

\pnum
如果$e$是结果类型\tcode{$T$ !! $E$},则\tcode{$e$?}在正确值的情况下使用该值,为错误值的情况下将错误传播到完整表达式。

\rSec2[expr.prev-next]{前驱后继运算符}
\indextext{运算符!前驱后继}

\begin{bnf}{PrevNextExpr}
    SuffixExpr \terminal{+!} \br
    SuffixExpr \terminal{-!}
\end{bnf}

\pnum
\tcode{$e$+!}和\tcode{$e$-!}分别表示$e$的后继和前驱。如果$e$的类型是算术类型,\tcode{$e$+!}等价于$e+1$,\tcode{$e$-!}等价于$e-1$。

\rSec2[expr.inc-dec]{自增自减运算符}

\begin{bnf}{IncDecExpr}
    SuffixExpr \terminal{++} \br
    SuffixExpr \terminal{--}
\end{bnf}

\pnum
自增运算符\tcode{$e$++}等价于\tcode{$e$ = $e$+!},自减运算符\tcode{$e$--}等价于\tcode{$e$ = $e$-!},但$e$只被求值一次。

\rSec1[expr.prefix]{前缀运算符}
\indextext{运算符!前缀}

\begin{bnf}{PrefixExpr}
    SuffixExpr \br
    \terminal{+} PrefixExpr \br
    \terminal{-} PrefixExpr \br
    \terminal{!} PrefixExpr \br
    \terminal{'\~} PrefixExpr \br
    \terminal{\&} \terminal{mut}\bnfq PrefixExpr \br
    \terminal{*} PrefixExpr \br
    SomeExpr
\end{bnf}

\pnum
前缀运算符$\mathbin{\tcode{+}}$和$\mathbin{\tcode{-}}$分别表示正号和负号。其中$\mathbin{\tcode{+}}$的值为其操作数的值,而$\mathbin{\tcode{-}}$的值为其相反数。操作数类型必须为算术类型。

\pnum
逻辑否运算符\tcode{!}用于对布尔值取反。如果操作数为\tcode{true},则结果为\tcode{false};如果操作数为\tcode{false},则结果为\tcode{true}。

\pnum
位取反运算符\tcode{'\~}进行按位取反。操作数的类型必须是定长整数类型。

\pnum
引用运算符\tcode{\&}获得操作数的引用。\tcode{\& mut}获得操作数的可变引用。

\pnum
解引用运算符\tcode{*}获得一个引用指向的值。

\rSec2[expr.some]{\tcode{some}运算符}
\indextext{运算符!\idxcode{some}}

\begin{bnf}{SomeExpr}
    \terminal{some} PrefixExpr
\end{bnf}

\pnum
\tcode{some}运算符用于将一个值转换为可空值。如果操作数的类型为$T$,则结果的类型为\tcode{$T$?}。

\rSec1[expr.mul]{乘法运算符}
\indextext{运算符!乘法}

\begin{bnf}{MulExpr}
    PrefixExpr \br
    MulExpr \terminal{*} PrefixExpr \br
    MulExpr \terminal{/} PrefixExpr \br
    MulExpr \terminal{\%} PrefixExpr
\end{bnf}

\pnum
运算符\tcode{*}、\tcode{/}和\tcode{\%}分别表示乘法、除法和余数。乘除法只对整数类型进行溢出检查,而不对定长整数类型和浮点类型进行。除零检测对整数类型和定长整数类型都生效。

\rSec1[expr.add]{加法运算符}
\indextext{运算符!加法}

\begin{bnf}{AddExpr}
    MulExpr \br
    AddExpr \terminal{+} MulExpr \br
    AddExpr \terminal{-} MulExpr
\end{bnf}

\pnum
运算符\tcode{+}和\ \tcode{-}\ 分别表示加法和减法。其操作必须为算术类型。加减法只对整数类型进行溢出检查,而不对定长整数类型和浮点类型进行。

\rSec1[expr.shift]{移位运算符}
\indextext{运算符!移位}

\begin{bnf}{ShiftExpr}
    AddExpr \br
    AddExpr \terminal{shl} AddExpr \br
    AddExpr \terminal{shr} AddExpr
\end{bnf}

\pnum
运算符\tcode{shl}和\tcode{shr}表示按位左移和右移。其操作数必须为定长整数类型。在同一个表达式中混合使用\tcode{shl}和\tcode{shr}是一个编译错误。

\rSec1[expr.bit]{位运算符}
\indextext{运算符!位}

\begin{bnf}{BitwiseExpr}
    ShiftExpr \br
    BitwiseExpr \terminal{'\&} ShiftExpr \br
    BitwiseExpr \terminal{'\^{}} ShiftExpr \br
    BitwiseExpr \terminal{'|} ShiftExpr
\end{bnf}

\pnum
运算符\tcode{'\&}、\tcode{'\^{}}和\tcode{'|}分别表示按位与、按位异或和按位或。其操作数必须为定长整数类型。在同一个表达式中混合使用\tcode{'\&}、\tcode{'\^{}}和\tcode{'|}是一个编译错误。

\rSec1[expr.range]{区间运算符}
\indextext{运算符!区间}

\begin{bnf}{RangeExpr}
    NullCoalExpr \br
    NullCoalExpr \terminal{..} NullCoalExpr \br
    NullCoalExpr \terminal{..=} NullCoalExpr
\end{bnf}

\pnum
运算符\tcode{..}生成左闭右开区间,结果类型是\tcode{Range}。
运算符\tcode{..=}生成左闭右闭区间,结果类型是\tcode{ClosedRange}。
参数类型必须是整数类型。参见~\ref{core.range}。

\rSec1[expr.connect]{连接运算符}
\indextext{运算符!连接}

\begin{bnf}{ConnectExpr}
    RangeExpr \br
    ConnectExpr \terminal{\~} RangeExpr
\end{bnf}

\pnum
运算符\tcode{\~}用于连接字符串或集合。其操作数的类型必须满足以下条件之一:

\begin{itemize}
    \item 两个操作数都是\tcode{string};
    \item 一个操作数满足\tcode{Sequence<$T$>},另一个是$T$;
    \item 两个操作数都满足\tcode{Sequence<$T$>}。
\end{itemize}

对第一种情况,结果等于将两个字符串左右连接得到的结果;对第二种情况,\tcode{$x$ \~ $y$}等于\tcode{[$x$, ...$y$]}或\tcode{[...$x$, $y$]},取决于哪个操作数是序列;对第三种情况,\tcode{$x$ \~ $y$}等于\tcode{[...$x$, ...$y$]}。

\rSec1[expr.null-coal]{空值合并运算符}
\indextext{运算符!空值合并}

\begin{bnf}{NullCoalExpr}
    BitwiseExpr \br
    BitwiseExpr \terminal{??} NullCoalExpr
\end{bnf}

\pnum
\tcode{$a$ ?? $b$}首先对$a$求值,如果其结果是\tcode{some $e$},则表达式的值为$e$,且$b$不会被求值;否则表达式的值为$b$。如果$a$的类型为\tcode{$A$?},$b$的类型为$B$,则表达式的类型为$A$和$B$的公共类型。

\pnum
\tcode{??}是右结合的。

\rSec1[expr.cmp-in]{比较运算符、包含运算符}

\begin{bnf}{BooleanExpr}
    RangeExpr \br
    CompareExpr \br
    IncludeExpr \br
    CastExpr \br
    MatchExpr
\end{bnf}

\pnum
本节中的运算符的结果都是\tcode{bool}。

\rSec2[expr.compare]{比较运算符}
\indextext{运算符!比较}

\begin{bnf}{CompareExpr}
    RangeExpr \terminal{!=} RangeExpr \br
    RangeExpr \terminal{cmp} RangeExpr \br
    LessChainExpr \br
    GreaterChainExpr
\end{bnf}

\begin{bnf}{LessChainExpr}
    RangeExpr LessChainOperator RangeExpr \br
    LessChainExpr LessChainOperator RangeExpr
\end{bnf}

\begin{bnf}{LessChainOperator}[\oneof]
    \terminal{< == <=}
\end{bnf}

\begin{bnf}{GreaterChainExpr}
    RangeExpr GreaterChainOperator RangeExpr \br
    GreaterChainExpr GreaterChainOperator RangeExpr
\end{bnf}

\begin{bnf}{GreaterChainOperator}[\oneof]
    \terminal{> == >=}
\end{bnf}

\begin{align*}
a\ \tcode{==}\ b &\iff a\ \tcode{cmp}\ b = \tcode{.equal} \\
a\ \tcode{!=}\ b &\iff a\ \tcode{cmp}\ b \ne \tcode{.equal} \\
a\ \tcode{<}\ b &\iff a\ \tcode{cmp}\ b = \tcode{.less} \\
a\ \tcode{>}\ b &\iff a\ \tcode{cmp}\ b = \tcode{.greater} \\
a\ \tcode{<=}\ b &\iff a\ \tcode{cmp}\ b = \tcode{.less}\ \mathrm{or} \ \tcode{.equal} \\
a\ \tcode{>=}\ b &\iff a\ \tcode{cmp}\ b = \tcode{.greater}\ \mathrm{or} \ \tcode{.equal} \\
\end{align*}

\pnum
$a\ \tcode{cmp}\ b$比较两个表达式,其结果类型为\tcode{Order}。其余比较运算符的结果类型为\tcode{bool}。

\pnum
\tcode{<}、\tcode{<=}和\tcode{==}可以连续使用。\tcode{$a$ < $b$ <= $c$}等价于\tcode{$a$ < $b$ \& $b$ <= $c$}。\tcode{>}、\tcode{>=}和\tcode{==}也可以用类似方式混合。以其他方式在一个表达式中使用超过一个比较运算符是一个编译错误。

\rSec2[expr.include]{包含运算符}
\indextext{运算符!包含}

\begin{bnf}{IncludeExpr}
    RangeExpr \terminal{in} RangeExpr \br
    RangeExpr \terminal{!in} RangeExpr
\end{bnf}

\pnum
\tcode{$a$ in $b$}检测$a$是否在$b$中。\tcode{$a$ !in $b$}等价于\tcode{!($a$ in $b$)}。表达式的类型为\tcode{bool}。

\pnum
\enternote \tcode{!in}中\tcode{!}和\tcode{in}之间必须没有空白,否则会被识别成两个分开的运算符。\exitnote

\rSec2[expr.cast]{类型转换运算符}
\indextext{运算符!类型转换}

\begin{bnf}{CastExpr}
    SuffixExpr \terminal{as} Type \br
    SuffixExpr \terminal{as?} Type \br
    SuffixExpr \terminal{as!} Type
\end{bnf}

\pnum
\tcode{$e$ as $T$}运算符用于进行显式的类型转换,结果的类型为\tcode{T}。

\pnum
\tcode{$e$ as? $T$}用于进行可能失败的类型转换,结果类型为可空类型或结果类型。
\tcode{$e$ as! $T$}用于进行强制类型转换,其等价于\tcode{($e$ as? $T$)!}。

\pnum
\enternote \tcode{as?}和\tcode{as!}中\tcode{as}和后面的符号之间必须没有空白,否则会被识别为两个分开的运算符。\exitnote

\rSec2[expr.match]{匹配运算符}
\indextext{运算符!匹配}

\begin{bnf}{MatchExpr}
    RangeExpr \terminal{is} Pattern \br
    RangeExpr \terminal{!is} Pattern
\end{bnf}

\pnum
$a \mathrel{\tcode{is}} p$检测$a$是否匹配模式$p$。$a \mathrel{\tcode{!is}} p$检测$a$是否不匹配模式$p$。表达式的类型为\tcode{bool}。模式中不能包含绑定模式。

\pnum
\enternote \tcode{!is}中\tcode{!}和\tcode{is}之间必须没有空白,否则会被识别成两个分开的运算符。\exitnote

\rSec1[expr.logic]{逻辑运算符}
\indextext{运算符!逻辑}

\begin{bnf}{LogicExpr}
    BooleanExpr \br
    LogicExpr \terminal{\&} BooleanExpr \br
    LogicExpr \terminal{|} BooleanExpr
\end{bnf}

\pnum
\tcode{\&}和\tcode{|}是逻辑运算符。两者的操作数都必须实现\tcode{Boolean}。它们都使用短路求值。在同一个表达式中混合使用两个运算符是一个编译错误。

\rSec1[expr.assign]{赋值运算符}
\indextext{运算符!赋值}

\begin{bnf}{AssignExpr}
    LogicExpr \br
    SuffixExpr \terminal{=} LogicExpr \br
    SuffixExpr \terminal{+=} LogicExpr \br
    SuffixExpr \terminal{-=} LogicExpr \br
    SuffixExpr \terminal{*=} LogicExpr \br
    SuffixExpr \terminal{/=} LogicExpr \br
    SuffixExpr \terminal{\%=} LogicExpr \br
    SuffixExpr \terminal{shl_eq} LogicExpr \br
    SuffixExpr \terminal{shr_eq} LogicExpr \br
    SuffixExpr \terminal{'\&=} LogicExpr \br
    SuffixExpr \terminal{'\^{}=} LogicExpr \br
    SuffixExpr \terminal{'|=} LogicExpr \br
    SuffixExpr \terminal{??=} LogicExpr \br
    SuffixExpr \terminal{<\~} LogicExpr \br
    LogicExpr \terminal{\~>} SuffixExpr
\end{bnf}

\pnum
赋值表达式的结果类型是\tcode{void}。

\pnum
\tcode{=}将左操作数的值更新为右操作数的值。左操作数必须是\tcode{mut}的,且右操作数必须能隐式转换到左操作数。

\pnum
复合赋值运算符\tcode{+=}、\tcode{-=}、\tcode{*=}、\tcode{/=}、\tcode{\%=}、\tcode{shl_eq}、\tcode{shr_eq}、\tcode{'\&=}、\tcode{'\^{}=}、\tcode{'|=}和\tcode{??=}分别表示加、减、乘、除、取余、左移、右移、按位与、按位异或、按位或和空值合并赋值。
对这些运算符而言,$a\ op\tcode{=}\ b$或$a\ op\tcode{_eq}\ b$等价于$a\ \tcode{=}\ a\ op\ b$,但$a$只被求值一次。

\pnum
追加运算符$e\ \tcode{<\~}\ v$等价于$e\ \tcode{=}\ e\ \tcode{\~}\ v$,$v\ \tcode{\~>}\ e$等价于$e\ \tcode{=}\ v\ \tcode{\~}\ e$,但$e$只被求值一次。

\rSec1[expr.semi]{分号运算符}
\indextext{运算符!分号}

\begin{bnf}{Expression}
    AssignExpr \br
    AssignExpr \terminal{;} Expression\br
    Binding \terminal{;} Expression
\end{bnf}

\pnum
分号表达式中,分号左侧可以为一个表达式或绑定。如果分号左侧为绑定,则该绑定会被插入到当前作用域中。如果左侧为表达式,则该表达式将被求值且结果会被丢弃。在那之后,将对右侧表达式进行求值并将其值作为整个表达式的值。
%!TEX root = x.tex

\rSec0[stmt]{语句}
\indextext{语句}

\begin{bnf}{Statement}
    Block \br
    BindingStmt \br
    IfStmt \br
    SwitchStmt \br
    AssertStmt
\end{bnf}

\pnum
语句是块的构成部分。如果语句包含子块,则这个块将优先作为语句的构成部分而不是其中的表达式的一部分。\enterexample

\begin{codeblock}
// 错误: \tcode{\{ true \}} 被认为是 \tcode{if} 语句的第一个子块,而不是它的条件表达式的一部分
if x.filter{ true }
\end{codeblock}

\exitexample

\rSec1[stmt.block]{块}
\indextext{语句!块}

\begin{bnf}{Block}
    \terminal{\{} BlockItem\bnfs\ \terminal{\}}
\end{bnf}

\begin{bnf}{BlockItem}
    Expression \terminal{;}\bnfq \br
    BlockDecl
\end{bnf}

\pnum
块是由大括号包裹的一系列声明和表达式的序列。块定义了一个块作用域。块的求值按照顺序进行,整个语句的值是最后一个项目的值。所有不是最后一项的表达式项的值被丢弃;这些表达式必须以\tcode{;}结尾。如果最后一个项目是一个声明,这个块的类型为 \tcode{void}。

\rSec1[stmt.bind]{绑定语句}
\indextext{语句!绑定}

\begin{bnf}{BindingStmt}
    Binding \terminal{;}
\end{bnf}

\begin{bnf}{Binding}
    Pattern \terminal{=} Expression \terminal{;}
\end{bnf}

\pnum
\term{绑定}形如$p\ \tcode{=}\ e$,其中$p$是包含至少一个绑定模式的模式。

\pnum
\term{绑定语句}将一个绑定插入当前作用域中。该绑定必须不能失败。

\rSec1[stmt.if]{\tcode{if} 语句}
\indextext{语句!if}

\begin{bnf}{IfStmt}
    \terminal{if} Condition \terminal{then} Expression \br
    \terminal{if} Condition \terminal{then} Expression \terminal{else} Expression \br
    \terminal{if} Condition Block \br
    \terminal{if} Condition Block \terminal{else} Block
\end{bnf}

\begin{bnf}{Condition}
    Expression \br
    Binding
\end{bnf}

\pnum
\term{条件}可以为任意对象声明后跟一个表达式或模式匹配。如果条件为表达式$e$\footnote{因为赋值表达式的类型是\tcode{void},形如\tcode{p = e}的程序文本将始终被看做一个模式匹配而不是赋值表达式。},那么这个表达式的类型必须实现\tcode{core.Boolean}。条件成立当且仅当$e$求值为真(\ref{core.oolean})。如果条件是绑定,那条件成立当且仅当绑定成功。该绑定必须可以失败。

\pnum
将\tcode{if}语句的第一个表达式记作$T$,第二个表达式(如果有)记作$F$。对\tcode{if}语句的求值按以下顺序进行:

\begin{itemize}

\item 如果条件成立,对$T$求值,然后将它的值作为整个语句的值。
\item 否则,如果存在$F$,那么对它求值,然后将它的值作为整个语句的值。
\item 否则,整个语句的值为\tcode{()}。

\end{itemize}

只有一个表达式会被求值。整个语句的类型是$T$和$F$的公共类型(如果$F$不存在的话视为\tcode{void})。

\rSec1[stmt.match]{\tcode{match} 语句}
\indextext{语句!match}

\begin{bnf}{MatchStmt}
    \terminal{match} Expression MatchBlock
\end{bnf}

\begin{bnf}{MatchBlock}
    \terminal{\{} BlockItem\bnfs MatchItem\bnfs \terminal{\}}
\end{bnf}

\begin{bnf}{MatchItem}
    Matcher Expression\bnfp
\end{bnf}

\begin{bnf}{Matcher}
    Pattern \terminal{->}
\end{bnf}

\pnum
\term{\tcode{match}语句}对其后跟的表达式进行模式匹配。整个语句的类型为每个匹配项表达式类型的公共类型。

\pnum
\tcode{match}语句的各项中的模式必须覆盖被匹配表达式的所有可能值,否则这是一个编译错误。

\pnum
对\tcode{match}语句的求值将按如下顺序进行:

\begin{itemize}
    \item 如果语句匹配块之前有项,执行这些项。他们的作用域是整个块。
    \item 按出现顺序对每个项进行匹配。如果某个项的模式匹配成功,则对其后的表达式进行求值,将其作为整个match表达式的值。所有其他项的表达式都不会进行求值。
\end{itemize}

\rSec1[stmt.while]{\tcode{while} 语句}
\indextext{语句!while}

\begin{bnf}{WhileStmt}
    \terminal{while} Expression\bnfq Block
\end{bnf}

\pnum
\term{\tcode{while}语句}处理循环,其中的表达式必须实现\tcode{core.Boolean}。整个语句的类型为\tcode{void}。

\pnum
\tcode{while}语句每次循环都会对控制表达式进行求值。如果求值为真,则继续循环,否则终止循环。如果表达式被省略,则等价于表达式为\tcode{true}。

\rSec1[stmt.for]{\tcode{for}语句}
\indextext{语句!for}

\begin{bnf}{ForStmt}
    \terminal{for} Pattern \terminal{:} Expression Block
\end{bnf}

\pnum
\term{\tcode{for}语句}进行明确的范围循环。形如$\tcode{for}\ p\ \tcode{:}\ e\ B$的\tcode{for}语句需满足:$e$实现了\tcode{core.Sequence};$\tcode{typeof(}e\tcode{).Item}$匹配$p$不会失败,否则这是一个编译错误。$p$中注入的变量在整个\tcode{for}语句的范围内生效。整个语句的类型为\tcode{void}。

\rSec1[stmt.control]{控制语句}
\indextext{语句!控制}

\begin{bnf}{BreakStmt}
    \terminal{break} \terminal{;}
\end{bnf}

\begin{bnf}{ContinueStmt}
    \terminal{continue} \terminal{;}
\end{bnf}

\begin{bnf}{ReturnStmt}
    \terminal{return} Expression\bnfq \terminal{;}
\end{bnf}

\pnum
控制语句包括\tcode{break}语句、\tcode{continue}语句和\tcode{return}语句。

\pnum
\tcode{break}语句只能在\tcode{while}或\tcode{for}语句中使用。它终止最内层的循环语句。

\pnum
\tcode{continue}语句只能在\tcode{while}或\tcode{for}语句中使用。它终止最内侧循环语句的本次循环。

\pnum
\tcode{return}语句只能在函数块中使用。它中止函数块的执行,并将后跟的表达式作为整个函数的返回值。如果表达式被省略,则等价于\tcode{()}。

%!TEX root = x.tex

\rSec0[pattern]{模式匹配}
\indextext{模式匹配}

\begin{bnf}{Pattern}
    PatternBody PatternAssertion\bnfs
\end{bnf}

\begin{bnf}{PatternBody}
    NullPattern \br
    ExprPattern \br
    BindPattern \br
    SomePattern \br
    ArrayPattern \br
    TuplePattern \br
    StructPattern \br
    AltPattern
\end{bnf}

\begin{bnf}{PatternAssertion}
    TypeAssertion \br
    IncludeAssertion \br
    CondAssertion
\end{bnf}

\pnum
模式匹配用于检验一个值是否符合特定的\term{模式},以及在符合特定的模式时从中提取某些成分。本节中,$\mathcal{v}$表示值,$\mathcal{p}$表示模式。值符合特定的模式称为这个值\term{匹配}这个模式,记作$\mathcal{v}\mathrel{\Vert}\mathcal{p}$。

\pnum
模式$\mathcal{p}$由\term{模式主体}和\term{模式断言}构成。模式主体规定匹配的结构与操作,模式断言则对值的特征进行断言。一个主体可以带有任意数量的断言。

\rSec1[pattern.null]{空模式}
\indextext{模式匹配!空}

\begin{bnf}{NullPattern}
    \tcode{_}
\end{bnf}

\pnum
空模式能够匹配任意值。匹配成功后,$\mathcal{v}$的值将被丢弃。

\rSec1[pattern.expr]{表达式模式}
\indextext{模式匹配!表达式}

\begin{bnf}{ExprPattern}
    RangeExpr
\end{bnf}

\pnum
$\mathcal{v}$和$\mathcal{e}$必须可比较。$\mathcal{v}\mathrel{\Vert}\mathcal{e}$当且仅当$\mathcal{v}\mathrel{\tcode{==}}\mathcal{e}$。

\rSec1[pattern.some]{\idxcode{some}模式}
\indextext{模式匹配!\idxcode{some}}

\begin{bnf}{SomePattern}
    \terminal{some} Pattern
\end{bnf}

\pnum
\tcode{some}模式匹配可空类型。如果该值为非空,则匹配成功。

\rSec1[pattern.array]{数组模式}
\indextext{模式匹配!数组}

\begin{bnf}{ArrayPattern}
    \terminal{[} AnyPattern \bnflp\terminal{,} AnyPattern\bnfrp\bnfs\ \terminal{]}
\end{bnf}

\begin{bnf}{AnyPattern}
    Pattern\br
    \terminal{...} \br
    BindKeyword \terminal{...} Identifier
\end{bnf}

\pnum
数组模式匹配序列中的元素。其中\tcode{...}项(称作\term{任意项模式})只能出现至多一次,否则这是一个编译错误。$\mathcal{v}$必须实现\tcode{Sequence},否则这是一个编译错误。

\begin{enumerate}
    \item 如果模式不包含任意项,且$\mathcal{v}$\tcode{.size}与模式中项的数量不相等,则匹配失败。
    \item 如果模式包含任意项,且$\mathcal{v}$\tcode{.size}小于模式中非任意项的数量,则匹配失败。
\end{enumerate}

\pnum
在那之后,将按如下规则依次对$\mathcal{v}$的元素进行匹配。如果每个匹配都成功,则整个模式$\mathcal{p}$匹配$\mathcal{v}$。

\begin{enumerate}
    \item 对任意项模式之前的模式(如果不存在任意项则对每个子模式),$\mathcal{p}_i$匹配$\mathcal{v}$\tcode{[}$i$\tcode{]},其中$i$是子模式的索引(从0开始)。
    \item 对任意项模式之后的模式,$\mathcal{p}_r$匹配$\mathcal{v}$\tcode{[\$-}$r$\tcode{]},其中$r$是子模式从后向前数的索引(从0开始)。
\end{enumerate}

\pnum
如果任意项包含一个绑定,则该任意项匹配到的元素将被绑定到相应的标识符上。

\rSec1[pattern.tuple]{元组模式}
\indextext{模式匹配!元组}

\begin{bnf}{TuplePattern}
    \terminal{(} AnyPattern \bnflp\terminal{,} AnyPattern\bnfrp\bnfs\ \terminal{)}
\end{bnf}

\pnum
与数组模式类似,\term{元组模式}匹配元组。

\rSec1[pattern.struct]{结构模式}
\indextext{模式匹配!结构}

\begin{bnf}{StructPattern}
    \terminal{\{} StructPatternBody \terminal{\}}
\end{bnf}

\begin{bnf}{StructPatternBody}
    StructItem \bnflp\terminal{,} StructItem\bnfrp\bnfs
\end{bnf}

\begin{bnf}{StructItem}
    Identifier \terminal{:} Pattern
\end{bnf}

\pnum
\term{结构模式}对结构进行匹配。如果对于每个对$(k, \mathcal{p}_k)$而言,$\mathcal{v}$\tcode{.}$k$匹配$\mathcal{p}_k$都成立,则整个模式匹配成功。

\pnum
与数组和元组匹配不同,结构匹配是开放的,即$\mathcal{v}$可以包含未在模式中列出的项。

\rSec1[pattern.bind]{绑定模式}
\indextext{模式匹配!绑定}

\begin{bnf}{BindPattern}
    BindKeyword PatternBind
\end{bnf}

\begin{bnf}{BindKeyword}
    \terminal{let} \br
    \terminal{let} \terminal{mut} \br
    \terminal{let} \terminal{const}
\end{bnf}

\begin{bnf}{PatternBind}
    Identifier PatternAssertion \br
    SomePatternBind \br
    ArrayPatternBind \br
    TuplePatternBind \br
    StructPatternBind
\end{bnf}

\begin{bnf}{SomePatternBind}
    \terminal{some} AnyPatternBind
\end{bnf}

\begin{bnf}{ArrayPatternBind}
    \terminal{[} AnyPatternBind \bnflp\terminal{,} AnyPatternBind\bnfrp\bnfs \terminal{]}
\end{bnf}

\begin{bnf}{TuplePatternBind}
    \terminal{(} AnyPatternBind \bnflp\terminal{,} AnyPatternBind\bnfrp\bnfs \terminal{)}
\end{bnf}

\begin{bnf}{AnyPatternBind}
    PatternBind \br
    \terminal{...} Identifier\bnfq \br
    NullPattern \br
    ExprPattern
\end{bnf}

\begin{bnf}{StructPatternBind}
    \terminal{\{} StructPatternBodyBind \terminal{\}}
\end{bnf}

\begin{bnf}{StructPatternBodyBind}
    StructItemBind \bnflp\terminal{,} StructItemBind\bnfrp\bnfs
\end{bnf}

\begin{bnf}{StructItemBind}
    Identifier \terminal{:} PatternBind \br
    Identifier
\end{bnf}

\pnum
绑定模式可以匹配任意值。匹配成功后,该标识符将作为一个变量插入到当前作用域中。

\pnum
如果绑定使用关键字\tcode{const},则这是一个常量绑定。参见~\ref{qual.const}。

\pnum
绑定模式可以使用简写,表~\ref{tab:binding-shorthand}~列出了一些常见的简写形式。

\begin{floattable}{绑定简写与其完整形式}{tab:binding-shorthand}{l|l}
    \topline
    \tcode{let [a, b]} & \tcode{[let a, let b]} \\
    \tcode{let (v, _)} & \tcode{(let v, _)} \\
    \tcode{let [x, ...y]} & \tcode{[let x, let... y]} \\
    \tcode{let \{ x \}} & \tcode{\{ x: let x \}} \\
\end{floattable}

\rSec2[pattern.alt]{选择模式}

\begin{bnf}{AltPattern}
    Pattern \terminal{|} Pattern \br
    AltPattern \terminal{|} Pattern
\end{bnf}

\pnum
选择模式同时匹配多个模式。如果其中有模式匹配成功,则整个模式匹配成功。匹配将从左到右进行。

\pnum
选择模式中使用的模式不能包含绑定模式。

\rSec1[pattern.type]{类型断言}
\indextext{模式匹配!类型断言}

\begin{bnf}{TypeAssertion}
    \terminal{is} Type \br
    \terminal{:} Type \br
    \terminal{as} Type
\end{bnf}

\pnum
\term{类型断言}对值的类型进行约束。它包括以下类型:

\begin{itemize}
    \item \tcode{is T}要求值的类型与\tcode{T}完全一致。
    \item \tcode{: T}要求值的类型是\tcode{T}的子类型。
    \item \tcode{as T}要求值的类型能够转换到\tcode{T},无论显式或隐式。
    \end{itemize}

\rSec1[pattern.include]{包含断言}
\indextext{模式匹配!包含断言}

\begin{bnf}{IncludeAssertion}
    \terminal{in} Expression
\end{bnf}

\pnum
\term{包含断言}要求值包含在某个集合$\mathcal{e}$中。如果$\mathcal{v}$\tcode{ !in }$\mathcal{e}$,则匹配失败。

\rSec1[pattern.cond]{条件断言}
\indextext{模式匹配!条件断言}

\begin{bnf}{CondAssertion}
    \terminal{if} Expression
\end{bnf}

\pnum
\term{条件断言}要求值满足某个条件。
\include{declaraion}
%!TEX root = x.tex

\rSec0[func]{函数}
\indextext{函数}

\begin{bnf}
\nontermdef{FuncDecl} \br
    FuncQual\bnfs \terminal{func} FuncName Parameter ReturnType\bnfq Block
\end{bnf}

\begin{bnf}
\nontermdef{FuncName} \br
    Identifier \br
    \terminal{init} \br
    \terminal{deinit} \br
    \terminal{operator} Operator
\end{bnf}

\begin{bnf}
\nontermdef{Parameter} \br
    \terminal{(} ParamList\bnfq \terminal{)}
\end{bnf}

\begin{bnf}
\nontermdef{ParamList} \br
    ParamDecl \br
    ParamList \terminal{,} ParamDecl
\end{bnf}

\begin{bnf}
\nontermdef{ParamDecl} \br
    ParamName TypeNotation\bnfq
\end{bnf}

\begin{bnf}
\nontermdef{ParamName} \br
    Identifier \br
    \terminal{this}
\end{bnf}
%!TEX root = x.tex

\rSec0[concept]{概念}
%!TEX root = x.tex

\rSec0[class]{类}
\indextext{类}

\begin{bnf}{ClassDecl}
    ClassQual\bnfs \terminal{class} Identifier ClassBody
\end{bnf}

\begin{bnf}{ClassQual}
    \terminal{const}
\end{bnf}

\begin{bnf}{ClassBody}
    \terminal{\{} ClassMember\bnfs \terminal{\}}
\end{bnf}

\begin{bnf}{ClassMember}
    FieldDecl \br
    PropertyDecl \br
    FuncDecl \br
    TypeDecl \br
    ClassDecl \br
    EnumDecl \br
    ConceptDecl
\end{bnf}

\pnum
\term{类}描述内部不透明的类型。

\rSec1[class.member]{字段}
\indextext{类!字段}

\begin{bnf}{FieldDecl}
    FieldQual\bnfs BindKeyword Identifier TypeNotation Initializer\bnfq \terminal{;}
    FieldQual\bnfs BindKeyword Identifier Initializer \terminal{;}
\end{bnf}

\begin{bnf}{TypeNotation}
    \terminal{:} Type
\end{bnf}

\begin{bnf}{Initializer}
    \terminal{=} Expression
\end{bnf}

\pnum
类中的字段表示类的内部状态,其默认访问级别为\tcode{private}。具有\tcode{mut}修饰的是可变字段。类字段可以显式指定类型,也可以通过初始值推导类型。

\rSec1[class.property]{属性}
\indextext{类!属性}

\begin{bnf}{PropertyDecl}
    PropertyQual\bnfs BindKeyword Identifier TypeNotation\bnfq Initializer\bnfq PropertyBody \terminal{;}
\end{bnf}

\begin{bnf}{PropertyQual}
    AccessQual
\end{bnf}

\begin{bnf}{PropertyBody}
    \terminal{\{} PropertyMember\bnfp \terminal{\}}
\end{bnf}

\begin{bnf}{PropertyMember}
    PropertyQual\bnfs PropertyKeyword PropertyBlockParam\bnfq Block \br
    PropertyQual\bnfs PropertyKeyword PropertyExprParam\bnfq \terminal{=>} Expression \terminal{;} \br
    PropertyQual\bnfs PropertyKeyword \terminal{;}
\end{bnf}

\begin{bnf}{PropertyKeyword}[\oneof]
    \terminal{get set willSet didSet}
\end{bnf}

\begin{bnf}{PropertyBlockParam}
    Identifier \br
    Identifier \terminal{,} Identifier
\end{bnf}

\begin{bnf}{PropertyExprParam}
    Identifier \br
    \terminal{(} Identifier \terminal{,} Identifier \terminal{)}
\end{bnf}

\pnum
类中还可以声明\term{属性}。属性是类对外暴露的接口,其默认访问级别为\tcode{public}。属性的定义至少需要包含一个访问器。访问器的块或表达式具有lambda作用域。

\pnum
属性的访问器可以以上下文关键字\tcode{get}、\tcode{set}、\tcode{willSet}或\tcode{didSet}开始。
\tcode{get}访问器不接受任何参数。
\tcode{set}访问器接受一个参数,其类型为该属性的类型。
\tcode{willSet}和\tcode{didSet}访问器可以接受一个或两个参数,其类型为属性的类型。

\pnum
如果\tcode{set}访问器不显式写出参数,则视为其具有lambda参数\tcode{\$value}。如果\tcode{willSet}或\tcode{didSet}不显式写出参数,则视为其具有lambda参数\tcode{\$oldValue}和\tcode{\$newValue}。

\pnum
如果属性$p$的声明中既不包含\tcode{get}也不包含\tcode{set}访问器,则视作该类具有字段$p'$,且具有访问器$\tcode{get}\mathrel{\tcode{=>}}\tcode{this.}p'$。
如果该属性有\tcode{mut}修饰,则还视为该属性具有访问器$\tcode{set => this.}p'\tcode{ = \$value}$。
如果\tcode{get}或\tcode{set}访问器后直接跟分号,则视作以上述方式生成访问器,且另一个对应的访问器若存在则必须采用这种省略方式声明。

\pnum
未使用\tcode{mut}声明的属性不能包含\tcode{set}、\tcode{willSet}和\tcode{didSet}访问器。
在考虑自动生成的访问器之后,如果一个属性缺少\tcode{get}访问器,或一个\tcode{mut}属性缺少\tcode{set}访问器,则这是一个编译错误。

\pnum
属性可以具有类型提示或初始化器。如果属性未按前文所述具有对应的字段,则不能具有初始化器。如果属性省略类型提示,则其类型将从\tcode{get}访问器中推导。如果它的\tcode{get}访问器是自动生成的,这是一个编译错误。

\pnum
当读取属性$p$时,会调用\tcode{get}访问器并将其返回值作为$p$的新值。

\pnum
当给属性$p$赋值时,首先将该值隐式转换到属性的类型,令结果为$v$:

\begin{itemize}
    \item 如果属性具有\tcode{willSet}访问器:
    \begin{itemize}
        \item 如果它接受一个参数,将以$v$调用;
        \item 如果它接受两个参数,将以$p$(旧值)和$v$调用。
    \end{itemize}
    \item 以$v$调用属性的\tcode{set}访问器。
    \item 如果属性具有\tcode{didSet}访问器:
    \begin{itemize}
        \item 如果它接受一个参数,将以$p$(旧值)调用;
        \item 如果它接受两个参数,将以$p$和$v$调用。
    \end{itemize}
\end{itemize}

\rSec1[method]{方法}
\indextext{方法}

\pnum
类中的函数声明称作\term{方法}。方法隐含了一个\tcode{this}参数,其为调用该方法的对象。

\enterexample
\begin{codeblock}
class A {
    let x: int;
    func set(x: int) {
        std.io.print(this.x); // 无需显式this参数
    }
};

func set(this: A, x: int) {
    std.io.print(this.x); // 与上面相同
}

\end{codeblock}
\exitexample

\rSec2[method.lookup]{方法查找}
\indextext{方法!查找}

\pnum

%!TEX root = x.tex

\rSec0[enum]{枚举}

\pnum
枚举类型是其实例全部是由枚举符表示的类型。一个枚举符表示枚举类型的一个或一系列实例。

\begin{bnf}
\nontermdef{EnumDecl} \br
    \terminal{enum} \terminal{\{} Enumerator\bnfl{}Separator \terminal{\}}
\end{bnf}
%!TEX root = x.tex

\rSec0[attr]{属性与修饰符}
\indextext{属性}
\indextext{修饰符}

\rSec1[attr.noreturn]{\tcode{noreturn}}
\indextext{属性!noreturn}

\pnum
\tcode{noreturn} 修饰的函数将不会返回。函数的返回类型必须省略或者是 \tcode{never}。

%!TEX root = x.tex

\rSec0[access]{访问控制}
\indextext{访问控制}

\begin{bnf}{AccessQual}[\oneof]
    \terminal{public private}
\end{bnf}
\include{package}

%!TEX root = x.tex

\rSec0[core]{\tcode{core} 库介绍}
\indextext{库}

\pnum
\tcode{core} 库是唯一与语言相互作用的库。实现了解 \tcode{core} 库的所有组件的接口以及(需要的话)内部细节。一个实现必须提供 \tcode{core} 库。
%!TEX root = x.tex

\rSec0[core.seq]{序列库}
\indextext{库!序列库}

\pnum
\tcode{core.seq}库定义了序列概念,并提供了一些操作序列的函数。

\rSec1[core.seq.traits]{概念}

\indexlibrary{\idxcode{Sequence}}
\begin{itemdecl}
trait Sequence<T> {
    type Item = T;
    type Iterator : core::Iterator<T>;

    let size: uint;
    let iter: Iterator;

    let isEmpty => this.size == 0;
}
\end{itemdecl}

\pnum
\tcode{Sequence}表示序列。

\pnum
\tcode{Item}是序列的元素类型,其始终为\tcode{T}。

\pnum
\tcode{Iterator}是序列的迭代器类型,其必须实现了\tcode{Iterator<T>}。

\pnum
\tcode{size}是序列的大小。序列实现了\tcode{Dollar},返回\tcode{size}。

\pnum
\tcode{iter}是序列的迭代器。

\pnum
\tcode{isEmpty}返回序列是否为空。它的默认实现将序列大小与0比较。

\indexlibrary{\idxcode{Sequence}!实现}
\begin{itemdecl}
impl<T> Sequence<T> : Dollar {
    func dollar() { this.size }
}
\end{itemdecl}

\pnum
序列实现了\tcode{Dollar},返回序列的大小。

\indexlibrary{\idxcode{Iterator}}
\begin{itemdecl}
trait Iterator<T> {
    type Item = T;

    func next(this: mut) -> T?;
}
\end{itemdecl}

\pnum
\tcode{Iterator}表示迭代器。

\pnum
\tcode{next}返回迭代器的下一个元素,如果迭代器已经到达末尾,则返回\tcode{nil}。

\rSec1[core.seq.helper]{辅助函数}


%%--------------------------------------------------
%% back matter
\backmatter
%!TEX root = x.tex

\renewcommand{\indexname}{索引}
\printindex[generalindex]

\clearpage
\renewcommand{\indexname}{语法产生式索引}
\printindex[grammarindex]

\clearpage
\renewcommand{\preindexhook}{}
\renewcommand{\indexname}{库名称索引}
\printindex[libraryindex]

\end{document}
