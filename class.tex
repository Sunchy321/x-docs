%!TEX root = x.tex

\rSec0[class]{类}
\indextext{类}

\begin{bnf}{ClassDecl}
    ClassQual\bnfs \terminal{class} Identifier ClassBody
\end{bnf}

\begin{bnf}{ClassQual}
    \terminal{const}
\end{bnf}

\begin{bnf}{ClassBody}
    \terminal{\{} ClassMember\bnfs \terminal{\}}
\end{bnf}

\begin{bnf}{ClassMember}
    FieldDecl \br
    PropertyDecl \br
    FuncDecl \br
    TypeDecl \br
    ClassDecl \br
    EnumDecl \br
    ConceptDecl
\end{bnf}

\pnum
\term{类}描述内部不透明的类型。

\rSec1[class.member]{类字段}
\indextext{类!字段}

\begin{bnf}{FieldDecl}
    FieldQual\bnfs BindKeyword Identifier TypeNotation Initializer\bnfq \terminal{;}
    FieldQual\bnfs BindKeyword Identifier Initializer \terminal{;}
\end{bnf}

\begin{bnf}{TypeNotation}
    \terminal{:} Type
\end{bnf}

\begin{bnf}{Initializer}
    \terminal{=} Expression
\end{bnf}

\pnum
类中的字段表示类的内部状态,其默认访问级别为\tcode{private}。使用\tcode{let}声明的表示不可变字段,使用\tcode{var}声明的表示可变字段。类字段可以显式指定类型,也可以通过初始值推导类型。

\rSec1[class.property]{类属性}
\indextext{类!属性}

\begin{bnf}{PropertyDecl}
    PropertyQual\bnfs BindKeyword Identifier TypeNotation\bnfq Initializer\bnfq PropertyBody \terminal{;}
\end{bnf}

\begin{bnf}{PropertyQual}
    AccessQual
\end{bnf}

\begin{bnf}{PropertyBody}
    \terminal{\{} PropertyMember\bnfp \terminal{\}}
\end{bnf}

\begin{bnf}{PropertyMember}
    PropertyQual\bnfs PropertyKeyword PropertyBlockParam\bnfq Block \br
    PropertyQual\bnfs PropertyKeyword PropertyExprParam\bnfq \terminal{=>} Expression \terminal{,}\bnfq \br
    PropertyQual\bnfs PropertyKeyword
\end{bnf}

\begin{bnf}{PropertyKeyword}[\oneof]
    \terminal{get set willSet didSet}
\end{bnf}

\begin{bnf}{PropertyBlockParam}
    Identifier \br
    Identifier \terminal{,} Identifier
\end{bnf}

\begin{bnf}{PropertyExprParam}
    Identifier \br
    \terminal{(} Identifier \terminal{,} Identifier \terminal{)}
\end{bnf}

\pnum
类中还可以声明\term{属性}。属性是类对外暴露的接口,其默认访问级别为\tcode{public}。属性的定义至少需要包含一个访问器。

\pnum
属性的访问器可以以上下文关键字\tcode{get}、\tcode{set}、\tcode{willSet}或\tcode{didSet}开始。
\tcode{get}访问器不接受任何参数。读取该属性的值将会调用它的\tcode{get}访问器。
\tcode{set}访问器接受一个参数,其类型为该属性的类型。\tcode{willSet}和\tcode{didSet}访问器接受两个参数,依次为旧值和新值。
在设置该属性的值时,将会依次调用\tcode{willSet}、\tcode{set}和\tcode{didSet}访问器。

\pnum
使用\tcode{let}声明的属性不能包含\tcode{set}、\tcode{willSet}和\tcode{didSet}访问器。

\pnum
如果属性$p$不包含\tcode{get}和\tcode{set}访问器,则视作该类具有一个同名的字段,且具有访问器$\tcode{get => this.}p$。如果该属性以\tcode{var}声明,则还视为该属性具有访问器$\tcode{set }v\tcode{ => this.}p\tcode{ = }v$,其中$v$是一个不与其他标识符冲突的标识符。


\rSec1[method]{方法}
\indextext{方法}

\pnum
类中的函数声明称作\term{方法}。方法隐含了一个\tcode{this}参数,其为调用该方法的对象。

\enterexample
\begin{codeblock}
class A {
    let x: int;
    func set(x: int) {
        std.io.print(this.x); // 无需显式this参数
    }
};

func set(this: A, x: int) {
    std.io.print(this.x); // 与上面相同
}

\end{codeblock}
\exitexample

\rSec2[method.lookup]{方法查找}
\indextext{方法!查找}

\pnum
