%!TEX root = x.tex

\rSec0[except]{异常处理}
\indextext{异常处理}

\rSec1[except.intro]{概述}
\indextext{异常处理!概述}

\pnum
\X 将程序中出现的问题分为三类:错误、异常与致命错误。错误是功能的预期结果的一部分,可能在使用该功能时频繁出现,例如\tcode{int.parse}解析了一个并非整数的字符串值;异常是功能预期结果之外的结果,通常较少出现,例如打开文件时文件不存在;致命错误是程序无法继续执行的错误,例如内存耗尽。

\pnum
如果函数可能会产生错误,则它应该返回可空类型或\tcode{core.Result}(参见\tcode{core.result})。

\pnum
如果函数可能会产生异常,则它应该使用异常处理机制,使用\tcode{throw}抛出异常并强制调用者处理异常。\X 不会隐式传播异常。

\pnum
如果函数产生了致命错误,则它应该调用\tcode{panic}直接终止程序的执行。

\rSec1[except.func.throw]{函数\tcode{throw}修饰符}

\begin{bnf}{ThrowQualifier}
    \terminal{throw} \terminal{(} TypeList\bnfs \terminal{)} \br
    \terminal{throw}
\end{bnf}

\begin{bnf}{TypeList}
    Type \br
    TypeList \terminal{,} Type
\end{bnf}

\pnum
\tcode{throw}修饰符显式指定了函数会产生什么类型的异常。

\rSec1[except.panic]{\tcode{panic}}
\indextext{异常处理!\idxcode{panic}}

\pnum
\tcode{core.panic}在程序遇到无法处理的错误时终止程序的执行。参见~\ref{core.panic}。