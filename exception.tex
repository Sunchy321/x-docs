%!TEX root = x.tex

\rSec0[except]{异常处理}
\indextext{异常处理}

\rSec1[except.intro]{概述}
\indextext{异常处理!概述}

\pnum
\X 将程序中出现的问题分为三类:错误、异常与致命错误。错误是功能的预期结果的一部分,可能在使用该功能时频繁出现,例如\tcode{int.parse}解析了一个并非整数的字符串值;异常是功能预期结果之外的结果,通常较少出现,例如打开文件时文件不存在;致命错误是程序无法继续执行的错误,例如内存耗尽。

\pnum
如果函数可能会产生错误,则它应该返回可空类型或\tcode{Result}(参见\tcode{core.result})。

\pnum
如果函数可能会产生异常,则它应该使用异常处理机制,使用\tcode{throw}抛出异常并强制调用者处理异常。

\pnum
如果函数产生了致命错误,则它应该调用\tcode{panic}直接终止程序的执行。

\rSec1[except.throw]{抛出异常}
\indextext{异常处理!抛出}

\pnum
\tcode{throw}语句(参见~\ref{stmt.control})抛出异常。整个函数将以抛出异常结束,并将异常传递给调用者。所有函数的局部变量将被销毁。

\pnum
异常的参数必须是一个不可变值,且它的生存期将被扩展到这个异常被处理。

\pnum
一个不带参数的\tcode{throw}语句只能在\tcode{catch}块中使用。这个语句可以将当前处理的异常重新抛出。

\pnum
\tcode{throw}也可以当作表达式使用。参见~\ref{expr.stmt}。

\rSec1[except.catch]{处理异常}
\indextext{异常处理!处理}

\pnum
一个被抛出的异常可能会被最近的\tcode{catch}块或\tcode{catch}表达式捕获。如果一个表达式的异常没有被捕获,则必须显式使用\tcode{try}关键字标识。

\pnum
一个\tcode{catch}块是跟在一个块后面的异常处理器。如果一个被抛出的异常经过带有\tcode{catch}块的块,则该异常将会依次对每个\tcode{catch}块进行模式匹配。如果某个模式匹配成功,则控制流转移到该\tcode{catch}块内部。如果没有\tcode{catch}块匹配,则异常将继续向外传播。

\pnum
如果一个异常传播到了最顶层函数的最外层,将会调用\tcode{core::terminate}。

\pnum
\tcode{catch}表达式捕获异常并将其包装到\tcode{Result}中。表达式的类型为\tcode{Result<T, E>},其中\tcode{T}为表达式的类型,\tcode{E}为异常的类型。如果表达式$e$抛出异常$e^\dagger$,则其值为$\tcode{.Error(}e^\dagger\tcode{)}$,否则其值为$\tcode{.OK(}e\tcode{)}$。

\rSec1[except.func.throw]{函数\tcode{throw}修饰符}

\begin{bnf}{ThrowQual}
    \terminal{throw} \terminal{(} TypeList\bnfs \terminal{)} \br
    \terminal{throw}
\end{bnf}

\begin{bnf}{TypeList}
    Type \br
    TypeList \terminal{,} Type
\end{bnf}

\pnum
\tcode{throw}修饰符显式指定了函数会产生什么类型的异常。\tcode{throw}后面的括号标记了可能抛出的异常类型。如果括号内部为空,则代表该函数不抛出任何异常。如果括号被省略,将会自动推导函数抛出的异常类型。

\rSec1[except.panic]{\tcode{panic}}
\indextext{异常处理!\idxcode{panic}}

\pnum
\tcode{core.panic}在程序遇到无法处理的错误时终止程序的执行。参见~\ref{core.panic}。