%!TEX root = x.tex

\rSec0[except]{异常处理}
\indextext{异常处理}

\rSec1[except.intro]{概述}
\indextext{异常处理!概述}

\pnum
\X 将程序中出现的问题分为一般错误与致命错误。一般错误能被上层代码处理,例如打开文件时文件不存在;致命错误则无法被处理,例如内存耗尽。
代码中的一般错误可以使用异常机制来处理。致命错误应该调用\tcode{panic}直接终止程序的执行。参见~\ref{except.panic}。

\rSec1[except.throw]{抛出异常}
\indextext{异常处理!抛出}

\pnum
\tcode{throw}表达式(参见~\ref{stmt.control})抛出异常。这等价于函数以结果类型的错误值返回。
在没有\tcode{throw}修饰符的函数中使用\tcode{throw}表达式是一个编译错误。

\pnum
\tcode{throw}表达式的参数必须实现了\tcode{ErrorCode}。

\pnum
不带参数的\tcode{throw}表达式只能在\tcode{catch}块中使用。它重新抛出当前捕获的异常。

\rSec1[except.catch]{处理异常}
\indextext{异常处理!处理}

\begin{bnf}{TryExpr}
    \terminal{try} Expression CatchBlock\bnfs
\end{bnf}

\begin{bnf}{CatchBlock}
    \terminal{catch} Pattern\bnfq Block
\end{bnf}

\pnum
\tcode{try}表达式可以解构结果类型$T\tcode{ throw }E$。如果其子表达式为正确值$e$,则整个表达式的值就为$e$。否则,将错误值与每个\tcode{catch}子句的模式进行匹配。如果找到了匹配的块,则整个表达式的值就为该块的值。如果没有找到匹配的块,则异常会继续向上抛出,等价于一个隐含的\tcode{catch \{ throw \}}。

\pnum
\tcode{try}表达式的类型$T$与所有\tcode{catch}块的类型的公共类型。

\pnum
如果一个异常传播到了最顶层函数的最外层,将会调用\tcode{core::terminate}。

\rSec1[except.func.throw]{函数\tcode{throw}修饰符}

\begin{bnf}{ThrowQual}
    \terminal{throw} \terminal{(} TypeList\bnfs \terminal{)} \br
    \terminal{throw}
\end{bnf}

\begin{bnf}{TypeList}
    Type \br
    TypeList \terminal{,} Type
\end{bnf}

\pnum
\tcode{throw}修饰符显式指定了函数会产生什么类型的异常。\tcode{throw}后面的括号标记了可能抛出的异常类型。如果括号内部为空,则代表该函数不抛出任何异常,其异常类型为\tcode{never}。如果括号被省略,将会自动推导函数抛出的异常类型。这些类型必须实现了\tcode{ErrorCode}。

\pnum
如果函数指定了\tcode{throw}修饰符,且修饰符指定或推导的类型为$E$,其显式指定或推导的返回值为$R$,则该函数实际的返回值为$R\tcode{ throw }E$。如果$R$或者$E$为结果类型,这是一个编译错误。

\rSec1[except.panic]{\tcode{panic}}
\indextext{异常处理!\idxcode{panic}}

\pnum
\tcode{core::panic}在程序遇到无法处理的错误时终止程序的执行。参见~\ref{core.panic}。