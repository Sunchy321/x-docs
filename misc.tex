%!TEX root = x.tex

\rSec0[core.misc]{杂项}

\rSec1[core.type]{类型}

\pnum
本节包含了若干类型,它们均为语言内建类型。

\indexlibrary{\idxcode{Symbol}}
\begin{itemdecl}
type Symbol = __intrinsic;
\end{itemdecl}

\pnum
\tcode{Symbol}是所有符号字面量的公共类型。它只能通过内建的符号字面量来获得值。

\indexlibrary{\idxcode{Optional}}
\begin{itemdecl}
enum Optional<T> {
    Some(T),
    Nil,
}
\end{itemdecl}

\pnum
\tcode{Optional}是可空类型的核心库定义;\tcode{Optional<$T$>}等价于\tcode{$T$?}。其中\tcode{Optional<$T$>.Some($v$)}对应\tcode{some $v$},\tcode{Optional<$T$>.Nil}对应\tcode{nil}。

\rSec1[core.order]{序}

\indexlibrary{\idxcode{Order}}
\begin{itemdecl}
enum Order {
    less,
    equal,
    greater
}
\end{itemdecl}

\pnum
\tcode{Order}定义了序关系。它是内建\tcode{cmp}运算符的返回值。自定义类型可以通过实现\tcode{cmp}运算符来支持序关系,参见~\ref{op.over.cmp}。

\pnum
\tcode{.less}代表左操作数小于右操作数。\tcode{.equal}代表左操作数等于右操作数。\tcode{.greater}代表左操作数大于右操作数。

\pnum
\tcode{cmp}也可以返回\tcode{Order?}。如果返回值为\tcode{nil},则代表两个操作数之间没有序关系。

\rSec1[core.range]{范围}

\indexlibrary{\idxcode{Range}}
\begin{itemdecl}
class Range<T>;
\end{itemdecl}

\indexlibrary{\idxcode{ClosedRange}}
\begin{itemdecl}
class ClosedRange<T>;
\end{itemdecl}

\pnum
\tcode{Range}表示左闭右开区间。它是内建\tcode{..}的结果类型。

\pnum
\tcode{ClosedRange}表示闭区间。它是内建\tcode{..=}的结果类型。

\rSec2[core.range.init]{构造器}

\indexlibrary{\idxcode{Range}!\idxcode{init}}
\begin{itemdecl}
func init<T>(start: T, end: T) -> Range<T>;
\end{itemdecl}

\indexlibrary{\idxcode{ClosedRange}!\idxcode{init}}
\begin{itemdecl}
func init<T>(start: T, end: T) -> ClosedRange<T>;
\end{itemdecl}

\pnum
\tcode{Range}和\tcode{ClosedRange}可以使用两个参数\tcode{start}与\tcode{end}构造,分别表示开头与结尾。如果\tcode{start > end}则会抛出\tcode{.InvalidBounds}。

\rSec2[core.range.impl]{实现}

\begin{itemdecl}
impl<T> Range<T> : Sequence<T> {
    type Iterator = RangeIterator<T>;
}
\end{itemdecl}

\begin{itemdecl}
impl<T> ClosedRange<T> : Sequence<T> {
    type Iterator = ClosedRangeIterator<T>;
}
\end{itemdecl}

\pnum
\tcode{Range}和\tcode{ClosedRange}是序列,其迭代器类型是\tcode{RangeIterator}。

\rSec2[core.range.iter]{迭代器}

\indexlibrary{\idxcode{Range}!\idxcode{Iterator}}
\begin{itemdecl}
class RangeIterator<T>;
\end{itemdecl}

\indexlibrary{\idxcode{ClosedRange}!\idxcode{Iterator}}
\begin{itemdecl}
class ClosedRangeIterator<T>;
\end{itemdecl}

\pnum
\tcode{RangeIterator}是\tcode{Range}的迭代器类型。

\pnum
\tcode{ClosedRangeIterator}是\tcode{ClosedRange}的迭代器类型。

\rSec1[core.error]{错误处理}

\rSec2[core.error.trait]{特征}

\indexlibrary{\idxcode{ErrorCode}}
\begin{itemdecl}
trait ErrorCode { }
\end{itemdecl}

\pnum
\tcode{ErrorCode}是一个特征,用于表示错误。它是一个空的trait,用于标记实现了错误的类型。

\indexlibrary{\idxcode{ErrorCode}!实现!\idxcode{never}}
\begin{itemdecl}
impl never : ErrorCode { }
\end{itemdecl}

\pnum
\tcode{never}实现了\tcode{ErrorCode},表示不会产生错误。

\indexlibrary{\idxcode{ErrorCode}!实现!\idxcode{void}}
\begin{itemdecl}
impl void : ErrorCode = never;
\end{itemdecl}

\pnum
\tcode{void}被禁止实现\tcode{ErrorCode},这是为了避免不带参数的\tcode{throw}表达式产生理解歧义。

\rSec2[core.error.impl]{实现}

\indexlibrary{结果类型!实现}
\begin{itemdecl}
impl<T, E> T !! E { /* ... */ }
\end{itemdecl}

\indexlibrary{结果类型!\idxcode{isOK}}
\indexlibrary{结果类型!\idxcode{isError}}
\begin{itemdecl}
let isOK: bool {
    get {
        try this catch { false } else { true }
    }
}

let isError: bool {
    get { !this.isOK }
}
\end{itemdecl}

\pnum
\tcode{isOK}当结果是正确值的时候返回\tcode{true},否则返回\tcode{false}。\tcode{isError}与之相反。

\indexlibrary{结果类型!\idxcode{takeOK}}
\indexlibrary{结果类型!\idxcode{takeError}}
\begin{itemdecl}
func takeOK() -> T? { this? }
func takeError() -> E? { try this catch let e { e } else { nil } }
\end{itemdecl}

\pnum
\tcode{takeOK}取出正确值。如果结果是错误值,则返回\tcode{nil}。\tcode{takeError}取出错误值。如果结果是正确值,则返回\tcode{nil}。

\indexlibrary{结果类型!\idxcode{map}}
\begin{itemdecl}
func map<U>(f: T -> U) throw(E) -> U {
    f(try this)
}

func map<U>(f: T -> U, or or: U) -> U {
    f(this? ?? or)
}
\end{itemdecl}

\pnum
\tcode{map}将正确值映射为其他值。错误值会被抛出。如果提供了\tcode{or}参数,则是错误值时会返回\tcode{or}。

\indexlibrary{结果类型!\idxcode{inspect}}
\begin{itemdecl}
func inspect(f: T -> void) -> void {
    f(this?)
}
\end{itemdecl}

\pnum
\tcode{inspect}将使用正确值调用函数\tcode{f}。如果结果是错误值,则不会调用\tcode{f}。

\indexlibrary{结果类型!\idxcode{expect}}
\begin{itemdecl}
func expect(failMessage: string) -> T {
    try this catch { panic(failMessage) }
}
\end{itemdecl}

\pnum
\tcode{expect}取出结果的值。如果结果是错误值,则调用\tcode{panic}并输出\tcode{failMessage}。

\indexlibrary{结果类型!\idxcode{expectError}}
\begin{itemdecl}
func expectError(failMessage: string) -> E {
    try this catch let e { e } else { panic(failMessage) }
}
\end{itemdecl}

\pnum
\tcode{expectError}取出结果的错误值。如果结果是正确值,则调用\tcode{panic}并输出\tcode{failMessage}。

\indexlibrary{结果类型!\idxcode{and}}
\begin{itemdecl}
func and<U>(other: U !! E) throw -> U {
    try this catch let e { throw e } else { try other }
}
\end{itemdecl}

\pnum
\tcode{and}返回第一个错误值。如果第一个结果是正确值,则返回第二个结果。

\indexlibrary{结果类型!\idxcode{andThen}}
\begin{itemdecl}
func andThen<U>(f: T -> U !! E) throw(E) -> U {
    try f(try this)
}
\end{itemdecl}

\pnum
如果结果是错误值,则直接抛出该值。否则以正确值调用函数\tcode{f}。\enternote 与\tcode{map}不同的是,\tcode{andThen}的\tcode{f}本身也可能产生错误。\exitnote

\indexlibrary{结果类型!\idxcode{or}}
\begin{itemdecl}
func or(other: T !! E) -> T {
    this? ?? try other
}
\end{itemdecl}

\pnum
\tcode{or}返回第一个正确值。如果第一个结果是错误值,则返回第二个结果。

\indexlibrary{结果类型!\idxcode{orElse}}
\begin{itemdecl}
func orElse<F>(f: E -> T !! F) throw(F) -> T {
    try this catch let e { try f(e) }
}
\end{itemdecl}

\pnum
如果结果是正确值,则直接返回该值。否则以错误值调用函数\tcode{f}。

\indexlibrary{结果类型!\idxcode{unwrapOr}}
\begin{itemdecl}
func unwrapOr(default: T) -> T {
    this? ?? default
}
\end{itemdecl}

\pnum
\tcode{unwrapOr}返回结果的正确值。如果结果是错误值,则返回\tcode{default}。

\indexlibrary{结果类型!\idxcode{unwrapOrElse}}
\begin{itemdecl}
impl<T, E> T !! E if T is default {
    func unwrapOrDefault() -> T {
        this? ?? self::default
    }
}
\end{itemdecl}

\pnum
如果$T$是\tcode{core::Default},则\tcode{unwrapOrDefault}等同于以\tcode{T::default}调用的\tcode{unwrapOr}。

\indexlibrary{结果类型!\idxcode{toOK}}
\begin{itemdecl}
impl<T> T !! never {
    func toOK() -> T {
        try this
    }
}
\end{itemdecl}

\pnum
如果错误类型为\tcode{never},则意味着永远不会发生错误。此时\tcode{toOK}将结果类型直接转换为正确值。

\indexlibrary{结果类型!\idxcode{toError}}
\begin{itemdecl}
impl<E> never !! E {
    func toError() -> E {
        try this
    }
}
\end{itemdecl}

\pnum
如果正确类型为\tcode{never},则意味着永远不会得到正确值。此时\tcode{toError}将结果类型直接转换为错误值。

\indexlibrary{结果类型!\idxcode{flatten}}
\begin{itemdecl}
impl<T, E> (T !! E) !! E {
    func flatten() throw(E) -> T {
        this!!
    }
}
\end{itemdecl}

\pnum
\tcode{flatten}将嵌套的结果类型展平。

\rSec2[core.panic]{\tcode{panic}}

\indexlibrary{\idxcode{panic}}
\begin{itemdecl}
func panic(message: string = "") -> never;
\end{itemdecl}

\pnum
\tcode{panic}在程序遇到无法处理的错误时终止程序的执行。它会向标准错误流打印错误信息\tcode{message},并析构当前线程的所有对象,然后终止程序的执行。

\pnum
如果\tcode{panic}在执行过程中再次调用了\tcode{panic},则会直接终止程序的执行而略过任何对象的析构。