%!TEX root = x.tex

\rSec0[core.misc]{杂项}

\rSec1[core.type]{类型}

\pnum
本节包含了若干类型,它们均为语言内建,但并不是单个关键字。

\indexlibrary{\idxcode{Symbol}}
\begin{itemdecl}
type Symbol = __intrinsic;
\end{itemdecl}

\pnum
\tcode{core.Symbol}是所有符号字面量的公共类型。它只能通过内建的符号字面量来获得值。

\rSec1[core.order]{序}

\indexlibrary{\idxcode{Order}}
\begin{itemdecl}
enum Order {
    less,
    equal,
    greater
}
\end{itemdecl}

\pnum
\tcode{core.Order}定义了序关系。它是内建\tcode{cmp}运算符的返回值。自定义类型可以通过实现\tcode{cmp}运算符来支持序关系,参见~\ref{op.over.cmp}。

\pnum
\tcode{.less}代表左操作数小于右操作数。\tcode{.equal}代表左操作数等于右操作数。\tcode{.greater}代表左操作数大于右操作数。\tcode{.unordered}代表左操作数和右操作数之间没有序关系。

\rSec1[core.range]{范围}

\indexlibrary{\idxcode{Range}}
\begin{itemdecl}
type Range<T> {
    let begin: T;
    let end: T;
}
\end{itemdecl}

\pnum
\tcode{core.Range}表示一个范围。它是内建\tcode{..}和\tcode{..=}的类型。

\rSec2[core.range.impl]{实现}

\begin{itemdecl}
impl<T> Range<T> : Sequence<T> {
    type Iterator = RangeIterator<T>;
}
\end{itemdecl}

\pnum
\tcode{Range}是一个序列,其迭代器类型是\tcode{core.RangeIterator}。

\rSec2[core.range.iter]{迭代器}

\indexlibrary{\idxcode{Range}!\idxcode{Iterator}}
\begin{itemdecl}
class RangeIterator<T> {
    let mut curr: T;
    let end: T;

    init(begin: T, end: T) {
        this.curr = begin;
        this.end = end;
    }
}

impl<T> RangeIterator<T> : Iterator<T> {
    func next(this: mut) {


        let prev = this.curr;
    }
}
\end{itemdecl}

\pnum
\tcode{core.RangeIterator}是\tcode{Range}的迭代器类型。