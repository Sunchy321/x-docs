%!TEX root = x.tex

\rSec0[module]{模块}
\indextext{模块}

\rSec1[import]{导入指令}
\indextext{导入指令}

\begin{bnf}{ImportDirective}
    \terminal{import} ImportPath \terminal{;} \br
    \terminal{import} ImportPath \terminal{:} ImportItems \terminal{;}
\end{bnf}

\begin{bnf}{ImportPath}
    ExternalImportPath \br
    InternalImportPath \br
    RelativeImportPath
\end{bnf}

\begin{bnf}{ExternalImportPath}
    ImportPathPart \br
    ExternalImportPath \terminal{.} ImportPathPart
\end{bnf}

\begin{bnf}{InternalImportPath}
    \terminal{root} \terminal{.} ImportPathPart \br
    InternalImportPath \terminal{.} ImportPathPart
\end{bnf}

\begin{bnf}{RelativeImportPath}
    \terminal{this} \terminal{.} ImportPathPart \br
    RelativeImportPath \terminal{.} ImportPathPart
\end{bnf}

\begin{bnf}{ImportPathPart}
    Identifier \br
    StringLiteral \br
    \terminal{super}
\end{bnf}

\begin{bnf}{ImportItem}
    \terminal{*} \br
    \terminal{operator} \br
    \terminal{operator} OperatorType\bnfq OperatorName \br
    Identifier \br
    Identifier \terminal{as} Identifier
\end{bnf}

\pnum
\term{导入指令}用于将其他模块的内容引入到当前模块中。

\pnum
被导入的模块可以通过三种方式指定:外部路径、内部路径或相对路径。
外部路径以模块名开始,用于导入外部模块。
内部路径以\tcode{root}开始,从当前模块根目录开始寻找模块。
相对路径以\tcode{this}开始,从当前模块文件开始寻找模块。

\pnum
模块的路径各部分可以使用一个标识符或者字符串标识。
上下文关键字\tcode{root}指代当前模块的根目录,但只能用于开头;上下文关键字\tcode{super}指代当前目录的父目录。
其他关键字或上下文关键字都将作为普通标识符处理。
模块目录名称可以与\tcode{this}、\tcode{root}或\tcode{super}相同,但此时必须使用字符串表示。

\pnum
模块的目录名称可以包含连字符,并可以通过将连字符转换为下划线的对应标识符引用。如果对应位置分别为下划线和连字符的目录都存在,则该标识符将会引用下划线的目录。以字符串方式引用目录名不会进行此类转换。

\enterexample
\begin{codeblock}
import root.foo.bar; // 导入\tcode{/foo/bar}模块
import this.baz; // 导入当前目录下的\tcode{baz}模块
import this.super.qux; // 导入父目录下的\tcode{qux}模块

import some_hypen; // 导入外部模块\tcode{some-hypen}
import "super"; // 导入外部模块\tcode{super}
\end{codeblock}
\exitexample

\pnum
\tcode{*}导入指定模块的所有内容,但是不包括运算符定义。

\rSec1[access]{访问控制}
\indextext{访问控制}

\begin{bnf}{AccessQualifier}[\oneof]
    \terminal{public internal private}
\end{bnf}

\pnum
模块的每个顶层声明都具有\term{访问可见性}。访问可见性有以下三种类型:公开(\tcode{public})、内部(\tcode{internal})、私有(\tcode{private})、局部,其级别依次降低。
实现成员也具有访问可见性。

\pnum
导入一个外部模块的内容时,只有公开的内容才能被导入。导入一个内部模块的内容时,只有公开和内部的内容才能被导入。私有模块只能被该模块内部访问。

\pnum
声明和实现成员的默认可见性为\tcode{public}。如果一个成员没有显式指定可见性,且其所在实体的可见性比其默认可见性要更低,则使用其所在实体的可见性。
实现的默认可见性与被实现实体的可见性相同。

\pnum
如果一个声明位于函数内部,则它具有局部可见性。具有局部可见性的实体只能在其所在函数内部访问。函数的参数也具有局部可见性。

\enterexample
\begin{codeblock}
class X {
    x: int; // 默认为\tcode{private}
    public y: int; // 显式指定为\tcode{public}
}

impl X {
    func f() {} // 默认为\tcode{public}
    private func g() {} // 显式指定为\tcode{private}
}

internal class Y {
    x: int; // 默认为\tcode{private}
    public y: int; // 显式指定为\tcode{public}
}

impl Y {
    func f() {} // 默认为\tcode{internal}
    public func g() {} // 显式指定为\tcode{public}
}
\end{codeblock}