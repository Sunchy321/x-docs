%!TEX root = x.tex

\rSec0[module]{模块}
\indextext{模块}

\pnum
\X{}将程序视为若干个\term{模块}的组织。模块可以是一个文件、一个目录或者一整个库;模块也可以在代码中定义。

\pnum
模块可以拥有\term{子模块}。子模块可以通过目录结构自然生成,也可以在代码中指定。

\pnum
对模块外部而言,模块的文件结构是不透明的。只能通过模块导出的内容访问模块内部的实体。

\pnum
模块导出的内容即其根文件导出的内容。一般情况下,根文件是根目录下名称为\tcode{index.x}的文件。

\rSec1[module.decl]{模块声明}
\indextext{模块声明}

\begin{bnf}{ModuleDecl}
    \terminal{module} ModuleID\bnfq \terminal{\{} ModuleContent \terminal{\}} \br
    \terminal{module} ModuleID\bnfq \terminal{;} \br
    \terminal{module} ModuleID\bnfq \terminal{=} ImportPath \terminal{;}
\end{bnf}

\begin{bnf}{ModuleID}
    Identifier \br
    ModuleID \terminal{.} Identifier
\end{bnf}

\pnum
模块声明用于定义一个模块。模块声明中的模块ID指定模块的名称。

\pnum
如果模块声明省略了名称,则定义的是当前文件的模块。否则,模块声明定义的是当前文件模块的子模块。

\rSec1[import]{导入指令}
\indextext{导入指令}

\begin{bnf}{ImportDirective}
    \terminal{public}\bnfq \terminal{import} ImportPath \terminal{;} \br
    \terminal{public}\bnfq \terminal{import} ImportPath \terminal{:} ImportItems \terminal{;}
\end{bnf}

\begin{bnf}{ImportPath}
    ExternalImportPath \br
    InternalImportPath \br
    RelativeImportPath
\end{bnf}

\begin{bnf}{ExternalImportPath}
    ImportPathPart \br
    ExternalImportPath \terminal{.} ImportPathPart
\end{bnf}

\begin{bnf}{InternalImportPath}
    \terminal{root} \terminal{.} ImportPathPart \br
    InternalImportPath \terminal{.} ImportPathPart
\end{bnf}

\begin{bnf}{RelativeImportPath}
    \terminal{this} \terminal{.} ImportPathPart \br
    RelativeImportPath \terminal{.} ImportPathPart
\end{bnf}

\begin{bnf}{ImportPathPart}
    Identifier \br
    StringLiteral \br
    \terminal{super}
\end{bnf}

\begin{bnf}{ImportItem}
    \terminal{*} \br
    \terminal{operator} \br
    \terminal{operator} OperatorType\bnfq OperatorName \br
    Identifier \br
    Identifier \terminal{as} Identifier
\end{bnf}

\pnum
\term{导入指令}用于将其他模块或文件的内容引入到当前模块中。

\pnum
被导入的模块或文件可以通过三种方式指定:外部路径、内部路径或相对路径。
外部路径以模块名开始,用于导入外部模块。
内部路径以\tcode{root}开始,从当前模块根目录开始寻找文件或子模块。
相对路径以\tcode{this}开始,从当前文件开始寻找文件或子模块。

\pnum
模块的路径各部分可以使用一个标识符或者字符串标识。
上下文关键字\tcode{root}指代当前模块的根目录,但只能用于开头;上下文关键字\tcode{super}指代当前目录的父目录。
其他关键字或上下文关键字都将作为普通标识符处理。
模块目录名称可以与\tcode{this}、\tcode{root}或\tcode{super}相同,但此时必须使用字符串表示。

\pnum
模块的目录名称可以包含连字符,并可以通过将连字符转换为下划线的对应标识符引用。如果对应位置分别为下划线和连字符的目录都存在,则该标识符将会引用下划线的目录。以字符串方式引用目录名不会进行此类转换。

\enterexample
\begin{codeblock}
import root.foo.bar; // 导入\tcode{/foo/bar}模块
import this.baz; // 导入当前目录下的\tcode{baz}模块
import this.super.qux; // 导入父目录下的\tcode{qux}模块

import some_hypen; // 导入外部模块\tcode{some-hypen}
import "super"; // 导入外部模块\tcode{super}
\end{codeblock}
\exitexample

\pnum
\tcode{*}导入指定模块的所有内容,但是不包括运算符定义。

\pnum
导入指令可以带有一个\tcode{public}修饰符。此时该指令除了将被导入的实体引入当前文件外,还将这些实体重新作为公开实体导出。

\rSec1[access]{访问控制}
\indextext{访问控制}

\begin{bnf}{AccessQualifier}[\oneof]
    \terminal{public internal private}
\end{bnf}

\pnum
声明具有\term{访问可见性},用于指定该程序实体能被什么范围的其他实体所访问。
访问可见性有以下四种类型:公开(\tcode{public})、内部(\tcode{internal})、私有(\tcode{private})、局部,其级别依次降低。

\pnum
公开级别的实体能被任何位置的程序实体访问。在导入外部模块时,只能导入公开级别的实体。

\pnum
内部级别的实体能被同一模块中的程序实体访问。

\pnum
私有级别的实体只能被同一文件中的实体访问。

\pnum
如果一个声明位于函数内部,则它具有局部可见性。不能为它指定访问修饰符。
函数参数也具有局部可见性。
只能在函数内部访问该声明。

公开级别可以被任何位置的程序实体访问。
内部级别可以被同一模块中的程序实体访问。
私有级别只能被同一文件中的程序实体访问。
局部级别只能被其所在的函数中的程序实体访问。

\pnum
声明的默认可见性为\tcode{public}。
如果一个成员没有显式指定可见性,且其所在实体的可见性比其默认可见性要更低,则使用其所在实体的可见性。
可见性对实现没有意义,但固有实现可以包含一个可见性修饰符,用于指定其所有项的可见性。

\pnum
特征的项具有的可见性与特征相同。不能为这些项单独指定可见性。
在特征实现中,包含在特征中的项的可见性与该特征相同;未包含在特征中的项的可见性总是为\tcode{private}。不能为这些项重新指定可见性。

\enterexample
\begin{codeblock}
class X {
    x: int; // 默认为\tcode{private}
    public y: int; // 显式指定为\tcode{public}
}

impl X {
    func f() {} // 默认为\tcode{public}
    private func g() {} // 显式指定为\tcode{private}
}

internal class Y {
    x: int; // 默认为\tcode{private}
    public y: int; // 显式指定为\tcode{public}
}

impl Y {
    func f() {} // 默认为\tcode{internal}
    public func g() {} // 显式指定为\tcode{public}
}
\end{codeblock}