%!TEX root = x.tex

\rSec0[op]{运算符}
\indextext{运算符}

\rSec1[op.name]{运算符名称}

\begin{bnf}{OperatorID}
    \terminal{operator} OperatorType\bnfq OperatorName \br
    \terminal{operator} StringLiteral \br
    \terminal{operator} \terminal{\$} \br
    \terminal{operator} \terminal{if} \br
    \terminal{operator} \terminal{is} Type \br
    \terminal{operator} \terminal{as} Type
\end{bnf}

\begin{bnf}{OperatorName}
    Operator \br
    Identifier \br
    OperatorKeyword \br
    \terminal{(} \terminal{)} \br
    \terminal{[} \terminal{]}
\end{bnf}

\begin{bnf}{OperatorKeyword}[\oneof]
    \terminal{shl shr cmp in}
\end{bnf}

\begin{bnf}{OperatorType}[\oneof]
    \terminal{infix prefix suffix}
\end{bnf}

\begin{bnf}{Operator}
    CustomOperator \br
    \terminal{;}
\end{bnf}

\begin{bnf}{CustomOperator}
    OperatorSymbol\bnfp
\end{bnf}

\begin{bnf}{OperatorSymbol}[\oneof]
    \terminal{\~ ! \# \% \^{} \& * - | + = / ? < > . '}
\end{bnf}


\pnum
运算符名称指代对应的运算符函数的名称,用来重载运算符。运算符名称由关键字\tcode{operator}后跟运算符组成。\tcode{operator}后面可以跟\tcode{infix}、\tcode{prefix}或\tcode{suffix}以消歧义。

\pnum
\tcode{operator}后跟的运算符可以是内建运算符或自定义运算符。表~\ref{tab:op}~列出了全部内建运算符,其中上半部分为可以重载的运算符,下半部分为不能重载的运算符。

\begin{floattable}{内建运算符}{tab:op}{lllll}
\topline
\tcode{+!} &
\tcode{-!} &
前缀\tcode{+} &
前缀\tcode{-} &
前缀\tcode{!} \\
\tcode{'\~} &
\tcode{*} &
\tcode{/} &
\tcode{\%} &
二元\tcode{+} \\
二元\tcode{-} &
\tcode{shl} &
\tcode{shr} &
\tcode{'\&} &
\tcode{'\^{}} \\
\tcode{'|} &
\tcode{..} &
\tcode{..=} &
\tcode{\~} &
\tcode{==} \\
\tcode{!=} &
\tcode{<} &
\tcode{<=} &
\tcode{>} &
\tcode{>=} \\
\tcode{!<} &
\tcode{!>} &
\tcode{<>} &
\tcode{cmp} &
\tcode{in} \\
\tcode{\&} &
\tcode{|} &&&\\
\hhline{|=====|}
\tcode{.} &
\tcode{await} &
\tcode{?} &
后缀\tcode{!} &
\tcode{??} \\
\tcode{!in} &
\tcode{is} &
\tcode{!is} &
\tcode{=} &
\tcode{+=} \\
\tcode{-=} &
\tcode{*=} &
\tcode{/=} &
\tcode{\%=} &
\tcode{shl_eq} \\
\tcode{shr_eq} &
\tcode{'\&=} &
\tcode{'\^{}=} &
\tcode{'|=} &
\tcode{??=} \\
\tcode{++} &
\tcode{--} &
\tcode{\~>} &
\tcode{<\~} &
\tcode{;} \\
\end{floattable}

\pnum
\tcode{operator}后面还能跟有一些特殊符号组合,表示特殊含义。

\pnum
\tcode{opeartor()}重载函数调用(\ref{expr.call})。\tcode{operator[]}重载下标运算符(\ref{expr.sub})。\tcode{opeartor\$}重载\tcode{\$}(\ref{expr.dollar})。\tcode{opeartor if}允许其它值被用于条件中。$\tcode{operatro is} T$创建自定义隐式转换。$\tcode{opeartor as}\ T$创建自定义显式转换。

\pnum
\tcode{operator}后跟一个字符串字面量用于自定义后缀。这个字符串字面量必须是一个空字符串,且不包含任何前缀。

\rSec1[op.user]{自定义运算符}
\indextext{运算符!自定义}

\begin{bnf}{OpeartorDeclaration}
    \terminal{operator} CustomOperator OperatorSpecifier \terminal{;} \br
    \terminal{operator} Identifier OperatorSpecifier \terminal{;}
\end{bnf}

\begin{bnf}{OperatorSpecifier}
    \terminal{prefix} \br
    \terminal{suffix} \br
    \terminal{infix} \br
    \terminal{infix} \terminal{:} OperatorName \br
    \terminal{infix} \terminal{:} \terminal{(} OperatorPrecedence \terminal{,} OperatorPrecedence \terminal{)}
\end{bnf}

\begin{bnf}{OperatorPrecedence}
    OperatorName \br
    \terminal{_}
\end{bnf}

\pnum
用户可以声明自己的运算符。自定义运算符的名称由关键字\tcode{operator}后跟一个标识符或标点符号组成。

\pnum
自定义的运算符不能与内建的运算符相同,也不能是\tcode{->}、\tcode{=>}或\tcode{...}。如果自定义的运算符是一个标识符,则它不能在任何作用域中与另一个标识符相同,否则这是一个编译错误。自定义的运算符也不能是\tcode{prefix}、\tcode{suffix}或\tcode{infix}。

\pnum
自定义的运算符可以是前缀、后缀或者中缀的,分别使用\tcode{prefix}、\tcode{suffix}和\tcode{infix}指定。
后缀与前缀运算符不能指定优先级。后缀运算符的优先级总是高于前缀运算符,并高于所有类型的中缀运算符。
中缀运算符可以指定其优先级。表~\ref{tab:op-precedence}~展示了可选择的中缀运算符优先级。

\begin{floattable}{中缀运算符优先级}{tab:op-precedence}{lll}
\topline
\tcode{*} & \tcode{/} & \tcode{\%} \\
\tcode{+} & \tcode{-} &\\
\tcode{shl} & \tcode{shr} &\\
\tcode{'\&} & \tcode{'\^{}} & \tcode{'|} \\
\tcode{..} & \tcode{..=} &\\
\tcode{\~} &&\\
\tcode{??} &&\\
\tcode{==} 等 &&\\
\tcode{\&} & \tcode{|} &\\
\tcode{=}等 &&\\
\end{floattable}

\pnum
中缀运算符可以指定自己的优先级与某个内建的运算符组相同,或指定两个相邻的运算符组,创建一个自定义的优先级。不能在\tcode{==}到\tcode{=}之间创建新的优先级。
此外,还可以使用\tcode{(_, *)}或\tcode{(=, _)}指定比表中运算符更低或更高的优先级。
\enternote 自定义的优先级无法指定结合性。然而,可以通过自定义运算符的重载函数自行处理优先级。参见\ref{op.over.infix}。\exitnote
中缀运算符也可以不指定优先级。这类运算符视为比\tcode{==}具有更高的优先级。
除了未指定优先级的中缀运算符,运算符的优先级关系是线性的。

\pnum
如果中缀运算符的优先级组处于\tcode{==}组,则其不能与任何内建运算符混用,且其返回值必须为布尔类型。
如果中缀运算符处于\tcode{\&}组,则其不能与任何内建运算符混用,且其参数和返回值必须为布尔类型。
如果中缀运算符处于\tcode{=}组,则其不能与任何内建运算符混用,且其返回值必须是\tcode{void}。
自定义的与\tcode{??}或\tcode{\&}优先级相同的运算符不具有短路求值特性。。
如果两个自定义运算符具有相同等级的优先级,且不是内建的优先级,则在一个表达式中将其混用是一个编译错误。
如果一个自定义运算符未指定优先级,则它与任何优先级高于\tcode{==}的中缀运算符混用是一个编译错误。

\enterexample
\begin{codeblock}
operator >> infix : +;
operator << infix : +;
operator !! infix : (*, +);
operator ## infix : (*, +);

a >> b + c; // OK,等价于\tcode{(a >> b) + c}
a >> b << c; // OK,等价于\tcode{(a >> b) << c}
a !! b * c; // OK,等价于\tcode{a !! (b * c)}
a !! b ## c; // 错误,不能在同一个表达式中混用具有新优先级的运算符

operator ** infix;
operator && infix;

a ** b * c; // 错误,未指定优先级的运算符不能与\tcode{*}混用
a = a ** b; // 可以,允许赋值运算符
a ** b && c; // 错误,相互之间也不能混用
\end{codeblock}
\exitexample

\pnum
一个运算符可以同时定义为前缀、后缀和中缀,但不能定义同一个类别的相同运算符多于一次。如果自定义运算符产生了语法歧义,则这是一个编译错误。

\enterexample
\begin{codeblock}
operator ** suffix;
operator ** infix;
operator && prefix;
operator && infix;

a ** && b; // 错误,无法确定谁为中缀运算符
(a**) && b; // 可以
a ** (&&b); // 可以
\end{codeblock}
\exitexample

\pnum
解析表达式的流程如下:

\begin{itemize}
    \item 构造基本表达式和运算符。运算符将尽可能长地构造,请在必要的时候使用空格分隔。
    \item 如果有运算符\tcode{.}、\tcode{as}或\tcode{is},将它及后面的必需成分一起替换成一个后缀运算符。\enternote \tcode{is}运算符之后的模式将会尽可能长地构造。请在需要的情况下加上括号。\exitnote
    \item 对每一组连续且多于一个的基本表达式,如果除了第一个以外都是元组字面量、数组字面量或块,将它们替换为后缀运算符。否则,这是一个编译错误。
    \item 确定每个运算符是前缀、后缀还是中缀的。
    \item 对每组连续的表达式-运算符-表达式组,如果存在唯一一个运算符满足:
    \begin{itemize}
        \item 它可以作为中缀运算符;
        \item 它前面的所有运算符都可以作为后缀运算符;
        \item 它后面的所有运算符都可以作为前缀运算符。
    \end{itemize}
    那么这个运算符是中缀运算符,之前的所有运算符是后缀运算符,之后的所有运算符是前缀运算符。如果存在多于一个或不存在这样的运算符,这是一个编译错误。
    \item 第一个子表达式之前的所有运算符都是前缀运算符。最后一个子表达式之后的所有运算符都是后缀运算符。如果有运算符不是这个类别,这是一个编译错误。
    \item 将基本表达式与其后的后缀运算符替换成一个后缀表达式。
    \item 将后缀表达式与其前的前缀运算符替换成一个前缀表达式。
    \item 此时,表达式将成为子表达式与中缀运算符交替的链,且总以子表达式开头和结尾。
    \item 如果表达式中同时包含未指定优先级的运算符和另一个优先级高于\tcode{==}的运算符,这是一个编译错误。
    \item 运算符按照优先级组依次选择其操作数,并结合成为子表达式。如果此时存在非法的运算符混用,这是一个编译错误。
    \item 如果此时仍然剩余多于一个子表达式,这是一个编译错误。否则,表达式解析完成。
\end{itemize}

\rSec1[op.over]{运算符重载}
\indextext{运算符!重载}

\pnum
运算符可以被重载,以为其他类似提供类似于内建类型的行为。当声明一个重载运算符函数时,视同内建的运算符具有一个对应的重载函数签名来进行重载检测。

\enterexample
\begin{codeblock}
func operator+(lhs: int?, rhs: int?) { lhs? + rhs? } // 可以
func operator+(lhs: int, rhs: int) { lhs + rhs } // 错误,与内建运算符冲突
\end{codeblock}
\exitexample

\rSec2[op.over.call]{函数调用运算符}
\indextext{运算符!重载!函数调用}

\pnum
函数调用运算符可以接受任意数量的参数并返回任意值。重载了这个运算符的类型可以隐式转换到对应的函数类型。

\pnum
函数调用运算符只能作为类方法被重载。

\rSec2[op.over.sub]{下标运算符}
\indextext{运算符!重载!下标}

\rSec2[op.over.prefix]{前缀运算符}
\indextext{运算符!重载!前缀}

\rSec2[op.over.suffix]{后缀运算符}
\indextext{运算符!重载!后缀}

\rSec2[op.over.infix]{中缀运算符}
\indextext{运算符!重载!中缀}

\pnum
对中缀运算符\tcode{@}而言,$e_1\mathbin{\tcode{@}}e_2\mathbin{\tcode{@}}\ldots\mathbin{\tcode{@}}e_n$将会调用
$\tcode{opeartor infix@(}e_1\tcode{,}e_2\tcode{,}\ldots\tcode{,}e_n\tcode{)}$或\\$\tcode{opeartor infix@([}e_1\tcode{,}e_2\tcode{,}\ldots\tcode{,}e_n\tcode{])}$,取先匹配的那一个。如果没有找到匹配的重载函数,且该运算符所处的优先级组未定义其结合性,则这是一个编译错误。否则,将会按照其结合性拆分为多次以两个参数调用的形式。若仍然无法找到匹配的重载函数,则这是一个编译错误。

\rSec2[op.over.assign]{赋值运算符}

\pnum
赋值运算符,及与其相同优先级的运算符的重载函数必须返回\tcode{void}。

\rSec2[op.over.dollar]{\tcode{\$}运算符}
\indextext{运算符!重载!\$}

\pnum
\tcode{\$}表达式将被转换为对当前序列$s$调用$s\tcode{.operator\$()}$。它只能接受一个\tcode{this}参数。

\rSec2[op.over.if]{条件运算符}
\indextext{运算符!重载!条件}

\pnum
\tcode{operator if}允许类型在条件中被使用。重载了\tcode{operator if}的类型和\tcode{bool}称为\term{布尔类型}。它只能接受一个\tcode{this}参数,且其返回值类型必须为布尔类型。

\pnum
当布尔类型的表达式$e$被用于条件中时,会不断调用$e.\tcode{operator if()}$,直到返回值为\tcode{bool}为止。然后这个值将被用于条件判断。

\rSec2[op.over.gen]{自动生成的运算符重载}
\indextext{运算符!重载!自动生成}

\pnum
部分运算符重载会在其他运算符重载之后自动生成。

\pnum
如果类型$T$重载了\tcode{operator cmp},则会自动生成其他的比较运算符:

\begin{codeblock}
func operator==(lhs: T, rhs: T) { (lhs cmp rhs) is .equal }
func operator!=(lhs: T, rhs: T) { (lhs cmp rhs) !is .equal }
func operator<(lhs: T, rhs: T) { (lhs cmp rhs) is .less }
func operator!<(lhs: T, rhs: T) { (lhs cmp rhs) !is .less }
func operator<=(lhs: T, rhs: T) { (lhs cmp rhs) is .less | .equal }
func operator>(lhs: T, rhs: T) { (lhs cmp rhs) is .greater }
func operator!>(lhs: T, rhs: T) { (lhs cmp rhs) !is .greater }
func operator>=(lhs: T, rhs: T) { (lhs cmp rhs) is .greater | .equal }
func operator<>(lhs: T, rhs: T) { (lhs cmp rhs) is .less | .greater }
\end{codeblock}

\pnum
如果类型$T$重载了\tcode{operator==},则会自动生成\tcode{operator!=}:

\begin{codeblock}
func operator!=(lhs: T, rhs: T) { !(lhs == rhs) }
\end{codeblock}

\tcode{operator<}与\tcode{opeartor!<}、\tcode{operator>}与\tcode{operator!>}也有类似的自动生成规则。\enternote 虽然\tcode{!in}不能被重载,但是它也有类似的语义。\exitnote