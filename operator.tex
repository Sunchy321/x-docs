%!TEX root = x.tex

\rSec0[op]{运算符}
\indextext{运算符}

\rSec1[op.name]{运算符名称}

\begin{bnf}{OperatorName}
    Operator \br
    Identifier \br
    OperatorKeyword \br
    \terminal{(} \terminal{)} \br
    \terminal{[} \terminal{]}
\end{bnf}

\begin{bnf}{OperatorKeyword}[\oneof]
    \terminal{shl shr cmp in}
\end{bnf}

\begin{bnf}{OperatorType}[\oneof]
    \terminal{infix prefix suffix}
\end{bnf}

\begin{bnf}{Operator}
    CustomOperator \br
    \terminal{;}
\end{bnf}

\begin{bnf}{CustomOperator}
    \terminal{'}\bnfq OperatorSymbol\bnfp
\end{bnf}

\begin{bnf}{OperatorSymbol}[\oneof]
    \terminal{\~ ! \% \^{} \& * - | + = / ? < > .}
\end{bnf}

\pnum
表~\ref{tab:op}~列出了全部内建运算符,其中上半部分为可以重载的运算符,下半部分为不能重载的运算符。

\begin{floattable}{内建运算符}{tab:op}{lllll}
\topline
\tcode{+!} & \tcode{-!} & 前缀\tcode{+} & 前缀\tcode{-} & 前缀\tcode{!} \\
\tcode{'\~} & 前缀\tcode{*} & \tcode{?} & 后缀\tcode{!} & \tcode{*} \\
\tcode{/} & \tcode{\%} & 二元\tcode{+} & 二元\tcode{-} & \tcode{shl} \\
\tcode{shr} & \tcode{'\&} & \tcode{'\^{}} & \tcode{'|} & \tcode{..} \\
\tcode{..=} & \tcode{\~} & \tcode{==} & \tcode{!=} & \tcode{<} \\
\tcode{<=} & \tcode{>} & \tcode{>=} & \tcode{cmp} & \tcode{in} \\
\tcode{\&} & \tcode{|} & & & \\
\hhline{|=====|}
\tcode{.} & \tcode{await} & 前缀\tcode{\&} & \tcode{!in} & \tcode{is} \\
\tcode{!is} & \tcode{=} & \tcode{+=} & \tcode{-=} & \tcode{*=} \\
\tcode{/=} & \tcode{\%=} & \tcode{shl_eq} & \tcode{shr_eq} & \tcode{'\&=} \\
\tcode{'\^{}=} & \tcode{'|=} & \tcode{??=} & \tcode{++} & \tcode{--} \\
\tcode{\~>} & \tcode{<\~} & \tcode{;} && \\
\end{floattable}

\rSec1[op.user]{自定义运算符}
\indextext{运算符!自定义}

\begin{bnf}{OpeartorDecl}
    \terminal{operator} CustomOperator OperatorSpecifier OperatorTraitBinder \terminal{;} \br
    \terminal{operator} Identifier OperatorSpecifier OperatorTraitBinder \terminal{;}
\end{bnf}

\begin{bnf}{OperatorSpecifier}
    \terminal{prefix} \br
    \terminal{suffix} \br
    \terminal{infix} \br
    \terminal{infix} \terminal{(} OperatorName \terminal{)} \br
    \terminal{infix} \terminal{(} OperatorPrecedence \terminal{,} OperatorPrecedence \terminal{)}
\end{bnf}

\begin{bnf}{OperatorPrecedence}
    OperatorName \br
    \terminal{_}
\end{bnf}

\begin{bnf}{OperatorTraitBinder}
    \terminal{=} PathList
\end{bnf}

\begin{bnf}{PathList}
    Path \br
    PathList \terminal{,} Path
\end{bnf}

\pnum
用户可以声明自己的运算符。自定义运算符需要指定一个或多个特征方法,实现此特征的类型可以使用这个运算符,且等价于调用该方法。

\pnum
自定义的运算符不能与内建的运算符相同,也不能是\tcode{//}、\tcode{/*}、\tcode{->}、\tcode{=>}或\tcode{...}。如果自定义的运算符是一个标识符,则它不能在任何作用域中与另一个标识符相同,否则这是一个编译错误。自定义的运算符也不能是\tcode{prefix}、\tcode{suffix}或\tcode{infix}。

\pnum
自定义的运算符可以是前缀、后缀或者中缀的,分别使用\tcode{prefix}、\tcode{suffix}和\tcode{infix}指定。
后缀与前缀运算符不能指定优先级。后缀运算符的优先级总是高于前缀运算符,并高于所有类型的中缀运算符。
中缀运算符可以指定其优先级。表~\ref{tab:op-precedence}~展示了可选择的中缀运算符优先级。

\begin{floattable}{中缀运算符优先级}{tab:op-precedence}{lll|c}
\topline
& 运算符 & & 结合性 \\
\capsep
\tcode{*} & \tcode{/} & \tcode{\%} & 左结合 \\
\tcode{+} & \tcode{-} & & 左结合 \\
\tcode{shl} & \tcode{shr} & & 不结合 \\
\tcode{'\&} & \tcode{'\^{}} & \tcode{'|} & 左结合,不能混用 \\
\tcode{..} & \tcode{..=} & & 不结合 \\
\tcode{\~} && & 左结合,不能混用 \\
\tcode{??} && & 右结合,不能混用 \\
\tcode{==} 等 && & 特殊结合性 \\
\tcode{\&} & \tcode{|} & & 左结合,不能混用 \\
\tcode{=}等 && & 不结合 \\
\end{floattable}

\pnum
中缀运算符可以指定自己的优先级与某个内建的运算符组相同,或指定两个相邻的运算符组,创建一个自定义的优先级。不能在\tcode{==}到\tcode{=}之间创建新的优先级。
此外,还可以使用\tcode{(_, *)}或\tcode{(=, _)}指定比表中运算符更低或更高的优先级。
\enternote 自定义的优先级无法指定结合性。然而,可以通过自定义运算符的重载函数自行处理优先级。参见\ref{op.over.infix}。\exitnote
中缀运算符也可以不指定优先级。这类运算符视为比\tcode{==}具有更高的优先级。
除了未指定优先级的中缀运算符,运算符的优先级关系是线性的。

\pnum
如果中缀运算符的优先级组处于\tcode{==}组,则其不能与任何内建运算符混用,且其返回值必须为布尔类型。
如果中缀运算符处于\tcode{\&}组,则其不能与任何内建运算符混用,且其参数和返回值必须为布尔类型。
如果中缀运算符处于\tcode{=}组,则其不能与任何内建运算符混用,且其返回值必须是\tcode{void}或\tcode{never}。
如果两个自定义运算符具有相同等级的优先级,且不是内建的优先级,则在一个表达式中将其混用是一个编译错误。
如果一个自定义运算符未指定优先级,则它与任何优先级高于\tcode{==}的中缀运算符混用是一个编译错误。

\enterexample
\begin{codeblock}
operator >> infix(+) = A::a;
operator << infix(+) = B::b;
operator !! infix(*, +) = C::c;
operator ~~ infix(*, +) = D::d;

a >> b + c; // OK,等价于\tcode{(a >> b) + c}
a >> b << c; // OK,等价于\tcode{(a >> b) << c}
a !! b * c; // OK,等价于\tcode{a !! (b * c)}
a !! b ~~ c; // 错误,不能在同一个表达式中混用具有新优先级的运算符

operator ** infix = E::e;
operator && infix = F::f;

a ** b * c; // 错误,未指定优先级的运算符不能与\tcode{*}混用
a = a ** b; // 可以,允许赋值运算符
a ** b && c; // 错误,相互之间也不能混用
\end{codeblock}
\exitexample

\pnum
一个运算符可以同时定义为前缀、后缀和中缀,但不能定义同一个类别的相同运算符多于一次。如果自定义运算符产生了语法歧义,则这是一个编译错误。

\enterexample
\begin{codeblock}
operator ** suffix = A::a;
operator ** infix = B::b;
operator && prefix = C::c;
operator && infix = D::d;

a ** && b; // 错误,无法确定谁为中缀运算符
(a**) && b; // 可以
a ** (&&b); // 可以
\end{codeblock}
\exitexample

\pnum
解析表达式的流程如下:

\begin{itemize}
    \item 构造基本表达式和运算符。运算符将尽可能长地构造,请在必要的时候使用空格分隔。
    \item 如果有运算符\tcode{.},将它及后面的必需成分一起替换成一个基本表达式。
    \item 如果有运算符\tcode{as}或\tcode{is},将它及后面的必需成分一起替换成一个后缀运算符。\enternote \tcode{is}运算符之后的模式将会尽可能长地构造。请在需要的情况下加上括号。\exitnote
    \item 对每一组连续且多于一个的基本表达式,如果除了第一个以外都是\tcode{.}后缀运算、元组字面量、数组字面量或块,将它们替换为后缀运算符。否则,这是一个编译错误。
    \item 确定每个运算符是前缀、后缀还是中缀的。
    \item 对每组连续的表达式-运算符-表达式组,如果存在唯一一个运算符满足:
    \begin{itemize}
        \item 它可以作为中缀运算符;
        \item 它前面的所有运算符都可以作为后缀运算符;
        \item 它后面的所有运算符都可以作为前缀运算符。
    \end{itemize}
    那么这个运算符是中缀运算符,之前的所有运算符是后缀运算符,之后的所有运算符是前缀运算符。如果存在多于一个或不存在这样的运算符,这是一个编译错误。
    \item 第一个子表达式之前的所有运算符都是前缀运算符。最后一个子表达式之后的所有运算符都是后缀运算符。如果有运算符不是这个类别,这是一个编译错误。
    \item 将基本表达式与其后的后缀运算符替换成一个后缀表达式。
    \item 将后缀表达式与其前的前缀运算符替换成一个前缀表达式。
    \item 此时,表达式将成为子表达式与中缀运算符交替的链,且总以子表达式开头和结尾。
    \item 如果表达式中同时包含未指定优先级的运算符和另一个优先级高于\tcode{==}的运算符,这是一个编译错误。
    \item 运算符按照优先级组依次选择其操作数,并结合成为子表达式。如果此时存在非法的运算符混用,这是一个编译错误。
    \item 如果此时仍然剩余多于一个子表达式,这是一个编译错误。否则,表达式解析完成。
\end{itemize}

\rSec1[op.over]{运算符重载}
\indextext{运算符!重载}

\pnum
运算符可以通过实现其对应特征的方式重载。如果一个类型为一个运算符实现了多个特征,则将会使用第一个可使用的特征。

\enterexample
\begin{codeblock}
trait A { func a(&mut this) -> int; }
trait B { func b(this) -> int; }

operator ** prefix = A::a, B::b;

class X = ();

impl X : A { func a(&mut this) => 1; }
impl X : B { func b(this) => 2; }

let x = () as X;

**x; // \tcode{A::a}不适用,调用\tcode{B::b},返回\tcode{2}

let mut x = () as X;

**x; // 调用\tcode{A::a},返回\tcode{1}
\end{codeblock}
\exitexample

\pnum
可重载的内建运算符也都有其对应的特征方法。所有特征方法都位于模块\tcode{core.ops}中。

\rSec2[op.over.suffix]{后缀运算符}
\indextext{运算符!重载!后缀}

\pnum
表~\ref{tab:trait-suffix}~列出了可重载的内建后缀运算符对应的特征。

\begin{floattable}{后缀运算符特征}{tab:trait-suffix}{ll}
\topline
运算符 & 特征 \\
\capsep
\tcode{[]} & \tcode{IndexMut::indexMut}, \tcode{Index::index} \\
\tcode{()} & \tcode{FunctorMut::callMut}, \tcode{Functor::call}, \tcode{FunctorOnce::callOnce} \\
\tcode{?} & \tcode{Failable::chain} \\
\tcode{!} & \tcode{Failable::unwrap} \\
\tcode{+?} & \tcode{Successor::next} \\
\tcode{-?} & \tcode{Predecessor::prev} \\
\tcode{++} & \tcode{Increment::inc} \\
\tcode{--} & \tcode{Decrement::dec} \\
\tcode{as} & \tcode{Into::into} \\
\tcode{as?} & \tcode{TryInto::tryInto} \\
\end{floattable}

\rSec2[op.over.index]{下标运算符}
\indextext{运算符!重载!下标}

\pnum
下标运算符对应四个特征方法:\tcode{Index::index}、\tcode{IndexAssign::indexAssign}、\tcode{IndexRef::indexRef}和\tcode{IndexRefMut::indexRefMut}。这四个特征都需要一个函数类型的泛型参数。

\pnum
对表达式\tcode{array[key]}(不作为赋值的左操作数),它等价于\tcode{*array.indexRef(key)}和\tcode{array.index(key)}中先成立的一个。

\pnum
对表达式\tcode{array[key] = value},它等价于\tcode{*array.indexRefMut(key) = value}和\tcode{array.indexAssign(key, value)}中先成立的一个。


\rSec2[op.over.call]{函数调用运算符}
\indextext{运算符!重载!函数调用}

\pnum
函数调用运算符有三个对应的特征方法:\tcode{FunctorMut::callMut}、\tcode{Functor::call}和\tcode{FunctorOnce::callOnce},分别对应可变、不可变和只能调用一次的情况。这三个特征都需要一个函数类型的泛型参数。

\rSec2[op.over.null]{空值检测与空值合并运算符}
\indextext{运算符!重载!空值检测与空值合并}

\pnum
后缀\tcode{?}和后缀\tcode{!}运算符是使用到了\tcode{FromResidual}及\tcode{Try}两个特征的语法糖。

\pnum
\tcode{$e$!}将被翻译为如下代码:

\begin{codeblock}

match Try::branch(\{$e$}) {
    case .Continue(\{$c$}) => \{$c$},
    case .Break(\{$b$}) => return FromResidual::fromResidual(\{$b$}),
}

\end{codeblock}

其中$c$和$b$是内部使用的匿名变量名称。

\pnum
后缀\tcode{?}与后缀\tcode{!}类似,但是残差值不会直接返回而是作为完整表达式的值。

\pnum
\tcode{$a$ ?? $b$}将被翻译为如下代码:

\begin{codeblock}

match Try::branch(\{$a$}) {
    case .Continue(\{$c$}) => \{$c$},
    case .Break(_) => \{$b$},
}

\end{codeblock}

其中$c$是内部使用的匿名变量名称。

\rSec2[op.over.as]{类型转换运算符}
\indextext{运算符!重载!类型转换}

\pnum
转换运算符\tcode{as}对应\tcode{Into::into}。转换运算符\tcode{as?}对应\tcode{TryInto::tryInto}。它们也可能由对应的\tcode{From}或\tcode{TryFrom}生成。

\pnum
转换运算符\tcode{as!}采用内建语义。它不能被显式重载以与\tcode{as?}采取不一致的语义。

\rSec2[op.over.prefix]{前缀运算符}
\indextext{运算符!重载!前缀}

\pnum
表~\ref{tab:trait-prefix}~列出了可重载的内建前缀运算符对应的特征。

\begin{floattable}{前缀运算符特征}{tab:trait-prefix}{ll}
\topline
运算符 & 特征 \\
\capsep
\tcode{+} & \tcode{Positive::positive} \\
\tcode{-} & \tcode{Negative::negative} \\
\tcode{!} & \tcode{Not::not} \\
\tcode{'\~} & \tcode{Complement::compl} \\
\tcode{*} & \tcode{DerefMut::derefMut}, \tcode{Deref::deref} \\
\end{floattable}

\rSec2[op.over.deref]{解引用运算符}
\indextext{运算符!重载!解引用}

\pnum
前缀\tcode{*}运算符对应\tcode{Deref::deref}和\tcode{DerefMut::derefMut},分别为不可变和可变版本。其返回类型必须是引用类型。

\rSec2[op.over.infix]{中缀运算符}
\indextext{运算符!重载!中缀}

\pnum
对中缀运算符\tcode{@}而言,\tcode{$e_1$ @ $e_2$ @ $\ldots$ @ $e_n$}将会调用
\tcode{Trait::method($e_1$, $e_2$, $\ldots$, $e_n$)}或\tcode{Trait::method([$e_1$, $e_2$, $\ldots$, $e_n$])},取先匹配的那一个。如果没有找到匹配的重载函数,且该运算符所处的优先级组未定义其结合性,则这是一个编译错误。否则,将会按照其结合性拆分为多次以两个参数调用的形式。若仍然无法找到匹配的重载函数,则这是一个编译错误。

\pnum
表~\ref{tab:trait-infix}~列出了可重载的内建中缀运算符对应的特征。

\begin{floattable}{中缀运算符特征}{tab:trait-infix}{ll}
\topline
运算符 & 特征 \\
\capsep
\tcode{*} & \tcode{Multiply::multiply} \\
\tcode{/} & \tcode{Divide::divide} \\
\tcode{\%} & \tcode{Modulo::modulo} \\
\tcode{+} & \tcode{Add::add} \\
\tcode{-} & \tcode{Subtract::subtract} \\
\tcode{shl} & \tcode{ShiftLeft::shiftLeft} \\
\tcode{shr} & \tcode{ShiftRight::shiftRight} \\
\tcode{'\&} & \tcode{BitAnd::bitAnd} \\
\tcode{'\^{}} & \tcode{BitXor::bitXor} \\
\tcode{'|} & \tcode{BitOr::bitOr} \\
\tcode{..} & \tcode{RangeTo::rangeTo} \\
\tcode{..=} & \tcode{ClosedRangeTo::closedrangeTo} \\
\tcode{\~} & \tcode{Concat::concat} \\
\tcode{in} & \tcode{Include::include} \\
\tcode{\&} & \tcode{LogicAnd::logicAnd} \\
\tcode{|} & \tcode{LogicOr::logicOr} \\
\tcode{+=} & \tcode{AddAssign::addAssign} \\
\tcode{-=} & \tcode{SubtractAssign::subtractAssign} \\
\tcode{*=} & \tcode{MultiplyAssign::multiplyAssign} \\
\tcode{/=} & \tcode{DivideAssign::divideAssign} \\
\tcode{\%=} & \tcode{ModuloAssign::moduloAssign} \\
\tcode{shl_eq} & \tcode{ShiftLeftAssign::shiftLeftAssign} \\
\tcode{shr_eq} & \tcode{ShiftRightAssign::shiftRightAssign} \\
\tcode{'\&=} & \tcode{BitAndAssign::bitAndAssign} \\
\tcode{'\^{}=} & \tcode{BitXorAssign::bitXorAssign} \\
\tcode{'|=} & \tcode{BitOrAssign::bitOrAssign} \\
\tcode{<\~} & \tcode{AppendRight::appendRight} \\
\tcode{\~>} & \tcode{AppendLeft::appendLeft} \\
\end{floattable}

\rSec2[op.over.cmp]{比较运算符}
\indextext{运算符!重载!比较}

\pnum
相等运算符\tcode{==}与\tcode{Equal::equal}或\tcode{PartialEqual::partialEqual}关联。

\pnum
不等运算符\tcode{!=}与\tcode{PartialEqual::partialNotEqual}关联。它提供了默认实现,因此重载了\tcode{==}的类型不需要显式重载\tcode{!=}。

\pnum
类型总是默认实现\tcode{Equal}。也可以通过显式实现来替换默认的实现。

\pnum
比较运算符\tcode{cmp}与\tcode{Ordered::compare}或\tcode{PartialOrdered::partialCompare}关联。

\pnum
\tcode{<}、\tcode{<=}、\tcode{>}、\tcode{>=}分别与\tcode{PartialOrdered::lessThan}、\tcode{PartialOrdered::lessEqual}、\tcode{PartialOrdered::greaterThan}、\tcode{PartialOrdered::greaterEqual}关联。它们提供了默认实现,因此重载了\tcode{cmp}的类型不需要显式重载这些运算符。

\rSec2[op.over.in]{包含运算符}
\indextext{运算符!重载!包含}

\pnum
\tcode{in}运算符与\tcode{Include::include}关联。与其他二元运算符相反的是,\tcode{$a$ in $b$}将被翻译为\tcode{$b$.include($a$)}。

\pnum
\tcode{!in}的语义总是保持与\tcode{in}相反。它不能被重载。

\rSec2[op.over.logic]{逻辑运算符}
\indextext{运算符!重载!逻辑}

\pnum
逻辑运算符的重载解析与一般的中缀运算符相同,但参数必须均以\tcode{lazy}修饰。重载调用时会隐式加入\tcode{lazy}修饰。

\pnum
自定义的逻辑运算符也必须遵循上述规则。

\rSec2[op.over.assign]{赋值运算符}
\indextext{运算符!重载!赋值}

\pnum
赋值运算符,及与其相同优先级的运算符的重载函数必须返回\tcode{void}或\tcode{never}。

\rSec2[op.over.dollar]{\tcode{\$}运算符}
\indextext{运算符!重载!\idxcode{\$}}

\pnum
\tcode{\$}表达式将被转换为对当前序列$s$调用\tcode{$s$.dollar()}。它只能接受一个\tcode{this}参数。当前序列必须实现\tcode{Dollar}。
