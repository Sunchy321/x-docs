%!TEX root = x.tex

\rSec0[attr]{特性与修饰符}
\indextext{特性}
\indextext{修饰符}

\begin{bnf}{Attribute}
    \terminal{@} Identifier \br
    \terminal{@} Identifier \terminal{(} Arguments\bnfq \terminal{)} \br
    \terminal{@} Identifier \terminal{[} ExprList\bnfq \terminal{]} \br
    \terminal{@} Identifier \terminal{\{} StructItems\bnfq \terminal{\}} \br
    \terminal{@} Identifier DictLiteral
\end{bnf}

\pnum
特性可以修饰特定的程序实体,以对其添加特定的描述或限制。

\rSec1[attr.noreturn]{\tcode{noreturn}}
\indextext{特性!\idxcode{noreturn}}

\pnum
\tcode{noreturn}修饰的函数将不会返回。函数的返回类型必须是\tcode{never}。编译器可以对实际控制流能到达函数结尾的\tcode{noreturn}函数提出一个警告或编译错误。

\rSec1[attr.deprecated]{\tcode{deprecated}}
\indextext{特性!\idxcode{deprecated}}

\pnum
\tcode{deprecated}特性用于标记已经过时的程序实体。编译器可以对使用已经过时的程序实体提出一个警告或编译错误。