%!TEX root = x.tex

\rSec0[trait]{特征}
\indextext{特征}

\begin{bnf}{TraitDecl}
    TraitQual\bnfs \terminal{trait} Identifier SuperTrait\bnfq TraitBody
\end{bnf}

\begin{bnf}{TraitQual}
    \terminal{const}
\end{bnf}

\begin{bnf}{SuperTrait}
    \terminal{:} EntityID \br
    SuperTrait \terminal{,} EntityID
\end{bnf}

\begin{bnf}{TraitBody}
    \terminal{\{} TraitMember\bnfs \terminal{\}}
\end{bnf}

\begin{bnf}{TraitMember}
    TypeConstraint \br
    ConstConstraint \br
    FuncConstraint \br
    PropConstraint
\end{bnf}

\begin{bnf}{TypeConstraint}
    \terminal{type} Identifier \terminal{;} \br
    \terminal{type} Identifier \terminal{:} Type \terminal{;} \br
    \terminal{type} Identifier \terminal{=} Type \terminal{;}
\end{bnf}

\begin{bnf}{ConstConstraint}
    \terminal{const} Identifier \terminal{;} \br
    \terminal{const} Identifier \terminal{=} Expression \terminal{;}
\end{bnf}

\begin{bnf}{FuncConstraint}
    FuncQual\bnfs \terminal{func} FuncName Parameter ThrowQual\bnfq ReturnType \terminal{;} \br
    FuncDecl
\end{bnf}

\begin{bnf}{PropConstraint}
    PropertyQual\bnfs BindKeyword Identifier TypeNotation\bnfq PropertyBody \terminal{;} \br
    PropertyDecl
\end{bnf}

\pnum
\term{特征}描述了类型可以满足的抽象接口,包括类型约束、常量约束、函数约束和属性约束。
如果这些约束提供了一个具体的类型、值或函数实现,则它会被用作该约束的默认实现。
如果一个约束没有提供默认实现,则实现该特征的类型必须提供一个实现。

\pnum
特征可以具有一个或数个\term{父特征}。实现特征的类型必须实现其所有父特征。