%!TEX root = x.tex

\rSec0[expr]{表达式}
\indextext{表达式}

\begin{bnf}{Expression}
    PrimaryExpr \br
    Operator Expression \br
    Expression Operator \br
    Expression Operator Expression
\end{bnf}

% \pnum
% \term{表达式}由\term{运算符}与操作数按照顺序组合在一起,它表示一个计算过程。运算符是若干标点符号的组合、一个标识符或是语言规定的特殊结构。运算符可以是前缀、后缀或者是中缀的,这决定了运算符与其操作数的结合方式。运算符组是运算符的集合,每一个运算符都属于某个运算符组。运算符组之间有一个弱偏序关系,决定了它们的\term{优先级}。完全由中缀运算符构成的运算符组有\term{结合性}:左结合的组每个运算符从左到右选择操作数;右结合的组从右到左;无结合的组两个运算符不能选择同一个操作数。

\pnum
\ref{expr.suffix}到\ref{expr.semi}各节按照优先级从高到低依次对运算符组进行描述。若无特别说明,每一节描述一个运算符组。用户也能定义新的运算符组或在当前的组中添加新的运算符,参见\ref{op.user}。

$$ \mathrm{\rhd}: \mathcal{E} \times \Omega \rightarrow (\mathcal{V} \cup \mathcal{V}^\dag \cup \{\ast\}) \times \Omega $$

\indextext{求值}
\pnum
$\mathrm{e \rhd \omega}$ 称作\term{在环境 $\omega$ 下对 $e$ 求值}。设 $\mathrm{e \rhd \omega} = \langle v, \omega^\prime \rangle$。如果 $v \in \mathcal{V}$,称\term{求值正常结束},$v$ 是 $e$ 的\term{值};否则,称\term{求值以抛出$v$异常结束}。$v = \ast$ 意味着求值过程中程序终止了。若无特别说明,对表达式$e$的子表达式$e_0$求值以抛出$v$异常结束也会导致对$e$的求值以抛出$v$异常结束。

\pnum
$\omega$ 是求值之前的环境,$\omega^\prime$ 是求值之后的环境。如果 $\omega = \omega^\prime$,称 $e$ 是\term{无副作用}的;如果 $e$ 是无副作用的且 $v$ 与 $\omega$ 无关,称 $e$ 是\term{纯}的。

\pnum
对含有子表达式的表达式求值时,总是先对其子表达式按出现次序从左到右求值。

\pnum
\term{丢弃表达式 $e$ 的结果}指,在决定 $e$ 的类型时,直接将它确定为 \tcode{never} 而跳过所有步骤;在对 $e$ 求值时,进行所有步骤,但是如果求值正常结束,丢弃最后的值。

\pnum
完整表达式指下列情况之一:

\begin{itemize}
    \item 表达式语句中的表达式;
    \item 一个语句表达式;
    \item 绑定语句中的初始化表达式;
    \item lambda表达式的函数体表达式;
    \item \tcode{if}、\tcode{match}、\tcode{while}的条件表达式;
    \item \tcode{for}语句的迭代表达式;
    \item \tcode{return}、\tcode{throw}、\tcode{break}表达式的子表达式;
    \item 函数参数的默认值表达式;
    \item 函数体与属性体表达式;
    \item 枚举的初始化表达式;
    \item 模式匹配与泛型的约束表达式;
    \item \tcode{typeof}的参数表达式;
    \item 泛型参数的非类型表达式。
\end{itemize}

\rSec1[expr.primary]{基本表达式}
\indextext{表达式!基本}

\begin{bnf}{PrimaryExpr}
    \terminal{(} Expression \terminal{)} \br
    LiteralExpr \br
    TypeLiteral \br
    Identifier \br
    DeductedEnumerator \br
    \terminal{this} \br
    \terminal{\$} \br
    StmtExpr \br
    LambdaParameter \br
    LambdaExpr \br
    DoExpr \br
    TryExpr
\end{bnf}

\pnum
表达式$\tcode{(}e\tcode{)}$等价于$e$,括号只作分组用途。\enternote 注意括号表达式不是一元元组。\exitnote

\rSec2[expr.lit]{字面量表达式}
\indextext{表达式!字面量}

\begin{bnf}{LiteralExpr}
    IntegerLiteral \br
    FloatingLiteral \br
    StringLiteral \br
    CharacterLiteral \br
    SymbolLiteral \br
    \terminal{'(} UnqualID \terminal{)} \br
    BooleanLiteral \br
    \terminal{nil} \br
    ArrayLiteral \br
    TupleLiteral \br
    StructLiteral \br
    DictLiteral
\end{bnf}

\begin{bnf}{ArrayLiteral}
    \terminal{[} ExprList\bnfq \terminal{]}
\end{bnf}

\begin{bnf}{TupleLiteral}
    \terminal{(} TupleExprList\bnfq \terminal{)}
\end{bnf}

\begin{bnf}{ExprList}
    ExprListNoComma \terminal{,}\bnfq
\end{bnf}

\begin{bnf}{ExprListNoComma}
    ExprItem \br
    ExprListNoComma \terminal{,} ExprItem
\end{bnf}

\begin{bnf}{TupleExprList}
    Expression \terminal{,} \br
    \terminal{...} Expression \terminal{,}\bnfq \br
    TupleExprListNoComma \terminal{,}\bnfq
\end{bnf}

\begin{bnf}{TupleExprListNoComma}
    ExprItem \terminal{,} ExprItem \br
    TupleExprListNoComma \terminal{,} ExprItem
\end{bnf}

\begin{bnf}{ExprItem}
    Expression \br
    \terminal{...} Expression
\end{bnf}

\begin{bnf}{StructLiteral}
    \terminal{\{} StructItems\bnfq \terminal{\}}
\end{bnf}

\begin{bnf}{StructItems}
    StructItemsNoComma \terminal{,}\bnfq
\end{bnf}

\begin{bnf}{StructItemsNoComma}
    StructItem \br
    StructItemsNoComma \terminal{,} StructItem
\end{bnf}

\begin{bnf}{StructItem}
    Identifier \terminal{:} Expression \br
    Identifier \br
    \terminal{...} Expression
\end{bnf}

\begin{bnf}{DictLiteral}
    \terminal{[} DictItems \terminal{]} \br
    \terminal{[} \terminal{:} \terminal{]}
\end{bnf}

\begin{bnf}{DictItems}
    DictItemsNoComma \terminal{,}\bnfq
\end{bnf}

\begin{bnf}{DictItemsNoComma}
    DictItem \br
    DictItemsNoComma \terminal{,} DictItem
\end{bnf}

\begin{bnf}{DictItem}
    Expression \terminal{:} Expression \br
    \terminal{...} Expression \br
    \terminal{...} Expression \terminal{:} \terminal{...} Expression
\end{bnf}

\pnum
字面量本身是基本表达式。整数字面量、浮点字面量、字符串字面量和布尔字面量的值分别参见\ref{literal.integer}、\ref{literal.floating}、\ref{literal.string}和\ref{literal.boolean}节的定义。

\pnum
$\tcode{'(}id\tcode{)}$表示与$id$对应的符号表达式。$id$本身不能是符号字面量。

\pnum
\tcode{nil}的类型为$T\tcode{?}$,其中$T$为待推导的类型参数。

\pnum
数组字面量$\tcode{[}e_1\tcode{,}\ldots\tcode{,}e_n\tcode{]}$表示一个显式写出其各元素的数组值。
其类型为$T\tcode{[]}$,其中$T$为各表达式的公共类型。
如果其中包含形如$\tcode{...}e$的项,则视同将$e$的各元素显式插入在该位置。$e$必须可迭代。
如果数组字面量不包含任何成员,则$T$是一个待推导的类型。

\pnum
元组字面量$\tcode{(}e_1\tcode{,}\ldots\tcode{,}e_n\tcode{)}$表示一个显式写出其各元素的元组值。
其类型为$\tcode{(}T_1\tcode{,}\ldots\tcode{,}T_n\tcode{)}$,其中$T_i$为$e_i$的类型。
如果其中包含形如$\tcode{...}e$的项,则视同将$e$的各元素显式插入在该位置。$e$必须也是一个元组。
特别的,$\tcode{(}e\tcode{,)}$表示一元元组,此时逗号不能省略。\tcode{()}的类型为\tcode{void},是\tcode{void}的唯一值。

\pnum
结构字面量$\tcode{\{}k_1\tcode{:}e_1\tcode{,}\ldots\tcode{,}x_n\tcode{:}e_n\tcode{\}}$表示一个显式写出其各元素的结构值。
其类型为$\tcode{\{}k_1\tcode{:}T_1\tcode{,}\ldots\tcode{,}x_n\tcode{:}T_n\tcode{\}}$,其中$T_i$为$e_i$的类型。
如果其中包含形如$\tcode{...}e$的项,则视同将$e$的各成员以相同标签显式插入在该位置。$e$必须是结构类型。
如果其中包含形如$e$的项,且$e$是一个标识符,则视同$e\mathbin{\tcode{:}}e$。
\enternote 结构字面量必须至少包含一个键值对。\tcode{\{\}}将被解析为一个块。\exitnote

\pnum
形如\tcode{\{\}}的表达式总是被视为一个空的结构字面量而不是块。\enternote 如果需要创建一个空的块,请使用\tcode{do! \{\}}。\exitnote
形如$\tcode{\{}e\tcode{\}}$的表达式总是被视为一个块而不是结构字面量。\enternote 如果需要创建结构字面量,请显式写为$\tcode{\{}e\tcode{:}e\tcode{\}}$。\exitnote

\pnum
字典字面量$\tcode{[}k_1\tcode{:} v_1\tcode{,} k_2\tcode{:} v_2\tcode{,} \ldots\tcode{,} k_n\tcode{:} v_n\tcode{]}$表示一个显式写出其各元素的字典值。
其类型为$T\tcode{[}K\tcode{]}$,其中$T$为各$v_i$的类型,$K$为各$k_i$的公共类型。
如果其中包含形如$\tcode{...}e$的项,则视同将$e$的各元素对显式插入在该位置,$e$必须是字典类型。
如果其中包含形如$\tcode{...}k\mathbin{\tcode{:}}\tcode{...}v$的项,则视同将$k$和$v$的各元素组成键值对显式插入在该位置。$k$和$v$必须是可迭代的。如果两者产生的值不能对应,则多出的值会被忽略。
如果如果字典字面量不包含任何表达式,则$T$和$K$是待推导的类型。

\pnum
如果以方括号界定的字面量只包含元素展开,则其总是被识别为数组字面量。\enternote 如果需要创建一个只包含元素展开的字典字面量,请将其中一个元素改写为键值对的展开。\exitnote

\rSec2[expr.lit.init]{初始化字面量}
\indextext{表达式!字面量!初始化}

\begin{bnf}{TypeLiteral}
    TypeParenLiteral \br
    TypeArrayLiteral \br
    TypeStructLiteral \br
    TypeDictLiteral
\end{bnf}

\begin{bnf}{TypeParenLiteral}
    TypeName \terminal{(} Arguments\bnfq \terminal{)} Block\bnfs \br
    TypeName Block
\end{bnf}

\begin{bnf}{TypeArrayLiteral}
    TypeName \terminal{[} ExprList\bnfq \terminal{]}
\end{bnf}

\begin{bnf}{TypeStructLiteral}
    TypeName \terminal{\{} StructItems\bnfq \terminal{\}}
\end{bnf}

\begin{bnf}{TypeDictLiteral}
    TypeName DictLiteral
\end{bnf}

\pnum
可以使用与字面量类似的语法来创建指定类型的值。被创建的类型必须是一个类型名称。

\enterexample
\begin{codeblock}

let a = int[] [1, 2, 3]; // 错误,不能用\tcode{[]}初始化\tcode{int}

type IntArray = int[];

let a = IntArray[1, 2, 3]; // 可以

let a = (int[])[1, 2, 3]; // 可以

\end{codeblock}
\exitexample

\pnum
$T\tcode{(}a_1\tcode{,}\ldots\tcode{,}a_n\tcode{)}$以参数给定的参数创建$T$的对象。创建对象时,会以这些参数调用给定的初始化器。以这种方式初始化的对象也可以带有lambda块。如果这个对象的初始化器只接受这一个参数,也可以省略括号。

\pnum
$T\tcode{[}v_1\tcode{,}\ldots\tcode{,}v_n\tcode{]}$以对应的数组字面量为参数创建$T$的对象。
$T\tcode{[}k_1\tcode{:} v_1\tcode{,} k_2\tcode{:} v_2\tcode{,} \ldots\tcode{,} k_n\tcode{:} v_n\tcode{]}$以对应的字典字面量创建$T$的对象。

\pnum
$T\tcode{\{}k_1\tcode{:}v_1\tcode{,}\ldots\tcode{,}k_n\tcode{:}v_n\tcode{\}}$以结构方式创建$T$的对象。

\pnum
如果$T$是结构类型,则每个$kv$对都会用来初始化对应的成员。如果$v$不能隐式转换到成员的类型,则这是一个编译错误。如果有成员未被显式指定初始值,且其不是可空类型,则这是一个编译错误;如果是可空类型,则其初始值为\tcode{nil}。

\pnum
如果$T$是类类型,则选择以下方式中第一个成功的:

\begin{itemize}
    \item 以具名参数初始化,等价于$T\tcode{(}k_1\tcode{:}v_1\tcode{,}\ldots\tcode{,}k_n\tcode{:}v_n\tcode{)}$;
    \item 以单个结构字面量的方式初始化,等价于$T\tcode{(\{}k_1\tcode{:}v_1\tcode{,}\ldots\tcode{,}k_n\tcode{:}v_n\tcode{\})}$。
\end{itemize}

\pnum
$T\tcode{\{\}}$、$T\tcode{\{}e\tcode{\}}$和$T\tcode{[...}e\tcode{]}$的语法歧义将按与普通字面量类似的规则解决(见上)。

\pnum
如果使用了内建的初始化方式或者采用了不抛出异常的初始化器,则结果类型为$T$;否则,结果类型为$T\mathbin{\tcode{throw}}E$,其中$E$为抛出的异常类型。

\rSec2[expr.enum]{匿名静态成员表达式}

\begin{bnf}{DeductedEnumerator}
    \terminal{.} Identifier \br
    \terminal{.} Identifier \terminal{(} Arguments\bnfq \terminal{)} Block\bnfs \br
    \terminal{.} Identifier Block \br
    \terminal{.} Identifier \terminal{[} ExprList\bnfq \terminal{]} \br
    \terminal{.} Identifier \terminal{\{} StructItems\bnfq \terminal{\}}
\end{bnf}

\pnum
静态成员和枚举符可以在适当的上下文中省略类型名称而使用自动推导。参见~\ref{deduct.static}。

\rSec2[expr.this]{\tcode{this}}
\indextext{表达式!\idxcode{this}}

\pnum
表达式\tcode{this}在类方法或扩展方法中表示当前方法的调用者。如果没有在参数中显式指定\tcode{this}的类型,则其类型为\tcode{self}。

\rSec2[expr.dollar]{\tcode{\$}}
\indextext{表达式!\idxcode{\$}}

\pnum
表达式\tcode{\$}只能在序列作用域(\ref{scope.sequence})中使用。
如果当前序列为$s$,则\tcode{\$}等价于$s\tcode{.opeartor\$()}$。
如果$s$实现了\tcode{Sequence}(例如内建数组$T\tcode{[]}$),其值为$s\tcode{.size}$。
在其他位置使用\tcode{\$}是一个编译错误。

\rSec2[expr.stmt]{语句表达式}
\indextext{表达式!语句}

\begin{bnf}{StmtExpr}
    Block \br
    IfExpr \br
    MatchStatement \br
    \terminal{break} SymbolLiteral\bnfq Expression\bnfq \br
    \terminal{continue} SymbolLiteral\bnfq \br
    \terminal{return} Expression\bnfq \br
    \terminal{throw} Expression\bnfq
\end{bnf}

\begin{bnf}{IfExpr}
    IfStatement \br
    \terminal{if} Condition \terminal{then} Expression \br
    \terminal{if} Condition \terminal{then} Expression \terminal{else} Expression \br
    \terminal{if} Condition \terminal{then} Expression \terminal{else} IfExpr
\end{bnf}

\pnum
部分语句也有对应的表达式。

\pnum
块表达式的值是其末尾表达式的值。如果该表达式被省略,其类型为\tcode{void}。

\pnum
\tcode{if}表达式的类型是其两个分支表达式的公共类型。如果\tcode{else}分支被省略,则视为\tcode{void}。其值为最终执行的那个分支的值。只有一个分支会被求值。可以使用\tcode{then}来用表达式代替块。

\pnum
\tcode{match}表达式的类型为其所有分支的公共类型,值为最终匹配成功的分支的值。只有一个分支会被求值。

\pnum
\tcode{break}、\tcode{continue}、\tcode{return}和\tcode{throw}表达式的语义与其对应的语句的语义相同,类型为\tcode{never}。

\rSec2[expr.lambda]{Lambda表达式}
\indextext{表达式!lambda}

\begin{bnf}{LambdaExpr}
    LambdaParameter LambdaQual\bnfs ReturnType\bnfq \terminal{=>} LambdaBody
\end{bnf}

\begin{bnf}{LambdaQual}
    \terminal{async} \br
    ThrowQual
\end{bnf}

\begin{bnf}{LambdaParameter}
    ParamDecl
\end{bnf}

\begin{bnf}{LambdaBody}
    Expression
\end{bnf}

\pnum
$\tcode{\$}$后跟数字$i$指代第$i$个参数。$\tcode{\$}$后跟标识符$n$指代具名参数$n$。

\rSec3[expr.do]{\tcode{do}表达式}
\indextext{表达式!\idxcode{do}}

\begin{bnf}{DoExpr}
    \terminal{do} LambdaQual\bnfs Block \br
    \terminal{do!} LambdaQual\bnfs Block
\end{bnf}

\pnum
\tcode{do}后跟一个块创建一个没有显式指定参数的lambda表达式。

\pnum
\tcode{do!}后跟一个块创建一个无参的lambda表达式并立即调用它,将其值作为整个表达式的值。\enternote \tcode{do}和后面的\tcode{!}之间必须没有空白,否则会被识别成两个分开的运算符。\exitnote

\enterexample
\begin{codeblock}
let arr = [1, 2, 3];

let first1 = arr.first(do { $0 > 2 }); // 获取第一个满足条件的元素

let firstFinder = do {
    for let v : $0 {
        if v > 2 { return v; }
    }
    nil
};

let first2 = firstFinder(arr); // 与上面等价

let first3 = do! {
    for let v : arr {
        if v > 2 { return v; }
    }
    nil
}; // 与上面等价

\end{codeblock}
\exitexample

\rSec1[expr.suffix]{后缀运算符}
\indextext{运算符!后缀}

\begin{bnf}{SuffixExpr}
    PrimaryExpr \br
    IndexExpr \br
    FuncCallExpr \br
    MemberAccessExpr \br
    AwaitExpr \br
    NullCheckExpr \br
    PrevNextExpr \br
    IncDecExpr \br
    CastExpr \br
    MatchExpr
\end{bnf}

\rSec2[expr.sub]{下标运算符}
\indextext{运算符!下标}

\begin{bnf}{IndexExpr}
    SuffixExpr \terminal{[} ExprList\bnfq \terminal{]}
\end{bnf}

\pnum
下标运算符用于对数组进行访问。$a\tcode{[}i\tcode{]}$表示数组$a$的第$i$个元素。用户自定义的下标运算符可以接受多于一个参数。

\rSec2[expr.call]{函数调用运算符}
\indextext{运算符!函数调用}

\begin{bnf}{FuncCallExpr}
    SuffixExpr \terminal{(} Arguments\bnfq \terminal{)} Block\bnfs \br
    SuffixExpr Block
\end{bnf}

\begin{bnf}{Arguments}
    UnnamedArgs \br
    NamedArgs \br
    UnnamedArgs \terminal{,} NamedArgs
\end{bnf}

\begin{bnf}{UnnamedArgs}
    Argument \br
    UnnamedArgs \terminal{,} Argument
\end{bnf}

\begin{bnf}{NamedArgs}
    Identifier \terminal{:} Argument \br
    NamedArgs \terminal{,} Identifier \terminal{:} Argument
\end{bnf}

\begin{bnf}{Argument}
    ParamQual\bnfq Expression
\end{bnf}

\pnum
函数调用运算符用于调用函数。括号内的项作为参数传递给函数。

\pnum
如果函数调用运算符的左操作数形如$o\tcode{.}f$,其中$f$是一个名称且不是$o$的成员名称,则这称作\term{方法调用}。此时,$o$将作为$f$的\tcode{this}参数传递给函数$f$。

\pnum
函数调用运算符可以后跟一个块。这个块将作为一个匿名lambda块,创建一个lambda表达式并作为函数的最后一个位置参数传递给函数。若此时函数没有任何其他参数,则函数调用的小括号可以省略。

\pnum
如果函数参数指定了修饰符,则传递实参时必须显式传递相同的修饰符。

\rSec2[expr.member]{成员访问运算符}
\indextext{运算符!成员访问}

\begin{bnf}{MemberAccessExpr}
    SuffixExpr \terminal{.} UnqualID \br
    SuffixExpr \terminal{.} IntegerLiteral \br
    SuffixExpr \terminal{.} \terminal{(} Expression \terminal{)}
\end{bnf}

\pnum
成员访问运算符用于访问对象的成员。$o\tcode{.}m$表示对象$o$的成员$m$。如果$o$没有成员$m$,若它后跟一个函数调用运算符,则作为方法调用处理;否则这是一个编译错误。

\pnum
$o\tcode{.}i$用于元组成员的访问,表示$o$的第$i$个元素。$i$必须是一个不包含前缀或后缀的十进制字面量。如果$i$大于元组的长度,这是一个编译错误。

\pnum
$o\tcode{.(}e\tcode{)}$首先对$e$求值。如果得到一个\tcode{symbol}类型的值,则对$o$进行成员访问。如果得到一个整数类型的值,则对$o$进行元组成员访问。此时$o$必须是一个常量表达式。否则,$o$必须是一个带有\tcode{this}参数的函数类型,且其必须后跟一个函数调用运算符,此时将视为对$e$调用方法$o$。

\enterexample
\begin{codeblock}
let o = { a: 1, b: 2 };
let s = 'a;

o.(s) // 等价于\tcode{o.a}

let t = (1, 2);
let k = 0;

t.(k) // 等价于\tcode{t.0}

impl int {
    func test(this) => this;
}

let f: (this: int) -> int = int::test;

0.(f)() // 等价于\tcode{0.test()}
\end{codeblock}
\exitexample

\rSec2[expr.await]{\tcode{await}运算符}
\indextext{运算符!\idxcode{await}}

\begin{bnf}{AwaitExpr}
    SuffixExpr \terminal{.} \terminal{await}
\end{bnf}

\pnum
\tcode{await}运算符用于等待一个异步表达式的结果。$e\tcode{.await}$挂起当前计算,直到$e$的值可用。如果$e$的类型是$\tcode{Future<}T\tcode{>}$,则$e\tcode{.await}$的类型是$T$。只能在具有\tcode{async}修饰的函数作用域中使用\tcode{await}运算符。

\rSec2[expr.null]{空值检测运算符}
\indextext{运算符!空值检测}

\begin{bnf}{NullCheckExpr}
    SuffixExpr \terminal{?} \br
    SuffixExpr \terminal{!}
\end{bnf}

\pnum
$e\tcode{?}$对$e$进行空值检测。如果$e$的值不为\tcode{nil},则$e\tcode{?}$的值为$e$;否则,该表达式直到空值检测运算符为止的整个表达式的值为\tcode{nil}。$e$的类型必须是$T\tcode{?}$。即使空值检测运算符检测为空,表达式的其它部分仍然会被求值。

\enterexample
\begin{codeblock}
let a: int? = nil;

a + 1 // 编译错误,不存在接受\tcode{int?}和\tcode{int}的加法运算符
a? + 1 // \tcode{nil}
(a + 1)? // \tcode{nil},多余的运算符
a? + 1 ?? 1 // $1$,\tcode{?}不会越过\tcode{??}运算符
(a? + 1) ?? 1 // 同上,但是更清晰
a? + 1 == b // 等号也会停止传播,等价于\tcode{(a? + 1) == b}

func f(i: int) => i + 1;

f(a?) // 整个表达式的类型为\tcode{int?},值为\tcode{nil}
f(a ?? 2) // 被空值检测运算符截断,整个表达式的类型为\tcode{int},值为\tcode{3}

let mut b: int? = 1;

b = a?; // 被赋值运算符截断,\tcode{b}成为\tcode{nil}

match a? { some let t -> true, nil -> false } // 被\tcode{match}的条件截断,不会传播到整个\tcode{match}表达式级别
\end{codeblock}
\exitexample

\pnum
后缀\tcode{!}将可空类型转换为非可空类型。对表达式而言,如果$e\tcode{!}$不为\tcode{nil},则结果为$e$;否则\tcode{panic}。参见~\ref{except.panic}。

\pnum
如果空值检测运算符的操作数类型是$T\tcode{?}$,则整个表达式的类型为$T$。
如果空值检测运算符的左操作数不是可空类型(且并未重载空值检测运算符),则空值检测运算符将被忽略。编译器可以为此提出一个警告。

\pnum
如果$e$是结果类型$T\mathbin{\tcode{throw}}E$,则$e\tcode{?}$与$\tcode{try }e\tcode{ catch \{ nil \} else let }e\tcode{ \{ some }e\tcode{ \}}$等价。

\pnum
如果$e$是结果类型,则$e!$在$e$为正确值的情况下返回$e$,否则抛出$e$的错误值,即等价于$\tcode{try }e$。

\rSec2[expr.prev-next]{前驱后继运算符}
\indextext{运算符!前驱后继}

\begin{bnf}{PrevNextExpr}
    SuffixExpr \terminal{+!} \br
    SuffixExpr \terminal{-!}
\end{bnf}

\pnum
$e\tcode{+!}$和$e\tcode{-!}$分别表示$e$的后继和前驱。如果$e$的类型是算术类型,$e\tcode{+!}$等价于$e+1$,$e\tcode{-!}$等价于$e-1$。

\rSec2[expr.inc-dec]{自增自减运算符}

\begin{bnf}{IncDecExpr}
    SuffixExpr \terminal{++} \br
    SuffixExpr \terminal{--}
\end{bnf}

\pnum
自增运算符$e\tcode{++}$等价于$e\ \tcode{=}\ e\tcode{+!}$,自减运算符$e\tcode{--}$等价于$e\ \tcode{=}\ e\tcode{-!}$,但$e$只被求值一次。

\rSec2[expr.cast]{类型转换运算符}
\indextext{运算符!类型转换}

\begin{bnf}{CastExpr}
    SuffixExpr \terminal{as} Type \br
    SuffixExpr \terminal{as?} Type \br
    SuffixExpr \terminal{as!} Type
\end{bnf}

\pnum
$e\mathbin{\tcode{as}}T$运算符用于进行显式的类型转换,结果的类型为\tcode{T}。

\pnum
$e\mathbin{\tcode{as?}}T$用于进行可能失败的类型转换,结果类型为可空类型或结果类型。
$e\mathbin{\tcode{as!}}T$用于进行强制类型转换,其等价于$\tcode{(}e\mathbin{\tcode{as?}}T\tcode{)!}$。
\enternote \tcode{as?}和\tcode{as!}中\tcode{as}和后面的符号之间必须没有空白,否则会被识别为两个分开的运算符。\exitnote

\rSec2[expr.match]{匹配运算符}
\indextext{运算符!匹配}

\begin{bnf}{MatchExpr}
    RangeExpr \terminal{is} Pattern \br
    RangeExpr \terminal{!is} Pattern
\end{bnf}

\pnum
$a \mathrel{\tcode{is}} p$检测$a$是否匹配模式$p$。$a \mathrel{\tcode{!is}} p$等价于$\tcode{!(}a \mathrel{\tcode{is}} p\tcode{)}$。表达式的类型为\tcode{bool}。

\pnum
\enternote \tcode{!is}中\tcode{!}和\tcode{is}之间必须没有空白,否则会被识别成两个分开的运算符。\exitnote

\rSec1[expr.prefix]{前缀运算符}
\indextext{运算符!前缀}

\begin{bnf}{PrefixExpr}
    SuffixExpr \br
    \terminal{+} PrefixExpr \br
    \terminal{-} PrefixExpr \br
    \terminal{!} PrefixExpr \br
    \terminal{'\~} PrefixExpr \br
    \terminal{\&} PrefixExpr \br
    \terminal{\&} \terminal{mut} PrefixExpr \br
    \terminal{*} PrefixExpr \br
    SomeExpr
\end{bnf}

\pnum
前缀运算符$\mathbin{\tcode{+}}$和$\mathbin{\tcode{-}}$分别表示正号和负号。其中$\mathbin{\tcode{+}}$的值为其操作数的值,而$\mathbin{\tcode{-}}$的值为其相反数。操作数类型必须为算术类型。

\pnum
逻辑否运算符\tcode{!}用于对布尔值取反。如果操作数为\tcode{true},则结果为\tcode{false};如果操作数为\tcode{false},则结果为\tcode{true}。

\pnum
位取反运算符\tcode{'\~}进行按位取反。操作数的类型必须是定长整数类型。

\pnum
引用运算符\tcode{\&}获得操作数的引用。\tcode{\& mut}获得操作数的可变引用。

\pnum
解引用运算符\tcode{*}获得一个引用指向的值。

\rSec2[expr.some]{\tcode{some}运算符}
\indextext{运算符!\idxcode{some}}

\begin{bnf}{SomeExpr}
    \terminal{some} PrefixExpr
\end{bnf}

\pnum
\tcode{some}运算符用于将一个值转换为可空值。如果操作数的类型为$T$,则结果的类型为$T\tcode{?}$。

\rSec1[expr.mul]{乘法运算符}
\indextext{运算符!乘法}

\begin{bnf}{MulExpr}
    PrefixExpr \br
    MulExpr \terminal{*} PrefixExpr \br
    MulExpr \terminal{/} PrefixExpr \br
    MulExpr \terminal{\%} PrefixExpr
\end{bnf}

\pnum
运算符\tcode{*}、\tcode{/}和\tcode{\%}分别表示乘法、除法和余数。乘除法只对整数类型进行溢出检查,而不对定长整数类型和浮点类型进行。除零检测对整数类型和定长整数类型都生效。

\rSec1[expr.add]{加法运算符}
\indextext{运算符!加法}

\begin{bnf}{AddExpr}
    MulExpr \br
    AddExpr \terminal{+} MulExpr \br
    AddExpr \terminal{-} MulExpr
\end{bnf}

\pnum
运算符\tcode{+}和\ \tcode{-}\ 分别表示加法和减法。其操作必须为算术类型。加减法只对整数类型进行溢出检查,而不对定长整数类型和浮点类型进行。

\rSec1[expr.shift]{移位运算符}
\indextext{运算符!移位}

\begin{bnf}{ShiftExpr}
    AddExpr \br
    AddExpr \terminal{shl} AddExpr \br
    AddExpr \terminal{shr} AddExpr
\end{bnf}

\pnum
运算符\tcode{shl}和\tcode{shr}表示按位左移和右移。其操作数必须为定长整数类型。在同一个表达式中混合使用\tcode{shl}和\tcode{shr}是一个编译错误。

\rSec1[expr.bit]{位运算符}
\indextext{运算符!位}

\begin{bnf}{BitwiseExpr}
    ShiftExpr \br
    BitwiseExpr \terminal{'\&} ShiftExpr \br
    BitwiseExpr \terminal{'\^{}} ShiftExpr \br
    BitwiseExpr \terminal{'|} ShiftExpr
\end{bnf}

\pnum
运算符\tcode{'\&}、\tcode{'\^{}}和\tcode{'|}分别表示按位与、按位异或和按位或。其操作数必须为定长整数类型。在同一个表达式中混合使用\tcode{'\&}、\tcode{'\^{}}和\tcode{'|}是一个编译错误。

\rSec1[expr.range]{区间运算符}
\indextext{运算符!区间}

\begin{bnf}{RangeExpr}
    NullCoalExpr \br
    NullCoalExpr \terminal{..} NullCoalExpr \br
    NullCoalExpr \terminal{..=} NullCoalExpr
\end{bnf}

\pnum
运算符\tcode{..}生成左闭右开区间,结果类型是\tcode{Range}。
运算符\tcode{..=}生成左闭右闭区间,结果类型是\tcode{ClosedRange}。
参数类型必须是整数类型。参见~\ref{core.range}。

\rSec1[expr.connect]{连接运算符}
\indextext{运算符!连接}

\begin{bnf}{ConnectExpr}
    RangeExpr \br
    ConnectExpr \terminal{\~} RangeExpr
\end{bnf}

\pnum
运算符\tcode{\~}用于连接字符串或集合。其操作数的类型必须满足以下条件之一:

\begin{itemize}
    \item 两个操作数都是\tcode{string};
    \item 一个操作数满足$\tcode{Sequence<}T\tcode{>}$,另一个是$T$;
    \item 两个操作数都满足$\tcode{Sequence<}T\tcode{>}$。
\end{itemize}

对第一种情况,结果等于将两个字符串左右连接得到的结果;对第二种情况,$x\ \tcode{\~}\ y$等于$\tcode{[}x\tcode{,}\ \tcode{...}y\tcode{]}$或$\tcode{[...}x\tcode{,}\ y\tcode{]}$,取决于哪个操作数是序列;对第三种情况,$x\ \tcode{\~}\ y$等于$\tcode{[...}x\tcode{,}\ \tcode{...}y\tcode{]}$。

\rSec1[expr.null-coal]{空值合并运算符}
\indextext{运算符!空值合并}

\begin{bnf}{NullCoalExpr}
    BitwiseExpr \br
    BitwiseExpr \terminal{??} NullCoalExpr
\end{bnf}

\pnum
$a\mathbin{\tcode{??}}b$首先对$a$求值,如果其结果是$\tcode{some }e$,则表达式的值为$e$,且$b$不会被求值;否则表达式的值为$b$。如果$a$的类型为$A\tcode{?}$,$b$的类型为$B$,则表达式的类型为$A$和$B$的公共类型。

\pnum
\tcode{??}是右结合的。

\rSec1[expr.cmp-in]{比较运算符、包含运算符}

\begin{bnf}{BooleanExpr}
    RangeExpr \br
    CompareExpr \br
    IncludeExpr
\end{bnf}

\pnum
本节中的运算符的结果都是\tcode{bool}。

\rSec2[expr.compare]{比较运算符}
\indextext{运算符!比较}

\begin{bnf}{CompareExpr}
    RangeExpr \terminal{!=} RangeExpr \br
    RangeExpr \terminal{cmp} RangeExpr \br
    LessChainExpr \br
    GreaterChainExpr
\end{bnf}

\begin{bnf}{LessChainExpr}
    RangeExpr LessChainOperator RangeExpr \br
    LessChainExpr LessChainOperator RangeExpr
\end{bnf}

\begin{bnf}{LessChainOperator}[\oneof]
    \terminal{< == <=}
\end{bnf}

\begin{bnf}{GreaterChainExpr}
    RangeExpr GreaterChainOperator RangeExpr \br
    GreaterChainExpr GreaterChainOperator RangeExpr
\end{bnf}

\begin{bnf}{GreaterChainOperator}[\oneof]
    \terminal{> == >=}
\end{bnf}

\begin{align*}
a\ \tcode{==}\ b &\iff a\ \tcode{cmp}\ b = \tcode{.equal} \\
a\ \tcode{!=}\ b &\iff a\ \tcode{cmp}\ b \ne \tcode{.equal} \\
a\ \tcode{<}\ b &\iff a\ \tcode{cmp}\ b = \tcode{.less} \\
a\ \tcode{>}\ b &\iff a\ \tcode{cmp}\ b = \tcode{.greater} \\
a\ \tcode{<=}\ b &\iff a\ \tcode{cmp}\ b = \tcode{.less}\ \mathrm{or} \ \tcode{.equal} \\
a\ \tcode{>=}\ b &\iff a\ \tcode{cmp}\ b = \tcode{.greater}\ \mathrm{or} \ \tcode{.equal} \\
\end{align*}

\pnum
$a\ \tcode{cmp}\ b$比较两个表达式,其结果类型为\tcode{Order}。其余比较运算符的结果类型为\tcode{bool}。

\pnum
\tcode{<}、\tcode{<=}和\tcode{==}可以连续使用。$a\ \tcode{<}\ b\ \tcode{<=}\ c$等价于$a\ \tcode{<}\ b\ \tcode{\&}\ b\ \tcode{<=}\ c$。\tcode{>}、\tcode{>=}和\tcode{==}也可以用类似方式混合。以其他方式在一个表达式中使用超过一个比较运算符是一个编译错误。

\rSec2[expr.include]{包含运算符}
\indextext{运算符!包含}

\begin{bnf}{IncludeExpr}
    RangeExpr \terminal{in} RangeExpr \br
    RangeExpr \terminal{!in} RangeExpr
\end{bnf}

\pnum
$a \mathrel{\tcode{in}} b$检测$a$是否在$b$中。$a \mathrel{\tcode{!in}} b$等价于$\tcode{!(}a \mathrel{\tcode{in}} b\tcode{)}$。表达式的类型为\tcode{bool}。

\pnum
\enternote \tcode{!in}中\tcode{!}和\tcode{in}之间必须没有空白,否则会被识别成两个分开的运算符。\exitnote

\rSec1[expr.logic]{逻辑运算符}
\indextext{运算符!逻辑}

\begin{bnf}{LogicExpr}
    BooleanExpr \br
    LogicExpr \terminal{\&} BooleanExpr \br
    LogicExpr \terminal{|} BooleanExpr
\end{bnf}

\pnum
\tcode{\&}和\tcode{|}是逻辑运算符。两者的操作数都必须实现\tcode{Boolean}。它们都使用短路求值。在同一个表达式中混合使用两个运算符是一个编译错误。

\rSec1[expr.assign]{赋值运算符}
\indextext{运算符!赋值}

\begin{bnf}{AssignExpr}
    LogicExpr \br
    SuffixExpr \terminal{=} LogicExpr \br
    SuffixExpr \terminal{+=} LogicExpr \br
    SuffixExpr \terminal{-=} LogicExpr \br
    SuffixExpr \terminal{*=} LogicExpr \br
    SuffixExpr \terminal{/=} LogicExpr \br
    SuffixExpr \terminal{\%=} LogicExpr \br
    SuffixExpr \terminal{shl_eq} LogicExpr \br
    SuffixExpr \terminal{shr_eq} LogicExpr \br
    SuffixExpr \terminal{'\&=} LogicExpr \br
    SuffixExpr \terminal{'\^{}=} LogicExpr \br
    SuffixExpr \terminal{'|=} LogicExpr \br
    SuffixExpr \terminal{??=} LogicExpr \br
    SuffixExpr \terminal{<\~} LogicExpr \br
    LogicExpr \terminal{\~>} SuffixExpr
\end{bnf}

\pnum
赋值表达式的结果类型是\tcode{void}。

\pnum
\tcode{=}将左操作数的值更新为右操作数的值。左操作数必须是\tcode{mut}的,且右操作数必须能隐式转换到左操作数。

\pnum
复合赋值运算符\tcode{+=}、\tcode{-=}、\tcode{*=}、\tcode{/=}、\tcode{\%=}、\tcode{shl_eq}、\tcode{shr_eq}、\tcode{'\&=}、\tcode{'\^{}=}、\tcode{'|=}和\tcode{??=}分别表示加、减、乘、除、取余、左移、右移、按位与、按位异或、按位或和空值合并赋值。
对这些运算符而言,$a\ op\tcode{=}\ b$或$a\ op\tcode{_eq}\ b$等价于$a\ \tcode{=}\ a\ op\ b$,但$a$只被求值一次。

\pnum
追加运算符$e\ \tcode{<\~}\ v$等价于$e\ \tcode{=}\ e\ \tcode{\~}\ v$,$v\ \tcode{\~>}\ e$等价于$e\ \tcode{=}\ v\ \tcode{\~}\ e$,但$e$只被求值一次。

\rSec1[expr.semi]{分号运算符}
\indextext{运算符!分号}

\begin{bnf}{Expression}
    AssignExpr \br
    AssignExpr \terminal{;} Expression\br
    Binding \terminal{;} Expression
\end{bnf}

\pnum
分号表达式中,分号左侧可以为一个表达式或绑定。如果分号左侧为绑定,则该绑定会被插入到当前作用域中。如果左侧为表达式,则该表达式将被求值且结果会被丢弃。在那之后,将对右侧表达式进行求值并将其值作为整个表达式的值。