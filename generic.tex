%!TEX root = x.tex

\rSec0[generic]{泛型}
\indextext{泛型}

\begin{bnf}{GenericSpecification}
    \terminal{<} GenericParameters \terminal{>}
\end{bnf}

\begin{bnf}{GenericParameters}
    GenericParameter \br
    GenericParameters \terminal{,} GenericParameter
\end{bnf}

\begin{bnf}{GenericParameter}
    \terminal{dyn}\bnfq Identifier \terminal{...}\bnfq GenericConstraint\bnfq GenericIfConstraint\bnfq
\end{bnf}

\begin{bnf}{GenericConstraint}
    \terminal{:} \terminal{type} \br
    \terminal{:} Type \br
    GenericTraitConstraint
\end{bnf}

\begin{bnf}{GenericTraitConstraint}
    \terminal{is} Type
\end{bnf}

\begin{bnf}{GenericIfConstraint}
    \terminal{if} Expression
\end{bnf}

\begin{bnf}{GenericArguments}
    GenericArgument \br
    GenericArguments \terminal{,} GenericArgument
\end{bnf}

\begin{bnf}{GenericArgument}
    Type \br
    \terminal{dyn}\bnfq Expression \br
    \terminal{dyn} \terminal{_}
\end{bnf}

\pnum
\X 中的实体可以带有编译器的类型或非类型参数进行泛化。

\rSec1[kind]{高阶类型}
\indextext{高阶类型}

\rSec1[generic.dynamic]{动态泛型}
\indextext{泛型!动态}

\pnum
非类型的泛型参数可以使用关键字\tcode{dyn}修饰。以这种方式修饰的参数在运行时可变,但可以对其使用静态分析。其对应的泛型实参也必须使用\tcode{dyn}修饰。

\enterexample
\begin{codeblock}

type<dyn I: int> T = { };

let a: T<dyn 0> = { };
let b: T<dyn 1> = a; // 错误,\tcode{a}和\tcode{b}的类型不匹配

func externFunc() -> int;

let n: int = externFunc();

let c: T<dyn n> = a; // 正确:\tcode{n}是运行时值,编译时值\tcode{0}与其兼容

\end{codeblock}
\exitexample

\pnum
动态参数可以特定方式进行约束。

\enterexample
\begin{codeblock}

type<dyn I: int> T = { };

func<dyn I: int> f(a: T<dyn I>, b: T<dyn I>) => a;

func externFunc() -> T<dyn _>; // 未知的动态类型

let a = externFunc();
let b = externFunc();

f(a, b); // 错误,无法保证\tcode{a}和\tcode{b}的动态类型参数相同

func<dyn I: int> extract(a: T<dyn I>) => I;

if extract(a) == extract(b) {
    f(a, b); // 正确,能发现\tcode{a}和\tcode{b}的动态类型参数相同
}

\end{codeblock}
\exitexample