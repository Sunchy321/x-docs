%!TEX root = x.tex

\rSec0[generic]{泛型}
\indextext{泛型}

\begin{bnf}{GenericSpecification}
    \terminal{<} GenericParameters \terminal{>}
\end{bnf}

\begin{bnf}{GenericParameters}
    GenericParameter \br
    GenericParameters \terminal{,} GenericParameter
\end{bnf}

\begin{bnf}{GenericParameter}
    Identifier \terminal{...}\bnfq GenericConstraint\bnfq GenericIfConstraint\bnfq
\end{bnf}

\begin{bnf}{GenericConstraint}
    \terminal{:} \terminal{type} \br
    \terminal{:} Type \br
    GenericTraitConstraint
\end{bnf}

\begin{bnf}{GenericTraitConstraint}
    \terminal{impl} Type
\end{bnf}

\begin{bnf}{GenericIfConstraint}
    \terminal{if} Expression
\end{bnf}

\begin{bnf}{GenericArguments}
    GenericArgument \br
    GenericArguments \terminal{,} GenericArgument
\end{bnf}

\begin{bnf}{GenericArgument}
    Type \br
    Expression
\end{bnf}

\pnum
\X 中的实体可以带有编译器的类型或非类型参数进行泛化。

\rSec1[kind]{高阶类型}
\indextext{高阶类型}

\rSec1[generic.some]{\tcode{some}泛型}
\indextext{泛型!\idxcode{some}}

\pnum
如果\tcode{some}被用于一个函数的参数类型中,则其相当于一个匿名的泛型参数。

\enterexample
\begin{codeblock}

func f(a: some A) { }

// 等价于
func<T impl A> f(a: T) { }

\end{codeblock}
\exitexample
