%!TEX root = x.tex

\rSec0[func]{函数}
\indextext{函数}

\begin{bnf}{FuncDecl}
    FuncQual\bnfs \terminal{func} FuncName Parameter ThrowQual\bnfq FuncBody \br
    FuncQual\bnfs \terminal{func} FuncName Parameter ThrowQual\bnfq ReturnType\bnfq \terminal{;}
\end{bnf}

\begin{bnf}{FuncName}
    UnqualID
\end{bnf}

\begin{bnf}{FuncQual}
    \terminal{async} \br
    \terminal{const} \br
    \terminal{extern} \br
    \terminal{unsafe}
\end{bnf}

\begin{bnf}{FuncBody}
    ReturnType\bnfq Block \br
    \terminal{=>} Expression \terminal{;}
\end{bnf}

\begin{bnf}{ReturnType}
    \terminal{->} Type
\end{bnf}

\pnum
\term{函数}声明将函数、方法或闭包引入当前作用域。函数声明由关键字\tcode{func}标记,后跟函数名、参数列表、异常修饰符、返回类型和函数体。

\pnum
在同一个作用域中可以定义两个名称相同的函数,但是这些函数必须具有不同名称的命名参数(忽略顺序)。
如果两个函数在省略某些具有默认值的命名参数之后,剩余的命名参数名称完全相同(忽略顺序),则这也是一个编译错误。

\pnum
函数的返回类型可以省略,此时函数的返回类型为函数体的类型与函数中\tcode{return}语句参数类型的公共类型。
如果函数包含异常修饰符,则返回类型还会继续受到异常修饰符的影响变为错误类型。参见~\ref{except.func.throw}。

\pnum
函数体为单个块或者\tcode{=>}后跟任意表达式和分号。如果采用后一种形式,则不能显式指定返回类型。

\pnum
实现中的函数可以为\term{方法}。参见~\ref{method}。

\pnum
函数中可以有另一个函数声明。此时如果内部函数使用了外部函数作用域中的值,该函数为\term{闭包},否则为普通函数。

\rSec1[param]{函数参数}

\begin{bnf}{Parameter}
    \terminal{(} ParamList\bnfq \terminal{)}
\end{bnf}

\begin{bnf}{ParamList}
    Params \terminal{,}\bnfq
\end{bnf}

\begin{bnf}{Params}
    ThisParamDecl \br
    ThisParamDecl \terminal{,} NamedParamList \br
    ThisParamDecl \terminal{,} UnnamedParamList \br
    ThisParamDecl \terminal{,} UnnamedParamList \terminal{,} NamedParamList \br
    UnnamedParamList \br
    UnnamedParamList \terminal{,} NamedParamList \br
    NamedParamList
\end{bnf}

\begin{bnf}{UnnamedParamList}
    UnnamedParamDecl \br
    UnnamedParamList \terminal{,} UnnamedParamDecl
\end{bnf}

\begin{bnf}{NamedParamList}
    NamedParamDecl \br
    NamedParamList \terminal{,} NamedParamDecl
\end{bnf}

\begin{bnf}{UnnamedParamDecl}
    Attribute\bnfs ParamQual\bnfq Pattern TypeNotation
\end{bnf}

\begin{bnf}{NamedParamDecl}
    Attribute\bnfs ParamQual\bnfs \terminal{(} Identifier \terminal{)} Pattern TypeNotation DefaultValue\bnfq
\end{bnf}

\begin{bnf}{ThisParamDecl}
    Attribute\bnfs \terminal{mut}\bnfq \terminal{this} \br
    Attribute\bnfs \terminal{\&} \terminal{mut}\bnfq \terminal{this} \br
    Attribute\bnfs \terminal{this} TypeNotation
\end{bnf}

\begin{bnf}{ParamQual}
    \terminal{mut} \br
    \terminal{ref} \br
    \terminal{lazy} \br
    \terminal{borrow}
\end{bnf}

\begin{bnf}{TypeNotation}
    \terminal{:} Type
\end{bnf}

\begin{bnf}{DefaultValue}
    \terminal{=} Expression
\end{bnf}

\pnum
\term{参数}为调用函数时传入其内的值。
如果函数不接受任何参数,则必须使用\tcode{()}标识而不能省略括号。
函数参数默认是不可变的,除非由\tcode{mut}修饰。

\pnum
参数可以为\tcode{this}参数、顺序参数或具名参数,且必须按照此顺序排列,并且\tcode{this}参数不能出现超过一次。
顺序参数和具名参数必须显式指定其类型,不能省略类型。
具名参数可以指定默认值。如果函数调用时没有提供对应的值,则视为传入该默认值。

\pnum
函数参数可以是一个模式,但视为处于绑定模式中,其中所有的标识符都视为一个绑定。这个模式不能是可失败的。

\enterexample
\begin{codeblock}

func foo((a, b): (int, bool)) {
    // \tcode{a} is \tcode{int}, \tcode{b} is \tcode{bool}
}

\end{codeblock}
\exitexample

\rSec1[param.qual]{参数修饰符}
\indextext{参数修饰符}

\pnum
参数修饰符修饰函数参数,指定参数的某些属性。

\rSec2[param.qual.mut]{\tcode{mut}}
\indextext{参数修饰符!\idxcode{mut}}

\pnum
\tcode{mut}修饰的参数是可变的,可以在函数内部被修改或重新赋值。\enternote 由于参数总是移动传递,所以除非使用\tcode{ref}修饰,否则调用处实参的值不会受影响。\exitnote

\rSec2[param.qual.ref]{\tcode{ref}}
\indextext{参数修饰符!\idxcode{ref}}

\pnum
\tcode{ref}修饰的参数以引用方式传递,调用处实参的值会受影响。除非使用\tcode{mut}修饰,否则参数是只读的。

\pnum
类型为$T$的\tcode{ref}参数实际上是类型为\tcode{$T$\&}(或\tcode{$T$ mut\&})的引用参数的语法糖。
在函数内部使用该参数时,将会自动对参数进行解引用。

\pnum
\tcode{ref}参数将会按如下方式脱糖:

\begin{codeblock}
func foo(ref a: int) {
    a = 1;
}

let mut a = 0;

foo(ref a);
\end{codeblock}
变为
\begin{codeblock}
func foo(\{$r_a$}: int mut&) {
    *\{$r_a$} = 1;
}

let mut a = 0;

foo(&mut a);
\end{codeblock}

\rSec2[param.qual.lazy]{\tcode{lazy}}
\indextext{参数修饰符!\idxcode{lazy}}

\pnum
\tcode{lazy}修饰的参数是惰性求值的,只有在函数内部使用时才会计算实参的值。
\tcode{lazy}不能与\tcode{mut}修饰符一起使用。

\pnum
类型为$T$的\tcode{lazy}参数实际上是类型为\tcode{() once -> $T$}的闭包参数的语法糖。
在函数内部首次读取该参数时,将会调用该闭包并将结果缓存,之后再次读取时直接使用缓存的值。

\pnum
\tcode{lazy}参数将会按如下方式脱糖:

\begin{codeblock}
func foo(lazy a: int) {
    bar(a);
}

let a = 0;

foo(a);
\end{codeblock}
变为
\begin{codeblock}
func foo(\{$c_a$}: () once -> int) {
    let \{$s_a$}: int? = nil;
    let \{$g_a$} = do {
        if let some \{$v_a$} = \{$s_a$} {
            \{$v_a$}
        } else {
            let \{$v_a$} = \{$c_a$}();
            \{$s_a$} = some \{$v_a$};
            \{$v_a$}
        }
    };

    bar(\{$g_a$}());
}

let a = 0;

foo(do { a });
\end{codeblock}

\rSec2[param.qual.borrow]{\tcode{borrow}}
\indextext{参数修饰符!\idxcode{borrow}}

\pnum
\tcode{borrow}修饰的参数为借用参数,它会将实参移入到函数中,然后在函数返回时将其返回并覆盖原实参。
\tcode{borrow}不能与\tcode{lazy}和\tcode{ref}修饰符一起使用。

\pnum
\tcode{borrow}参数将会按如下方式脱糖:

\begin{codeblock}
func foo(borrow a: int) {
    let a = 1;
}

let a = 0;

foo(borrow a);
\end{codeblock}
变为
\begin{codeblock}
func foo(a: int) -> (void, int) {
    let a = 1;
    ((), a)
}

let a = 0;

let (_, a) = foo(a);
\end{codeblock}

\rSec1[func.async]{异步函数}

\pnum
使用\tcode{async}修饰的函数称为异步函数。