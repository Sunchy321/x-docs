%!TEX root = x.tex

\rSec0[func]{函数}
\indextext{函数}

\pnum
函数调用按照以下流程进行:

\begin{itemize}
    \item 将形参的值按照传参方式设置为对应的值。
    \item 对函数最外层的块进行求值。
    \begin{itemize}
        \item 如果这个求值异常结束(因此,这个函数具有 \tcode{throws} 修饰符),将这个异常传递到调用处;
        \item 否则:
        \begin{itemize}
            \item 如果函数的返回类型是 \tcode{void},这个值被丢弃;
            \item 如果函数的返回类型是 \tcode{never},这是一个编译错误 \enternote 只要有一条控制流可能到达结尾就会导致编译错误。\exitnote;
            \item 否则,将这个值隐式转换到返回类型。(如果不能的话,这是一个编译错误)
        \end{itemize}
    \end{itemize}
\end{itemize}

\rSec1[func.inout]{函数前置/后置条件}
