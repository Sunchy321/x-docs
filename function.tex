%!TEX root = x.tex

\rSec0[func]{函数}
\indextext{函数}

\begin{bnf}{FuncDecl}
    FuncQual\bnfs \terminal{func} FuncName Parameter ThrowQual\bnfq ReturnType\bnfq Block \br
    FuncQual\bnfs \terminal{func} FuncName Parameter ThrowQual\bnfq \terminal{=>} Expression \terminal{;}
\end{bnf}

\begin{bnf}{FuncName}
    UnqualID \br
    \terminal{init} \br
    \terminal{deinit}
\end{bnf}

\begin{bnf}{FuncQual}
    \terminal{async} \br
    \terminal{const} \br
    \terminal{extern} \br
    \terminal{unsafe}
\end{bnf}

\begin{bnf}{ReturnType}
    \terminal{->} Type
\end{bnf}

\pnum
\term{函数}声明将函数、方法或闭包引入当前作用域。函数声明由关键字\tcode{func}标记,后跟函数名、参数列表、异常修饰符、返回类型和函数体。

\pnum
函数的返回类型可以省略,此时函数的返回类型为函数体的类型与函数中\tcode{return}语句参数类型的公共类型。
如果函数包含异常修饰符,则返回类型还会继续受到异常修饰符的影响变为错误类型。参见~\ref{except.func.throw}。

\pnum
函数体为单个块或者\tcode{=>}后跟任意表达式和分号。如果采用后一种形式,则不能显式指定返回类型。

\pnum
实现中的函数可以为\term{方法}。参见~\ref{method}。

\pnum
函数中可以有另一个函数声明。此时如果内部函数使用了外部函数作用域中的值,该函数为\term{闭包},否则为普通函数。

\rSec1[param]{函数参数}

\begin{bnf}{Parameter}
    \terminal{(} ParamList\bnfq \terminal{)}
\end{bnf}

\begin{bnf}{ParamList}
    ThisParamDecl \br
    ThisParamDecl \terminal{,} NamedParamList \br
    ThisParamDecl \terminal{,} UnnamedParamList \br
    ThisParamDecl \terminal{,} UnnamedParamList \terminal{,} NamedParamList \br
    UnnamedParamList \br
    UnnamedParamList \terminal{,} NamedParamList \br
    NamedParamList
\end{bnf}

\begin{bnf}{UnnamedParamList}
    UnnamedParamDecl \br
    UnnamedParamList \terminal{,} UnnamedParamDecl
\end{bnf}

\begin{bnf}{NamedParamList}
    NamedParamDecl \br
    NamedParamList \terminal{,} NamedParamDecl
\end{bnf}

\begin{bnf}{UnnamedParamDecl}
    Attribute\bnfs ParamQual\bnfq Pattern TypeNotation\bnfq
\end{bnf}

\begin{bnf}{NamedParamDecl}
    Attribute\bnfs ParamQual\bnfq \terminal{(} Identifier \terminal{)} Pattern TypeNotation\bnfq DefaultValue\bnfq
\end{bnf}

\begin{bnf}{ThisParamDecl}
    Attribute\bnfs \terminal{mut}\bnfq \terminal{this} \br
    Attribute\bnfs \terminal{\&} \terminal{mut}\bnfq \terminal{this} \br
    Attribute\bnfs \terminal{this} TypeNotation
\end{bnf}

\begin{bnf}{ParamQual}
    \terminal{mut} \br
    \terminal{ref} \br
    \terminal{ref} \terminal{mut} \br
    \terminal{lazy}
\end{bnf}

\begin{bnf}{TypeNotation}
    \terminal{:} Type
\end{bnf}

\begin{bnf}{DefaultValue}
    \terminal{=} Expression
\end{bnf}

\pnum
\term{参数}为调用函数时传入其内的值。
如果函数不接受任何参数,则必须使用\tcode{()}标识而不能省略括号。
函数参数默认是不可变的,除非由\tcode{mut}修饰。

\pnum
参数可以为\tcode{this}参数、顺序参数或具名参数,且必须按照此顺序排列,并且\tcode{this}参数不能出现超过一次。
顺序参数和具名参数必须显式指定其类型,不能省略类型。
具名参数可以指定默认值。如果函数调用时没有提供对应的值,则视为传入该默认值。

\pnum
函数参数可以是一个模式,但视为处于绑定模式中,其中所有的标识符都视为一个绑定。这个模式不能是可失败的。

\enterexample
\begin{codeblock}

func foo((a, b): (int, bool)) {
    // \tcode{a} is \tcode{int}, \tcode{b} is \tcode{bool}
}

\end{codeblock}
\exitexample

\rSec1[param.qual]{参数修饰符}
\indextext{参数修饰符}

\pnum
参数修饰符修饰函数参数,指定参数的某些属性。

\rSec2[param.qual.mut]{\tcode{mut}}
\indextext{参数修饰符!\idxcode{mut}}

\pnum
\tcode{mut}修饰的参数是可变的,可以在函数内部被修改或重新赋值。\enternote 由于参数总是移动传递,所以除非使用\tcode{ref}修饰,否则调用处实参的值不会受影响。\exitnote

\rSec2[param.qual.ref]{\tcode{ref}}
\indextext{参数修饰符!\idxcode{ref}}

\pnum
\tcode{ref}修饰的参数以引用方式传递,调用处实参的值会受影响。除非使用\tcode{mut}修饰,否则参数是只读的。

\pnum
\tcode{this}参数总是以引用方式传递,即使没有显式指定。

\rSec2[param.qual.lazy]{\tcode{lazy}}
\indextext{参数修饰符!\idxcode{lazy}}

\pnum
\tcode{lazy}修饰的参数是惰性求值的,只有在函数内部使用时才会计算实参的值。

\rSec1[func.async]{异步函数}

\pnum
