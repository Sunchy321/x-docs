%!TEX root = x.tex

\rSec0[macro]{宏}
\indextext{宏}

\begin{bnf}{MacroInvocation}
    \terminal{\#} MacroName TokenDelimited\bnfq
\end{bnf}

\begin{bnf}{MacroName}
    Identifier
\end{bnf}

\pnum
\term{宏}提供了一种在编译期生成代码的能力。宏可以接受参数,并展开成为特定的标记序列。宏的参数由一对对应的分隔符包围,且内部的分隔符也必须成对。

\pnum
宏名称是一个标识符,并且可以与关键字相同。

\rSec1[macro.def]{宏定义}

\begin{bnf}{MacroDefinition}
    \terminal{macro} MacroName \terminal{\{} MacroRule\bnfs \terminal{\}}
\end{bnf}

\begin{bnf}{MacroRule}
    TokenDelimited \terminal{->} TokenDelimited
\end{bnf}

\rSec1[macro.builtin]{内建宏}
\indextext{内建宏}

\pnum
一些宏内建于语言之中,它们提供了无法通过用户实现的功能。

\rSec2[macro.src-pos]{源代码位置}
\indextext{内建宏!源代码位置}

\pnum
\tcode{\#file}扩展为标识当前文件的路径的常量字符串。\tcode{\#line}扩展为当前行号。\tcode{\#column}扩展为当前列号。如果这些宏由一个宏展开调用,则使用的是最外层宏展开的对应位置。

\rSec2[macro.nameof]{\tcode{nameof}}
\indextext{内建宏!\idxcode{nameof}}

\pnum
$\tcode{\#nameof(}id\tcode{)}$扩展为值为$id$的常量字符串。