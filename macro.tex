%!TEX root = x.tex

\rSec0[macro]{宏}
\indextext{宏}

\begin{bnf}{MacroInvocation}
    \terminal{\#} MacroName TokenDelimited\bnfq
\end{bnf}

\begin{bnf}{MacroName}
    Identifier
\end{bnf}

\pnum
\term{宏}提供了一种在编译期生成代码的能力。宏可以接受参数,并展开成为特定的标记序列。宏的参数由一对对应的分隔符包围,且内部的分隔符也必须成对。

\pnum
宏名称是一个标识符,并且可以与关键字相同。

\rSec1[macro.def]{宏定义}

\begin{bnf}{MacroDefinition}
    \terminal{macro} MacroName \terminal{\{} MacroRule\bnfs \terminal{\}}
\end{bnf}

\begin{bnf}{MacroRule}
    MacroPattern \terminal{->} TokenDelimited
\end{bnf}

\begin{bnf}{MacroPattern}
    \terminal{(} MacroItem\bnfs \terminal{)} \br
    \terminal{[} MacroItem\bnfs \terminal{]} \br
    \terminal{\{} MacroItem\bnfs \terminal{\}}
\end{bnf}

\begin{bnf}{MacroItem}
    MacroToken \br
    MacroPattern \br
    \terminal{\#} Identifier \terminal{:} MacroItemType \br
    \terminal{\#} \terminal{(} MacroItem\bnfp \terminal{)} MacroSep\bnfs MacroRepeat
\end{bnf}

\begin{bnf}{MacroItemType}{\oneof}
    \terminal{id expr block stmt decl pat type path qual attr tokenTree }
\end{bnf}

\begin{bnf}{MacroToken}
    Token \textnormal{但不是} \terminal{( ) [ ] \{ \}}
\end{bnf}

\begin{bnf}{MacroSep}
    Token \textnormal{但不是} \terminal{( ) [ ] \{ \} + * ?}
\end{bnf}

\begin{bnf}{MacroToken}{\oneof}
    \terminal{+ * ?}
\end{bnf}

\begin{bnf}{MacroReplacer}
    \terminal{\{} MacroReplacerItem\bnfs \terminal{\}}
\end{bnf}

\begin{bnf}{MacroReplacerItem}
    MacroToken \br
    TokenDelimited \br
    MacroInvocation \br
    \terminal{\#} \terminal{(} MacroReplacerItem\bnfp \terminal{)} MacroSep\bnfs MacroRepeat
\end{bnf}

\pnum
宏定义通过称为\term{声明宏}的方式创建宏。宏定义在关键字后跟\tcode{macro}和宏名称,然后是一对大括号,其中包含宏规则。
宏规则由宏模式和宏转换两部分组成。宏模式和宏转换是由一对定界符包括的标记序列。

\pnum
宏模式定义了宏替换的模式。宏模式由一系列项组成。每个项可以是标记、定界符包含的宏模式、宏匹配项或者宏重复项。
标记匹配完全相同的单个标记,但不能指定匹配定界符。
定界符包含的宏模式匹配成对的定界符以及内部的模式。最外层的宏模式也按照相同方式匹配。
宏匹配项匹配一个特定的程序项。
宏重复项匹配括号内部的程序项,但是可以以指定的分隔符分隔,并可以指定为重复任意次、重复至少一次或者重复零次或一次。

\pnum
宏匹配项指定的特定程序项的含义如表~\ref{tab:macro-item-type}~所示。

\begin{floattable}{整数字面量后缀}{tab:macro-item-type}{c|c}
    \topline
    类型 & 匹配的程序项 \\
    \capsep
    \tcode{id} & 单个标识符 \\
    \tcode{expr} & 表达式 \\
    \tcode{block} & 块 \\
    \tcode{stmt} & 语句 \\
    \tcode{decl} & 声明 \\
    \tcode{pat} & 模式 \\
    \tcode{type} & 类型 \\
    \tcode{path} & 路径 \\
    \tcode{qual} & 单个修饰符 \\
    \tcode{attr} & 属性 \\
    \tcode{tokenTree} & 成对定界符及其内部的标记,或单个非定界符标记 \\
\end{floattable}

\pnum
当宏展开时,首先将参数中的所有宏递归展开(在展开点)。然后对宏定义中每条规则的模式依次进行匹配。如果某一项的模式匹配成功,则按照对应的宏转换将输入的标记序列转换为结果标记序列。如果所有规则都无法匹配,则这是一个编译错误。

\pnum
宏转换指定了宏展开的结果。其中的普通标记项及定界符将按原样输出。宏展开项如果其名称与模式中的匹配项相同,则展开为特定的展开项;否则按照一般的宏展开规则递归展开(在定义点)。
宏转换中的重复项表示重复展开。其中被匹配了多次的匹配项也必须在转换中被使用了等量次数,否则这是一个编译错误。
产生的标记序列(不包括宏展开最外层的定界符)被插入到宏展开点。

\rSec1[macro.builtin]{内建宏}
\indextext{内建宏}

\pnum
一些宏内建于语言之中,它们提供了无法通过用户实现的功能。

\rSec2[macro.src-pos]{源代码位置}
\indextext{内建宏!源代码位置}

\pnum
\tcode{\#file}扩展为标识当前文件的路径的常量字符串。\tcode{\#line}扩展为当前行号。\tcode{\#column}扩展为当前列号。如果这些宏由一个宏展开调用,则使用的是最外层宏展开的对应位置。

\rSec2[macro.nameof]{\tcode{nameof}}
\indextext{内建宏!\idxcode{nameof}}

\pnum
$\tcode{\#nameof(}id\tcode{)}$扩展为值为$id$的常量字符串。