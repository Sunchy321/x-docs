%!TEX root = x.tex

\rSec0[basic]{基本概念}

\pnum
\term{实体}包括对象、函数、类型、模块、运算符、扩展。

\rSec1[scope]{作用域}
\indextext{作用域}

\pnum
\term{作用域}是一段程序文本。特定的语言功能可能只能在特定的作用域中生效。不同的作用域具有不同的类型,分别被不同的语言功能所引用。作用域可以互相包含。

\pnum
\term{全局作用域}是整个程序文本代表的作用域,包含所有其他作用域。

\rSec2[scope.decl]{声明作用域}
\indextext{作用域!声明}

\pnum
声明作用域限制声明的范围。一个声明或绑定将会被插入到最近的声明作用域中,并且在该作用域内可以使用该名称引用被声明的实体。
在离开该作用域之后,被声明的实体将不能被使用该方式引用。

\pnum
所有的语句都具有声明作用域。全局作用域也是声明作用域。

\rSec2[scope.func]{函数作用域}
\indextext{作用域!函数}

\pnum
函数作用域限制\tcode{return}语句的使用。参见~\ref{stmt.control}。

\pnum
函数定义的块、lambda表达式的块或表达式、\tcode{do}表达式的块和函数调用表达式的lambda块具有函数作用域。

\rSec2[scope.lambda]{Lambda 作用域}
\indextext{作用域!lambda}

\pnum
Lambda作用域限制lambda参数的使用。参见~\ref{expr.lambda.param}。

\pnum
lambda表达式的块或表达式以及函数调用表达式的lambda块具有lambda作用域。

\rSec2[scope.sequence]{序列作用域}
\indextext{作用域!序列}

\pnum
序列作用域限制\tcode{\$}的使用,参见~\ref{expr.dollar}。序列作用域有一个关联的当前序列(下文记作$s$),无论该$s$是否实现\tcode{core.Sequence}。

\pnum
下标运算符$s\tcode{[}\ldots\tcode{]}$的两个方括号之间具有序列作用域。其当前序列为$s$。

\pnum
函数调用表达式$s\tcode{(}\ldots\tcode{)}$的括号之间具有序列作用域。如果$s$形如$o\tcode{.}f$,且$f$是一个方法,则其当前序列为$o$,否则当前序列为$s$。\enternote 此处只能进行一次拆分,即$a\tcode{.}b\tcode{.}c$的当前序列不可能为$a$。\exitnote

\pnum
如果函数调用表达式带有一个lambda块,则这整个块也具有序列作用域,其当前序列确定方法同上。

\enterexample

\begin{codeblock}
let a = [1, 2, 3, 4, 5];
let o = { a };
func operator()(this: int[]) -> int[] { [] }
func v(this: int[], index: int) -> int[] { [] }

a[$ - 1] // 当前序列为a
a($ - 1) // 当前序列为a
o.a[$ - 1] // 成员访问,当前序列为o.a
a.v($ - 1) // 方法调用,当前序列为a

\end{codeblock}

\exitexample

\rSec1[name]{名称}
\indextext{名称}

\begin{bnf}{UnqualID}
    Identifier \br
    SymbolLiteral \br
    \terminal{init} \br
    \terminal{deinit} \br
    \terminal{operator} Operator \br
    \terminal{operator} StringLiteral \br
    \terminal{operator} \terminal{[]} \br
    \terminal{operator} \terminal{as}
\end{bnf}

\pnum
\term{名称}用于引用程序实体。一个名称可能是一个标识符、\tcode{init}、\tcode{deinit}或一个运算符名称。

\rSec2[name.lookup]{名称查找}
\indextext{名称查找}

\pnum
\term{名称查找}用于解析一个 \grammarterm{IDExpr} 具体指代的实体。
