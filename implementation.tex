%!TEX root = x.tex

\rSec0[impl]{实现}
\indextext{实现}

\begin{bnf}{ImplDecl}
    \terminal{impl} Type TraitSpec\bnfq ImplBody
\end{bnf}

\begin{bnf}{TraitSpec}
    \terminal{:} Type
\end{bnf}

\begin{bnf}{ImplBody}
    \terminal{\{} ImplMember\bnfs \terminal{\}} \br
    \terminal{;}
    \terminal{=} \terminal{never} \terminal{;}
\end{bnf}

\begin{bnf}{ImplMember}
    TypeDecl \br
    FuncDecl \br
    PropertyDecl
\end{bnf}

\pnum
\term{实现}用于为类型添加额外的特性。实现可以显式指定以令类型满足某个概念。

% \rSec1[impl.member]{字段}
% \indextext{类!字段}

% \begin{bnf}{FieldDecl}
%     FieldQual\bnfs BindKeyword Identifier TypeNotation Initializer\bnfq \terminal{;}
%     FieldQual\bnfs BindKeyword Identifier Initializer \terminal{;}
% \end{bnf}

% \begin{bnf}{TypeNotation}
%     \terminal{:} Type
% \end{bnf}

% \begin{bnf}{Initializer}
%     \terminal{=} Expression
% \end{bnf}

% \pnum
% 类中的字段表示类的内部状态,其默认访问级别为\tcode{private}。具有\tcode{mut}修饰的是可变字段。类字段可以显式指定类型,也可以通过初始值推导类型。

\rSec1[impl.props]{属性}
\indextext{实现!属性}

\begin{bnf}{PropertyDecl}
    PropertyQual\bnfs BindKeyword Identifier TypeNotation\bnfq Initializer\bnfq PropertyBody \terminal{;} \br
    PropertyQual\bnfs BindKeyword Identifier TypeNotation\bnfq \terminal{=>} Expression \terminal{;}
\end{bnf}

\begin{bnf}{PropertyQual}
    AccessQual
\end{bnf}

\begin{bnf}{PropertyBody}
    \terminal{\{} PropertyMember\bnfp \terminal{\}}
\end{bnf}

\begin{bnf}{PropertyMember}
    PropertyQual\bnfs PropertyKeyword PropertyBlockParam\bnfq Block \br
    PropertyQual\bnfs PropertyKeyword PropertyExprParam\bnfq \terminal{=>} Expression \terminal{;} \br
    PropertyQual\bnfs PropertyKeyword \terminal{;}
\end{bnf}

\begin{bnf}{PropertyKeyword}[\oneof]
    \terminal{get set willSet didSet}
\end{bnf}

\begin{bnf}{PropertyBlockParam}
    Identifier \br
    Identifier \terminal{,} Identifier
\end{bnf}

\begin{bnf}{PropertyExprParam}
    Identifier \br
    \terminal{(} Identifier \terminal{,} Identifier \terminal{)}
\end{bnf}

\pnum
实现可以给类型添加\term{属性}。属性的定义至少需要包含一个访问器。访问器的块或表达式具有lambda作用域,且隐含\tcode{this}参数。

\pnum
属性的访问器可以以上下文关键字\tcode{get}、\tcode{set}、\tcode{willSet}或\tcode{didSet}开始。
\tcode{get}访问器不接受任何参数。
\tcode{set}访问器接受一个参数,其类型为该属性的类型。
\tcode{willSet}和\tcode{didSet}访问器可以接受一个或两个参数,其类型为属性的类型。

\pnum
如果\tcode{set}访问器不显式写出参数,则视为其具有lambda参数\tcode{\$value}。如果\tcode{willSet}或\tcode{didSet}不显式写出参数,则视为其具有lambda参数\tcode{\$oldValue}和\tcode{\$newValue}。

\pnum
如果属性以直接后跟$\mathrel{\tcode{=>}}e$的方式声明,则视作该属性具有访问器$\tcode{get} \mathrel{\tcode{=>}} e$。

\pnum
未使用\tcode{mut}声明的属性不能包含\tcode{set}、\tcode{willSet}和\tcode{didSet}访问器。
在考虑自动生成的访问器之后,如果一个属性缺少\tcode{get}访问器,或一个\tcode{mut}属性缺少\tcode{set}访问器,则这是一个编译错误。

\pnum
属性可以具有类型提示或初始化器。如果属性未按前文所述具有对应的字段,则不能具有初始化器。如果属性省略类型提示,则其类型将从\tcode{get}访问器中推导。如果它的\tcode{get}访问器是自动生成的,这是一个编译错误。

\pnum
当读取属性$p$时,会调用\tcode{get}访问器并将其返回值作为$p$的新值。

\pnum
当给属性$p$赋值时,首先将该值隐式转换到属性的类型,令结果为$v$:

\begin{itemize}
    \item 如果属性具有\tcode{willSet}访问器:
    \begin{itemize}
        \item 如果它接受一个参数,将以$v$调用;
        \item 如果它接受两个参数,将以$p$(旧值)和$v$调用。
    \end{itemize}
    \item 以$v$调用属性的\tcode{set}访问器。
    \item 如果属性具有\tcode{didSet}访问器:
    \begin{itemize}
        \item 如果它接受一个参数,将以$p$(旧值)调用;
        \item 如果它接受两个参数,将以$p$和$v$调用。
        \item 如果函数体内没有使用新值,则不会调用\tcode{get}访问器并使用第一种形式调用。
    \end{itemize}
\end{itemize}

\rSec1[method]{方法}
\indextext{方法}

\pnum
实现中可以包含函数声明,其中含有\tcode{this}参数的称为方法。\tcode{this}参数指代调用该方法的对象。

\pnum
方法可以显式指定\tcode{this}参数以为其添加修饰符。\tcode{this}参数要么省略类型指示,要么包含一个仅具有修饰符的类型指示,这些修饰符就是\tcode{this}的类型。

\enterexample
\begin{codeblock}
impl A {
    f(this) { } // \#1
    f(&mut this) { } // \#2,可以修改\tcode{A}的成员
}

\end{codeblock}
\exitexample

\rSec1[impl.init]{构造器}
\indextext{实现!构造器}

\pnum
构造器是名称为关键字\tcode{init}的函数。构造器可以接受任意类型的参数并且返回一个\tcode{self}类型的值,或者返回一个\tcode{self}类型的值的同时抛出一个异常。构造器不能是方法。

\pnum
具有泛型参数的构造器可以指定与\tcode{self}不同的返回类型,这称为\term{参与推导的构造器}。

\rSec1[impl.deinit]{析构器}
\indextext{实现!析构器}

\pnum
析构器是名称为关键字\tcode{deinit}的方法。析构器不接受任何非\tcode{this}参数且不返回值。析构器不能抛出异常。只能为不透明类型的实现提供析构器。

\pnum
当一个对象被销毁时,将会调用析构器。