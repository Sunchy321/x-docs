%!TEX root = x.tex

\rSec0[typesystem]{类型系统}

\rSec1[type]{类型、值和对象}
\indextext{类型}

\begin{bnf}{Type}
    FundaType \br
    SpecialType \br
    CompType
\end{bnf}

$$ \mathcal{V} = \{ \langle v, T, Q \rangle \mid T \in \mathcal{T}, v \in T, Q \subset \mathcal{Q} \} $$

\pnum
\term{类型}是一个集合。\term{值}是类型、类型的成员和修饰符集合的元组。$T$称为值$v$的\term{类型}。

\rSec2[type.funda]{基本类型}
\indextext{类型!基本类型}

\begin{bnf}{FundaType}
    \terminal{void} \br
    \terminal{never} \br
    \terminal{bool} \br
    \terminal{int} \br
    \terminal{uint} \br
    \terminal{int} \terminal{<} Expression \terminal{>} \br
    \terminal{uint} \terminal{<} Expression \terminal{>} \br
    \terminal{float} \br
    \terminal{float} \terminal{<} Expression \terminal{>}
\end{bnf}

$$\tcode{void} \coloneqq \{ \mathrm{void} \}$$

\pnum
\tcode{void}标识只有唯一一个值的类型。

$$\tcode{never} \coloneqq \{ \}$$

\pnum
\tcode{never}标识没有值的类型。

$$\tcode{bool} \coloneqq \{ \mathrm{true}, \mathrm{false} \}$$

\pnum
\tcode{bool}标识具有真或假两个值的类型。

$$\tcode{int}_{l,h} \coloneqq \{ x \in \mathcal{Z} \mid l \le x \le h \} $$

\pnum
$\tcode{int}_{l,h}$称作\term{整数类型},其中$l$和$h$为待推导常数。在本规范中,如果$l = h$,则记作$\tcode{int}_l$。存在实现定义的常数$m$和$M$。$l$和$h$须满足
$$ l \ge m $$
$$ h \le M $$
$$ 0 \le h - l \le M $$

\tcode{uint}是$\tcode{int}_{l,h}$的别名,但满足$l\ge0$。

\pnum
$\tcode{int<}w\tcode{>}$是$\tcode{int}_{l,h}$的别名,但满足$l\ge-2^{w-1}$且$h\le2^{w-1}-1$。$\tcode{uint<}w\tcode{>}$是$\tcode{int}_{l,h}$的别名,但满足$l\ge0$且$h\le2^w-1$。其中$w$可以取8、16、32、64或128。它们称作\tcode{定长整数类型},表示长度固定的整数。只能在定长整数类型上进行位运算。

$$ \tcode{float<}s\tcode{>}^\ast \subset \mathcal{R} $$
$$ \tcode{float<}s\tcode{>}^\dagger \subset \{ +\infty, -\infty, \mathrm{NaN} \} $$
$$ \tcode{float<}s\tcode{>} \coloneqq \tcode{float}^\ast \cup \tcode{float}^\dagger $$

\pnum
$\tcode{float<}s\tcode{>}$称作\term{浮点类型}。其中$s$为32或64。\tcode{float}为\tcode{float<64>}的别名。

\pnum
整数类型、定长整数类型和浮点类型称为\term{算术类型}。

\rSec2[type.special]{特殊类型}
\indextext{类型!特殊类型}

\begin{bnf}{SpecialType}
    \terminal{self}
\end{bnf}

\pnum
\tcode{self}用于在方法或概念中指示类型自身。

\rSec2[type.comp]{复合类型}
\indextext{类型!复合类型}

\begin{bnf}{CompType}
    Type \terminal{?} \br
    Type \terminal{[} \terminal{]} \br
    Type \terminal{[} Type \terminal{]} \br
    \terminal{(} TupleTypes\bnfs \terminal{)} \br
    \terminal{\{} ObjectTypes \terminal{\}} \br
    \terminal{(} TupleTypes\bnfs \terminal{)} \terminal{->} Type \br
    UnionType
\end{bnf}

\begin{bnf}{TupleTypes}
    Type \br
    TupleTypes \terminal{,} Type
\end{bnf}

\begin{bnf}{ObjectTypes}
    ObjectType \br
    ObjectTypes \terminal{,} ObjectType
\end{bnf}

\begin{bnf}{ObjectType}
    ObjectTypeQualifier\bnfs Identifier \terminal{:} Type
\end{bnf}

\begin{bnf}{ObjectTypeQualifier}
    \terminal{let} \br
    \terminal{var}
\end{bnf}

\begin{bnf}{UnionType}
    Type \terminal{|} Type \br
    UnionType \terminal{|} Type
\end{bnf}

$$ T\tcode{?} \coloneqq \{ \langle t \rangle \mid t \in T \} \cup \{ \mathrm{nil} \} $$

\pnum
$T\tcode{?}$为\term{可空类型}。

$$ T\tcode{[]} \coloneqq \bigcup^\infty_{n=0} T^n $$

\pnum
$T\tcode{[]}$为\term{数组类型}。

$$ T\tcode{[}K\tcode{]} \coloneqq T^K $$

\pnum
$T\tcode{[}K\tcode{]}$为\term{字典类型}。

$$ \tcode{(}T_1\tcode{,} \ldots\tcode{,} T_n\tcode{,}\tcode{)} \coloneqq \prod^n_{i=1} T_i $$
$$ \tcode{()} \coloneqq \tcode{void} $$

\pnum
$\tcode{(}T_1\tcode{,} \ldots\tcode{,} T_n\tcode{)}$称作\term{元组类型}。特别地,只有一个元素的元组与其内部元素类型等价;没有元素的元组与\tcode{void}等价。

$$ \tcode{\{} K_1\tcode{:}T_1\tcode{,} \ldots\tcode{,} K_n\tcode{:}T_n \tcode{\}} \coloneqq \prod^n_{i=1} T_i $$

\pnum
$\tcode{\{}K_1\tcode{:}T_1\tcode{,} \ldots\tcode{,} K_n\tcode{:}T_n\tcode{\}}$称作\term{对象类型}。

\pnum
$\tcode{(}T_1\tcode{,} \ldots\tcode{,} T_n\tcode{) -> }R$称作\term{函数类型}。

$$ T_1 \tcode{|} \ldots \tcode{|} T_n \coloneqq \bigcup^n_{i=1} \{ \langle v_i, T_i \rangle \mid v_i \in T_i \} $$

\pnum
$T_1\tcode{|}\ldots\tcode{|}T_n$ 称作\term{联合类型}。其中各个$T_i$是无序的。相同的$T_i$将被视为一个。

\rSec2[type.named]{具名类型}

\begin{bnf}{NamedType}
    EntityID \br
    FundaType \br
    SpecialType
\end{bnf}

\rSec1[type.common]{公共类型}
\indextext{类型!公共类型}

\pnum
存在函数

$$ \otimes: \mathcal{T} \times \mathcal{T} \rightarrow \mathcal{T} \cup \{ \ast \} $$

满足交换律。如果对于类型 $T_1$ 和 $T_2$,$T_1 \otimes T_2 \ne \ast$,称 $T_1 \otimes T_2$ 为 $T_1$ 和 $T_2$ 的\term{公共类型}。

\pnum
函数 $\otimes$ 按照如下顺序确定:

\begin{itemize}
\item $T \otimes T \coloneqq T$
\item $\tcode{never} \otimes T \coloneqq T$
\item $\tcode{void} \otimes T \coloneqq \tcode{void}$
\item $\tcode{int}_{l_1,h_1} \otimes \tcode{int}_{l_2,h_2} \coloneqq \tcode{int}_{\min\{l_1, l_2\},\max\{h_1, h_2\}}$
\item $\tcode{int}_{l,h} \otimes \tcode{float<}s\tcode{>} \coloneqq \tcode{float<}s\tcode{>}$
\item $\tcode{float<}s_1\tcode{>} \otimes \tcode{float<}s_2\tcode{>} \coloneqq \tcode{float<}\max\{s_1, s_2\}\tcode{>}$
\item $T \otimes T\tcode{?} \coloneqq T\tcode{?}$
\item $T \otimes T \tcode{|} T_1 \tcode{|} \ldots \tcode{|} T_{n-1} \coloneqq T \tcode{|} T_1 \tcode{|} \ldots \tcode{|} T_{n-1}$
\item $T \otimes T_1 \tcode{|} T_2 \tcode{|} \ldots \tcode{|} T_n \coloneqq T \tcode{|} T_1 \tcode{|} \ldots \tcode{|} T_n$
\item $T \otimes U \coloneqq T \tcode{|} U$
\end{itemize}

\rSec1[qualifier]{修饰符}
\indextext{修饰符}

\pnum
值除了总是具有类型之外,还可能带有一个或数个修饰符。修饰符指示了值的其他属性。

\rSec2[qual.mut]{\tcode{mut}}
\indextext{修饰符!mut}

\pnum
\tcode{mut} 表示该值是可变的。