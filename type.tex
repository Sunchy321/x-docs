%!TEX root = x.tex

\rSec0[typequal]{类型与修饰符}

\rSec1[type]{类型、值和对象}
\indextext{类型}

\begin{bnf}{Type}
    FundaType \br
    SpecialType \br
    CompType
\end{bnf}

$$ \mathcal{V} = \{ \langle T, Q, v \rangle \mid T \in \mathcal{T}, Q \subset \mathcal{Q}, v \in T \} $$

\pnum
\term{类型}是一个有限集合。\term{修饰符}是一个标识符。\term{值}是类型、修饰符集合和这个类型的成员的三元组。\enternote 实际上,值并不由这三个项完全表征。例如 \tcode{lvalue} 表示有与这个值相关联的对象,这个对象不能由这个值确定。 \exitnote $T$ 称为值 $v$ 的\term{类型}。

\rSec2[type.funda]{基本类型}
\indextext{类型!基本类型}

\begin{bnf}{FundaType}
    \terminal{void} \br
    \terminal{never} \br
    \terminal{bool} \br
    \terminal{int} \br
    \terminal{uint} \br
    \terminal{int} \terminal{<} Expression \terminal{,} Expression \terminal{>} \br
    \terminal{float} \br
    \terminal{float} \terminal{<} Expression \terminal{>} \br
\end{bnf}

\pnum
\tcode{void}标识只有唯一一个值的类型。

$$\tcode{void} \coloneqq \{ \mathrm{void} \}$$

\pnum
\tcode{never}标识没有值的类型。

$$\tcode{never} \coloneqq \{ \mathrm{never} \}$$

\pnum
\tcode{bool} 标识布尔值。

$$\tcode{bool} \coloneqq \{ \mathrm{true}, \mathrm{false} \}$$

\pnum
\tcode{int} 称作\term{整数类型}。

$$\tcode{int} \coloneqq \{ x \in \mathcal{Z} \mid l \le x \le h \} $$

其中$l$和$h$为待推导常数。在本规范中,带有特定$l$和$h$的\tcode{int}类型将记作$\tcode{int}_{l, h}$。如果$l = h$,则记作$\tcode{int}_l$。存在实现定义的常数$m$和$M$。$l$和$h$须满足
$$ l \ge m $$
$$ h \le M $$
$$ 0 \le h - l \le M $$

\tcode{uint}是$\tcode{int}_{l,h}$的别名,但满足$l>=0$。

\tcode{byte}是$\tcode{int}_{l,h}$的别名,但满足$l>=0$且$h<=255$。

\pnum
\tcode{float}称作\term{浮点类型}。

$$ \tcode{float<}s\tcode{>}^\ast \subset \mathcal{R} $$
$$ \tcode{float<}s\tcode{>}^\dagger \subset \{ +\infty, -\infty, \mathrm{NaN} \} $$
$$ \tcode{float<}s\tcode{>} \coloneqq \tcode{float}^\ast \cup \tcode{float}^\dagger $$

其中$s$为32或64。\tcode{float}为\tcode{float<64>}的别名。

\rSec2[type.special]{特殊类型}
\indextext{类型!特殊类型}

\begin{bnf}{SpecialType}
    \terminal{self} \br
\end{bnf}

\rSec2[type.comp]{复合类型}
\indextext{类型!复合类型}

\begin{bnf}{CompType}
    Type \terminal{?} \br
    % Type \terminal{[} Expression\bnfq  \terminal{]} \br
    Type \terminal{[} Type \terminal{]}
    \terminal{(} UnnamedTypes \terminal{)} \br
    \terminal{(} NamedTypes \terminal{)} \br
    \terminal{(} UnnamedTypes \terminal{,} NamedTypes \terminal{)} \br
    \terminal{\{} UnnamedTypes \terminal{\}}
    Type \terminal{*}
\end{bnf}

\begin{bnf}{UnnamedTypes}
    Type \br
    UnnamedTypes \terminal{,} Type
\end{bnf}

\begin{bnf}{NamedTypes}
    Identifier \terminal{:} Type \br
    NamedTypes \terminal{,} Identifier \terminal{:} Type
\end{bnf}

\pnum
\tcode{T?}为\term{可空类型}。

$$ T\tcode{?} \coloneqq \{ \langle t \rangle \mid t \in T \} \cup \{ \mathrm{nil} \} $$

\pnum
\tcode{T[]}为\term{数组类型}。

$$ T\tcode{[]} \coloneqq \bigcup^\infty_{n=0} T^n $$

\pnum
\tcode{T[U]}为\term{字典类型}。

$$ T\tcode{[}U\tcode{]} \coloneqq T^U $$

\pnum
\tcode{(T1, \ldots, Tm)} 称作\term{元组类型}。

$$ \tcode{(}T_1\tcode{,} \ldots\tcode{,} T_m\tcode{,}\tcode{)} \coloneqq \prod^m_{i=1} T_i$$

\pnum
\tcode{T1|}\ldots\tcode{|Tn} 称作\term{联合类型}。其中各个 \tcode{T$_i$} 是无序的。重复的 \tcode{T$_i$} 将被视为一个。

$$ T_1 \tcode{|} \ldots \tcode{|} T_n \coloneqq \bigcup^n_{i=1} \{ \langle T_i, t_i \rangle \mid t_i \in T_i \} $$

\rSec2[type.common]{公共类型}
\indextext{类型!公共类型}

\pnum
存在函数

$$ \otimes: \mathcal{T} \times \mathcal{T} \rightarrow \mathcal{T} \cup \{ \ast \} $$

满足交换律。如果对于类型 $T_1$ 和 $T_2$,$T_1 \otimes T_2 \ne \ast$,称 $T_1 \otimes T_2$ 为 $T_1$ 和 $T_2$ 的\term{公共类型}。

\pnum
函数 $\otimes$ 按照如下顺序确定:

\begin{itemize}
\item $T \otimes T \coloneqq T$
\item $\tcode{never} \otimes T \coloneqq T$
\item $\tcode{void} \otimes T \coloneqq \ast$
\item $\tcode{int}_{l_1,h_1} \otimes \tcode{int}_{l_2,h_2} \coloneqq \tcode{int}_{\min\{l_1, l_2\},\max\{h_1, h_2\}}$
\item $\tcode{int}_{l,h} \otimes \tcode{float} \coloneqq \tcode{float}$
\item $\tcode{int}_{l,h} \otimes \tcode{float32} \coloneqq \tcode{float32}$
\item $\tcode{float} \otimes \tcode{float32} \coloneqq \tcode{float}$
\item $T \otimes T\tcode{?} \coloneqq T\tcode{?}$
\item $T \otimes T \tcode{|} T_1 \tcode{|} \ldots \tcode{|} T_{n-1} \coloneqq T \tcode{|} T_1 \tcode{|} \ldots \tcode{|} T_{n-1}$
\item $T \otimes T_1 \tcode{|} T_2 \tcode{|} \ldots \tcode{|} T_n \coloneqq T \tcode{|} T_1 \tcode{|} \ldots \tcode{|} T_n$
\item $T \otimes U \coloneqq T \tcode{|} U$
\end{itemize}

\rSec1[qualifier]{修饰符}
\indextext{修饰符}

\rSec2[qual.mut]{\tcode{mut}}
\indextext{修饰符!mut}

\pnum
\tcode{mut} 表示该值是可变的。
