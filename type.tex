%!TEX root = x.tex

\rSec0[typesystem]{类型系统}

\rSec1[type]{类型、值和对象}
\indextext{类型}

\begin{bnf}{Type}
    NormalType \br
    ResultType
\end{bnf}

\begin{bnf}{NormalType}
    \terminal{(} Type \terminal{)} \br
    FundaType \br
    SpecialType \br
    CompType \br
    OpaqueType \br
    SomeType \br
    AnyType \br
    \terminal{typeof} \terminal{(} Expression \terminal{)} \br
    Type TypeQualifier
\end{bnf}

\begin{bnf}{TypeQualifier}
    \terminal{mut} \br
    \terminal{const}
\end{bnf}

$$ \mathcal{V} = \{ \langle v, T, Q \rangle \mid T \in \mathcal{T}, v \in T, Q \subset \mathcal{Q} \} $$

\pnum
\term{类型}是一个集合。\term{值}是类型、类型的成员和修饰符集合的元组。$T$称为值$v$的\term{类型}。

\rSec2[type.funda]{基本类型}
\indextext{类型!基本类型}

\begin{bnf}{FundaType}
    \terminal{void} \br
    \terminal{never} \br
    \terminal{bool} \br
    \terminal{int} \br
    \terminal{uint} \br
    \terminal{int} \terminal{<} Expression \terminal{>} \br
    \terminal{uint} \terminal{<} Expression \terminal{>} \br
    \terminal{float} \br
    \terminal{float} \terminal{<} Expression \terminal{>} \br
    \terminal{char} \br
    SymbolType
\end{bnf}

$$\tcode{void} \coloneqq \{ \mathrm{void} \}$$

\pnum
\tcode{void}标识只有唯一一个值的类型。

$$\tcode{never} \coloneqq \{ \}$$

\pnum
\tcode{never}标识没有值的类型。

$$\tcode{bool} \coloneqq \{ \mathrm{true}, \mathrm{false} \}$$

\pnum
\tcode{bool}标识具有真或假两个值的类型。

$$\tcode{int}_{l,h} \coloneqq \{ x \in \mathcal{Z} \mid l \le x \le h \} $$

\pnum
$\tcode{int}_{l,h}$称作\term{整数类型},其中$l$和$h$为待推导常数。在本规范中,如果$l = h$,则记作$\tcode{int}_l$。存在实现定义的常数$m$和$M$。$l$和$h$须满足
$$ l \ge m $$
$$ h \le M $$
$$ 0 \le h - l \le M $$

\tcode{uint}是$\tcode{int}_{l,h}$的别名,但满足$l\ge0$。

\pnum
$\tcode{int<}w\tcode{>}$是$\tcode{int}_{l,h}$的别名,但满足$l\ge-2^{w-1}$且$h\le2^{w-1}-1$。$\tcode{uint<}w\tcode{>}$是$\tcode{int}_{l,h}$的别名,但满足$l\ge0$且$h\le2^w-1$。其中$w$可以取8、16、32、64或128。它们称作\tcode{定长整数类型},表示长度固定的整数。只能在定长整数类型上进行位运算。

$$ \tcode{float<}s\tcode{>}^\ast \subset \mathcal{R} $$
$$ \tcode{float<}s\tcode{>}^\dagger \subset \{ +\infty, -\infty, \mathrm{NaN} \} $$
$$ \tcode{float<}s\tcode{>} \coloneqq \tcode{float<}s\tcode{>}^\ast \cup \tcode{float<}s\tcode{>}^\dagger $$

\pnum
$\tcode{float<}s\tcode{>}$称作\term{浮点类型}。其中$s$为32或64。\tcode{float}为\tcode{float<64>}的别名。

\pnum
整数类型、定长整数类型和浮点类型称为\term{算术类型}。

$$\tcode{char} \coloneqq \left\{ \textnormal{Any UTF-32 Code Unit} \right\}$$

\pnum
\tcode{char}是字符类型,其表示一个UTF-32码元。

\rSec3[type.symbol]{符号类型}
\indextext{类型!符号}

\begin{bnf}{SymbolType}
    SymbolLiteral \br
    \terminal{'(} UnqualID \terminal{)}
\end{bnf}

$$ \tcode{'}s \coloneqq \left\{ \langle \tcode{'}s, \tcode{'}s \rangle \right\} $$

$$ \tcode{Symbol} \coloneqq \bigcup_{\tcode{'}s} \left\{  \tcode{'}s \right\} $$

\pnum
符号字面量的类型与其值相同。\tcode{Symbol}是符号字面量的公共类型。参见~\ref{core.type}。

\rSec2[type.special]{特殊类型}
\indextext{类型!特殊类型}

\begin{bnf}{SpecialType}
    \terminal{self}
\end{bnf}

\pnum
\tcode{self}用于在方法或概念中指示类型自身。

\rSec2[type.comp]{复合类型}
\indextext{类型!复合类型}

\begin{bnf}{CompType}
    Type \terminal{?} \br
    Type \terminal{[} \terminal{]} \br
    Type \terminal{[} Type \terminal{]} \br
    \terminal{(} TupleTypes\bnfs \terminal{)} \br
    \terminal{\{} StructTypes \terminal{\}} \br
    \terminal{(} TupleTypes\bnfs \terminal{)} \terminal{->} Type \br
    UnionType \br
    FuncType \br
    OpaqueType
\end{bnf}

\rSec3[type.optional]{可空类型}
\indextext{类型!可空}

$$ T\tcode{?} \coloneqq \{ \langle t \rangle \mid t \in T \} \cup \{ \mathrm{nil} \} $$

\pnum
$T\tcode{?}$为\term{可空类型}。$T\tcode{?}$包含$T$的所有值(使用\tcode{some}标识)和空值\tcode{nil}。

\rSec3[type.array]{数组类型}
\indextext{类型!数组}

$$ T\tcode{[]} \coloneqq \bigcup^\infty_{i=1} T^i $$

\pnum
$T\tcode{[]}$为\term{数组类型},代表有限个类型$T$的值的序列。

\rSec3[type.dict]{字典类型}
\indextext{类型!字典}

$$ T\tcode{[}K\tcode{]} \coloneqq T^K $$

\pnum
$T\tcode{[}K\tcode{]}$为\term{字典类型},代表键类型$K$到值类型$T$的映射。

\rSec3[type.tuple]{元组类型}
\indextext{类型!元组}

\begin{bnf}{TupleTypes}
    Type \br
    TupleTypes \terminal{,} Type
\end{bnf}

$$ \tcode{(}T_1\tcode{,} \ldots\tcode{,} T_n\tcode{,}\tcode{)} \coloneqq \prod^n_{i=1} T_i $$
$$ \tcode{()} \coloneqq \tcode{void} $$

\pnum
$\tcode{(}T_1\tcode{,} \ldots\tcode{,} T_n\tcode{)}$称作\term{元组类型},代表有限个值的序列。

\pnum
特别地,只有一个元素的元组与其内部元素类型等价;没有元素的元组与\tcode{void}等价。

\rSec3[type.struct]{结构类型}
\indextext{类型!结构}

\begin{bnf}{StructTypes}
    StructType \br
    StructTypes \terminal{,} StructType
\end{bnf}

\begin{bnf}{StructType}
    StructTypeQualifier\bnfs Identifier \terminal{:} Type
\end{bnf}

\begin{bnf}{StructTypeQualifier}
    \terminal{mut}
\end{bnf}

$$ \tcode{\{} K_1\tcode{:}T_1\tcode{,} \ldots\tcode{,} K_n\tcode{:}T_n \tcode{\}} \coloneqq \prod^n_{i=1} T_i $$

\pnum
$\tcode{\{}K_1\tcode{:}T_1\tcode{,} \ldots\tcode{,} K_n\tcode{:}T_n\tcode{\}}$称作\term{结构类型}。结构类型可以使用标识符访问其成员。

\rSec3[type.func]{函数类型}
\indextext{类型!函数}

\begin{bnf}{FuncType}
    Parameter ThrowQual\bnfq ReturnType \br
    Type ThrowQual\bnfq ReturnType
\end{bnf}

\pnum
$\tcode{(}T_1\tcode{,} \ldots\tcode{,} T_n\tcode{) -> }R$称作\term{函数类型}。

\rSec2[type.opaque]{不透明类型}
\indextext{类型!不透明}

\begin{bnf}{OpaqueType}
    \terminal{class} Type
\end{bnf}

\pnum
不透明类型用于从现有的类型出发构造一个相同但不能混用的类型。类型$T$与其构造的不透明类型$U$之间不具有隐式转换,但可以显式转换。同一个类型构造的复数个不透明类型之间也不能混用。

\enterexample
\begin{codeblock}

type T = class int;
type U = class int;

let i = 0;
let j: T = i; // 错误,\tcode{int}不能隐式转换为\tcode{class int}
let k: T = i as T; // 可以,显式转换
let l: U = j; // 错误,\tcode{T}和\tcode{U}之间没有隐式转换

\end{codeblock}
\exitexample

\pnum
如果创建一个元组或结构类型的不透明类型,则其各成员的默认访问级别为\tcode{private}。

\rSec2[type.impl]{\tcode{impl}类型}
\indextext{类型!\tcode{impl}}

\begin{bnf}{SomeType}
    \terminal{impl} Type \br
    \terminal{impl} \terminal{_}
\end{bnf}

\pnum
$\tcode{impl}\ T$标识一个静态待推导类型,但保证该类型为$T$的子类型。
\tcode{impl _}代表一个基础类型待推导的\tcode{impl}类型。

\pnum
\tcode{impl}还能用于简化泛型函数声明。参见~\ref{generic.impl}。

\enterexample
\begin{codeblock}

trait T { }
type A { }
type B { }

impl A : T { }
impl B : T { }

let x: some T = A { }; // \tcode{x}的类型是\tcode{A}
let mut y: some T = B { }; // \tcode{y}的类型是\tcode{B}

y = x; // 错误,\tcode{B}不能赋给\tcode{A}

\end{codeblock}
\exitexample

\rSec2[type.any]{\tcode{any}类型}
\indextext{类型!\tcode{any}}

\begin{bnf}{AnyType}
    \terminal{any} Type \br
    \terminal{any} \terminal{_} \br
    \terminal{any}
\end{bnf}

\pnum
$\tcode{any}\ T$对$T$的子类型进行包装,保证在运行时可以接受任何为$T$的子类型的值。
\tcode{any _}代表一个基础类型待推导的\tcode{any}类型。
$\tcode{any}$表示对任意类型的包装。

\enterexample
\begin{codeblock}

trait T { }
type A { }
type B { }

impl A : T { }
impl B : T { }

let x: any T = A { }; // \tcode{x}的类型是\tcode{any T}
let mut y: any T = B { }; // \tcode{y}的类型是\tcode{any T}

y = x; // 正确,\tcode{any T}之间可以互相赋值

\end{codeblock}
\exitexample

\rSec2[type.result]{结果类型}
\indextext{类型!结果类型}

\begin{bnf}{ResultType}
    NormalType \terminal{throw} NormalType
\end{bnf}

\pnum
结果类型$T\tcode{ throw }E$标识一个可能产生错误的值,其中$T$是正常的返回值类型,$E$是错误类型,且必须实现\tcode{ErrorCode}。

\rSec2[type.named]{具名类型}

\begin{bnf}{TypeName}
    \terminal{(} Type \terminal{)} \br
    EntityID \br
    FundaType \br
    SpecialType
\end{bnf}

\pnum
\term{类型名称}在特定的语法位置表示类型,以避免潜在的语法歧义。

\rSec1[subtype]{子类型}
\indextext{类型!子类型}

\pnum
类型$A$可能是类型$B$的\term{子类型},记作$A \preceq B$。$A$可以在需要$B$的上下文中隐式转换到$B$。

\pnum
子类型关系具有自反性和传递性,即对任意类型$A$、$B$和$C$有$A \preceq A$和$A \preceq B \land B \preceq C \implies A \preceq C$成立。

\begin{equation*}
\begin{aligned}
    T \preceq \tcode{void},& T \in \mathcal{T} \\
    \tcode{never} \preceq T,& T \in \mathcal{T}
\end{aligned}
\end{equation*}

\pnum
任何类型都是\tcode{void}的子类型。\tcode{never}是任何类型的子类型。

\begin{equation*}
\begin{aligned}
I_{l_1,h_1} &\preceq J_{l_2,h_2} \mathrel{\mathrm{if}} l_1 \geq l_2 \lor h_1 \leq h_2 \\
    \tcode{float<}s_1\tcode{>} &\preceq \tcode{float<}s_2\tcode{>} \mathrel{\mathrm{if}} s_1 \leq s_2 \\
I_{l,h} &\preceq \tcode{float<}s\tcode{>} \\
    I, J &\in \left\{ \tcode{int}, \tcode{uint}, \tcode{int<}w\tcode{>}, \tcode{uint<}w\tcode{>} \right\}
\end{aligned}
\end{equation*}

\pnum
范围更小的整数类型是范围更大的整数类型的子类型。长度更小的浮点类型是长度更大的浮点类型的子类型。

\pnum
整数类型是浮点类型的子类型。当整数被隐式转换为浮点数时,其值将被转换为最接近的浮点数。\enternote 虽然整数转换到浮点数可能无法保持值不变,但出于习惯仍然保持这个隐式转换。 \exitnote \enternote 浮点类型不能隐式转换到整数类型,但可以显式转换。 \exitnote

$$ \tcode{'}s \preceq \tcode{Symbol} $$

\pnum
所有符号字面量类型都是\tcode{Symbol}的子类型。

\pnum
\tcode{bool}没有语言内建的子类型约束。但是,其他类型的值可以在\tcode{if}表达式中当作条件使用,这通过重载\tcode{operator if}运算符实现。

\begin{equation*}
\begin{aligned}
T &\preceq T\tcode{?} \\
T\tcode{[]} &\preceq U\tcode{[]} \mathrel{\mathrm{if}} T \preceq U \\
T\tcode{[}K\tcode{]} &\preceq U\tcode{[}L\tcode{]} \mathrel{\mathrm{if}} T \preceq U \mathrel{\mathrm{and}} K \preceq L \\
\tcode{(}T_1\tcode{,} \ldots\tcode{,} T_n\tcode{)} &\preceq \tcode{(}U_1\tcode{,} \ldots\tcode{,} U_n\tcode{)} \mathrel{\mathrm{if}} T_i \preceq U_i \mathrel{\mathrm{for}} 1 \leq i \leq n \\
\end{aligned}
\end{equation*}

\pnum
任意类型是其对应可空类型的子类型。如果$T_i$是$U_i$的子类型,则$T_0\tcode{[]}$、$T_0\tcode{[}T_1\tcode{]}$、$\tcode{(}T_1\tcode{,} \ldots\tcode{,} T_n\tcode{)}$分别是$U_0\tcode{[]}$、$U_0\tcode{[}U_1\tcode{]}$、$\tcode{(}U_1\tcode{,} \ldots\tcode{,} U_n\tcode{)}$的子类型。\enternote 这意味着数组、字典和元组的元素类型是协变的。 \exitnote

\pnum
对两个结构类型$T$和$U$而言,如果$U$的每个成员都有对应的$T$的成员且类型是该成员的子类型,则$T$是$U$的子类型。

\pnum
对两个函数类型$T$和$U$而言,如果$U$的每个参数类型都是对应$T$的参数的子类型,且$T$的返回类型是$U$的返回类型的子类型,则$T$是$U$的子类型。\enternote 这意味着函数类型的参数类型是逆变的,返回类型是协变的。 \exitnote $T$可以拥有比$U$更多的顺序参数,或者$U$不包含的命名参数。

\pnum
类类型可以定义类型转换函数。每个这样的函数定义了一个子类型关系。

\pnum
对类型表达式而言,$T\mathbin{\tcode{\&}}U$是$T$和$U$的子类型;$T$和$U$是$T\mathbin{\tcode{|}}U$的子类型。

\rSec1[type.common]{公共类型}
\indextext{类型!公共类型}

\pnum
对两个类型$A$和$B$而言,存在一个唯一的类型$C$称为$A$和$B$的\term{公共类型}。$C$满足:

\begin{equation}
\begin{aligned}
    A &\preceq C \\
    B &\preceq C \\
    \forall D \in \mathcal{T}, A &\preceq D \land B \preceq D \Rightarrow C \preceq D
\end{aligned}
\end{equation}

记作 $C = A \otimes B$。公共类型满足交换律。

\pnum
如果$A \preceq B$,则$A \otimes B = B$。

\pnum
如果$A$和$B$之间没有子类型关系,则$A \otimes B = A \mathbin{\tcode{|}} B$。

\rSec1[qualifier]{修饰符}
\indextext{修饰符}

\pnum
值除了总是具有类型之外,还可能带有一个或数个修饰符。修饰符指示了值的其他属性。

\pnum
类型可以带有修饰符,指定该值需有特定的修饰符约束。

\rSec2[qual.mut]{\tcode{mut}}
\indextext{修饰符!mut}

\pnum
\tcode{mut}表示该值是可变的。具有\tcode{mut}的值才能成为赋值操作符的左操作数。参见~\ref{expr.assign}。

\rSec2[qual.const]{\tcode{const}}

\pnum
\tcode{const}表示该值是一个常量值,于编译期间确定且不可变。部分语言功能只允许常量值。

\pnum
\tcode{const}绑定创建一个常量并插入当前作用域。该绑定的初始值必须是常量表达式。
