%!TEX root = x.tex

\rSec0[access]{访问控制}
\indextext{访问控制}

\pnum
实体 $x$ 可以\term{访问}实体 $y$ 意味着,$x$ 的声明中指代 $y$ 的 \grammarterm{IDExpr} (无论是显式还是隐式)是合法的。\footnote{某些情况可能不存在这样的 \grammarterm{IDExpr},这并不意味着 $x$ 能访问或不能访问 $y$。}

\pnum
变量、函数、类型、模块的声明 $e$ 可以包含以下的\term{访问控制符}:

\begin{itemize}
\item \tcode{public},任意实体 $f$ 都可以访问它;
\item \tcode{internal},只有与 $e$ 处于同一个包中的实体才能访问它;
\item \tcode{private},只有与 $e$ 定义在同一个文件中的实体才可以访问它。
\end{itemize}

块作用域的声明不能具有访问控制符。如果一个非块作用域的声明不包含访问控制符,那么

\begin{itemize}
\item 如果它对应的 \grammarterm{UnqualID} 不是标识符,它是 \tcode{public} 的;
\item 如果它的标识符以下划线开头,那么它是 \tcode{private} 的;否则
\item 它是 \tcode{public} 的。
\end{itemize}

扩展不能具有访问控制符(因为它不能被使用第二次)。声明块的声明控制符应用到其内部的声明。
