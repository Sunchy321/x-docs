%!TEX root = x.tex

\rSec0[stmt]{语句}
\indextext{语句}

\begin{bnf}
\nontermdef{Statement} \br
    Block \br
    BindingStmt \br
    IfStmt \br
    SwitchStmt \br
    AssertStmt
\end{bnf}

\pnum
语句是一类特殊的基本表达式。在语句中的子块将优先作为语句的一部分而不是其中的表达式的一部分。\enterexample

\begin{codeblock}
// 错误: \tcode{\{ true \}} 被认为是 \tcode{if} 语句的第一个子块,而不是它的条件表达式的一部分
if x.filter{ true }
\end{codeblock}

\exitexample

\rSec1[stmt.block]{块}
\indextext{语句!块}

\begin{bnf}
\nontermdef{Block} \br
    \terminal{\{} BlockItem\bnfs\ \terminal{\}}
\end{bnf}

\begin{bnf}
\nontermdef{BlockItem} \br
    Expression \terminal{;}\bnfq \br
    BlockDecl
\end{bnf}

\pnum
块是由大括号包裹的一系列声明和表达式的序列。块定义了一个块作用域。块的求值按照顺序进行,整个语句的值是最后一个项目的值。所有不是最后一项的表达式项的值被丢弃;这些表达式必须以\tcode{;}结尾。如果最后一个项目是一个声明,这个块的类型为 \tcode{void}。

\rSec1[stmt.bind]{绑定语句}
\indextext{语句!绑定}

\begin{bnf}
\nontermdef{BindingStmt} \br
    Binding \terminal{;}
\end{bnf}

\begin{bnf}
\nontermdef{Binding} \br
    Pattern \terminal{=} Expression \terminal{;}
\end{bnf}

\pnum
\term{绑定}形如$p\ \tcode{=}\ e$,其中$p$是包含至少一个绑定模式的模式。

\pnum
\term{绑定语句}将一个绑定插入当前作用域中。该绑定必须不能失败。

\rSec1[stmt.if]{\tcode{if} 语句}
\indextext{语句!if}

\begin{bnf}
\nontermdef{IfStmt} \br
    \terminal{if} Condition \terminal{then} Expression \br
    \terminal{if} Condition \terminal{then} Expression \terminal{else} Expression \br
    \terminal{if} Condition Block \br
    \terminal{if} Condition Block \terminal{else} Block
\end{bnf}

\begin{bnf}
\nontermdef{Condition} \br
    Expression \br
    Binding
\end{bnf}

\pnum
\term{条件}可以为任意对象声明后跟一个表达式或模式匹配。如果条件为表达式$e$\footnote{因为赋值表达式的类型是\tcode{void},形如\tcode{p = e}的程序文本将始终被看做一个模式匹配而不是赋值表达式。},那么这个表达式的类型必须实现\tcode{core.Boolean}。条件成立当且仅当$e$求值为真(\ref{core.oolean})。如果条件是绑定,那条件成立当且仅当绑定成功。该绑定必须可以失败。

\pnum
将\tcode{if}语句的第一个表达式记作$T$,第二个表达式(如果有)记作$F$。对\tcode{if}语句的求值按以下顺序进行:

\begin{itemize}

\item 如果条件成立,对$T$求值,然后将它的值作为整个语句的值。
\item 否则,如果存在$F$,那么对它求值,然后将它的值作为整个语句的值。
\item 否则,整个语句的值为\tcode{()}。

\end{itemize}

只有一个表达式会被求值。整个语句的类型是$T$和$F$的公共类型(如果$F$不存在的话视为\tcode{void})。

\rSec1[stmt.match]{\tcode{match} 语句}
\indextext{语句!match}

\begin{bnf}
\nontermdef{MatchStmt} \br
    \terminal{match} Expression MatchBlock
\end{bnf}

\begin{bnf}
\nontermdef{MatchBlock} \br
    \terminal{\{} BlockItem\bnfs MatchItem\bnfs \terminal{\}}
\end{bnf}

\begin{bnf}
\nontermdef{MatchItem} \br
    Matcher Expression\bnfp
\end{bnf}

\begin{bnf}
\nontermdef{Matcher} \br
    Pattern \terminal{->}
\end{bnf}

\pnum
\term{\tcode{match}语句}对其后跟的表达式进行模式匹配。整个语句的类型为每个匹配项表达式类型的公共类型。

\pnum
\tcode{match}语句的各项中的模式必须覆盖被匹配表达式的所有可能值,否则这是一个编译错误。

\pnum
对\tcode{match}语句的求值将按如下顺序进行:

\begin{itemize}
    \item 如果语句匹配块之前有项,执行这些项。他们的作用域是整个块。
    \item 按出现顺序对每个项进行匹配。如果某个项的模式匹配成功,则对其后的表达式进行求值,将其作为整个match表达式的值。所有其他项的表达式都不会进行求值。
\end{itemize}

\rSec1[stmt.while]{\tcode{while} 语句}
\indextext{语句!while}

\begin{bnf}
\nontermdef{WhileStmt} \br
    \terminal{while} Expression\bnfq Block
\end{bnf}

\pnum
\term{\tcode{while}语句}处理循环,其中的表达式必须实现\tcode{core.Boolean}。整个语句的类型为\tcode{void}。

\pnum
\tcode{while}语句每次循环都会对控制表达式进行求值。如果求值为真,则继续循环,否则终止循环。如果表达式被省略,则等价于表达式为\tcode{true}。

\rSec1[stmt.for]{\tcode{for}语句}
\indextext{语句!for}

\begin{bnf}
\nontermdef{ForStmt} \br
    \terminal{for} Pattern \terminal{:} Expression Block
\end{bnf}

\pnum
\term{\tcode{for}语句}进行明确的范围循环。形如$\tcode{for}\ p\ \tcode{:}\ e\ B$的\tcode{for}语句需满足:$e$实现了\tcode{core.Sequence};$\tcode{typeof(}e\tcode{).Item}$匹配$p$不会失败,否则这是一个编译错误。$p$中注入的变量在整个\tcode{for}语句的范围内生效。整个语句的类型为\tcode{void}。

\rSec1[stmt.control]{控制语句}
\indextext{语句!控制}

\begin{bnf}
\nontermdef{BreakStmt} \br
    \terminal{break} \terminal{;}
\end{bnf}

\begin{bnf}
\nontermdef{ContinueStmt} \br
    \terminal{continue} \terminal{;}
\end{bnf}

\begin{bnf}
\nontermdef{ReturnStmt} \br
    \terminal{return} Expression\bnfq \terminal{;}
\end{bnf}

\pnum
控制语句包括\tcode{break}语句、\tcode{continue}语句和\tcode{return}语句。

\pnum
\tcode{break}语句只能在\tcode{while}或\tcode{for}语句中使用。它终止最内层的循环语句。

\pnum
\tcode{continue}语句只能在\tcode{while}或\tcode{for}语句中使用。它终止最内侧循环语句的本次循环。

\pnum
\tcode{return}语句只能在函数块中使用。它中止函数块的执行,并将后跟的表达式作为整个函数的返回值。如果表达式被省略,则等价于\tcode{()}。
