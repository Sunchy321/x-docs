%!TEX root = x.tex

\rSec0[stmt]{语句}
\indextext{语句}

\begin{bnf}{Statement}
    Block \br
    BindingStatement \br
    IfStatement \br
    MatchStatement \br
    WhileStatement \br
    ForStatement \br
    BreakStatement \br
    ContinueStatement \br
    ReturnStatement \br
    ThrowStatement
\end{bnf}

\pnum
语句是一类特殊的表达式。

\pnum
语句是块的构成部分。如果语句包含子块,则这个块将优先作为语句的构成部分而不是其中的表达式的一部分。\enterexample

\begin{codeblock}
// 错误: \tcode{\{ true \}} 被认为是 \tcode{if} 语句的第一个子块,而不是它的条件表达式的一部分
if x.filter{ true }
\end{codeblock}

\exitexample

\rSec1[stmt.block]{块}
\indextext{语句!块}

\begin{bnf}{Block}
    \terminal{\{} BlockItem\bnfs\ \terminal{\}}
\end{bnf}

\begin{bnf}{BlockItem}
    Expression \terminal{;}\bnfq \br
    BlockDecl \br
    Statement
\end{bnf}

\pnum
块是由大括号包裹的一系列声明和表达式的序列。块定义了一个块作用域。块的求值按照顺序进行,整个语句的值是最后一个项目的值。所有不是最后一项的表达式项的值被丢弃;这些表达式必须以\tcode{;}结尾。如果最后一个项目是一个声明,这个块的类型为\tcode{void}。

\rSec1[stmt.bind]{绑定语句}
\indextext{语句!绑定}

\begin{bnf}{BindingStatement}
    Binding \terminal{;}
\end{bnf}

\begin{bnf}{Binding}
    Pattern \terminal{=} Expression \terminal{;}
\end{bnf}

\pnum
\term{绑定}形如$p\ \tcode{=}\ e$,其中$p$是包含至少一个绑定模式的模式。

\pnum
\term{绑定语句}将一个绑定插入当前作用域中。该绑定必须不能失败。

\rSec1[stmt.if]{\tcode{if} 语句}
\indextext{语句!\idxcode{if}}

\begin{bnf}{IfStatement}
    \terminal{if} Condition Block \br
    \terminal{if} Condition Block \terminal{else} Block \br
    \terminal{if} Condition Block \terminal{else} IfStatement
\end{bnf}

\begin{bnf}{Condition}
    Expression \br
    Binding
\end{bnf}

\pnum
如果条件为表达式$e$\footnote{因为赋值表达式的类型是\tcode{void},形如\tcode{p = e}的程序文本将始终被看做一个模式匹配而不是赋值表达式。},那么这个表达式的类型必须是布尔类型(参见~\ref{op.over.if})。条件成立当且仅当$e$求值为真。如果条件是绑定,那条件成立当且仅当绑定成功。该绑定必须可以失败。

\pnum
如果\tcode{if}语句的条件成立,执行第一个子块。否则,若有\tcode{else}子块,执行\tcode{else}子块。

\rSec1[stmt.match]{\tcode{match} 语句}
\indextext{语句!\idxcode{match}}

\begin{bnf}{MatchStatement}
    \terminal{match} Expression MatchBlock
\end{bnf}

\begin{bnf}{MatchBlock}
    \terminal{\{} BlockItem\bnfs MatchItem\bnfp \terminal{\}}
\end{bnf}

\begin{bnf}{MatchItem}
    Pattern \terminal{->} Statement
\end{bnf}

\pnum
\term{\tcode{match}语句}对其后跟的表达式进行模式匹配。
\tcode{match}语句的各项中的模式必须覆盖被匹配表达式的所有可能值,否则这是一个编译错误。
对\tcode{match}语句的求值将按如下顺序进行:

\begin{itemize}
    \item 如果语句匹配块之前有项,执行这些项。他们的作用域是整个块。
    \item 按出现顺序对每个项进行匹配。如果某个项的模式匹配成功,则执行其后的语句。所有其他项都不会进行求值。
\end{itemize}

\rSec1[stmt.while]{\tcode{while} 语句}
\indextext{语句!\idxcode{while}}

\begin{bnf}{WhileStatement}
    \terminal{while} SymbolLiteral\bnfq Condition\bnfq Block \br
    \terminal{while} SymbolLiteral\bnfq Condition\bnfq Block \terminal{else} Block
\end{bnf}

\pnum
\term{\tcode{while}语句}处理循环。\tcode{while}语句每次循环都会对控制表达式进行求值。如果求值为真,则继续循环,否则终止循环。如果表达式被省略,则等价于表达式为\tcode{true}。

\pnum
如果\tcode{while}语句包含一个\tcode{else}块,则在循环结束时执行该块,并将该块的值作为整个语句的值。

\pnum
\tcode{while}语句的类型为语句主体块包含的所有终止该循环的\tcode{break}语句值的类型以及\tcode{else}块的类型的公共类型。
如果语句不含\tcode{else}块,则主体块的类型为\tcode{void}。
如果语句的条件被省略,则主体块的类型为\tcode{never}。\tcode{else}块的类型将被忽略。

\pnum
\tcode{while}语句可以带有一个符号字面量作为标签,用于在嵌套循环中终止指定的循环。\enternote \tcode{while} 之后紧跟的符号字面量总是被解析为一个标签。如果你需要在循环条件中使用符号字面量,请使用括号将其包裹起来。 \exitnote

\rSec1[stmt.for]{\tcode{for}语句}
\indextext{语句!\idxcode{for}}

\begin{bnf}{ForStatement}
    \terminal{for} SymbolLiteral\bnfq \terminal{await}\bnfq Pattern \terminal{:} Expression Block \br
    \terminal{for} SymbolLiteral\bnfq \terminal{await}\bnfq Pattern \terminal{:} Expression Block \terminal{else} Block
\end{bnf}

\pnum
\term{\tcode{for}语句}进行明确的范围循环。形如$\mathrel{\tcode{for}} p\mathrel{\tcode{:}} e\ B$的\tcode{for}语句需满足:$e$实现了\tcode{Sequence}且
$\tcode{typeof(}e\tcode{).Item}$匹配$p$不会失败,否则这是一个编译错误。$p$中注入的变量在整个\tcode{for}语句的范围内生效。

\pnum
如果\tcode{for}语句包含一个\tcode{else}块,则在循环结束时执行该块,并将该块的值作为整个语句的值。

\pnum
\tcode{for}语句的类型为语句主体块包含的所有终止该循环的\tcode{break}语句值的类型以及\tcode{else}块的类型的公共类型。
如果语句不含\tcode{else}块,则视为其类型为\tcode{void}。

\pnum
\tcode{for}语句将被展开为如下形式:

\begin{codeblock}
{
    let mut \{$i$} = \{$e$}.iter;

    while let \{$v$} = \{$i$}.next(); \{$v$} != nil {
        \{$p$} = \{$v$};
        \{$B$}
    } /* else E */
}
\end{codeblock}

其中$i$和$v$是内部使用的匿名变量名称。参见~\ref{core.seq}。

\pnum
与\tcode{while}语句相同,\tcode{for}语句也可以带有标签。

\pnum
\tcode{for}语句可以带有一个\tcode{await}标识,代表该语句进行异步迭代。

\rSec1[stmt.control]{控制语句}
\indextext{语句!控制}

\begin{bnf}{BreakStatement}
    \terminal{break} SymbolLiteral\bnfq Expression\bnfq \terminal{;}
\end{bnf}

\begin{bnf}{ContinueStatement}
    \terminal{continue} SymbolLiteral\bnfq \terminal{;}
\end{bnf}

\begin{bnf}{ReturnStatement}
    \terminal{return} Expression\bnfq \terminal{;}
\end{bnf}

\begin{bnf}{ThrowStatement}
    \terminal{throw} Expression\bnfq \terminal{;}
\end{bnf}

\pnum
控制语句包括\tcode{break}语句、\tcode{continue}语句、\tcode{return}语句和\tcode{throw}语句。

\pnum
\tcode{break}语句只能在\tcode{while}或\tcode{for}语句中使用。它终止最内层的循环语句,将控制流移动到该语句之后。
\tcode{break}语句可以带有一个表达式,此时该表达式将作为整个循环语句的值被返回。
如果它带有一个符号字面量,则它会终止具有对应符号字面量的循环语句。如果没有对应的符号字面量,则这是一个编译错误。
\enternote 如果希望将符号字面量作为返回表达式,则需要使用括号将其包裹起来。 \exitnote

\pnum
\tcode{continue}语句只能在\tcode{while}或\tcode{for}语句中使用。它终止最内侧循环语句的本次循环。将控制流移动到该语句的下一次循环开始。
如果它带有一个符号字面量,则它会终止具有对应符号字面量的循环语句。如果没有对应的符号字面量,则这是一个编译错误。

\pnum
\tcode{return}语句只能在函数作用域(\ref{scope.func})中使用。它中止当前函数的执行,并将后跟的表达式作为整个函数的返回值。如果表达式被省略,则等价于\tcode{()}。

\pnum
\tcode{throw}语句只能在函数作用域中使用。它终止当前函数的执行,并将后跟的表达式作为异常抛出。如果表达式被省略,除非语句当前处于\tcode{catch}块中,此时将会重新抛出原异常;否则这是一个编译错误。
