%!TEX root = x.tex

\rSec0[core.seq]{序列库}
\indextext{库!序列库}

\pnum
\tcode{core.seq}库定义了序列特征,并提供了一些操作序列的函数。

\rSec1[core.seq.traits]{特征}

\indexlibrary{\idxcode{Sequence}}
\begin{itemdecl}
trait Sequence<T> {
    type Item = T;
    type Iterator : core::Iterator<T>;

    let size: uint;
    let iter: Iterator;

    let isEmpty => this.size == 0;
}
\end{itemdecl}

\pnum
\tcode{Sequence}表示序列。

\pnum
\tcode{Item}是序列的元素类型,其始终为\tcode{T}。

\pnum
\tcode{Iterator}是序列的迭代器类型,其必须实现了\tcode{Iterator<T>}。

\pnum
\tcode{size}是序列的大小。序列实现了\tcode{Dollar},返回\tcode{size}。

\pnum
\tcode{iter}是序列的迭代器。

\pnum
\tcode{isEmpty}返回序列是否为空。它的默认实现将序列大小与0比较。

\indexlibrary{\idxcode{Sequence}!实现}
\begin{itemdecl}
impl<T> Sequence<T> : Dollar {
    func dollar() { this.size }
}
\end{itemdecl}

\pnum
序列实现了\tcode{Dollar},返回序列的大小。

\indexlibrary{\idxcode{Iterator}}
\begin{itemdecl}
trait Iterator<T> {
    type Item = T;

    func next(this: mut) -> T?;
}
\end{itemdecl}

\pnum
\tcode{Iterator}表示迭代器。

\pnum
\tcode{next}返回迭代器的下一个元素,如果迭代器已经到达末尾,则返回\tcode{nil}。

\rSec1[core.seq.helper]{辅助函数}
