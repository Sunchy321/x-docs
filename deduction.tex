%!TEX root = x.tex

\rSec0[deduct]{类型推导}
\indextext{类型推导}

\pnum
\X 可以在适当的地方不提供显式类型,而是让类型自动推导。

\rSec1[deduct.static]{自动推导的静态成员}

\pnum
使用类型的静态成员和枚举符时,可以省略类型名称。

\pnum
匿名静态成员表达式$e$按照如下方法匹配类型$T$:

\begin{itemize}
    \item 如果$T$是枚举类型,且具有枚举符$e$,且枚举符参数与$e$匹配,则匹配成功。
    \item 否则,如果$T$具有类型为$T$的静态成员$e$,且$e$不含有枚举符参数,则匹配成功。
    \item 否则,如果$e$是函数类型的表达式,且$T$具有名称为$e$、返回类型为$T$的静态方法,且$T\tcode{.}e$的参数类型与$e$的兼容,则匹配成功。
    \item 其他情况匹配失败。
\end{itemize}

\enterexample
\begin{codeblock}
enum E { A, B(int), C };

class X {
    static let x = X();
    static func v(i: int) => X();
    init() { }
}

func f(e: E) { }
func g(x: X) { }

f(.A); // 可以
f(.B); // 错误,枚举符参数不一致
f(.B(0)); // 可以
g(.x); // 可以,等价于\tcode{g(X.x)}
g(.v(0)); // 可以,等价于\tcode{g(X.v(0))}

\end{codeblock}
\exitexample

\pnum
匿名静态成员表达式匹配可以跨越方法调用。对方法调用$o\tcode{.}f$而言,如果$T$有方法$f$其类型与$o\tcode{.}f$兼容,且返回类型为$T$,则$o\tcode{.}f$匹配$T$当且仅当$o$匹配$T$。

\enterexample
\begin{codeblock}
enum E { A, B(int), C };
enum F { X, Y, Z };

impl E {
    func f() => self.C;
    func g() => F.X;
}

impl F {
    func f() => E.A;
    func g() => E.C;
}

func f(e: E) { }

f(.C.f()); // 可以,推导可以穿过方法调用,且只会检查\tcode{E.f}
f(.Y.g()); // 错误,\tcode{E}上的方法\tcode{g}不返回\tcode{E}

\end{codeblock}
\exitexample