%!TEX root = x.tex

\rSec0[decl]{声明}
\indextext{声明}

\begin{bnf}{Declaration}
    BlockDecl
\end{bnf}

\begin{bnf}{BlockDecl}
    Binding \br
    FuncDecl \br
    TypeDecl \br
    ClassDecl \br
    EnumDecl \br
    TraitDecl \br
    ImplDecl \br
    OperatorDecl \br
    MacroDecl \br
    StaticAssertion \br
    QualifierDirective
\end{bnf}

\pnum
\term{声明}将名称插入当前作用域,或者使用特定的程序结构实现程序功能。声明在插入名称时也可能\term{定义}了该名称对应的程序项。

\rSec1[decl.type]{类型声明}
\indextext{声明!类型}

\begin{bnf}{TypeDecl}
    TypeQual\bnfs \terminal{type} Identifier TypeBody
\end{bnf}

\begin{bnf}{TypeDeclName}
    Identifier \br
    \terminal{self}
\end{bnf}

\begin{bnf}{TypeQual}
    \terminal{const}
\end{bnf}

\begin{bnf}{TypeBody}
    StructType \br
    \terminal{\{} \terminal{\}} \br
    \terminal{\{} StructTypeList \terminal{,}\bnfq \terminal{\}} \br
    \terminal{=} Type
\end{bnf}

\pnum
类型声明用于为类型创建别名。在类型声明之后,可以使用该名称代表所关联的类型。
在类型声明中,可以使用大括号代替结构类型的小括号。

\rSec1[decl.class]{类声明}
\indextext{声明!类}

\begin{bnf}{ClassDecl}
    ClassQual\bnfs \terminal{class} Identifier ClassBody
\end{bnf}

\begin{bnf}{ClassQual}
    \terminal{const}
\end{bnf}

\begin{bnf}{ClassBody}
    StructType \br
    \terminal{\{} \terminal{\}} \br
    \terminal{\{} StructTypeList \terminal{,}\bnfq \terminal{\}} \br
    \terminal{=} Type
\end{bnf}

\pnum
类声明创建不透明类型。\tcode{class $C$ $B$}等价于\tcode{type $C$ = class $B$}。参见~\ref{type.opaque}。
在类声明中,可以使用大括号代替结构类型的小括号。

\rSec1[decl.static-assert]{静态断言}
\indextext{声明!静态断言}

\begin{bnf}{StaticAssertion}
    \terminal{static} \terminal{assert} Expression \terminal{;} \br
    \terminal{static} \terminal{assert} Expression \terminal{:} Expression \terminal{;}
\end{bnf}

\pnum
\term{静态断言}用于在编译时对条件进行检查。如果条件为假,则会产生一个编译错误。静态断言可以在函数、模块、枚举定义、特征定义、实现定义中出现。

\pnum
静态断言可以带有一个可选的错误消息。如果提供了错误消息,当条件为假时,编译器会输出该错误消息。

\pnum
静态断言的条件和错误消息都必须是常量。

\rSec1[qual.dir]{限定符指令}
\indextext{限定符指令}

\begin{bnf}{QualifierDirective}
    DirectiveQualifiers \terminal{::}
\end{bnf}

\begin{bnf}{DirectiveQualifiers}
    DirectiveQualifier\bnfp
\end{bnf}

\begin{bnf}{DirectiveQualifier}
    AccessQualifier
\end{bnf}

\pnum
可以使用限定符后跟\tcode{::}的方式为多个声明指定限定符。直到下一个限定符指令或该声明作用域结束为止,所有出现的声明都会受所指定的修饰符修饰。忽略无效的限定符。

\enterexample
\begin{codeblock}
public:: // 后面所有声明都是\tcode{public}的
type A = int;
const let size = 0;
\end{codeblock}